% Document: ./rm_1dvar
% Source: ./mywork/
% Generated with ROBODoc Version 4.99.36 (Jul 28 2009)
\section{1DVar/ropp\_1dvar\_cost}
\textsl{[ Functions ]}

\label{ch:robo0}
\label{ch:1DVar_ropp_1dvar_cost}
\index{unsorted!ropp\_1dvar\_cost}\index{Functions!ropp\_1dvar\_cost}
\textbf{NAME:}\hspace{0.08in}\begin{Verbatim}
    ropp_1dvar_cost - Evaluate a cost function for a one-dimensional 
                      variational retrieval of radio occultation data.
\end{Verbatim}
\textbf{SYNOPSIS:}\hspace{0.08in}\begin{Verbatim}
    call ropp_1dvar_cost(yo, bg, control, precon, J, control_ad, config, indic)
\end{Verbatim}
\textbf{DESCRIPTION:}\hspace{0.08in}\begin{Verbatim}
    This subroutine evaluates a quadratic cost function for a
    variational data assimilation procedure. 

    More specifically, this routine calculates a cost function

           1  /         |  -1 |         \
       J = - < y - H(x) | O   | y - H(x) > + 
           2  \         |     |         /

                       1  /       |  -1 |       \
                       - < x - x  | B   | x - x  >
                       2  \     b |     |      b/

    where the background state x_b is given by the state vector
    state.

    On exit, control_ad contains the gradient of J with respect to the 
    variables in control.
\end{Verbatim}
\textbf{INPUTS:}\hspace{0.08in}\begin{Verbatim}
    yo                 Observation data structure.
    bg                 Background data structure.
    control            Control variable structure.
    precon             Preconditioning matrix.
    J                  Cost function value.
    config             Configuration structure.
    indic              State variable used for the communication with the
                         calling program and the minimiser. The following
                         values are recognised as input:
                           1       Initialise arrays if additional 
                                     convergence checks are to be applied.
                           other:  Ignored
                         The following values are output if additional
                         convergence checks are to be applied:
                           0       Iteration converged
                           4       Iteration not converged
\end{Verbatim}
\textbf{OUTPUT:}\hspace{0.08in}\begin{Verbatim}
    J 
    control_ad
\end{Verbatim}
\textbf{NOTES:}\hspace{0.08in}\begin{Verbatim}
    If called with indic < 0 (which should never happen during the 
    minimization), variables for the convergence checks are reset. The 
    calculations for cost function and its gradient are nevertheless 
    performed. This allows the simultanious initialization
    of the cost function minimization and the convergence checks with a single
    call to ropp_1dvar_cost.
\end{Verbatim}
\textbf{EXAMPLE:}\hspace{0.08in}\textbf{SEE ALSO:}\hspace{0.08in}\textbf{REFERENCES:}\hspace{0.08in}\section{1DVar/ropp\_1dvar\_diag2roprof}
\textsl{[ Subroutines ]}

\label{ch:robo1}
\label{ch:1DVar_ropp_1dvar_diag2roprof}
\index{unsorted!ropp\_1dvar\_diag2roprof}\index{Subroutines!ropp\_1dvar\_diag2roprof}
\textbf{NAME:}\hspace{0.08in}\begin{Verbatim}
    ropp_1dvar_diag2roprof - Add diagnostic information gathered during a
                             1DVar retrieval to an ROProf data structure.
\end{Verbatim}
\textbf{SYNOPSIS:}\hspace{0.08in}\begin{Verbatim}
    call ropp_1dvar_diag2roprof(obs, diag, ro_data, config)
\end{Verbatim}
\textbf{INPUTS:}\hspace{0.08in}\begin{Verbatim}
    diag        - Diagnostic information structure
    ro_data     - RO data file structure
    config      - configuration options
\end{Verbatim}
\textbf{OUTPUT:}\hspace{0.08in}\begin{Verbatim}
    ro_data     - Updated RO data file structure containing diagnostics
\end{Verbatim}
\section{1DVar/ropp\_1dvar\_levmarq}
\textsl{[ Functions ]}

\label{ch:robo2}
\label{ch:1DVar_ropp_1dvar_levmarq}
\index{unsorted!ropp\_1dvar\_levmarq}\index{Functions!ropp\_1dvar\_levmarq}
\textbf{NAME:}\hspace{0.08in}\begin{Verbatim}
    ropp_1dvar_levmarq - Solve the 1DVar for background data using the
                         Levenberg-Marquardt minimiser
\end{Verbatim}
\textbf{SYNOPSIS:}\hspace{0.08in}\begin{Verbatim}
    call ropp_1dvar_levmarq(obs, bg, state, config, diag)
\end{Verbatim}
\textbf{DESCRIPTION:}\hspace{0.08in}\begin{Verbatim}
    This subroutine evaluates a quadratic cost function for a
    variational data assimilation procedure. 

    More specifically, this routine calculates a cost function

           1  /         |  -1 |         \
       J = - < y - H(x) | O   | y - H(x) > + 
           2  \         |     |         /

                       1  /       |  -1 |       \
                       - < x - x  | B   | x - x  >
                       2  \     b |     |      b/

    where the background state x_b is given by the state vector
    state. 

    A solution for x is obtained by minimising J using the Levenberg-Marquardt
    minimisation method.
\end{Verbatim}
\textbf{INPUTS:}\hspace{0.08in}\begin{Verbatim}
    obs                Observation data structure.
    bg                 Background data structure.
    state              State vector structure.
    config             Configuration structure.
    diag               Diagnostics structure.
\end{Verbatim}
\textbf{OUTPUT:}\hspace{0.08in}\begin{Verbatim}
    state 
    diag
\end{Verbatim}
\textbf{REFERENCES:}\hspace{0.08in}\begin{Verbatim}
   W.H. Press, S.A. Teukolsjy, W.T. Vetterling and B.P. Flannery,
   Numerical Recipes in C - The Art of Scientific Computing.
   2nd Ed., Cambridge University Press, 1992.
\end{Verbatim}
\section{1DVar/ropp\_1dvar\_minropp}
\textsl{[ Parameters ]}

\label{ch:robo3}
\label{ch:1DVar_ropp_1dvar_minropp}
\index{unsorted!ropp\_1dvar\_minropp}\index{Parameters!ropp\_1dvar\_minropp}
\textbf{NAME:}\hspace{0.08in}\begin{Verbatim}
    ropp_1dvar_minropp - Minimisation routine using Quasi-Newton method
\end{Verbatim}
\textbf{SYNOPSIS:}\hspace{0.08in}\begin{Verbatim}
    ropp_1dvar_minropp(state, J_grad, J_dir, dJ, gconv, niter, indic, 
                       miter, maxstore)
\end{Verbatim}
\textbf{DESCRIPTION:}\hspace{0.08in}\textbf{INPUTS:}\hspace{0.08in}\begin{Verbatim}
    state      - control vector (1st guess)
    J_grad     - penalty function gradient
    J_dir      - quasi-Newton method direction
    dJ         - expected decrease of cost function
    gconv      - decrease in norm of penalty function for convergence
    niter      - iteration counter
    indic      - communication flag
    miter      - maximum number of iterations allowed
    maxstore   - size of storage available
\end{Verbatim}
\textbf{OUTPUT:}\hspace{0.08in}\begin{Verbatim}
    state      - control vector solution
    J_grad     - penalty function gradient
    J_dir      - quasi-Newton method direction
    niter      - iteration counter
    indic      - communication flag
\end{Verbatim}
\textbf{NOTES:}\hspace{0.08in}\begin{Verbatim}
       Based on limited-memory quasi-Newton method with diagonal scaling.
       Further details on algorithm methods in  
       Gilbert, J. C., Lemarechal, C. (1989) Some numerical experiments with 
          variable storage quasi-Newton algorithms. Mathematical Programming, 
          45, 407-435  
      Nocedal, J. (1980) Updating quasi-Newton matrices with limited
          storage. Mathematics of Computation, 35, 773-782
      Based on original code written by Don Allan.

      Cost function and gradient must be computed external to this routine.
      Communication with external routines achieved using indic flag
            indic = 0   : convergence achieved, no further minimisation
            indic = 1   : first call in loop
            indic = 4   : calling program has to compute new values for J and 
                          gradient of J based for updated control vector
\end{Verbatim}
\section{1DVar/ropp\_1dvar\_read\_config}
\textsl{[ Subroutines ]}

\label{ch:robo4}
\label{ch:1DVar_ropp_1dvar_read_config}
\index{unsorted!ropp\_1dvar\_read\_config}\index{Subroutines!ropp\_1dvar\_read\_config}
\textbf{NAME:}\hspace{0.08in}\begin{Verbatim}
    ropp_1dvar_read_config - Read a configuration file for a 1DVar retrieval.
\end{Verbatim}
\textbf{SYNOPSIS:}\hspace{0.08in}\begin{Verbatim}
    call ropp_1dvar_read_config(file, config)
\end{Verbatim}
\textbf{DESCRIPTION:}\hspace{0.08in}\textbf{INPUTS:}\hspace{0.08in}\begin{Verbatim}
      file        - configuration file name
\end{Verbatim}
\textbf{OUTPUT:}\hspace{0.08in}\begin{Verbatim}
     config       - configuration options
\end{Verbatim}
\section{1DVar/ropp\_1dvar\_solve}
\textsl{[ Parameters ]}

\label{ch:robo5}
\label{ch:1DVar_ropp_1dvar_solve}
\index{unsorted!ropp\_1dvar\_solve}\index{Parameters!ropp\_1dvar\_solve}
\textbf{NAME:}\hspace{0.08in}\begin{Verbatim}
    ropp_1dvar_solve - Solve the 1DVar for background data.
\end{Verbatim}
\textbf{SYNOPSIS:}\hspace{0.08in}\begin{Verbatim}
    ropp_1dvar_solve(obs, bg, state, config, diag)
\end{Verbatim}
\textbf{DESCRIPTION:}\hspace{0.08in}\begin{Verbatim}
   Minimisation of the cost function is achieved by an iterative method which
   computes the cost function value and updates the state vector on each of
   n = 1, ..., n_iter iterations until convergence to a solution is achieved.
   The state vector is updated using the ROPP-specific minimiser minROPP. This
   implements a quasi-Newton method using a BFGS algorithm.
\end{Verbatim}
\textbf{INPUTS:}\hspace{0.08in}\begin{Verbatim}
    obs         - Observation vector
    bg          - Background vector
    state       - First guess state vector
    config      - Configuration options
\end{Verbatim}
\textbf{OUTPUT:}\hspace{0.08in}\begin{Verbatim}
    state       - Modified state vector with 1dVar solution
    diag        - Output diagnostics
\end{Verbatim}
\section{common/ropp\_1dvar\_version}
\textsl{[ Functions ]}

\label{ch:robo6}
\label{ch:common_ropp_1dvar_version}
\index{unsorted!ropp\_1dvar\_version}\index{Functions!ropp\_1dvar\_version}
\textbf{NAME:}\hspace{0.08in}\begin{Verbatim}
   ropp_1dvar_version
\end{Verbatim}
\textbf{SYNOPSIS:}\hspace{0.08in}\begin{Verbatim}
   Return ROPP_1DVAR version ID string

   USE ropp_1dvar
   version = ropp_1dvar_version()
\end{Verbatim}
\textbf{DESCRIPTION:}\hspace{0.08in}\begin{Verbatim}
   This function returns the (common) version string for the ROPP_1DVAR
   module. By default, this function should be called by all ROPP_1DVAR
   tools to display a version ID when the '-v' command-line switch is
   used.
\end{Verbatim}
\section{Datatypes/KeyConfig}
\textsl{[ Structures ]}

\label{ch:robo7}
\label{ch:Datatypes_KeyConfig}
\index{unsorted!KeyConfig}\index{Structures!KeyConfig}
\textbf{NAME:}\hspace{0.08in}\begin{Verbatim}
    KeyConfig - Key-value pairs defined in 1dVar config files
\end{Verbatim}
\textbf{SYNOPSIS:}\hspace{0.08in}\begin{Verbatim}
    use ropp_1dvar_types
      ...
    type(KeyConfig) :: key_value
\end{Verbatim}
\textbf{NOTES:}\hspace{0.08in}\textbf{SEE ALSO:}\hspace{0.08in}\textbf{SOURCE:}\hspace{0.08in}\begin{Verbatim}
  TYPE KeyConfig

     CHARACTER(len = 1024), DIMENSION(:), POINTER :: keys   => null()
     CHARACTER(len = 4096), DIMENSION(:), POINTER :: values => null()

  END TYPE KeyConfig
\end{Verbatim}
\section{Datatypes/minROPPConfig}
\textsl{[ Structures ]}

\label{ch:robo8}
\label{ch:Datatypes_minROPPConfig}
\index{unsorted!minROPPConfig}\index{Structures!minROPPConfig}
\textbf{NAME:}\hspace{0.08in}\begin{Verbatim}
    minROPPConfig - 
\end{Verbatim}
\textbf{SYNOPSIS:}\hspace{0.08in}\begin{Verbatim}
    use ropp_1dvar_types
      ...
    type(VarConfig) :: config
\end{Verbatim}
\textbf{NOTES:}\hspace{0.08in}\textbf{SEE ALSO:}\hspace{0.08in}\begin{Verbatim}
    VarConfig
\end{Verbatim}
\textbf{SOURCE:}\hspace{0.08in}\begin{Verbatim}
  TYPE minROPPConfig
     CHARACTER(len = nmax_fname) :: method       = "MINROPP"
     CHARACTER(len = nmax_fname) :: log_file     = "screen"
     INTEGER                     :: impres       =    0
     INTEGER                     :: n_iter       = 1500
     INTEGER                     :: mode         =    0
     INTEGER                     :: n_updates    =   50
     REAL(wp)                    :: eps_grad     = 1.0e-8_wp
     REAL(wp)                    :: dx_min       = 1.0e-16_wp
  END TYPE minROPPConfig
\end{Verbatim}
\section{Datatypes/VarConfig}
\textsl{[ Structures ]}

\label{ch:robo9}
\label{ch:Datatypes_VarConfig}
\index{unsorted!VarConfig}\index{Structures!VarConfig}
\textbf{NAME:}\hspace{0.08in}\begin{Verbatim}
    VarConfig - 
\end{Verbatim}
\textbf{SYNOPSIS:}\hspace{0.08in}\begin{Verbatim}
    use ropp_1dvar_types
      ...
    type(VarConfig) :: config
\end{Verbatim}
\textbf{NOTES:}\hspace{0.08in}\begin{Verbatim}
    The VarConfigType structure is composed out of several other structures; 
    see the user guide for a breakdown of the actual element names.
\end{Verbatim}
\textbf{SEE ALSO:}\hspace{0.08in}\begin{Verbatim}
    minROPPConfig
\end{Verbatim}
\textbf{SOURCE:}\hspace{0.08in}\begin{Verbatim}
  TYPE VarConfig

!    Input and output files

     ! Input observation file
     CHARACTER(len = nmax_fname) :: obs_file         = "ropp_obs.nc"      
     ! Observation error covariance method
     CHARACTER(len = nmax_fname) :: obs_covar_method = "VSDC"             
     ! Observation error correlation file
     CHARACTER(len = nmax_fname) :: obs_corr_file    = "ropp_obs_corr.nc" 
     ! Input background file
     CHARACTER(len = nmax_fname) :: bg_file          = "ropp_bg.nc"       
     ! Output file
     CHARACTER(len = nmax_fname) :: out_file         = "ropp_out.nc"      
     ! Background error covariance method
     CHARACTER(len = nmax_fname) :: bg_covar_method  = "VSFC"             
     ! Background error correlation file
     CHARACTER(len = nmax_fname) :: bg_corr_file     = "ropp_bg_corr.nc"  

!    Observation cutoff range

     ! Minimum obs height to include in 1dVar analysis (km)
     REAL(wp)                    :: min_1dvar_height         = -10.0_wp 
     ! Maximum obs height to include in 1dVar analysis (km)
     REAL(wp)                    :: max_1dvar_height         = 60.0_wp   

!    Generic quality control

     ! Apply obs vs. bg colocation check?
     LOGICAL                     :: genqc_colocation_apply     = .TRUE.      
     ! Maximum obs vs. bg great circle distance (km)
     REAL(wp)                    :: genqc_max_distance         = 300.0_wp    
     ! Maximum obs vs. bg temporal seperation (sec)
     REAL(wp)                    :: genqc_max_time_sep         = 10800.0_wp    
     ! Minimium temperature value (K)
     REAL(wp)                    :: genqc_min_temperature      = 150.0_wp
     ! Maximium temperature value (K)
     REAL(wp)                    :: genqc_max_temperature      = 350.0_wp
     ! Minimium specific humidity value (g/kg)
     REAL(wp)                    :: genqc_min_spec_humidity    =  0.0_wp
     ! Maximium specific humidity value (g/kg)
     REAL(wp)                    :: genqc_max_spec_humidity    = 50.0_wp
     ! Minimium impact parameter value (m)
     REAL(wp)                    :: genqc_min_impact           = 6.2e6_wp   
     ! Maximium impact parameter value (m)
     REAL(wp)                    :: genqc_max_impact           = 6.6e6_wp  
     ! Minimium bending angle value (rad)
     REAL(wp)                    :: genqc_min_bangle           = -1.0e-4_wp
     ! Maximium bending angle value (rad)
     REAL(wp)                    :: genqc_max_bangle           =  0.1_wp 
     ! Minimium geopotential height for refractivity value (m)
     REAL(wp)                    :: genqc_min_geop_refrac      = -1.0e3_wp
     ! Maximum geopotential height for refractivity value (m)
     REAL(wp)                    :: genqc_max_geop_refrac      = 1.e5_wp  
     ! Minimium refractivity value (N-units)
     REAL(wp)                    :: genqc_min_refractivity     = 0.0_wp 
     ! Maximium refractivity value (N-units)
     REAL(wp)                    :: genqc_max_refractivity     = 500.0_wp  
     ! Minimum threshold observation height (m)
     REAL(wp)                    :: genqc_min_obheight         = 20000.0_wp

     !    Background quality control

     ! Apply background quality control?
     LOGICAL                     :: bgqc_apply                 = .TRUE.    
     ! Data rejected if O-B > factor * sigma
     REAL(wp)                    :: bgqc_reject_factor         = 10.0_wp    
     ! Maximum percentage of data rejected
     REAL(wp)                    :: bgqc_reject_max_percent    = 50.0_wp     

!    Probability of Gross Error

     ! Apply PGE for quality control?
     LOGICAL                     :: pge_apply                  = .FALSE.    
     ! First guess PGE
     REAL(wp)                    :: pge_fg                     = 0.001_wp  
     ! Width of gross error plateau
     REAL(wp)                    :: pge_d                      = 10.0_wp   

!    Minimiser

     TYPE(minROPPConfig)         :: minROPP
     LOGICAL                     :: use_precond                = .TRUE.

!    Additional convergence checks

     ! Apply additional convergence checks?
     LOGICAL                     :: conv_check_apply           = .TRUE.    
     ! Minimum number of iterations required
     INTEGER                     :: conv_check_n_previous      = 2         
     ! State vector must change less than this * bg error
     REAL(wp)                    :: conv_check_max_delta_state = 0.1_wp    
     ! Cost function must change less than this
     REAL(wp)                    :: conv_check_max_delta_J     = 0.1_wp     

!    Additional output

     LOGICAL                     :: extended_1dvar_diag     = .FALSE.

!    Log(pressure) and Log(humidity) calculations

     LOGICAL                     :: use_logp                = .FALSE.
     LOGICAL                     :: use_logq                = .FALSE.

!    compressibility 
     
     LOGICAL                     :: compress                = .FALSE. 

  END TYPE VarConfig
\end{Verbatim}
\section{Datatypes/VarDiag}
\textsl{[ Structures ]}

\label{ch:robo10}
\label{ch:Datatypes_VarDiag}
\index{unsorted!VarDiag}\index{Structures!VarDiag}
\textbf{NAME:}\hspace{0.08in}\begin{Verbatim}
    VarDiag - Diagnostics of a 1DVar retrieval.
\end{Verbatim}
\textbf{SYNOPSIS:}\hspace{0.08in}\begin{Verbatim}
    use ropp_1dvar_types
      ...
    type(VarDiag) :: diag
\end{Verbatim}
\textbf{NOTES:}\hspace{0.08in}\begin{Verbatim}
    The VarDiag structure is composed out of several other structures; 
    see the user guide for a breakdown of the actual element names.
\end{Verbatim}
\textbf{SEE ALSO:}\hspace{0.08in}\textbf{SOURCE:}\hspace{0.08in}\begin{Verbatim}
  TYPE VarDiag
     INTEGER                         :: n_data
     INTEGER                         :: n_bgqc_reject
     INTEGER                         :: n_pge_reject
     TYPE(Obs1DBangle)               :: bg_bangle
     TYPE(Obs1DRefrac)               :: bg_refrac
     REAL(wp), DIMENSION(:), POINTER :: OmB         => null() ! O - B
     REAL(wp), DIMENSION(:), POINTER :: OmB_sigma   => null() ! O - B std dev
     REAL(wp)                        :: pge_gamma             ! PGE gamma value
     REAL(wp), DIMENSION(:), POINTER :: pge         => null() ! PGE of obs
     REAL(wp), DIMENSION(:), POINTER :: pge_weights => null() ! PGE weights
     LOGICAL                         :: ok             ! Overall quality flag
     REAL(wp)                        :: J              ! Value of cost function
     REAL(wp)                        :: J_scaled ! Value of cost (scaled, 2J/m)
     REAL(wp)                        :: J_init ! Initial cost function
     REAL(wp), DIMENSION(:), POINTER :: J_bgr => null() ! BG cost profile
     REAL(wp), DIMENSION(:), POINTER :: J_obs => null() ! OB cost profile
     REAL(wp), DIMENSION(:), POINTER :: B_sigma => null() ! Bg obs std dev
     INTEGER                         :: n_iter         ! Number of iterations
     INTEGER                         :: n_simul        ! Number of simulations
     INTEGER                         :: min_mode       ! Minimiser exit mode
     TYPE(Obs1DBangle)               :: res_bangle     ! Analysis bending angle
     TYPE(Obs1DRefrac)               :: res_refrac     ! Analysis refractivity
     REAL(wp), DIMENSION(:), POINTER :: OmA         => null() ! O - A
     REAL(wp), DIMENSION(:), POINTER :: OmA_sigma   => null() ! O - A std dev
  END TYPE VarDiag
\end{Verbatim}
\section{Diagnostics/ropp\_1dvar\_diagnostics}
\textsl{[ Subroutines ]}

\label{ch:robo11}
\label{ch:Diagnostics_ropp_1dvar_diagnostics}
\index{unsorted!ropp\_1dvar\_diagnostics}\index{Subroutines!ropp\_1dvar\_diagnostics}
\textbf{NAME:}\hspace{0.08in}\begin{Verbatim}
    ropp_1dvar_diagnostics - 1DVar postprocessing and diagnostics.
\end{Verbatim}
\textbf{SYNOPSIS:}\hspace{0.08in}\begin{Verbatim}
    call rop_1dvar_diagnostics(obs, state, config, diag)
\end{Verbatim}
\textbf{DESCRIPTION:}\hspace{0.08in}\begin{Verbatim}
    This subroutine performs some postprocessing and diagnostic
    calculations for a 1DVar. In particular,

       - O - A differences for the solution
       - error estimates for the solution

    are calculated and stored into the respective members of the
    diagnostic structure 'diag' as well as into the state vector's
    error covariance matrix.
\end{Verbatim}
\textbf{INPUTS:}\hspace{0.08in}\begin{Verbatim}
    obs
    bg
    config
\end{Verbatim}
\textbf{OUTPUT:}\hspace{0.08in}\begin{Verbatim}
    bg
    diag
\end{Verbatim}
\textbf{NOTES:}\hspace{0.08in}\begin{Verbatim}
    The calculation of the expected errors in O - A is currently open.
\end{Verbatim}
\textbf{EXAMPLE:}\hspace{0.08in}\textbf{SEE ALSO:}\hspace{0.08in}\textbf{REFERENCES:}\hspace{0.08in}\section{Error covariances/ropp\_1dvar\_covar\_bangle}
\textsl{[ Subroutines ]}

\label{ch:robo12}
\label{ch:Error_covariances_ropp_1dvar_covar_bangle}
\index{unsorted!ropp\_1dvar\_covar\_bangle}\index{Subroutines!ropp\_1dvar\_covar\_bangle}
\textbf{NAME:}\hspace{0.08in}\begin{Verbatim}
    ropp_1dvar_covar_bangle - Set up covariance matrices for a bending angle
                                observation profile.
\end{Verbatim}
\textbf{SYNOPSIS:}\hspace{0.08in}\begin{Verbatim}
    call ropp_1dvar_covar(obs, obs_covar_config)
\end{Verbatim}
\textbf{DESCRIPTION:}\hspace{0.08in}\begin{Verbatim}
    This subroutine sets up an error covariance matrix for a vector of bending
    angle observations. This particular routine provides the implementation for
    an overloaded interface to various background and observation data types.
\end{Verbatim}
\textbf{INPUTS:}\hspace{0.08in}\begin{Verbatim}
    type(Obs1dBangle)   :: obs  Bending angle observation vector
\end{Verbatim}
\textbf{OUTPUT:}\hspace{0.08in}\begin{Verbatim}
    The observation vector's error covariance elements are
    filled or updated according to the settings in the configuration
    structures. 
\end{Verbatim}
\textbf{NOTES:}\hspace{0.08in}\begin{Verbatim}
    Bending angle error covariances can be constructed using the following 
    methods:

       FSFC   Fixed Sigmas, Fixed correlations: Both error correlations and
                and error standard deviations are read from an observation
                error correlation file. The error correlation file must 
                contain both the error correlation matrix as well as the
                standard deviations (errors) for all observation vector
                elements.

       VSDC   Variable Sigmas, Diagonal Correlations: A diagonal error 
                correlation structure (i.e., no error correlations) is assumed,
                while per profile error estimates as contained in the 
                observation data file are used. In this case, no error 
                correlation / covariance data file is required.

       VSFC   Variable Sigmas, Fixed Correlations: Error correlations are read
                from an error correlation file, while per profile error 
                estimates as contained in the observation data file are used.
                In this case, the error correlation / covariance data files
                only require to contain the error correlations.

    Note that error correlation files may contain latitudinally binned error
    correlations and standard deviations, allowing to have latitudinally 
    varying error correlation structures and standard deviations in the FSFC  
    and VSFC scenarios.
\end{Verbatim}
\textbf{EXAMPLE:}\hspace{0.08in}\textbf{SEE ALSO:}\hspace{0.08in}\begin{Verbatim}
    ropp_1dvar_covar_bg
    ropp_1dvar_covar_refrac
    ropp_1dvar_covar_bangle
\end{Verbatim}
\textbf{REFERENCES:}\hspace{0.08in}\section{Error covariances/ropp\_1dvar\_covar\_bg}
\textsl{[ Subroutines ]}

\label{ch:robo13}
\label{ch:Error_covariances_ropp_1dvar_covar_bg}
\index{unsorted!ropp\_1dvar\_covar\_bg}\index{Subroutines!ropp\_1dvar\_covar\_bg}
\textbf{NAME:}\hspace{0.08in}\begin{Verbatim}
    ropp_1dvar_covar_bg - Set up covariance matrices for background data.
\end{Verbatim}
\textbf{SYNOPSIS:}\hspace{0.08in}\begin{Verbatim}
    call ropp_1dvar_covar(bg,  bg_covar_config)
    call ropp_1dvar_covar(obs, obs_covar_config)
\end{Verbatim}
\textbf{DESCRIPTION:}\hspace{0.08in}\begin{Verbatim}
    This subroutine sets up an error covariance matrix for a background
    or an observation vector. It is an overloaded interface to various 
    specialised error covariance routines.
\end{Verbatim}
\textbf{INPUTS:}\hspace{0.08in}\begin{Verbatim}
    type(State1dFM) :: bg   Background state vector
\end{Verbatim}
\textbf{OUTPUT:}\hspace{0.08in}\begin{Verbatim}
    The state vector's error covariance elements are
    filled or updated according to the settings in the configuration
    structures. 
\end{Verbatim}
\textbf{NOTES:}\hspace{0.08in}\begin{Verbatim}
    Background covariances can be constructed using the following methods:

       FSFC   Fixed Sigmas, Fixed correlations: Both error correlations and
                and error standard deviations are read from a background
                error correlation file. The error correlation file must 
                contain both the error correlation matrix as well as the
                standard deviations (errors) for all background state vector
                elements.

       VSFC   Variable Sigmas, Fixed Correlations: Error correlations are read
                from an error correlation file, while per profile error 
                estimates as contained in the background data file are used.
                In this case, the error correlation / covariance data files
                only require to contain the error correlations.

    Note that error correlation files may contain latitudinally binned error
    correlations and standard deviations, allowing to have latitudinally 
    varying error correlation structures and standard deviations even in the 
    FCFS scenario.
\end{Verbatim}
\textbf{EXAMPLE:}\hspace{0.08in}\textbf{SEE ALSO:}\hspace{0.08in}\begin{Verbatim}
    ropp_1dvar_covar_bg
    ropp_1dvar_covar_refrac
    ropp_1dvar_covar_bangle
\end{Verbatim}
\textbf{REFERENCES:}\hspace{0.08in}\section{Error covariances/ropp\_1dvar\_covar\_refrac}
\textsl{[ Subroutines ]}

\label{ch:robo14}
\label{ch:Error_covariances_ropp_1dvar_covar_refrac}
\index{unsorted!ropp\_1dvar\_covar\_refrac}\index{Subroutines!ropp\_1dvar\_covar\_refrac}
\textbf{NAME:}\hspace{0.08in}\begin{Verbatim}
    ropp_1dvar_covar_refrac - Set up covariance matrices for a refractivity
                                observation profile.
\end{Verbatim}
\textbf{SYNOPSIS:}\hspace{0.08in}\begin{Verbatim}
    call ropp_1dvar_covar(obs, obs_covar_config)
\end{Verbatim}
\textbf{DESCRIPTION:}\hspace{0.08in}\begin{Verbatim}
    This subroutine sets up an error covariance matrix for a vector of 
    refractivity observations. This particular routine provides the 
    implementation for an overloaded interface to various background and 
    observation data types.
\end{Verbatim}
\textbf{INPUTS:}\hspace{0.08in}\begin{Verbatim}
    type(Obs1dRefrac)   :: obs  Refractivity observation vector
\end{Verbatim}
\textbf{OUTPUT:}\hspace{0.08in}\begin{Verbatim}
    The observation vector's error covariance elements are
    filled or updated according to the settings in the configuration
    structures. 
\end{Verbatim}
\textbf{NOTES:}\hspace{0.08in}\begin{Verbatim}
    Background covariances can be constructed using the following methods:

       FSFC   Fixed Sigmas, Fixed correlations: Both error correlations and
                and error standard deviations are read from an observation
                error correlation file. The error correlation file must 
                contain both the error correlation matrix as well as the
                standard deviations (errors) for all observation vector
                elements.

       VSDC   Variable Sigmas, Diagonal Correlations: A diagonal error 
                correlation structure (i.e., no error correlations) is assumed,
                while per profile error estimates as contained in the 
                observation data file are used. In this case, no error 
                correlation / covariance data file is required.

       VSFC   Variable Sigmas, Fixed Correlations: Error correlations are read
                from an error correlation file, while per profile error 
                estimates as contained in the observation data file are used.
                In this case, the error correlation / covariance data files
                only require to contain the error correlations.

    Note that error correlation files may contain latitudinally binned error
    correlations and standard deviations, allowing to have latitudinally 
    varying error correlation structures and standard deviations in the FSFC  
    and VSFC scenarios.
\end{Verbatim}
\textbf{EXAMPLE:}\hspace{0.08in}\textbf{SEE ALSO:}\hspace{0.08in}\begin{Verbatim}
    ropp_1dvar_covar_bg
    ropp_1dvar_covar_refrac
    ropp_1dvar_covar_bangle
\end{Verbatim}
\textbf{REFERENCES:}\hspace{0.08in}\section{Interface/Modules}
\textsl{[ Topics ]}

\label{ch:robo15}
\label{ch:Interface_Modules}
\index{unsorted!Modules}\index{Topics!Modules}
\textbf{SYNOPSIS:}\hspace{0.08in}\begin{Verbatim}
    use ropp_1dvar

    use ropp_fm_types
    use ropp_fm_constants
\end{Verbatim}
\textbf{DESCRIPTION:}\hspace{0.08in}\begin{Verbatim}
    Access to the routines in the ROPP 1dVar library ropp_1dvar is
    provided by the single Fortran 90 module ropp_1dvar. The ropp_1dvar module
    also includes the module ropp_1dvar_types, which contains derived type
    (structure) declarations used by the 1dVar routines.

    Meteorological and physical constants are provided in the module
    ropp_1dvar_constants; this module is loaded with the ropp_1dvar module.
\end{Verbatim}
\textbf{SEE ALSO:}\hspace{0.08in}\begin{Verbatim}
    ropp_1dvar
    ropp_1dvar_types
    ropp_1dvar_constants
    ropp_1dvar_copy
\end{Verbatim}
\subsection{Modules/matrix}
\textsl{[ Modules ]}
\textsl{[ Modules ]}

\label{ch:robo35}
\label{ch:Modules_matrix}
\index{unsorted!matrix}\index{Modules!matrix}
\textbf{NAME:}\hspace{0.08in}\begin{Verbatim}
    matrix - Matrix routines and functions.
\end{Verbatim}
\textbf{SYNOPSIS:}\hspace{0.08in}\begin{Verbatim}
    use matrix
\end{Verbatim}
\textbf{DESCRIPTION:}\hspace{0.08in}\begin{Verbatim}
    This Fortran module provides interfaces and data types required for
    matrix operations.
\end{Verbatim}
\textbf{NOTES:}\hspace{0.08in}\textbf{SEE ALSO:}\hspace{0.08in}\subsection{Modules/ropp\_1dvar}
\textsl{[ Modules ]}
\textsl{[ Modules ]}

\label{ch:robo36}
\label{ch:Modules_ropp_1dvar}
\index{unsorted!ropp\_1dvar}\index{Modules!ropp\_1dvar}
\textbf{NAME:}\hspace{0.08in}\begin{Verbatim}
    ropp_1dvar - Interface module for the ROPP 1DVar implementations.
\end{Verbatim}
\textbf{SYNOPSIS:}\hspace{0.08in}\begin{Verbatim}
    use ropp_1dvar
\end{Verbatim}
\textbf{DESCRIPTION:}\hspace{0.08in}\begin{Verbatim}
    This module provides interfaces for all 1DVar functions and routines
    in the ROPP 1DVar library.
\end{Verbatim}
\textbf{NOTES:}\hspace{0.08in}\textbf{SEE ALSO:}\hspace{0.08in}\subsection{Modules/ropp\_1dvar\_copy}
\textsl{[ Modules ]}
\textsl{[ Modules ]}

\label{ch:robo37}
\label{ch:Modules_ropp_1dvar_copy}
\index{unsorted!ropp\_1dvar\_copy}\index{Modules!ropp\_1dvar\_copy}
\textbf{NAME:}\hspace{0.08in}\begin{Verbatim}
    ropp_1dvar_copy - Interface module for the ROPP 1dVar
\end{Verbatim}
\textbf{SYNOPSIS:}\hspace{0.08in}\begin{Verbatim}
    use ropp_1dVar_copy
\end{Verbatim}
\textbf{DESCRIPTION:}\hspace{0.08in}\begin{Verbatim}
    Data type/structure copying functions using ROprof structures used by the
    forward models of ROPP and converting units within the ROprof structure.
\end{Verbatim}
\textbf{SEE ALSO:}\hspace{0.08in}\begin{Verbatim}
    ropp_1dvar_diag2roprof
\end{Verbatim}
\subsection{Modules/ropp\_1dvar\_types}
\textsl{[ Modules ]}
\textsl{[ Modules ]}

\label{ch:robo38}
\label{ch:Modules_ropp_1dvar_types}
\index{unsorted!ropp\_1dvar\_types}\index{Modules!ropp\_1dvar\_types}
\textbf{NAME:}\hspace{0.08in}\begin{Verbatim}
    ropp_1dvar_types - Type declarations for the ROPP 1DVAR libary.
\end{Verbatim}
\textbf{SYNOPSIS:}\hspace{0.08in}\begin{Verbatim}
    use ropp_1dvar
\end{Verbatim}
\textbf{DESCRIPTION:}\hspace{0.08in}\begin{Verbatim}
    This Fortran module supports the ROPP 1DVar library and provides
    derived data types used by the ropp_1dvar library. It must be loaded
    for all applications using the ropp_io library / package. Note that
    loading the ropp_1dvar module includes the ropp_1dvar_types module.
\end{Verbatim}
\textbf{NOTES:}\hspace{0.08in}\textbf{SEE ALSO:}\hspace{0.08in}\subsection{Modules/ropp\_qc}
\textsl{[ Modules ]}
\textsl{[ Modules ]}

\label{ch:robo39}
\label{ch:Modules_ropp_qc}
\index{unsorted!ropp\_qc}\index{Modules!ropp\_qc}
\textbf{NAME:}\hspace{0.08in}\begin{Verbatim}
    ropp_qc - Interface module for the ROPP Quality Control implementations.
\end{Verbatim}
\textbf{SYNOPSIS:}\hspace{0.08in}\begin{Verbatim}
    use ropp_qc
\end{Verbatim}
\textbf{DESCRIPTION:}\hspace{0.08in}\begin{Verbatim}
    This module provides interfaces for all quality control routines contained
    in the ROPP 1DVar library.
\end{Verbatim}
\textbf{NOTES:}\hspace{0.08in}\textbf{SEE ALSO:}\hspace{0.08in}\section{Matrices/matrix\_bm2full}
\textsl{[ Subroutines ]}

\label{ch:robo16}
\label{ch:Matrices_matrix_bm2full}
\index{unsorted!matrix\_bm2full}\index{Subroutines!matrix\_bm2full}
\textbf{NAME:}\hspace{0.08in}\begin{Verbatim}
    matrix_bm2full - Convert a symmetric or hermitian matrix from
                     Lapack's banded into full (general) form.
\end{Verbatim}
\textbf{SYNOPSIS:}\hspace{0.08in}\begin{Verbatim}
    call matrix_bm2full(banded, ku, matrix)
\end{Verbatim}
\textbf{DESCRIPTION:}\hspace{0.08in}\begin{Verbatim}
    This subroutine copies a symmetric matrix in Lapack's packed form
    into full (or general) form.
\end{Verbatim}
\textbf{INPUTS:}\hspace{0.08in}\begin{Verbatim}
    banded    Matrix in banded form.
    ku        Number of upper side diagonals.
\end{Verbatim}
\textbf{OUTPUT:}\hspace{0.08in}\begin{Verbatim}
    matrix    Matrix in full (general) form.
\end{Verbatim}
\textbf{NOTES:}\hspace{0.08in}\begin{Verbatim}
    The number of lower side diagonals will be computed from the shape
    of the full matrix. If the dimensions of the full matrix do not equal
    the shape of the original matrix, the conversion will give a wrong
    result.

    See the Lapack User Guide for details on matrix storage systems 
    supported by Lapack. 
\end{Verbatim}
\textbf{EXAMPLE:}\hspace{0.08in}\begin{Verbatim}
    To convert a general symmetric matrix with ku side diagonals into 
    banded form, use

       call matrix_full2bm(gen_matrix, ku, banded_matrix)

    A full matrix can be obtained from a banded symmetric matrix via

       call matrix_bm2full(banded_matrix, ku, gen_matrix)
\end{Verbatim}
\textbf{SEE ALSO:}\hspace{0.08in}\begin{Verbatim}
    matrix_pp2full
    matrix_pp2full_alloc

    matrix_full2pp
    matrix_full2pp_alloc

    matrix_bm2full
    matrix_bm2full_alloc

    matrix_full2bm
    matrix_full2bm_alloc
\end{Verbatim}
\textbf{REFERENCES:}\hspace{0.08in}\begin{Verbatim}
    E. Anderson, Z. Bai, C. Bischof, S. Blackford, J. Demmel, J. Dongarra,
       J. Du Croz, A. Greenbaum, S. Hammarling, A. McKenney, D. Sorensen,
       LAPACK Users' Guide, 3rd Ed., SIAM, 1999.
\end{Verbatim}
\section{Matrices/matrix\_bm2full\_alloc}
\textsl{[ Subroutines ]}

\label{ch:robo17}
\label{ch:Matrices_matrix_bm2full_alloc}
\index{unsorted!matrix\_bm2full\_alloc}\index{Subroutines!matrix\_bm2full\_alloc}
\textbf{NAME:}\hspace{0.08in}\begin{Verbatim}
    matrix_bm2full_alloc - Convert a symmetric or hermitian matrix from 
                           Lapack's packed into full (general) form.
\end{Verbatim}
\textbf{SYNOPSIS:}\hspace{0.08in}\begin{Verbatim}
    call matrix_bm2full_alloc(banded, m, ku, matrix)
\end{Verbatim}
\textbf{DESCRIPTION:}\hspace{0.08in}\begin{Verbatim}
    This subroutine copies a symmetric matrix in Lapack's banded form 
    into full (or general) form. The general matrix to be filled is allocated.
\end{Verbatim}
\textbf{INPUTS:}\hspace{0.08in}\begin{Verbatim}
    banded    Matrix in banded form.
    m         Number rows of the full matrix.
    ku        Number of upper side diagonals.
\end{Verbatim}
\textbf{OUTPUT:}\hspace{0.08in}\begin{Verbatim}
    matrix    Pointer to a matrix in full (general) form.
\end{Verbatim}
\textbf{NOTES:}\hspace{0.08in}\begin{Verbatim}
    The number of lower side diagonals will be computed from the number m
    of rows of the full matrix. If the number of rows does not equal the
    shape of the original matrix, the conversion will give a wrong result.

    The matrix argument is a 2d Fortran pointer which is allocated within
    this routine. If the pointer has been used before, all data will be lost.

    See the Lapack User Guide for details on matrix storage systems supported 
    by Lapack. 
\end{Verbatim}
\textbf{EXAMPLE:}\hspace{0.08in}\begin{Verbatim}
    To convert a general symmetric matrix into packed form, use

       call matrix_full2pp(gen_matrix, packed_matrix)

    A full matrix can be obtained from a packed positive definitive matrix via

       call matrix_pp2full(packed_matrix, gen_matrix)
\end{Verbatim}
\textbf{SEE ALSO:}\hspace{0.08in}\begin{Verbatim}
    matrix_pp2full

    matrix_full2pp
    matrix_full2pp_alloc

    matrix_bm2full
    matrix_bm2full_alloc

    matrix_full2bm
    matrix_full2bm_alloc
\end{Verbatim}
\textbf{REFERENCES:}\hspace{0.08in}\begin{Verbatim}
    E. Anderson, Z. Bai, C. Bischof, S. Blackford, J. Demmel, J. Dongarra,
       J. Du Croz, A. Greenbaum, S. Hammarling, A. McKenney, D. Sorensen,
       LAPACK Users' Guide, 3rd Ed., SIAM, 1999.
\end{Verbatim}
\section{Matrices/matrix\_full2bm}
\textsl{[ Subroutines ]}

\label{ch:robo18}
\label{ch:Matrices_matrix_full2bm}
\index{unsorted!matrix\_full2bm}\index{Subroutines!matrix\_full2bm}
\textbf{NAME:}\hspace{0.08in}\begin{Verbatim}
    matrix_full2bm - Convert a symmetric or hermitian matrix in full
                     (or general) form into Lapack's banded form.
\end{Verbatim}
\textbf{SYNOPSIS:}\hspace{0.08in}\begin{Verbatim}
    call matrix_full2bm(matrix, ku, banded)
\end{Verbatim}
\textbf{DESCRIPTION:}\hspace{0.08in}\begin{Verbatim}
    This subroutine copies a symmetric matrix into Lapack's banded form.
\end{Verbatim}
\textbf{INPUTS:}\hspace{0.08in}\begin{Verbatim}
    matrix    Matrix in full (general) form.
    ku        Number of upper side diagonals.
\end{Verbatim}
\textbf{OUTPUT:}\hspace{0.08in}\begin{Verbatim}
    banded    Matrix in banded form.
\end{Verbatim}
\textbf{NOTES:}\hspace{0.08in}\begin{Verbatim}
    See the Lapack User Guide for details on matrix storage systems
    supported by Lapack. 
\end{Verbatim}
\textbf{EXAMPLE:}\hspace{0.08in}\begin{Verbatim}
    To convert a general symmetric matrix with ku side diagonals into 
    banded form, use

       call matrix_full2bm(gen_matrix, ku, banded_matrix)

    A full matrix can be obtained from a banded symmetric matrix via

       call matrix_bm2full(banded_matrix, ku, gen_matrix)
\end{Verbatim}
\textbf{SEE ALSO:}\hspace{0.08in}\begin{Verbatim}
    matrix_pp2full
    matrix_pp2full_alloc

    matrix_full2pp
    matrix_full2pp_alloc

    matrix_bm2full
    matrix_bm2full_alloc

    matrix_full2bm
    matrix_full2bm_alloc
\end{Verbatim}
\textbf{REFERENCES:}\hspace{0.08in}\begin{Verbatim}
    E. Anderson, Z. Bai, C. Bischof, S. Blackford, J. Demmel, J. Dongarra,
       J. Du Croz, A. Greenbaum, S. Hammarling, A. McKenney, D. Sorensen,
       LAPACK Users' Guide, 3rd Ed., SIAM, 1999.
\end{Verbatim}
\section{Matrices/matrix\_full2bm\_alloc}
\textsl{[ Subroutines ]}

\label{ch:robo19}
\label{ch:Matrices_matrix_full2bm_alloc}
\index{unsorted!matrix\_full2bm\_alloc}\index{Subroutines!matrix\_full2bm\_alloc}
\textbf{NAME:}\hspace{0.08in}\begin{Verbatim}
    matrix_full2bm_alloc - Convert a symmetric or hermitian matrix in full
                           (or general) form into Lapack's banded form.
\end{Verbatim}
\textbf{SYNOPSIS:}\hspace{0.08in}\begin{Verbatim}
    call matrix_full2bm_alloc(matrix, ku, kl, banded)
\end{Verbatim}
\textbf{DESCRIPTION:}\hspace{0.08in}\begin{Verbatim}
    This subroutine copies a symmetric matrix into Lapack's banded form. 
    The banded matrix to be filled is allocated.
\end{Verbatim}
\textbf{INPUTS:}\hspace{0.08in}\begin{Verbatim}
    matrix    Matrix in full (general) form.
    ku        Number of upper side diagonals.
    kl        Number of lower side diagonals.
\end{Verbatim}
\textbf{OUTPUT:}\hspace{0.08in}\begin{Verbatim}
    banded    Matrix in banded form.
\end{Verbatim}
\textbf{NOTES:}\hspace{0.08in}\begin{Verbatim}
    The banded argument is a 2d Fortran pointer which is allocated within
    this routine. If the pointer has been used before, all data will be lost.

    See the Lapack User Guide for details on matrix storage systems 
    supported by Lapack. 
\end{Verbatim}
\textbf{EXAMPLE:}\hspace{0.08in}\begin{Verbatim}
    To convert a general symmetric matrix with ku side diagonals into 
    banded form, use

       call matrix_full2bm_alloc(gen_matrix, ku, kl, banded_matrix)

    A full matrix can be obtained from a banded symmetric matrix via

       call matrix_bm2full(banded_matrix, ku, gen_matrix)
\end{Verbatim}
\textbf{SEE ALSO:}\hspace{0.08in}\begin{Verbatim}
    matrix_pp2full
    matrix_pp2full_alloc

    matrix_full2pp
    matrix_full2pp_alloc

    matrix_bm2full
    matrix_bm2full_alloc

    matrix_full2bm
    matrix_full2bm_alloc
\end{Verbatim}
\textbf{REFERENCES:}\hspace{0.08in}\begin{Verbatim}
    E. Anderson, Z. Bai, C. Bischof, S. Blackford, J. Demmel, J. Dongarra,
       J. Du Croz, A. Greenbaum, S. Hammarling, A. McKenney, D. Sorensen,
       LAPACK Users' Guide, 3rd Ed., SIAM, 1999.
\end{Verbatim}
\section{Matrices/matrix\_full2pp}
\textsl{[ Subroutines ]}

\label{ch:robo20}
\label{ch:Matrices_matrix_full2pp}
\index{unsorted!matrix\_full2pp}\index{Subroutines!matrix\_full2pp}
\textbf{NAME:}\hspace{0.08in}\begin{Verbatim}
    matrix_full2pp - Convert a symmetric or hermitian matrix into full
                     (or general) form into Lapack's packed form.
\end{Verbatim}
\textbf{SYNOPSIS:}\hspace{0.08in}\begin{Verbatim}
    call matrix_full2pp(matrix, packed [, uplo])
\end{Verbatim}
\textbf{DESCRIPTION:}\hspace{0.08in}\begin{Verbatim}
    This subroutine copies a symmetric matrix into Lapack's packed form.
\end{Verbatim}
\textbf{INPUTS:}\hspace{0.08in}\begin{Verbatim}
    matrix    Matrix in full (general) form.
    uplo      'UPLO' parameter; determines if the packed matrix has been packed
                 from an upper ('U') or lower ('L') full matrix. This parameter
                 is optional; it defaults to 'U'.
\end{Verbatim}
\textbf{OUTPUT:}\hspace{0.08in}\begin{Verbatim}
    packed    Matrix in packed form.
\end{Verbatim}
\textbf{NOTES:}\hspace{0.08in}\begin{Verbatim}
    See the Lapack User Guide for details on matrix storage systems
    supported by Lapack. 
\end{Verbatim}
\textbf{EXAMPLE:}\hspace{0.08in}\begin{Verbatim}
    To convert a general symmetric matrix into a packed form, use

       call matrix_full2pp(gen_matrix, packed_matrix)

    A full matrix can be obtained from a packed symmetric matrix via

       call matrix_pp2full(packed_matrix, gen_matrix)
\end{Verbatim}
\textbf{SEE ALSO:}\hspace{0.08in}\begin{Verbatim}
    matrix_pp2full
    matrix_pp2full_alloc

    matrix_full2pp
    matrix_full2pp_alloc

    matrix_bm2full
    matrix_bm2full_alloc

    matrix_full2bm
    matrix_full2bm_alloc
\end{Verbatim}
\textbf{REFERENCES:}\hspace{0.08in}\begin{Verbatim}
    E. Anderson, Z. Bai, C. Bischof, S. Blackford, J. Demmel, J. Dongarra,
       J. Du Croz, A. Greenbaum, S. Hammarling, A. McKenney, D. Sorensen,
       LAPACK Users' Guide, 3rd Ed., SIAM, 1999.
\end{Verbatim}
\section{Matrices/matrix\_full2pp\_alloc}
\textsl{[ Subroutines ]}

\label{ch:robo21}
\label{ch:Matrices_matrix_full2pp_alloc}
\index{unsorted!matrix\_full2pp\_alloc}\index{Subroutines!matrix\_full2pp\_alloc}
\textbf{NAME:}\hspace{0.08in}\begin{Verbatim}
    matrix_full2pp_alloc - Convert a symmetric or hermitian matrix into full
                     (or general) form into Lapack's packed form.
\end{Verbatim}
\textbf{SYNOPSIS:}\hspace{0.08in}\begin{Verbatim}
    call matrix_full2pp_alloc(matrix, packed [, uplo])
\end{Verbatim}
\textbf{DESCRIPTION:}\hspace{0.08in}\begin{Verbatim}
    This subroutine copies a symmetric matrix into Lapack's packed form.
    The packed matrix to be filled is allocated.
\end{Verbatim}
\textbf{INPUTS:}\hspace{0.08in}\begin{Verbatim}
    matrix    Matrix in full (general) form.
    uplo      'UPLO' parameter; determines if the packed matrix has been packed
                 from an upper ('U') or lower ('L') full matrix. This parameter
                 is optional; it defaults to 'U'.
\end{Verbatim}
\textbf{OUTPUT:}\hspace{0.08in}\begin{Verbatim}
    packed    Matrix in packed form.
\end{Verbatim}
\textbf{NOTES:}\hspace{0.08in}\begin{Verbatim}
    See the Lapack User Guide for details on matrix storage systems
    supported by Lapack. 
\end{Verbatim}
\textbf{EXAMPLE:}\hspace{0.08in}\begin{Verbatim}
    To convert a general symmetric matrix with ku side diagonals into 
    banded form, use

       call matrix_full2pp(gen_matrix, packed_matrix)

    A full matrix can be obtained from a packed symmetric matrix via

       call matrix_pp2full(packed_matrix, gen_matrix)
\end{Verbatim}
\textbf{SEE ALSO:}\hspace{0.08in}\begin{Verbatim}
    matrix_pp2full
    matrix_pp2full_alloc

    matrix_full2pp
    matrix_full2pp_alloc

    matrix_bm2full
    matrix_bm2full_alloc

    matrix_full2bm
    matrix_full2bm_alloc
\end{Verbatim}
\textbf{REFERENCES:}\hspace{0.08in}\begin{Verbatim}
    E. Anderson, Z. Bai, C. Bischof, S. Blackford, J. Demmel, J. Dongarra,
       J. Du Croz, A. Greenbaum, S. Hammarling, A. McKenney, D. Sorensen,
       LAPACK Users' Guide, 3rd Ed., SIAM, 1999.
\end{Verbatim}
\section{Matrices/matrix\_pp2full}
\textsl{[ Subroutines ]}

\label{ch:robo22}
\label{ch:Matrices_matrix_pp2full}
\index{unsorted!matrix\_pp2full}\index{Subroutines!matrix\_pp2full}
\textbf{NAME:}\hspace{0.08in}\begin{Verbatim}
    matrix_pp2full - Convert a symmetric or hermitian matrix from Lapack's
                        packed into full (general) form.
\end{Verbatim}
\textbf{SYNOPSIS:}\hspace{0.08in}\begin{Verbatim}
    call matrix_pp2full(packed, matrix [, uplo])
\end{Verbatim}
\textbf{DESCRIPTION:}\hspace{0.08in}\begin{Verbatim}
    This subroutine copies a matrix in Lapack's packed form into full (or 
    general) form.
\end{Verbatim}
\textbf{INPUTS:}\hspace{0.08in}\begin{Verbatim}
    packed    Matrix in packed form.
    uplo      'UPLO' parameter; determines if the packed matrix has been packed
                 from an upper ('U') or lower ('L') full matrix. This parameter
                 is optional; it defaults to 'U'.
\end{Verbatim}
\textbf{OUTPUT:}\hspace{0.08in}\begin{Verbatim}
    matrix    Matrix in full (general) form.
\end{Verbatim}
\textbf{NOTES:}\hspace{0.08in}\begin{Verbatim}
    Packed matrix storage in Lapack is reserved for symmetric or Hermitian
    postive definite matrices. See the Lapack User Guide for details on this
    and other matrix storage systems supported by Lapack. 
\end{Verbatim}
\textbf{EXAMPLE:}\hspace{0.08in}\begin{Verbatim}
    To convert a general symmetric matrix into packed form, use

       call matrix_full2pp(gen_matrix, packed_matrix)

    A full matrix can be obtained from a packed positive definitive matrix via

       call matrix_pp2full(packed_matrix, gen_matrix)
\end{Verbatim}
\textbf{SEE ALSO:}\hspace{0.08in}\begin{Verbatim}
    matrix_pp2full
    matrix_pp2full_alloc

    matrix_full2pp
    matrix_full2pp_alloc

    matrix_bm2full
    matrix_bm2full_alloc

    matrix_full2bm
    matrix_full2bm_alloc
\end{Verbatim}
\textbf{REFERENCES:}\hspace{0.08in}\begin{Verbatim}
    E. Anderson, Z. Bai, C. Bischof, S. Blackford, J. Demmel, J. Dongarra,
       J. Du Croz, A. Greenbaum, S. Hammarling, A. McKenney, D. Sorensen,
       LAPACK Users' Guide, 3rd Ed., SIAM, 1999.
\end{Verbatim}
\section{Matrices/matrix\_pp2full\_alloc}
\textsl{[ Subroutines ]}

\label{ch:robo23}
\label{ch:Matrices_matrix_pp2full_alloc}
\index{unsorted!matrix\_pp2full\_alloc}\index{Subroutines!matrix\_pp2full\_alloc}
\textbf{NAME:}\hspace{0.08in}\begin{Verbatim}
    matrix_pp2full_alloc - Convert a symmetric or hermitian positive definite
                           matrix from Lapack's packed into full (general)
                           form.
\end{Verbatim}
\textbf{SYNOPSIS:}\hspace{0.08in}\begin{Verbatim}
    call matrix_pp2full(packed, matrix [, uplo])
\end{Verbatim}
\textbf{DESCRIPTION:}\hspace{0.08in}\begin{Verbatim}
    This subroutine copies a matrix in Lapack's packed form into full (or 
    general) form. The general matrix to be filled is allocated.
\end{Verbatim}
\textbf{INPUTS:}\hspace{0.08in}\begin{Verbatim}
    packed    Matrix in packed form.
    uplo      'UPLO' parameter; determines if the packed matrix has been packed
                 from an upper ('U') or lower ('L') full matrix. This parameter
                 is optional; it defaults to 'U'.
\end{Verbatim}
\textbf{OUTPUT:}\hspace{0.08in}\begin{Verbatim}
    matrix    Pointer to a matrix in full (general) form.
\end{Verbatim}
\textbf{NOTES:}\hspace{0.08in}\begin{Verbatim}
    The matrix argument is a 2d Fortran pointer which is allocated within
    this routine. If the pointer has been used before, all data will be lost.

    Packed matrix storage in Lapack is reserved for symmetric or Hermitian
    postive definite matrices. See the Lapack User Guide for details on this
    and other matrix storage systems supported by Lapack. 
\end{Verbatim}
\textbf{EXAMPLE:}\hspace{0.08in}\begin{Verbatim}
    To convert a general symmetric matrix into packed form, use

       call matrix_full2pp(gen_matrix, packed_matrix)

    A full matrix can be obtained from a packed positive definitive matrix via

       call matrix_pp2full(packed_matrix, gen_matrix)
\end{Verbatim}
\textbf{SEE ALSO:}\hspace{0.08in}\begin{Verbatim}
    matrix_pp2full
    matrix_pp2full_alloc

    matrix_full2pp
    matrix_full2pp_alloc

    matrix_bm2full
    matrix_bm2full_alloc

    matrix_full2bm
    matrix_full2bm_alloc
\end{Verbatim}
\textbf{REFERENCES:}\hspace{0.08in}\begin{Verbatim}
    E. Anderson, Z. Bai, C. Bischof, S. Blackford, J. Demmel, J. Dongarra,
       J. Du Croz, A. Greenbaum, S. Hammarling, A. McKenney, D. Sorensen,
       LAPACK Users' Guide, 3rd Ed., SIAM, 1999.
\end{Verbatim}
\section{Matrices/matrix\_pp2full\_subset}
\textsl{[ Subroutines ]}

\label{ch:robo24}
\label{ch:Matrices_matrix_pp2full_subset}
\index{unsorted!matrix\_pp2full\_subset}\index{Subroutines!matrix\_pp2full\_subset}
\textbf{NAME:}\hspace{0.08in}\begin{Verbatim}
    matrix_pp2full_subset - Convert a subset of a symmetric or hermitian 
                            matrix from Lapack's packed into full (general) 
                            form.
\end{Verbatim}
\textbf{SYNOPSIS:}\hspace{0.08in}\begin{Verbatim}
    call matrix_pp2full_subset(packed, matrix [, uplo])
\end{Verbatim}
\textbf{DESCRIPTION:}\hspace{0.08in}\begin{Verbatim}
    This subroutine copies part of a matrix in Lapack's packed form into full 
    (or general) form.
\end{Verbatim}
\textbf{INPUTS:}\hspace{0.08in}\begin{Verbatim}
    packed    Matrix in packed form.
    uplo      'UPLO' parameter; determines if the packed matrix has been packed
                 from an upper ('U') or lower ('L') full matrix. This parameter
                 is optional; it defaults to 'U'.
\end{Verbatim}
\textbf{OUTPUT:}\hspace{0.08in}\begin{Verbatim}
    matrix    Matrix in full (general) form.
\end{Verbatim}
\textbf{NOTES:}\hspace{0.08in}\begin{Verbatim}
    Packed matrix storage in Lapack is reserved for symmetric or Hermitian
    postive definite matrices. See the Lapack User Guide for details on this
    and other matrix storage systems supported by Lapack. 
\end{Verbatim}
\textbf{EXAMPLE:}\hspace{0.08in}\begin{Verbatim}
    To convert a general symmetric matrix into packed form, use

       call matrix_full2pp(gen_matrix, packed_matrix)

    A full matrix can be obtained from a packed positive definitive matrix via

       call matrix_pp2full(packed_matrix, gen_matrix)
\end{Verbatim}
\textbf{SEE ALSO:}\hspace{0.08in}\begin{Verbatim}
    matrix_pp2full
    matrix_pp2full_alloc

    matrix_full2pp
    matrix_full2pp_alloc

    matrix_bm2full
    matrix_bm2full_alloc

    matrix_full2bm
    matrix_full2bm_alloc
\end{Verbatim}
\textbf{REFERENCES:}\hspace{0.08in}\begin{Verbatim}
    E. Anderson, Z. Bai, C. Bischof, S. Blackford, J. Demmel, J. Dongarra,
       J. Du Croz, A. Greenbaum, S. Hammarling, A. McKenney, D. Sorensen,
       LAPACK Users' Guide, 3rd Ed., SIAM, 1999.
\end{Verbatim}
\section{Matrices/matrix\_types}
\textsl{[ Modules ]}

\label{ch:robo25}
\label{ch:Matrices_matrix_types}
\index{unsorted!matrix\_types}\index{Modules!matrix\_types}
\textbf{NAME:}\hspace{0.08in}\begin{Verbatim}
    matrix_types - Type declaration for positive definite matrices.
\end{Verbatim}
\textbf{SYNOPSIS:}\hspace{0.08in}\begin{Verbatim}
    use matrix_types
\end{Verbatim}
\textbf{DESCRIPTION:}\hspace{0.08in}\begin{Verbatim}
    This module provides derived data types for matrices.
\end{Verbatim}
\textbf{NOTES:}\hspace{0.08in}\begin{Verbatim}
    The main purpose of these matrix types is to be used with the matrix
    equation solving routines of this library for the respective matrix
    types. The derived matrix types store decomposition information as
    obtained by Lapack in addition to the actual matrix. For example, if
    a positive definite packed matrix is used for the first time with
    matrix_solve(), a Cholesky decomposition of the matrix is calculated
    and used for the solving the given linear equation. The decomposition
    is also stored in the same matrix data structure. In the next call
    to matrix_solve, the Cholesky decomposition obtained previously is
    reused, saving significant amounts of computations when solving
    linear equations with many different right hand sides.

    At present, only double precision matrices are supported.
\end{Verbatim}
\textbf{SEE ALSO:}\hspace{0.08in}\begin{Verbatim}
    matrix_ge
    matrix_pp
    matrix_pb
    matrix_sq
\end{Verbatim}
\section{Matrix/delete}
\textsl{[ Subroutines ]}

\label{ch:robo26}
\label{ch:Matrix_delete}
\index{unsorted!delete}\index{Subroutines!delete}
\textbf{NAME:}\hspace{0.08in}\begin{Verbatim}
    delete - Delete the data contained in the matrix type.
\end{Verbatim}
\textbf{SYNOPSIS:}\hspace{0.08in}\begin{Verbatim}
    call delete(matrix)
\end{Verbatim}
\textbf{DESCRIPTION:}\hspace{0.08in}\begin{Verbatim}
    This subroutine deletes all data / memory hold in the given matrix
    type variable
\end{Verbatim}
\textbf{INPUTS:}\hspace{0.08in}\begin{Verbatim}
    matrix
\end{Verbatim}
\textbf{OUTPUT:}\hspace{0.08in}\textbf{NOTES:}\hspace{0.08in}\textbf{EXAMPLE:}\hspace{0.08in}\textbf{SEE ALSO:}\hspace{0.08in}\textbf{REFERENCES:}\hspace{0.08in}\section{Matrix/matrix\_assign}
\textsl{[ Modules ]}

\label{ch:robo27}
\label{ch:Matrix_matrix_assign}
\index{unsorted!matrix\_assign}\index{Modules!matrix\_assign}
\textbf{NAME:}\hspace{0.08in}\begin{Verbatim}
    matrix_assign - Implement matrix assignment (i.e., the operator =).
\end{Verbatim}
\textbf{SYNOPSIS:}\hspace{0.08in}\textbf{DESCRIPTION:}\hspace{0.08in}\begin{Verbatim}
    This file contains implementations for the assignment operator (=) for
    the various types in the matrix class.
\end{Verbatim}
\textbf{INPUTS:}\hspace{0.08in}\textbf{OUTPUT:}\hspace{0.08in}\textbf{NOTES:}\hspace{0.08in}\textbf{EXAMPLE:}\hspace{0.08in}\textbf{SEE ALSO:}\hspace{0.08in}\textbf{REFERENCES:}\hspace{0.08in}\section{Matrix/matrix\_invert}
\textsl{[ Functions ]}

\label{ch:robo28}
\label{ch:Matrix_matrix_invert}
\index{unsorted!matrix\_invert}\index{Functions!matrix\_invert}
\textbf{NAME:}\hspace{0.08in}\begin{Verbatim}
    matrix_invert - Invert a square matrix 
\end{Verbatim}
\textbf{SYNOPSIS:}\hspace{0.08in}\begin{Verbatim}
    use matrix
      ...
    x = matrix_invert(A)
\end{Verbatim}
\textbf{DESCRIPTION:}\hspace{0.08in}\begin{Verbatim}
    This function calls matrix_solve to solve a linear equation of the form

       A x = b

    for matrix A. It is assumed that the matrix A is positive 
    definite, and the solution is obtained using a Cholesky decomposition. 
    The matrix b is defined zero everywhere except for one element equal to 1.
\end{Verbatim}
\textbf{INPUTS:}\hspace{0.08in}\begin{Verbatim}
    A     Matrix to be inverted.
\end{Verbatim}
\textbf{OUTPUT:}\hspace{0.08in}\begin{Verbatim}
    x     Inverted matrix.
\end{Verbatim}
\textbf{NOTES:}\hspace{0.08in}\begin{Verbatim}
    The matrix A may be in either full or packed form. If the matrix is not 
    positive definite, the attempt to solve will fail with an error message.
    These routines are currently available in double precision only.
\end{Verbatim}
\textbf{EXAMPLE:}\hspace{0.08in}\begin{Verbatim}
    To invert a positive definite matrix A,

       A_inverted = matrix_invert(A)
\end{Verbatim}
\textbf{SEE ALSO:}\hspace{0.08in}\begin{Verbatim}
    matrix_types
    matrix_solve
\end{Verbatim}
\textbf{REFERENCES:}\hspace{0.08in}\section{Matrix/matrix\_operators}
\textsl{[ Subroutines ]}

\label{ch:robo29}
\label{ch:Matrix_matrix_operators}
\index{unsorted!matrix\_operators}\index{Subroutines!matrix\_operators}
\textbf{NAME:}\hspace{0.08in}\begin{Verbatim}
    matrix_operators - Implement matrix operators (i.e., the operators +, -, *, /).
\end{Verbatim}
\textbf{SYNOPSIS:}\hspace{0.08in}\textbf{DESCRIPTION:}\hspace{0.08in}\begin{Verbatim}
    This file implements various operators for the types of the matrix clas.
    In particular, these operations are supported:

      - Addition
      - Subtraction
      - Scalar multiplication
      - Scalar division
\end{Verbatim}
\textbf{INPUTS:}\hspace{0.08in}\textbf{OUTPUT:}\hspace{0.08in}\textbf{NOTES:}\hspace{0.08in}\textbf{EXAMPLE:}\hspace{0.08in}\textbf{SEE ALSO:}\hspace{0.08in}\textbf{REFERENCES:}\hspace{0.08in}\section{Matrix/matrix\_solve}
\textsl{[ Functions ]}

\label{ch:robo30}
\label{ch:Matrix_matrix_solve}
\index{unsorted!matrix\_solve}\index{Functions!matrix\_solve}
\textbf{NAME:}\hspace{0.08in}\begin{Verbatim}
    matrix_solve - Solve a linear matrix equation
\end{Verbatim}
\textbf{SYNOPSIS:}\hspace{0.08in}\begin{Verbatim}
    use matrix
      ...
    x = matrix_solve(A, b)
\end{Verbatim}
\textbf{DESCRIPTION:}\hspace{0.08in}\begin{Verbatim}
    This function calls matrix_solve to solve a linear equation of the form

       A x = b

    for arbitrary matrices A and arbitrary right hand side vectors b. 
    It is assumed that the matrix A is positive 
    definite, and the solution is obtained using a Cholesky decomposition. 
\end{Verbatim}
\textbf{INPUTS:}\hspace{0.08in}\begin{Verbatim}
    A     Matrix defining the linear equation system.
    b     Right hand side of the linear equation system.
\end{Verbatim}
\textbf{OUTPUT:}\hspace{0.08in}\begin{Verbatim}
    x     Solution of the linear equation system.
\end{Verbatim}
\textbf{NOTES:}\hspace{0.08in}\begin{Verbatim}
    The matrix A may be in either full or packed form. If the matrix is not 
    positive definite, the attempt to solve will fail with an error message.
    These routines are currently available in double precision only.
\end{Verbatim}
\textbf{EXAMPLE:}\hspace{0.08in}\begin{Verbatim}
    The solution of a linear equation with right hand side vector b is

       x = solve(A, b)
\end{Verbatim}
\textbf{SEE ALSO:}\hspace{0.08in}\begin{Verbatim}
    matrix_types
\end{Verbatim}
\textbf{REFERENCES:}\hspace{0.08in}\section{Matrix/matrix\_sqrt}
\textsl{[ Functions ]}

\label{ch:robo31}
\label{ch:Matrix_matrix_sqrt}
\index{unsorted!matrix\_sqrt}\index{Functions!matrix\_sqrt}
\textbf{NAME:}\hspace{0.08in}\begin{Verbatim}
    matrix_sqrt - Square root and it's inverse of a symmetric matrix
\end{Verbatim}
\textbf{SYNOPSIS:}\hspace{0.08in}\begin{Verbatim}
    call matrix_sqrt(A, R)
\end{Verbatim}
\textbf{DESCRIPTION:}\hspace{0.08in}\begin{Verbatim}
    This subroutine calculates the (symmetric) square root of symmetric
    matrix from it's SVD along with the inverse square root.
\end{Verbatim}
\textbf{INPUTS:}\hspace{0.08in}\begin{Verbatim}
    A   Matrix to be used for calculating the preconditioner.
\end{Verbatim}
\textbf{OUTPUT:}\hspace{0.08in}\begin{Verbatim}
    R   Symmetric square root and its inverse.
\end{Verbatim}
\textbf{NOTES:}\hspace{0.08in}\begin{Verbatim}
   The routine works for regular (square) matrices as well as for positive
   definite matrices in packed form. Matrix types matrix_ge and matrix_pp
   are also supported.

   If square roots have been calculated before, the matrices will silently
   be deallocated, and all data will be lost.
\end{Verbatim}
\textbf{EXAMPLE:}\hspace{0.08in}\begin{Verbatim}
   use matrix
     ...
   type(matrix_sq) :: R
\end{Verbatim}
\textbf{SEE ALSO:}\hspace{0.08in}\textbf{REFERENCES:}\hspace{0.08in}\section{Matrix/matrix\_svd}
\textsl{[ Functions ]}

\label{ch:robo32}
\label{ch:Matrix_matrix_svd}
\index{unsorted!matrix\_svd}\index{Functions!matrix\_svd}
\textbf{NAME:}\hspace{0.08in}\begin{Verbatim}
    matrix_svd - Compute singular value decomposition (SVD) of a matrix.
\end{Verbatim}
\textbf{SYNOPSIS:}\hspace{0.08in}\begin{Verbatim}
    call matrix_svd(A, W, U, V)
\end{Verbatim}
\textbf{DESCRIPTION:}\hspace{0.08in}\begin{Verbatim}
    This subroutine calculates the SVD of a matrix.
               A = U diag(W) V^T
    where U and V are matrices containing eigenvectors of A and
    the diagonal elements of W give the singular values.
\end{Verbatim}
\textbf{INPUTS:}\hspace{0.08in}\begin{Verbatim}
    A   Matrix to be used for calculating the SVD
\end{Verbatim}
\textbf{OUTPUT:}\hspace{0.08in}\begin{Verbatim}
    W   Diagonal elements of singular values of A
    U   Left-hand singular vector of A
    V   Right-hand singular vector of A. (Note routine returns V and NOT its 
        transpose V^T).
\end{Verbatim}
\textbf{NOTES:}\hspace{0.08in}\begin{Verbatim}
   The routine is based on the Numerical Recipes (Press et al. 1992) algorithm
   for computing the SVD. Only full matrices are supported. Matrices in packed
   form should be converted to full type before calling matrix_svd.
\end{Verbatim}
\textbf{EXAMPLE:}\hspace{0.08in}\textbf{SEE ALSO:}\hspace{0.08in}\begin{Verbatim}
      matrix_sqrt
\end{Verbatim}
\textbf{REFERENCES:}\hspace{0.08in}\begin{Verbatim}
   W.H. Press, S.A. Teukolsjy, W.T. Vetterling and B.P. Flannery,
   Numerical Recipes in C - The Art of Scientific Computing.
   2nd Ed., Cambridge University Press, 1992.

    G.H. Golub, C.H. van Loan
    Matrix computations. 
    3rd Ed., The Johns Hopkins University Press, 1996.
\end{Verbatim}
\section{Matrix/matrix\_toast}
\textsl{[ Functions ]}

\label{ch:robo33}
\label{ch:Matrix_matrix_toast}
\index{unsorted!matrix\_toast}\index{Functions!matrix\_toast}
\textbf{NAME:}\hspace{0.08in}\begin{Verbatim}
    matrix_toast - Toast a matrix with a second one.
\end{Verbatim}
\textbf{SYNOPSIS:}\hspace{0.08in}\begin{Verbatim}
    call matrix_toast(A, B [, BABt])
\end{Verbatim}
\textbf{DESCRIPTION:}\hspace{0.08in}\begin{Verbatim}
    This routine calculates the 'toast product' of two matrices. It is
    possible to toast with a non-square matrix B if the optional result
    matrix is provided in the call to toast_transpose and dimensioned
    appropriately.
\end{Verbatim}
\textbf{INPUTS:}\hspace{0.08in}\begin{Verbatim}
    A     Matrix to be toasted.
    B     Toaster matrix.
\end{Verbatim}
\textbf{OUTPUT:}\hspace{0.08in}\begin{Verbatim}
    BABt  Toasted matrix.
\end{Verbatim}
\textbf{NOTES:}\hspace{0.08in}\begin{Verbatim}
    If no optional argument is given, calculations are done in place. The
    result is then returned in A. This is only possible if the toaster is
    a square matrix.
\end{Verbatim}
\textbf{EXAMPLE:}\hspace{0.08in}\textbf{SEE ALSO:}\hspace{0.08in}\section{Matrix/ropp\_state2control}
\textsl{[ Functions ]}

\label{ch:robo34}
\label{ch:Matrix_ropp_state2control}
\index{unsorted!ropp\_state2control}\index{Functions!ropp\_state2control}
\textbf{NAME:}\hspace{0.08in}\begin{Verbatim}
    ropp_state2control - Convert a state to a control variable.
\end{Verbatim}
\textbf{SYNOPSIS:}\hspace{0.08in}\begin{Verbatim}
    call ropp_state2control(precon, state, control)
\end{Verbatim}
\textbf{DESCRIPTION:}\hspace{0.08in}\begin{Verbatim}
    This routine performs the variable transform from a control variable to
    the original variable, given the preconditioner P.
\end{Verbatim}
\textbf{INPUTS:}\hspace{0.08in}\begin{Verbatim}
    precon     preconditioner
    state      variable in state space
\end{Verbatim}
\textbf{OUTPUT:}\hspace{0.08in}\begin{Verbatim}
    control    variable in control space
\end{Verbatim}
\textbf{NOTES:}\hspace{0.08in}\begin{Verbatim}
    The preconditioner must be set up previously by a call to preconditioner().
\end{Verbatim}
\textbf{EXAMPLE:}\hspace{0.08in}\begin{Verbatim}
    use matrix
    use ropp_1dvar
      ...
    type(matrix_pp) :: B
    type(matrix_sq) :: precon
      ...
    call matrix_sqrt(B, precon)
      ...
    call ropp_state2control(precon, state, control)
      ... [do something in control space] ...
    call ropp_control2state(precon, control, state)
\end{Verbatim}
\textbf{SEE ALSO:}\hspace{0.08in}\begin{Verbatim}
    matrix_sqrt
    ropp_controle2state
\end{Verbatim}
\textbf{REFERENCES:}\hspace{0.08in}\section{Preconditioning/ropp\_control2state}
\textsl{[ Functions ]}

\label{ch:robo40}
\label{ch:Preconditioning_ropp_control2state}
\index{unsorted!ropp\_control2state}\index{Functions!ropp\_control2state}
\textbf{NAME:}\hspace{0.08in}\begin{Verbatim}
    control2state - Convert the control variable back to a state.
\end{Verbatim}
\textbf{SYNOPSIS:}\hspace{0.08in}\begin{Verbatim}
    call ropp_control2state(precon, control, state)
\end{Verbatim}
\textbf{DESCRIPTION:}\hspace{0.08in}\begin{Verbatim}
    This routine performs the variable transform from a control variable to
    the original variable, given the preconditioner P.
\end{Verbatim}
\textbf{INPUTS:}\hspace{0.08in}\begin{Verbatim}
    precon     preconditioner
    control    variable in control space
\end{Verbatim}
\textbf{OUTPUT:}\hspace{0.08in}\begin{Verbatim}
    state      variable in state space
\end{Verbatim}
\textbf{NOTES:}\hspace{0.08in}\begin{Verbatim}
    The preconditioner must be set up previously by a call to preconditioner().
\end{Verbatim}
\textbf{EXAMPLE:}\hspace{0.08in}\begin{Verbatim}
    use matrix
    use ropp_1dvar
      ...
    type(matrix_pp) :: B
    type(matrix_sq) :: precon
      ...
    call matrix_sqrt(B, precon)
      ...
    call ropp_state2control(precon, state, control)
      ... [do something in control space] ...
    call ropp_control2state(precon, control, state)
\end{Verbatim}
\textbf{SEE ALSO:}\hspace{0.08in}\begin{Verbatim}
    matrix_sqrt
    ropp_state2control
\end{Verbatim}
\textbf{REFERENCES:}\hspace{0.08in}\section{Preconditioning/ropp\_control2state\_ad}
\textsl{[ Functions ]}

\label{ch:robo41}
\label{ch:Preconditioning_ropp_control2state_ad}
\index{unsorted!ropp\_control2state\_ad}\index{Functions!ropp\_control2state\_ad}
\textbf{NAME:}\hspace{0.08in}\begin{Verbatim}
    ropp_control2state_ad - Adjoint of ropp_control2state.
\end{Verbatim}
\textbf{SYNOPSIS:}\hspace{0.08in}\begin{Verbatim}
    call ropp_control2state_ad(precon, control_ad, state_ad)
\end{Verbatim}
\textbf{DESCRIPTION:}\hspace{0.08in}\begin{Verbatim}
    This routine performs the variable transform from a control variable to
    the original variable, given the preconditioner P.
\end{Verbatim}
\textbf{INPUTS:}\hspace{0.08in}\begin{Verbatim}
    precon        preconditioner
    state_ad      adjoint state variable
\end{Verbatim}
\textbf{OUTPUT:}\hspace{0.08in}\begin{Verbatim}
    control_ad    adjoint control variable
\end{Verbatim}
\textbf{NOTES:}\hspace{0.08in}\begin{Verbatim}
    The preconditioner must be set up previously by a call to preconditioner().
\end{Verbatim}
\textbf{EXAMPLE:}\hspace{0.08in}\begin{Verbatim}
    use matrix
    use ropp_1dvar
      ...
    type(matrix_pp) :: B
    type(matrix_sq) :: precon
      ...
    call matrix_sqrt(B, precon)
      ...
    call ropp_state2control(precon, state, control)
      ... [do something in control space] ...
    call ropp_control2state(precon, control, state)
\end{Verbatim}
\textbf{SEE ALSO:}\hspace{0.08in}\begin{Verbatim}
    matrix_sqrt
    ropp_state2control
\end{Verbatim}
\textbf{REFERENCES:}\hspace{0.08in}\section{Programs/ropp\_1dvar\_add\_bangle\_error}
\textsl{[ Parameters ]}

\label{ch:robo42}
\label{ch:Programs_ropp_1dvar_add_bangle_error}
\index{unsorted!ropp\_1dvar\_add\_bangle\_error}\index{Parameters!ropp\_1dvar\_add\_bangle\_error}
\textbf{NAME:}\hspace{0.08in}\begin{Verbatim}
    ropp_1dvar_add_bangle_error - Read a ROPP format netCDF radio
                                  occultation bending angle
                                  observation data file and add an error
                                  description (sigma) to the data. Write the
                                  resulting file
\end{Verbatim}
\textbf{SYNOPSIS:}\hspace{0.08in}\begin{Verbatim}
    ropp_1dvar_add_bangle_error <obs_file> -Omod <Omodel> -o <out_file> [-h]
\end{Verbatim}
\textbf{DESCRIPTION:}\hspace{0.08in}\textbf{ARGUMENTS:}\hspace{0.08in}\begin{Verbatim}
    <obs_file>       Name of input observation file.
    -Omod <Omodel>   Observational error model:
                        '1%': 1% at 0 km, 0.1% from 12 km,min(std(BA))=6urad
                        '2%': 2% at 0 km, 0.2% from 12 km,min(std(BA))=6urad
                        '3%': 3% at 0 km, 0.3% from 12 km,min(std(BA))=6urad
                        'MO : Met Office operational implementation
                              latitudinally varying
                        'EC': ECMWF operational implementation
    -o <out_file>    Name of output file (default for single profile files
                     to write to input file)
    -h               Help.
\end{Verbatim}
\textbf{NOTES:}\hspace{0.08in}\begin{Verbatim}
    Default is for the output file to overwrite the input file.
\end{Verbatim}
\textbf{SEE ALSO:}\hspace{0.08in}\begin{Verbatim}
    ropp_1dvar_add_refrac_error.f90
    ropp_1dvar_add_bgr_error.f90
\end{Verbatim}
\section{Programs/ropp\_1dvar\_add\_bgr\_error}
\textsl{[ Parameters ]}

\label{ch:robo43}
\label{ch:Programs_ropp_1dvar_add_bgr_error}
\index{unsorted!ropp\_1dvar\_add\_bgr\_error}\index{Parameters!ropp\_1dvar\_add\_bgr\_error}
\textbf{NAME:}\hspace{0.08in}\begin{Verbatim}
    ropp_1dvar_add_bgr_error - Read a ROPP netCDF background file containing
                               p, T, q and add an error description (sigma
                               values) to the data. Write resulting file.
\end{Verbatim}
\textbf{SYNOPSIS:}\hspace{0.08in}\begin{Verbatim}
    ropp_1dvar_add_bgr_error <bg_file> -c <cov_file> [-o <out_file>] [-h]
\end{Verbatim}
\textbf{DESCRIPTION:}\hspace{0.08in}\begin{Verbatim}
       This program adds profile-by-profile background error values to a
       ROPP format file by reading sigma values from the provided error
       covariance matrix files (in the errors/ directory).
       Users should modify this program to utilise background error
       descriptions from other sources.
\end{Verbatim}
\textbf{ARGUMENTS:}\hspace{0.08in}\begin{Verbatim}
    <bg_file>        Name of input background file.
    -c <cov_file>    Name of background error covariance matrix file
    -o <out_file>    Name of output file (default for single profile files
                     to write to input file)
    -h               Help.
\end{Verbatim}
\textbf{SEE ALSO:}\hspace{0.08in}\begin{Verbatim}
    ropp_1dvar_add_refrac_error.f90
    ropp_1dvar_add_bangle_error.f90
\end{Verbatim}
\section{Programs/ropp\_1dvar\_add\_refrac\_error}
\textsl{[ Parameters ]}

\label{ch:robo44}
\label{ch:Programs_ropp_1dvar_add_refrac_error}
\index{unsorted!ropp\_1dvar\_add\_refrac\_error}\index{Parameters!ropp\_1dvar\_add\_refrac\_error}
\textbf{NAME:}\hspace{0.08in}\begin{Verbatim}
    ropp_1dvar_add_refrac_error - Read a ROPP format netCDF radio
                                  occultation refractivity
                                  observation data file and add an error
                                  description (sigma) to the data. Write the
                                  resulting file
\end{Verbatim}
\textbf{SYNOPSIS:}\hspace{0.08in}\begin{Verbatim}
    ropp_1dvar_add_refrac_error <obs_file> -Omod <Omodel> -o <out_file> [-b <bg_file>] [-c <corr_file>] [-h]
\end{Verbatim}
\textbf{DESCRIPTION:}\hspace{0.08in}\textbf{ARGUMENTS:}\hspace{0.08in}\begin{Verbatim}
    <obs_file>       Name of input observation file.
    -Omod <Omodel>   Observational error model:
                        '1%': 1% at 0 km, 0.1% from 12 km, min(std(N))=0.02
                        '2%': 2% at 0 km, 0.2% from 12 km, min(std(N))=0.02
                        '3%': 3% at 0 km, 0.3% from 12 km, min(std(N))=0.02
                        'TP : ROM SAF operational implementation. As '2%'
                              model with dynamic tropopause height (default)
                              calculation from bg data rather than 12 km.
                        'MO : Met Office operational implementation
                              latitudinally varying
                        'SK': according to Steiner-Kirchengast [2005]
    -o <out_file>    Name of output file
    -b <bg_file>     Name of background file (only required for 'TP' model).
    -c <corr_file>   Name of optional output correlation file
    -h               Help.
\end{Verbatim}
\textbf{NOTES:}\hspace{0.08in}\begin{Verbatim}
    Default is for the output file to overwrite the input file.
\end{Verbatim}
\textbf{SEE ALSO:}\hspace{0.08in}\begin{Verbatim}
    ropp_1dvar_add_bgr_error.f90
\end{Verbatim}
\section{Programs/ropp\_1dvar\_bangle}
\textsl{[ Parameters ]}

\label{ch:robo45}
\label{ch:Programs_ropp_1dvar_bangle}
\index{unsorted!ropp\_1dvar\_bangle}\index{Parameters!ropp\_1dvar\_bangle}
\textbf{NAME:}\hspace{0.08in}\begin{Verbatim}
    ropp_1dvar_bangle
\end{Verbatim}
\textbf{SYNOPSIS:}\hspace{0.08in}\begin{Verbatim}
   Perform a 1DVar retrieval of radio occultation data using bending angle
   observations and a background

   > ropp_1dvar_bangle [-c <cfg_file>] [-y <obs_file>] [-b <bg_file>]
                       [-o <out_file>] [-no_ranchk] [-comp] [-check_qsat]
                       [-d] [-h] [-v]
\end{Verbatim}
\textbf{OPTIONS:}\hspace{0.08in}\begin{Verbatim}
   -c <cfg_file>               name of configuation file
   -y <obs_file>               name of observation file
   --obs-corr <obs_corr_file>  name of observation covariance file
   -b <bg_file>                name of background file
   --bg-corr <bg_corr_file>    name of background covariance file
   -o <output_file>            name of output file
   -no-ranchk                  do not range check input or output
   -comp                       include non ideal gas compressibility
   -check_qsat                 include check against saturation
   -d                          output additional diagnostics
   -h                          help
   -v                          version information
\end{Verbatim}
\textbf{DESCRIPTION:}\hspace{0.08in}\begin{Verbatim}
   Perform a 1DVar retrieval of radio occultation data using bending angle
   observations and a background
\end{Verbatim}
\textbf{NOTES:}\hspace{0.08in}\begin{Verbatim}
    Names of input and output files can be specified in both the configuration
    file and on the command line; command line arguments will overwrite
    configuration file settings.

    If the input file is a multifile, both input files must be multifiles, and
    the output file is also a multifile. It this case, it is assumed that
    observation and background profiles have been arranged in identical order
    in their respective input files.

    Already existing output files will be overwritten.
\end{Verbatim}
\textbf{SEE ALSO:}\hspace{0.08in}\begin{Verbatim}
   ropp_1dvar_refrac
\end{Verbatim}
\section{Programs/ropp\_1dvar\_refrac}
\textsl{[ Parameters ]}

\label{ch:robo46}
\label{ch:Programs_ropp_1dvar_refrac}
\index{unsorted!ropp\_1dvar\_refrac}\index{Parameters!ropp\_1dvar\_refrac}
\textbf{NAME:}\hspace{0.08in}\begin{Verbatim}
   ropp_1dvar_refrac
\end{Verbatim}
\textbf{SYNOPSIS:}\hspace{0.08in}\begin{Verbatim}
   Perform a 1DVar retrieval of radio occultation data using refractivity
   observations and a background

   > ropp_1dvar_refrac [-c <cfg_file>] [-y <obs_file>] [-b <bg_file>]
                       [-o <out_file>] [-no_ranchk] [-comp] [-check_qsat]
                       [-d] [-h] [-v]
\end{Verbatim}
\textbf{OPTIONS:}\hspace{0.08in}\begin{Verbatim}
   -c <cfg_file>               name of configuation file
   -y <obs_file>               name of observation file
   --obs-corr <obs_corr_file>  name of observation covariance file
   -b <bg_file>                name of background file
   --bg-corr <bg_corr_file>    name of background covariance file
   -o <output_file>            name of output file
   -no-ranchk                  do not range check input or output
   -comp                       include non-ideal gas compressibility
   -check_qsat                 include check against saturation
   -d                          output additional diagnostics
   -h                          help
   -v                          version information
\end{Verbatim}
\textbf{DESCRIPTION:}\hspace{0.08in}\begin{Verbatim}
   Perform a 1DVar retrieval of radio occultation data using refractivity
   observations and a background.
\end{Verbatim}
\textbf{NOTES:}\hspace{0.08in}\begin{Verbatim}
   Names of input and output files can be specified in both the configuration
   file and on the command line; command line arguments will overwrite
   configuration file settings.

   If the input file is a multifile, both input files must be multifiles, and
   the output file is also a multifile. It this case, it is assumed that
   observation and background profiles have been arranged in identical order
   in their respective input files.

   Already existing output files will be overwritten.
\end{Verbatim}
\textbf{SEE ALSO:}\hspace{0.08in}\begin{Verbatim}
   ropp_1dvar_bangle
\end{Verbatim}
\section{QC/ropp\_qc\_bgqc}
\textsl{[ Subroutines ]}

\label{ch:robo47}
\label{ch:QC_ropp_qc_bgqc}
\index{unsorted!ropp\_qc\_bgqc}\index{Subroutines!ropp\_qc\_bgqc}
\textbf{NAME:}\hspace{0.08in}\begin{Verbatim}
    ropp_qc_bgqc - Background quality control.
\end{Verbatim}
\textbf{SYNOPSIS:}\hspace{0.08in}\begin{Verbatim}
    call ropp_qc_bgqc(obs, config, diag)
\end{Verbatim}
\textbf{DESCRIPTION:}\hspace{0.08in}\begin{Verbatim}
    This subroutine performs a background quality control on bending
    angle or refractivity observation data.
\end{Verbatim}
\textbf{INPUTS:}\hspace{0.08in}\begin{Verbatim}
    obs         Observation vector
    config      Configuration options
    diag        Diagnostic output structure
\end{Verbatim}
\textbf{OUTPUT:}\hspace{0.08in}\begin{Verbatim}
    diag        Diagnostic structure with updated variables
\end{Verbatim}
\textbf{NOTES:}\hspace{0.08in}\begin{Verbatim}
    ropp_qc_bgqc requires that some fields in the diag structure have
    been properly filled, e.g. by a previous call to ropp_qc_OmB().
\end{Verbatim}
\textbf{EXAMPLE:}\hspace{0.08in}\textbf{SEE ALSO:}\hspace{0.08in}\begin{Verbatim}
    ropp_qc_OmB
    ropp_qc_pge
\end{Verbatim}
\textbf{REFERENCES:}\hspace{0.08in}\section{QC/ropp\_qc\_cutoff}
\textsl{[ Subroutines ]}

\label{ch:robo48}
\label{ch:QC_ropp_qc_cutoff}
\index{unsorted!ropp\_qc\_cutoff}\index{Subroutines!ropp\_qc\_cutoff}
\textbf{NAME:}\hspace{0.08in}\begin{Verbatim}
    ropp_qc_cutoff - Down-weight data outside required observation
                     height range [min_1dvar_height to max_1dvar_height]
\end{Verbatim}
\textbf{SYNOPSIS:}\hspace{0.08in}\begin{Verbatim}
    call ropp_qc_cutoff(obs, config)
\end{Verbatim}
\textbf{DESCRIPTION:}\hspace{0.08in}\begin{Verbatim}
    This subroutine down-weights all observations outside the height range
    interval specified by configuration parameters min_1dvar_height and
    max_1dvar_height
\end{Verbatim}
\textbf{INPUTS:}\hspace{0.08in}\begin{Verbatim}
    obs         Observation vector
    config      Configuration options
\end{Verbatim}
\textbf{OUTPUT:}\hspace{0.08in}\begin{Verbatim}
    obs         Updated observation vector
\end{Verbatim}
\textbf{NOTES:}\hspace{0.08in}\textbf{EXAMPLE:}\hspace{0.08in}\textbf{SEE ALSO:}\hspace{0.08in}\textbf{REFERENCES:}\hspace{0.08in}\section{QC/ropp\_qc\_genqc}
\textsl{[ Subroutines ]}

\label{ch:robo49}
\label{ch:QC_ropp_qc_genqc}
\index{unsorted!ropp\_qc\_genqc}\index{Subroutines!ropp\_qc\_genqc}
\textbf{NAME:}\hspace{0.08in}\begin{Verbatim}
    ropp_qc_genqc - Generic quality control checks.
\end{Verbatim}
\textbf{SYNOPSIS:}\hspace{0.08in}\begin{Verbatim}
    call ropp_qc_genqc(obs, bg, config, diag)
\end{Verbatim}
\textbf{DESCRIPTION:}\hspace{0.08in}\begin{Verbatim}
    This subroutine performs generic quality control checks, e.g. for the 
    proper co-location of observation and background data in both space and 
    time, for fullfilling monotonicity constraints, and also basic unit checks.
\end{Verbatim}
\textbf{INPUTS:}\hspace{0.08in}\begin{Verbatim}
    obs         Observation vector
    bg          Background vector
    config      Configuration options
\end{Verbatim}
\textbf{OUTPUT:}\hspace{0.08in}\begin{Verbatim}
    diag        Diagnostics structure with updated status
\end{Verbatim}
\textbf{NOTES:}\hspace{0.08in}\textbf{EXAMPLE:}\hspace{0.08in}\textbf{SEE ALSO:}\hspace{0.08in}\textbf{REFERENCES:}\hspace{0.08in}\section{QC/ropp\_qc\_OmB}
\textsl{[ Subroutines ]}

\label{ch:robo50}
\label{ch:QC_ropp_qc_OmB}
\index{unsorted!ropp\_qc\_OmB}\index{Subroutines!ropp\_qc\_OmB}
\textbf{NAME:}\hspace{0.08in}\begin{Verbatim}
    ropp_qc_OmB - Calculate O minus B.
\end{Verbatim}
\textbf{SYNOPSIS:}\hspace{0.08in}\begin{Verbatim}
    call ropp_qc_OmB(obs, bg, diag)
\end{Verbatim}
\textbf{DESCRIPTION:}\hspace{0.08in}\begin{Verbatim}
    This subroutine calculates the O minus B (i.e., the difference between
    the actual observations and the background forward modelled into
    observation space) along with the expected uncertainty of this difference.
    The results are placed in the diag structure.
\end{Verbatim}
\textbf{INPUTS:}\hspace{0.08in}\begin{Verbatim}
    obs      Observation vector
    bg       Background vector
\end{Verbatim}
\textbf{OUTPUT:}\hspace{0.08in}\begin{Verbatim}
    diag     Diagnostic structure updated with O - B field
\end{Verbatim}
\section{QC/ropp\_qc\_pge}
\textsl{[ Subroutines ]}

\label{ch:robo51}
\label{ch:QC_ropp_qc_pge}
\index{unsorted!ropp\_qc\_pge}\index{Subroutines!ropp\_qc\_pge}
\textbf{NAME:}\hspace{0.08in}\begin{Verbatim}
    ropp_qc_pge - Probability of gross error.
\end{Verbatim}
\textbf{SYNOPSIS:}\hspace{0.08in}\begin{Verbatim}
    call rop_qc_pge(obs, config, diag)
\end{Verbatim}
\textbf{DESCRIPTION:}\hspace{0.08in}\begin{Verbatim}
    This subroutine calculates the Probability of Gross Error for bending
    angle or refractivity observation data, and optionally uses the result
    for quality control.
\end{Verbatim}
\textbf{INPUTS:}\hspace{0.08in}\begin{Verbatim}
    obs          Observation vector
    config       Configuration options
    diag         Diagnostic structure
\end{Verbatim}
\textbf{OUTPUT:}\hspace{0.08in}\begin{Verbatim}
    diag         Diagnostic structure with updated PGE variable
\end{Verbatim}
\textbf{NOTES:}\hspace{0.08in}\begin{Verbatim}
    ropp_qc_pge requires that some fields in the diag structure have
    been properly filled, e.g. by a previous call to ropp_qc_OmB().
\end{Verbatim}
\textbf{SEE ALSO:}\hspace{0.08in}\begin{Verbatim}
    ropp_qc_OmB
    ropp_qc_bgqc
\end{Verbatim}
