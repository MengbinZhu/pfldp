% Document: ./rm_pp
% Source: ./mywork/
% Generated with ROBODoc Version 4.99.36 (Jul 28 2009)
\section{BendingAngle/ropp\_pp\_abel\_EXP}
\textsl{[ Subroutines ]}

\label{ch:robo0}
\label{ch:BendingAngle_ropp_pp_abel_EXP}
\index{unsorted!ropp\_pp\_abel\_EXP}\index{Subroutines!ropp\_pp\_abel\_EXP}
\textbf{NAME:}\hspace{0.08in}\begin{Verbatim}
    ropp_pp_abel_EXP - Calculate a one dimensional bending angle
                       profile from refractivity / impact parameter profile at
                       required output levels using a Fast Abel Transform
                       Assume exponential variation of ln(n) with height 
                       between successive impact parameter levels.
\end{Verbatim}
\textbf{SYNOPSIS:}\hspace{0.08in}\begin{Verbatim}
    call ropp_pp_abel_EXP(nr, refrac, impact, bangle)
\end{Verbatim}
\textbf{DESCRIPTION:}\hspace{0.08in}\begin{Verbatim}
    This routine calculates bending angles at a given set of impact parameters
    from a vertical profile of refractivity given at the observed set of 
    x = nr levels.
\end{Verbatim}
\textbf{INPUTS:}\hspace{0.08in}\begin{Verbatim}
    real(wp), dimension(:) :: nr          ! x=nr product
    real(wp), dimension(:) :: refrac      ! Refractivity values
    real(wp), dimension(:) :: impact      ! Impact parameters
\end{Verbatim}
\textbf{OUTPUT:}\hspace{0.08in}\begin{Verbatim}
    real(wp), dimension(:) :: bangle      ! Calculated bending angles
\end{Verbatim}
\textbf{NOTES:}\hspace{0.08in}\begin{Verbatim}
    The interpolation of bending angles calculated at the input data
    geopotential levels to the output impact parameter levels is
    carried out assuming that bending angle varies exponentially with
    impact parameter.
\end{Verbatim}
\textbf{SEE ALSO:}\hspace{0.08in}\begin{Verbatim}
     ropp_pp_invert_EXP
\end{Verbatim}
\section{BendingAngle/ropp\_pp\_abel\_LIN}
\textsl{[ Subroutines ]}

\label{ch:robo1}
\label{ch:BendingAngle_ropp_pp_abel_LIN}
\index{unsorted!ropp\_pp\_abel\_LIN}\index{Subroutines!ropp\_pp\_abel\_LIN}
\textbf{NAME:}\hspace{0.08in}\begin{Verbatim}
    ropp_pp_abel_LIN - Calculate a one dimensional bending angle
                       profile from refractivity / impact parameter profile at
                       required output levels using a Fast Abel Transform
                       Assume linear variation of ln(n) with impact height.  
\end{Verbatim}
\textbf{SYNOPSIS:}\hspace{0.08in}\begin{Verbatim}
    call ropp_pp_abel(nr, refrac, impact, bangle, dlndx, scale)
\end{Verbatim}
\textbf{DESCRIPTION:}\hspace{0.08in}\begin{Verbatim}
    This routine calculates bending angles at a given set of impact parameters
    from a vertical profile of refractivity given at the observation set of
    impact parameters.
\end{Verbatim}
\textbf{INPUTS:}\hspace{0.08in}\begin{Verbatim}
    real(wp), dimension(:) :: nr          ! x=nr product
    real(wp), dimension(:) :: refrac      ! Refractivity values
    real(wp), dimension(:) :: impact      ! Impact parameters
    real(wp), dimension(:) :: dlndx       ! Refractive index gradient
    real(wp),              :: scale       ! Scale height
\end{Verbatim}
\textbf{OUTPUT:}\hspace{0.08in}\begin{Verbatim}
    real(wp), dimension(:) :: bangle       ! Calculated bending angles
\end{Verbatim}
\textbf{NOTES:}\hspace{0.08in}\begin{Verbatim}
    The interpolation of bending angles calculated at the input data
    geopotential levels to the output impact parameter levels is
    carried out assuming that bending angle varies exponentially with
    impact parameter.
\end{Verbatim}
\textbf{SEE ALSO:}\hspace{0.08in}\begin{Verbatim}
    ropp_pp_abel_EXP
\end{Verbatim}
\section{BendingAngle/ropp\_pp\_invert\_EXP}
\textsl{[ Subroutines ]}

\label{ch:robo2}
\label{ch:BendingAngle_ropp_pp_invert_EXP}
\index{unsorted!ropp\_pp\_invert\_EXP}\index{Subroutines!ropp\_pp\_invert\_EXP}
\textbf{NAME:}\hspace{0.08in}\begin{Verbatim}
    ropp_pp_invert_EXP - Calculate a one dimensional refractivity profile from
                         bending angle / impact parameter profile using a 
                         Fast Abel Transform. Assume exponential variation of 
                         bending angle with height between successive impact 
                         parameter levels.
\end{Verbatim}
\textbf{SYNOPSIS:}\hspace{0.08in}\begin{Verbatim}
    call ropp_pp_invert_EXP(impact, bangle, nr, refrac)
\end{Verbatim}
\textbf{DESCRIPTION:}\hspace{0.08in}\begin{Verbatim}
    This routine calculates refractivity at a given set of x=nr levels
    from a vertical profile of bending angles given at a set 
    of impact parameters.
\end{Verbatim}
\textbf{INPUTS:}\hspace{0.08in}\begin{Verbatim}
    real(wp), dimension(:) :: impact      ! Input impact parameters 
    real(wp), dimension(:) :: bangle      ! Bending angles
    real(wp), dimension(:) :: nr          ! x = nr product
\end{Verbatim}
\textbf{OUTPUT:}\hspace{0.08in}\begin{Verbatim}
    real(wp), dimension(:) :: refrac      ! Refractivity values
\end{Verbatim}
\textbf{NOTES:}\hspace{0.08in}\begin{Verbatim}
    The interpolation of refractivity calculated at the input data
    impact height levels to the output geopotential levels is
    carried out assuming that dln(n)/dx varies exponentially with x.
\end{Verbatim}
\textbf{SEE ALSO:}\hspace{0.08in}\begin{Verbatim}
    ropp_pp_invert_LIN
\end{Verbatim}
\section{BendingAngle/ropp\_pp\_invert\_LIN}
\textsl{[ Subroutines ]}

\label{ch:robo3}
\label{ch:BendingAngle_ropp_pp_invert_LIN}
\index{unsorted!ropp\_pp\_invert\_LIN}\index{Subroutines!ropp\_pp\_invert\_LIN}
\textbf{NAME:}\hspace{0.08in}\begin{Verbatim}
    ropp_pp_invert_LIN - Calculate a one dimensional refractivity profile from
                         bending angle / impact parameter profile
                         using a Fast Abel Transform
                         Assume linear variation of bending angle with height
                         between successive impact parameter levels
\end{Verbatim}
\textbf{SYNOPSIS:}\hspace{0.08in}\begin{Verbatim}
    call ropp_pp_invert_LIN(impact, bangle, nr, refrac, scale)
\end{Verbatim}
\textbf{DESCRIPTION:}\hspace{0.08in}\begin{Verbatim}
    This routine calculates refractivity at a given set of x=nr levels
    from a vertical profile of bending angles given at a set of
    impact parameters.
\end{Verbatim}
\textbf{INPUTS:}\hspace{0.08in}\begin{Verbatim}
    real(wp), dimension(:) :: impact      ! Input impact parameters 
    real(wp), dimension(:) :: bangle      ! Bending angles
    real(wp), dimension(:) :: nr          ! x=nr product
    real(wp), optional     :: scale       ! Vertical scale height
\end{Verbatim}
\textbf{OUTPUT:}\hspace{0.08in}\begin{Verbatim}
    real(wp), dimension(:) :: refrac      ! Refractivity values
\end{Verbatim}
\textbf{NOTES:}\hspace{0.08in}\begin{Verbatim}
    The interpolation of refractivity calculated at the input data
    impact height levels to the output geopotential levels is
    carried out assuming that dln(n)/dx varies linearly with x.
\end{Verbatim}
\textbf{SEE ALSO:}\hspace{0.08in}\begin{Verbatim}
    ropp_pp_invert_LIN
\end{Verbatim}
\section{common/ropp\_pp\_version}
\textsl{[ Functions ]}

\label{ch:robo4}
\label{ch:common_ropp_pp_version}
\index{unsorted!ropp\_pp\_version}\index{Functions!ropp\_pp\_version}
\textbf{NAME:}\hspace{0.08in}\begin{Verbatim}
   ropp_pp_version
\end{Verbatim}
\textbf{SYNOPSIS:}\hspace{0.08in}\begin{Verbatim}
   Return ROPP_PP version ID string

   USE ropp_pp
   version = ropp_pp_version()
\end{Verbatim}
\textbf{DESCRIPTION:}\hspace{0.08in}\begin{Verbatim}
   This function returns the (common) version string for the ROPP_PP
   module. By default, this function should be called by all ROPP_PP
   tools to display a version ID when the '-v' command-line switch is
   used.
\end{Verbatim}
\section{Constants/Earth}
\textsl{[ Topics ]}

\label{ch:robo5}
\label{ch:Constants_Earth}
\index{unsorted!Earth}\index{Topics!Earth}
\textbf{DESCRIPTION:}\hspace{0.08in}\begin{Verbatim}
    Physical constants related to the Earth as used in the ROPP Forward Model
    library. 
\end{Verbatim}
\textbf{SEE ALSO:}\hspace{0.08in}\begin{Verbatim}
    g_wmo
    c_light
    R_dry
    R_vap
    C_p
    mw_dry_air
    mw_water
    epsilon_water
\end{Verbatim}
\subsection{Earth/g\_wmo}
\textsl{[ Earth ]}
\textsl{[ Parameters ]}

\label{ch:robo12}
\label{ch:Earth_g_wmo}
\index{unsorted!g\_wmo}\index{Parameters!g\_wmo}
\textbf{NAME:}\hspace{0.08in}\begin{Verbatim}
    g_wmo - Gravitational acceleration (WMO standard value)
\end{Verbatim}
\textbf{SOURCE:}\hspace{0.08in}\begin{Verbatim}
  REAL(wp), PARAMETER :: g_wmo = 9.80665_wp         !   m s^{-2}
\end{Verbatim}
\section{Constants/Maths}
\textsl{[ Topics ]}

\label{ch:robo6}
\label{ch:Constants_Maths}
\index{unsorted!Maths}\index{Topics!Maths}
\textbf{DESCRIPTION:}\hspace{0.08in}\begin{Verbatim}
    Mathematical constants as used in the ROPP Forward Model library. 
\end{Verbatim}
\textbf{SEE ALSO:}\hspace{0.08in}\begin{Verbatim}
    pi
\end{Verbatim}
\subsection{Maths/pi}
\textsl{[ Maths ]}
\textsl{[ Parameters ]}

\label{ch:robo32}
\label{ch:Maths_pi}
\index{unsorted!pi}\index{Parameters!pi}
\textbf{NAME:}\hspace{0.08in}\begin{Verbatim}
    pi  
\end{Verbatim}
\textbf{SOURCE:}\hspace{0.08in}\begin{Verbatim}
  REAL(wp), PARAMETER :: pi = 3.141592653589793238_wp   
\end{Verbatim}
\section{Constants/RadioOccultation}
\textsl{[ Topics ]}

\label{ch:robo7}
\label{ch:Constants_RadioOccultation}
\index{unsorted!RadioOccultation}\index{Topics!RadioOccultation}
\textbf{DESCRIPTION:}\hspace{0.08in}\begin{Verbatim}
    Physical constants specific for radio occultation measurement as used in
    the ROPP Forward Model library.
\end{Verbatim}
\textbf{SEE ALSO:}\hspace{0.08in}\begin{Verbatim}
    kappa1
    kappa2
\end{Verbatim}
\subsection{RadioOccultation/kappa1}
\textsl{[ RadioOccultation ]}
\textsl{[ Parameters ]}

\label{ch:robo93}
\label{ch:RadioOccultation_kappa1}
\index{unsorted!kappa1}\index{Parameters!kappa1}
\textbf{NAME:}\hspace{0.08in}\begin{Verbatim}
    kappa1 - Coefficient to calculate dry refractivity contribution.
\end{Verbatim}
\textbf{SOURCE:}\hspace{0.08in}\begin{Verbatim}
  REAL(wp), PARAMETER :: kappa1 = 77.60e-2_wp      !   K Pa^{-1}
\end{Verbatim}
\subsection{RadioOccultation/kappa2}
\textsl{[ RadioOccultation ]}
\textsl{[ Parameters ]}

\label{ch:robo94}
\label{ch:RadioOccultation_kappa2}
\index{unsorted!kappa2}\index{Parameters!kappa2}
\textbf{NAME:}\hspace{0.08in}\begin{Verbatim}
    kappa2 - Coefficient to calculate moist refractivity contribution.
\end{Verbatim}
\textbf{SOURCE:}\hspace{0.08in}\begin{Verbatim}
  REAL(wp), PARAMETER :: kappa2 = 3.73e3_wp        !   K^2 Pa^{-1}
\end{Verbatim}
\section{Datatypes/KeyConfig}
\textsl{[ Structures ]}

\label{ch:robo9}
\label{ch:Datatypes_KeyConfig}
\index{unsorted!KeyConfig}\index{Structures!KeyConfig}
\textbf{NAME:}\hspace{0.08in}\begin{Verbatim}
    KeyConfig - Key-value pairs defined in PP config files
\end{Verbatim}
\textbf{SYNOPSIS:}\hspace{0.08in}\begin{Verbatim}
    use ropp_pp_types
      ...
    type(KeyConfig) :: key_value
\end{Verbatim}
\textbf{NOTES:}\hspace{0.08in}\textbf{SEE ALSO:}\hspace{0.08in}\textbf{SOURCE:}\hspace{0.08in}\begin{Verbatim}
  TYPE KeyConfig

     CHARACTER(len = 1024), DIMENSION(:), POINTER :: keys   => null()
     CHARACTER(len = 4096), DIMENSION(:), POINTER :: values => null()

  END TYPE KeyConfig
\end{Verbatim}
\section{Datatypes/PPConfig}
\textsl{[ Structures ]}

\label{ch:robo10}
\label{ch:Datatypes_PPConfig}
\index{unsorted!PPConfig}\index{Structures!PPConfig}
\textbf{NAME:}\hspace{0.08in}\begin{Verbatim}
    PPConfig - Configuration options for excess phase to bending angle and 
               ionospheric correction processing
\end{Verbatim}
\textbf{SYNOPSIS:}\hspace{0.08in}\begin{Verbatim}
    use ropp_pp
      ...
    type(PPConfig) :: config
\end{Verbatim}
\textbf{SOURCE:}\hspace{0.08in}\begin{Verbatim}
  TYPE ppConfig

     LOGICAL  :: obs_ok   = .TRUE.      ! Observations QC flag
     
     LOGICAL  :: output_tdry = .TRUE.  ! Flag to output 'dry' parameters

     LOGICAL  :: output_diag = .FALSE.  ! Flag to output additional diagnostics

     !! Occultation (excess phase to bending angle) processing

     ! Excess phase to bangle method (GO = geometric optics, WO = wave optics)
     CHARACTER(len=10) :: occ_method = "WO"
     ! Filter type for phase differentiation (slpoly = sliding polynomial, optest = optimal estimation filter)
     CHARACTER(len=10) :: filter_method = "slpoly"
     ! Filter width for smoothed GO bending angles (m)
     REAL(wp) :: fw_go_smooth =  3000.0_wp  
     ! Filter width for computing full resolution GO bending angles (m)
     REAL(wp) :: fw_go_full   =  3000.0_wp  
     ! Filter width for wave optics bending angle above 7 km (m)
     REAL(wp) :: fw_wo        =  2000.0_wp 
     ! Filter width for wave optics bending angle below 7 km (m)
     REAL(wp) :: fw_low       =  -1000.0_wp 
     ! Maximum height for wave optics processing (m)
     REAL(wp) :: hmax_wo      = 25000.0_wp
     ! Fractional cut-off limit for amplitude
     REAL(wp) :: Acut     = 0.0_wp      
     ! Cut-off limit for impact height (m)
     REAL(wp) :: Pcut     = -2000.0_wp      
     ! Cut-off limit for bending angle (rad)
     REAL(wp) :: Bcut     = 0.1_wp 
     ! Cut-off limit for straight-line tangent altitude (m)
     REAL(wp) :: Hcut     = -250000.0_wp 
     ! Complex filter flag
     INTEGER  :: CFF = 3
     ! Shadow border width (m)
     REAL(wp) :: dsh = 200.0_wp
     ! Degraded L2 data flag
     LOGICAL  :: opt_DL2  = .TRUE.      
     ! Calculate and output spectra flag
     LOGICAL  :: opt_spectra = .FALSE.
     ! Path to EGM96 model coefficients file
     CHARACTER(len=80) :: egm96 = "egm96.dat"
     ! Path to EGM96 model corrections file
     CHARACTER(len=80) :: corr_egm96 = "corrcoef.dat"
     ! Path to external navigation bit file
     CHARACTER(len=80) :: navbit_file = " "

     !! Inversion (bending angle to refractivity) processing

     ! Ionospheric correction method (NONE = linear combination, MSIS = full,
     !                                GMSIS = full with global MSIS search,
     !                                BG = full with background profile)
     CHARACTER(len=10) :: method = "GMSIS"
     ! Statistical optimization method (so = statistical optimisation (default)
     !                                  lcso = linear combination+stat opt)
     CHARACTER(len=10) :: so_method = "so"
     ! Abel integral algorithm (LIN = linear, EXP = exponential)
     CHARACTER(len=10) :: abel = "LIN" 
     ! Model coefficients file path
     CHARACTER(len=80) :: mfile = "MSIS_coeff.nc"      
     ! Background atmospheric profile file path
     CHARACTER(len=80) :: bfile = "n/a"  
     ! Local radius of curvature (m)
     REAL(wp) :: r_curve               
     ! Number of input data points
     INTEGER  :: npoints               
     ! Step of standard impact parameter grid (m)
     REAL(wp) :: dpi      = 100.0_wp    

     ! Minimum impact parameter (m)
     REAL(wp) :: pmin                  
     ! Maximum impact parameter (m)
     REAL(wp) :: pmax                  
     ! Polynomial degree for smoothing regression
     INTEGER  :: np_smooth = 3         
     ! Filter width for smoothing profile
     REAL(wp) :: fw_smooth = 1000.0_wp 

     ! Number of parameters used for model fit regression
     REAL(wp) :: nparm_fit = 2
     ! Lower limit for model fit regression (m)
     REAL(wp) :: hmin_fit = 20000.0_wp 
     ! Upper limit for model fit regression (m)
     REAL(wp) :: hmax_fit = 70000.0_wp 
     ! a priori standard dev of regression factor
     REAL(wp) :: omega_fit = 0.3_wp    

     ! Ionospheric correction filter width (m)
     REAL(wp) :: f_width  = 2000.0_wp   
     ! Step of homogeneous impact p. grid (m)
     REAL(wp) :: delta_p  =  20.0_wp   
     ! External ionospheric smoothing scale (m)  
     REAL(wp) :: s_smooth = 2000.0_wp  
     ! Lower limit of ionospheric signal (m)
     REAL(wp) :: z_ion    = 50000.0_wp 
     ! Upper limit of stratospheric signal (m)
     REAL(wp) :: z_str    = 35000.0_wp 
     ! Upper limit of tropospheric signal (m)
     REAL(wp) :: z_ltr    = 12000.0_wp 
     ! Number of points for smoothing (odd)
     INTEGER  :: n_smooth = 11         

     ! Height of atmosphere for inversion (m)
     REAL(wp) :: ztop_invert = 150000.0_wp 
     ! Step of highest part inversion grid (m)
     REAL(wp) :: dzh_invert = 50.0_wp      
     ! Interval for regression in inversion (m)
     REAL(wp) :: dzr_invert = 20000.0_wp   

  END TYPE ppConfig
\end{Verbatim}
\section{Datatypes/PPDiag}
\textsl{[ Structures ]}

\label{ch:robo11}
\label{ch:Datatypes_PPDiag}
\index{unsorted!PPDiag}\index{Structures!PPDiag}
\textbf{NAME:}\hspace{0.08in}\begin{Verbatim}
    PPDiag -
\end{Verbatim}
\textbf{SYNOPSIS:}\hspace{0.08in}\begin{Verbatim}
    use ropp_pp
      ...
    type(PPDiag) :: diag
\end{Verbatim}
\textbf{SOURCE:}\hspace{0.08in}\begin{Verbatim}
  TYPE ppDiag
     REAL(wp) :: L2_badness    ! L2 phase correction badness score
     REAL(wp) :: sq            ! Statistical optimisation badness score
     ! CT processing impact parameter grid (m)
     REAL(wp), DIMENSION(:), POINTER :: CTimpact => null()
     ! CT processing amplitude 
     REAL(wp), DIMENSION(:), POINTER :: CTamplitude => null()
     ! CT processing smoothed amplitude 
     REAL(wp), DIMENSION(:), POINTER :: CTamplitude_smt => null()
     ! CT processing impact parameter grid (m)
     REAL(wp), DIMENSION(:), POINTER :: CTimpactL2 => null()
     ! CT processing amplitude 
     REAL(wp), DIMENSION(:), POINTER :: CTamplitudeL2 => null()
     ! CT processing smoothed amplitude 
     REAL(wp), DIMENSION(:), POINTER :: CTamplitudeL2_smt => null()
     ! Ionospheric bending angle in L1
     REAL(wp), DIMENSION(:), POINTER :: ba_ion => null()    
     ! Error covariance of ionospheric bending angle (rad**2)
     REAL(wp), DIMENSION(:), POINTER :: err_ion => null()   
     ! Error covariance of neutral bending angle (rad**2)
     REAL(wp), DIMENSION(:), POINTER :: err_neut => null()
     ! Weight of data (data:data+clim) in profile
     REAL(wp), DIMENSION(:), POINTER :: wt_data => null()

  END TYPE ppDiag
\end{Verbatim}
\section{FFT/ropp\_pp\_FFT\_complex}
\textsl{[ Subroutines ]}

\label{ch:robo13}
\label{ch:FFT_ropp_pp_FFT_complex}
\index{unsorted!ropp\_pp\_FFT\_complex}\index{Subroutines!ropp\_pp\_FFT\_complex}
\textbf{NAME:}\hspace{0.08in}\begin{Verbatim}
    ropp_pp_FFT - Compute Fast Fourier Transform of complex data
                  using the Danielson-Lanczos Lemma
\end{Verbatim}
\textbf{SYNOPSIS:}\hspace{0.08in}\begin{Verbatim}
    call ropp_pp_FFT(data, isign)
\end{Verbatim}
\textbf{DESCRIPTION:}\hspace{0.08in}\begin{Verbatim}
    This subroutine computes the Fast Fourier transform (or its inverse)
    of complex data using the Danielson-Lanczos Lemma. Based on dfour.f
    routine provided in Numerical Recipes.
\end{Verbatim}
\textbf{INPUTS:}\hspace{0.08in}\begin{Verbatim}
    complex(wp), dim(:)  :: data      Input complex data signal
    integer           :: isign     FFT direction
                                             > 0 - forward discrete FT
                                             < 0 - inverse discrete FT
\end{Verbatim}
\textbf{OUTPUT:}\hspace{0.08in}\begin{Verbatim}
    complex(wp), dim(:)  :: data      Transformed sequence
\end{Verbatim}
\textbf{REFERENCES:}\hspace{0.08in}\begin{Verbatim}
   W.H. Press, S.A. Teukolsjy, W.T. Vetterling and B.P. Flannery,
   Numerical Recipes in C - The Art of Scientific Computing.
   2nd Ed., Cambridge University Press, 1992.
\end{Verbatim}
\section{FFT/ropp\_pp\_FFT\_real}
\textsl{[ Subroutines ]}

\label{ch:robo14}
\label{ch:FFT_ropp_pp_FFT_real}
\index{unsorted!ropp\_pp\_FFT\_real}\index{Subroutines!ropp\_pp\_FFT\_real}
\textbf{NAME:}\hspace{0.08in}\begin{Verbatim}
    ropp_pp_FFT - Compute Fast Fourier Transform of complex data
                  (real representation) using the Danielson-Lanczos Lemma
\end{Verbatim}
\textbf{SYNOPSIS:}\hspace{0.08in}\begin{Verbatim}
    call ropp_pp_FFT(data, isign)
\end{Verbatim}
\textbf{DESCRIPTION:}\hspace{0.08in}\begin{Verbatim}
    This subroutine computes the Fast Fourier transform (or its inverse)
    of complex data using the Danielson-Lanczos Lemma. Based on dfour.f
    routine provided in Numerical Recipes.
\end{Verbatim}
\textbf{INPUTS:}\hspace{0.08in}\begin{Verbatim}
    real(wp), dim(:)  :: data      Input complex data signal
    integer           :: isign     FFT direction
                                             > 0 - forward discrete FT
                                             < 0 - inverse discrete FT
\end{Verbatim}
\textbf{OUTPUT:}\hspace{0.08in}\begin{Verbatim}
    real(wp), dim(:)  :: data      Transformed sequence
\end{Verbatim}
\textbf{NOTES:}\hspace{0.08in}\begin{Verbatim}
   The signal array contains complex data stored as
     data(1) = real part f(1)
     data(2) = imag part f(1)
     data(3) = real part f(2)
     data(4) = imag part f(2)
      ...
     data(2*nn-1) = real part f(nn)
     data(2*nn)   = imag part f(nn)
   etc....
\end{Verbatim}
\textbf{REFERENCES:}\hspace{0.08in}\begin{Verbatim}
   W.H. Press, S.A. Teukolsjy, W.T. Vetterling and B.P. Flannery,
   Numerical Recipes in C - The Art of Scientific Computing.
   2nd Ed., Cambridge University Press, 1992.
\end{Verbatim}
\section{FFT/ropp\_pp\_filter}
\textsl{[ Subroutines ]}

\label{ch:robo15}
\label{ch:FFT_ropp_pp_filter}
\index{unsorted!ropp\_pp\_filter}\index{Subroutines!ropp\_pp\_filter}
\textbf{NAME:}\hspace{0.08in}\begin{Verbatim}
    ropp_pp_filter - Filtering and differentitation of a signal
\end{Verbatim}
\textbf{SYNOPSIS:}\hspace{0.08in}\begin{Verbatim}
    call ropp_pp_filter(dt, s, w, nd, fs, ds)
\end{Verbatim}
\textbf{DESCRIPTION:}\hspace{0.08in}\begin{Verbatim}
    Optimal solution of integral equation
\end{Verbatim}
\textbf{INPUTS:}\hspace{0.08in}\begin{Verbatim}
    real(wp)             :: dt      Time step
    real(wp), dim([:],:) :: s       Signal samples ([channel],time)
    integer              :: w       Window width [npoints]
    integer              :: nd      Number of points for differentiation 
\end{Verbatim}
\textbf{OUTPUT:}\hspace{0.08in}\begin{Verbatim}
    real(wp), dim([:],:), optional  :: fs      Filtered signal
    real(wp), dim([:],:), optional  :: ds      Signal derivative
\end{Verbatim}
\section{FFT/ropp\_pp\_fourier\_filter}
\textsl{[ Subroutines ]}

\label{ch:robo16}
\label{ch:FFT_ropp_pp_fourier_filter}
\index{unsorted!ropp\_pp\_fourier\_filter}\index{Subroutines!ropp\_pp\_fourier\_filter}
\textbf{NAME:}\hspace{0.08in}\begin{Verbatim}
    ropp_pp_fourier_filter - Gaussian filter in spectral space
\end{Verbatim}
\textbf{SYNOPSIS:}\hspace{0.08in}\begin{Verbatim}
    call ropp_pp_fourier_filter(data, window)
\end{Verbatim}
\textbf{DESCRIPTION:}\hspace{0.08in}\begin{Verbatim}
    This routine filters an input complex signal by computing its spectrum,
    filtering the signal in spectral space, and inverse transforming back
    to the data space
\end{Verbatim}
\textbf{INPUTS:}\hspace{0.08in}\begin{Verbatim}
    complex(wp), dim(:)  :: data      Input complex data signal
    integer              :: window    Filtering window width (npoints)
\end{Verbatim}
\textbf{OUTPUT:}\hspace{0.08in}\begin{Verbatim}
    complex(wp), dim(:)  :: data      Filtered sequence
\end{Verbatim}
\textbf{REFERENCES:}\hspace{0.08in}\begin{Verbatim}
   W.H. Press, S.A. Teukolsjy, W.T. Vetterling and B.P. Flannery,
   Numerical Recipes in C - The Art of Scientific Computing.
   2nd Ed., Cambridge University Press, 1992.
\end{Verbatim}
\section{FFT/ropp\_pp\_sliding\_polynomial}
\textsl{[ Subroutines ]}

\label{ch:robo17}
\label{ch:FFT_ropp_pp_sliding_polynomial}
\index{unsorted!ropp\_pp\_sliding\_polynomial}\index{Subroutines!ropp\_pp\_sliding\_polynomial}
\textbf{NAME:}\hspace{0.08in}\begin{Verbatim}
    ropp_pp_sliding_poly - Least-square fitting polynomial in sliding
                           windows
\end{Verbatim}
\textbf{SYNOPSIS:}\hspace{0.08in}\begin{Verbatim}
    call ropp_pp_sliding_polynomial(t, s, w, np, fs, ds)
\end{Verbatim}
\textbf{DESCRIPTION:}\hspace{0.08in}\begin{Verbatim}
    Least-square fitting polynomial in sliding windows 
\end{Verbatim}
\textbf{INPUTS:}\hspace{0.08in}\begin{Verbatim}
    real(wp)             :: t       Time
    real(wp), dim([:],:) :: s       Signal samples ([channel],time)
    integer, [dim(:)],   :: w       Window width [npoints]
    integer              :: np      Polynomial degree
\end{Verbatim}
\textbf{OUTPUT:}\hspace{0.08in}\begin{Verbatim}
    real(wp), dim([:],:), optional  :: fs      Filtered signal
    real(wp), dim([:],:), optional  :: ds      Signal derivative
\end{Verbatim}
\section{GeometricOptics/ropp\_pp\_bending\_angle\_go}
\textsl{[ Subroutines ]}

\label{ch:robo18}
\label{ch:GeometricOptics_ropp_pp_bending_angle_go}
\index{unsorted!ropp\_pp\_bending\_angle\_go}\index{Subroutines!ropp\_pp\_bending\_angle\_go}
\textbf{NAME:}\hspace{0.08in}\begin{Verbatim}
    ropp_pp_bending_angle_go - Calculate L1 and L2 bending angle profiles from
                               occultation data by GEOMETRIC OPTICS
\end{Verbatim}
\textbf{SYNOPSIS:}\hspace{0.08in}\begin{Verbatim}
    call ropp_pp_bending_angle_go(time, r_leo, r_gns, r_coc, phase_L1,
                                  phase_L2, w_smooth, filter, impact, bangle)
\end{Verbatim}
\textbf{DESCRIPTION:}\hspace{0.08in}\begin{Verbatim}
    This routine calculates L1 and L2 bending angles.
      1) Detrend excess phase using spline regression
      2) Numerical differentiation of detrended excess phase
      3) Calculation of impact parameter and bending angle from Doppler shift
         using Snell's Law - GEOMETRIC OPTICS method
\end{Verbatim}
\textbf{INPUTS:}\hspace{0.08in}\begin{Verbatim}
    real(wp), dimension(:)   :: time      ! Relative time of samples (s)
    real(wp), dimension(:,:) :: r_leo     ! Cartesian LEO coordinates (m)
    real(wp), dimension(:,:) :: r_gns     ! Cartesian GPS coordinates (m)
    real(wp), dimension(:)   :: r_coc     ! Centre curvature coordinates (m)
    integer                  :: w_smooth  ! Smoothing window (points)
    character(len=*)         :: filter    ! Filter type (optest/slpoly)
    real(wp), dimension(:)   :: phase_L1  ! L1 excess phase (m)
    real(wp), dimension(:)   :: phase_L2  ! L2 excess phase (m)
\end{Verbatim}
\textbf{OUTPUT:}\hspace{0.08in}\begin{Verbatim}
    real(wp), dimension(:)   :: impact_L1 ! L1 impact parameters (m)
    real(wp), dimension(:)   :: bangle_L1 ! L1 bending angles (rad)
    real(wp), dimension(:)   :: impact_L2 ! L2 impact parameters (m)
    real(wp), dimension(:)   :: bangle_L2 ! L2 bending angles (rad)
\end{Verbatim}
\textbf{NOTES:}\hspace{0.08in}\begin{Verbatim}
    Method:
        1. Detrending excess phase using spline regression.
        2. Numerical differentiation of detrended phase using
           optimal solution of integral equation in matrix form.
        3. Calculation of impact parameter and bending
           angle from Doppler shift using Snell's law.
\end{Verbatim}
\textbf{REFERENCES:}\hspace{0.08in}\begin{Verbatim}
   Vorob'ev, V.V. and Krasil'nikova T.G. 1994, 
   Estimation of the accuracy of the atmospheric refractive index recovery
   from doppler Sshift measurements at frequencies used in the NAVSTAR system
   Physics of the Atmosphere and Ocean (29) 602-609

  Gorbunov, M.E. and Kornblueh, L. 2003,
  Principles of variational assimilation of GNSS radio occultation data
  Max Planck Institute Report 350 
  http://www.mpimet.mpg.de/fileadmin/publikationen/Reports/max_scirep_350.pdf
\end{Verbatim}
\section{GeometricOptics/ropp\_pp\_geometric\_optics}
\textsl{[ Subroutines ]}

\label{ch:robo19}
\label{ch:GeometricOptics_ropp_pp_geometric_optics}
\index{unsorted!ropp\_pp\_geometric\_optics}\index{Subroutines!ropp\_pp\_geometric\_optics}
\textbf{NAME:}\hspace{0.08in}\begin{Verbatim}
    ropp_pp_geometric_optics - Calculate bending angle and impact parameter 
                               from relative Doppler frequency shift.
\end{Verbatim}
\textbf{SYNOPSIS:}\hspace{0.08in}\begin{Verbatim}
    call ropp_pp_geometric_optics(r_leo, v_leo, r_gns, v_gns, doppler,
                                  impact, bangle)
\end{Verbatim}
\textbf{DESCRIPTION:}\hspace{0.08in}\begin{Verbatim}
    This routine calculates bending angle and impact parameter from relative
    Doppler frequency shift.
    Iterative solution of the system of equations:
               (c - (v_leo, U_leo))/(c - (v_gns, U_gns)) - 1 = doppler
               [r_leo, U_leo] - [r_gns, U_gns] = 0
               (U_leo, U_leo) = 1
               (U_gns, U_gns) = 1
   where U_leo and U_gns are the ray directions at the receiver and 
   transmitter respectively.
\end{Verbatim}
\textbf{INPUTS:}\hspace{0.08in}\begin{Verbatim}
    real(wp), dimension(:) :: r_leo     ! relative LEO position (m) [ECI]
    real(wp), dimension(:) :: v_leo     ! LEO velocity (m/s) [ECI]  
    real(wp), dimension(:) :: r_gns     ! relative GPS position (m) [ECI]
    real(wp), dimension(:) :: v_gns     ! GPS velocity (m/s) [ECI]  
    real(wp)               :: doppler   ! relative Doppler frequency shift
\end{Verbatim}
\textbf{OUTPUT:}\hspace{0.08in}\begin{Verbatim}
    real(wp)               :: impact    ! impact parameter (m)
    real(wp)               :: bangle    ! bending angle (rad)
\end{Verbatim}
\textbf{NOTES:}\hspace{0.08in}\textbf{REFERENCES:}\hspace{0.08in}\begin{Verbatim}
   Vorob'ev and Krasil'nikova 1994, 
   Estimation of the accuracy of the atmospheric refractive index recovery
   from doppler Sshift measurements at frequencies used in the NAVSTAR system
   Physics of the Atmosphere and Ocean (29) 602-609

  Gorbunov, M.E. and Kornblueh, L. 2003,
  Principles of variational assimilation of GNSS radio occultation data
  Max PlancK Institute Report 350 
  http://www.mpimet.mpg.de/fileadmin/publikationen/Reports/max_scirep_350.pdf
\end{Verbatim}
\section{GeometricOptics/ropp\_pp\_geometric\_optics\_adj}
\textsl{[ Subroutines ]}

\label{ch:robo20}
\label{ch:GeometricOptics_ropp_pp_geometric_optics_adj}
\index{unsorted!ropp\_pp\_geometric\_optics\_adj}\index{Subroutines!ropp\_pp\_geometric\_optics\_adj}
\textbf{NAME:}\hspace{0.08in}\begin{Verbatim}
    ropp_pp_geometric_optics_adj - Calculate bending angle and impact 
                                   parameter from relative Doppler frequency 
                                   shift. ADJOINT VERSION.
\end{Verbatim}
\textbf{SYNOPSIS:}\hspace{0.08in}\begin{Verbatim}
    call ropp_pp_geometric_optics_adj(r_leo, v_leo, r_gns, v_gns, doppler,
                                      impact, bangle, impact_dd, impact_dr, 
                                      bangle_dd, bangle_dr)
\end{Verbatim}
\textbf{DESCRIPTION:}\hspace{0.08in}\begin{Verbatim}
    This routine is the adjoint of ropp_pp_geometric_optics, which calculates 
    bending angle and impact parameter from relative Doppler frequency shift.
    Iterative solution of the system of equations:
               (c - (v_leo, U_leo))/(c - (v_gns, U_gns)) - 1 = doppler
               [r_leo, U_leo] - [r_gns, U_gns] = 0
               (U_leo, U_leo) = 1
               (U_gns, U_gns) = 1
   where U_leo and U_gns are the ray directions at the receiver and 
   transmitter respectively.
\end{Verbatim}
\textbf{INPUTS:}\hspace{0.08in}\begin{Verbatim}
    real(wp), dimension(:) :: r_leo     ! relative LEO position (ECI)
    real(wp), dimension(:) :: v_leo     ! LEO velocity (ECI)  
    real(wp), dimension(:) :: r_gns     ! relative GPS position (ECI)
    real(wp), dimension(:) :: v_gns     ! GPS velocity (ECI)  
    real(wp)               :: doppler   ! relative Doppler frequency shift
\end{Verbatim}
\textbf{OUTPUT:}\hspace{0.08in}\begin{Verbatim}
    real(wp)               :: impact    ! impact parameter (m)
    real(wp)               :: bangle    ! bending angle (rad)
    real(wp)               :: impact_dd ! d(IP)/d(d) 
    real(wp), dimension(:) :: impact_dr ! d(IP)/d(r_leo,r_gns)
    real(wp)               :: bangle_dd ! d(BA)/d(d) 
    real(wp), dimension(:) :: bangle_dr ! d(BA)/d(r_leo,r_gns) 
\end{Verbatim}
\textbf{NOTES:}\hspace{0.08in}\textbf{REFERENCES:}\hspace{0.08in}\begin{Verbatim}
   Vorob'ev and Krasil'nikova 1994, 
   Estimation of the accuracy of the atmospheric refractive index recovery
   from doppler Sshift measurements at frequencies used in the NAVSTAR system
   Physics of the Atmosphere and Ocean (29) 602-609

  Gorbunov, M.E. and Kornblueh, L. 2003,
  Principles of variational assimilation of GNSS radio occultation data
  Max PlancK Institute Report 350 
  http://www.mpimet.mpg.de/fileadmin/publikationen/Reports/max_scirep_350.pdf
\end{Verbatim}
\section{GPSRO/L1}
\textsl{[ Parameters ]}

\label{ch:robo21}
\label{ch:GPSRO_L1}
\index{unsorted!L1}\index{Parameters!L1}
\textbf{NAME:}\hspace{0.08in}\begin{Verbatim}
    f_L1 - Carrier frequency for L1 signal 
\end{Verbatim}
\textbf{SOURCE:}\hspace{0.08in}\begin{Verbatim}
  REAL(wp), PARAMETER :: f_L1 = 1.57542e9_wp         !   Hz
\end{Verbatim}
\section{GPSRO/L1L2}
\textsl{[ Topics ]}

\label{ch:robo22}
\label{ch:GPSRO_L1L2}
\index{unsorted!L1L2}\index{Topics!L1L2}
\textbf{DESCRIPTION:}\hspace{0.08in}\begin{Verbatim}
    Carrier frequencies L1 and L2 as used in ROPP pre-processor library.
\end{Verbatim}
\section{GPSRO/L2}
\textsl{[ Parameters ]}

\label{ch:robo23}
\label{ch:GPSRO_L2}
\index{unsorted!L2}\index{Parameters!L2}
\textbf{NAME:}\hspace{0.08in}\begin{Verbatim}
    f_L2 - Carrier frequency for L2 signal 
\end{Verbatim}
\textbf{SOURCE:}\hspace{0.08in}\begin{Verbatim}
  REAL(wp), PARAMETER :: f_L2 = 1.22760e9_wp         !   Hz
\end{Verbatim}
\section{Interface/Modules}
\textsl{[ Topics ]}

\label{ch:robo24}
\label{ch:Interface_Modules}
\index{unsorted!Modules}\index{Topics!Modules}
\textbf{SYNOPSIS:}\hspace{0.08in}\begin{Verbatim}
    use ropp_pp
    use ropp_pp_types
    use ropp_pp_constants
    use ropp_pp_utils
    use ropp_pp_spline
    use ropp_pp_MSIS
\end{Verbatim}
\subsection{Modules/ropp\_pp}
\textsl{[ Modules ]}
\textsl{[ Modules ]}

\label{ch:robo42}
\label{ch:Modules_ropp_pp}
\index{unsorted!ropp\_pp}\index{Modules!ropp\_pp}
\textbf{NAME:}\hspace{0.08in}\begin{Verbatim}
    ropp_pp - Interface module for the ROPP pre-processor
\end{Verbatim}
\textbf{SYNOPSIS:}\hspace{0.08in}\begin{Verbatim}
    use ropp_pp
\end{Verbatim}
\textbf{DESCRIPTION:}\hspace{0.08in}\begin{Verbatim}
    This module provides interfaces for all pre-processor routines in the
    ROPP Preprocessor library.
\end{Verbatim}
\textbf{NOTES:}\hspace{0.08in}\textbf{SEE ALSO:}\hspace{0.08in}\begin{Verbatim}
    ropp_pp_constants
\end{Verbatim}
\subsection{Modules/ropp\_pp\_constants}
\textsl{[ Modules ]}
\textsl{[ Modules ]}

\label{ch:robo43}
\label{ch:Modules_ropp_pp_constants}
\index{unsorted!ropp\_pp\_constants}\index{Modules!ropp\_pp\_constants}
\textbf{NAME:}\hspace{0.08in}\begin{Verbatim}
    ropp_pp_constants - Module providing meteorological and physical constants.
\end{Verbatim}
\textbf{SYNOPSIS:}\hspace{0.08in}\begin{Verbatim}
    use ropp_pp_constants
\end{Verbatim}
\textbf{DESCRIPTION:}\hspace{0.08in}\begin{Verbatim}
    This module provides numerical values for meteorological and other
    physical constants which are used throughout the ROPP package.
\end{Verbatim}
\textbf{SEE ALSO:}\hspace{0.08in}\begin{Verbatim}
    Thermodynamical constants:  R_dry, R_vap, C_p,
                                mw_dry_air, mw_water, epsilon_water
    GPSRO carrier frequencies:  f_L1, f_L2
    Refractivity constants:     kappa1, kappa2
    Light speed in vacuum:      c_light
    Gravitational acceleration: g_wmo 
    Mathematical constants:     pi
\end{Verbatim}
\subsection{Modules/ropp\_pp\_copy}
\textsl{[ Modules ]}
\textsl{[ Modules ]}

\label{ch:robo44}
\label{ch:Modules_ropp_pp_copy}
\index{unsorted!ropp\_pp\_copy}\index{Modules!ropp\_pp\_copy}
\textbf{NAME:}\hspace{0.08in}\begin{Verbatim}
    ropp_pp_copy - Interface module for the ROPP PP copying function to ROprof
\end{Verbatim}
\textbf{SYNOPSIS:}\hspace{0.08in}\begin{Verbatim}
    use ropp_pp_copy
\end{Verbatim}
\textbf{DESCRIPTION:}\hspace{0.08in}\begin{Verbatim}
    Data type/structure copying functions using ROprof structures used by the
    ROPP preprocessing module.
\end{Verbatim}
\textbf{SEE ALSO:}\hspace{0.08in}\begin{Verbatim}
    ropp_pp_diag2roprof
\end{Verbatim}
\subsection{Modules/ropp\_pp\_MSIS}
\textsl{[ Modules ]}
\textsl{[ Modules ]}

\label{ch:robo45}
\label{ch:Modules_ropp_pp_MSIS}
\index{unsorted!ropp\_pp\_MSIS}\index{Modules!ropp\_pp\_MSIS}
\textbf{NAME:}\hspace{0.08in}\begin{Verbatim}
    ropp_pp_MSIS - Interface module for routines to obtain MSIS bending angle
                   profiles in the ROPP pre-processor
\end{Verbatim}
\textbf{SYNOPSIS:}\hspace{0.08in}\begin{Verbatim}
    use ropp_pp_MSIS
\end{Verbatim}
\textbf{DESCRIPTION:}\hspace{0.08in}\begin{Verbatim}
    This module provides interfaces for all MSIS bending angle routines in the
    ROPP Preprocessor library.
\end{Verbatim}
\textbf{SEE ALSO:}\hspace{0.08in}\begin{Verbatim}
    ropp_pp
\end{Verbatim}
\subsection{Modules/ropp\_pp\_preproc}
\textsl{[ Modules ]}
\textsl{[ Modules ]}

\label{ch:robo46}
\label{ch:Modules_ropp_pp_preproc}
\index{unsorted!ropp\_pp\_preproc}\index{Modules!ropp\_pp\_preproc}
\textbf{NAME:}\hspace{0.08in}\begin{Verbatim}
    ropp_pp_preproc - Interface module for the ROPP pre-processing module.
\end{Verbatim}
\textbf{SYNOPSIS:}\hspace{0.08in}\begin{Verbatim}
    use ropp_pp_preproc
\end{Verbatim}
\textbf{DESCRIPTION:}\hspace{0.08in}\begin{Verbatim}
    Data type/structure copying functions using ROprof structures used by the
    ropp_pp module.
\end{Verbatim}
\textbf{SEE ALSO:}\hspace{0.08in}\begin{Verbatim}
    ropp_pp_set_coordinates
\end{Verbatim}
\subsection{Modules/ropp\_pp\_spline}
\textsl{[ Modules ]}
\textsl{[ Modules ]}

\label{ch:robo47}
\label{ch:Modules_ropp_pp_spline}
\index{unsorted!ropp\_pp\_spline}\index{Modules!ropp\_pp\_spline}
\textbf{NAME:}\hspace{0.08in}\begin{Verbatim}
    ropp_pp_spline - Spline interpolation routines
\end{Verbatim}
\textbf{SYNOPSIS:}\hspace{0.08in}\begin{Verbatim}
    use ropp_pp_spline
\end{Verbatim}
\textbf{DESCRIPTION:}\hspace{0.08in}\begin{Verbatim}
    This module provides spline interpolation routines used by
    the ROPP pre-processor package.
\end{Verbatim}
\subsection{Modules/ropp\_pp\_types}
\textsl{[ Modules ]}
\textsl{[ Modules ]}

\label{ch:robo48}
\label{ch:Modules_ropp_pp_types}
\index{unsorted!ropp\_pp\_types}\index{Modules!ropp\_pp\_types}
\textbf{NAME:}\hspace{0.08in}\begin{Verbatim}
    ropp_pp_types - Type declarations for ROPP PP library.
\end{Verbatim}
\textbf{SYNOPSIS:}\hspace{0.08in}\begin{Verbatim}
    use ropp_pp_types
\end{Verbatim}
\textbf{DESCRIPTION:}\hspace{0.08in}\begin{Verbatim}
    This module provides dervived type / structure definitions for the
    pre-processor routines in ROPP.
\end{Verbatim}
\textbf{NOTES:}\hspace{0.08in}\begin{Verbatim}
    The ropp_pp_types module is loaded automatically with the ropp_pp module.
\end{Verbatim}
\subsection{Modules/ropp\_pp\_utils}
\textsl{[ Modules ]}
\textsl{[ Modules ]}

\label{ch:robo49}
\label{ch:Modules_ropp_pp_utils}
\index{unsorted!ropp\_pp\_utils}\index{Modules!ropp\_pp\_utils}
\textbf{NAME:}\hspace{0.08in}\begin{Verbatim}
    ropp_pp_utils - Utility routines for ionospheric correction 
\end{Verbatim}
\textbf{SYNOPSIS:}\hspace{0.08in}\begin{Verbatim}
    use ropp_pp_utils
\end{Verbatim}
\textbf{DESCRIPTION:}\hspace{0.08in}\begin{Verbatim}
    This module provides signal and matrix processing routines used by
    the ROPP pre-processor ionospheric correction package.
\end{Verbatim}
\section{Interpolation/ropp\_pp\_interpol}
\textsl{[ Subroutines ]}

\label{ch:robo25}
\label{ch:Interpolation_ropp_pp_interpol}
\index{unsorted!ropp\_pp\_interpol}\index{Subroutines!ropp\_pp\_interpol}
\textbf{NAME:}\hspace{0.08in}\begin{Verbatim}
    ropp_pp_interpol - Interpolate linearly.
\end{Verbatim}
\textbf{SYNOPSIS:}\hspace{0.08in}\begin{Verbatim}
    call ropp_pp_interpol(x, newx, array, interp)
\end{Verbatim}
\textbf{DESCRIPTION:}\hspace{0.08in}\begin{Verbatim}
    This subroutine interpolates an array assuming it varies linearly in x.
\end{Verbatim}
\textbf{INPUTS:}\hspace{0.08in}\begin{Verbatim}
    real(wp), dim(:) :: x       Coordinate values.
    real(wp), dim(:) :: newx    New coordinate values.
    real(wp), dim(:) :: array   Data to be interpolated (lives on x).
    logical,optional :: Cext    Constant or linear (default) extrapolation
\end{Verbatim}
\textbf{OUTPUT:}\hspace{0.08in}\begin{Verbatim}
    real(wp), dim(:) :: interp  Interpolated data (lives on newx).
\end{Verbatim}
\textbf{NOTES:}\hspace{0.08in}\begin{Verbatim}
    The coordinate array x must be strictly monotonically increasing. If
    elements of newx are outside the range of x, data will be extrapolated.

    None of the above conditions are checked for, but wrong or unexpected
    results will be obtained if thone of them is not met.
\end{Verbatim}
\section{Interpolation/ropp\_pp\_interpol\_log}
\textsl{[ Subroutines ]}

\label{ch:robo26}
\label{ch:Interpolation_ropp_pp_interpol_log}
\index{unsorted!ropp\_pp\_interpol\_log}\index{Subroutines!ropp\_pp\_interpol\_log}
\textbf{NAME:}\hspace{0.08in}\begin{Verbatim}
    ropp_pp_interpol - Interpolate logarithmically.
\end{Verbatim}
\textbf{SYNOPSIS:}\hspace{0.08in}\begin{Verbatim}
    call ropp_pp_interpol_log(x, newx, array, interp)
\end{Verbatim}
\textbf{DESCRIPTION:}\hspace{0.08in}\begin{Verbatim}
    This subroutine interpolates an array assuming it varies exponentially
    in x i.e. assuming its log varies linearly as a function of x.  
\end{Verbatim}
\textbf{INPUTS:}\hspace{0.08in}\begin{Verbatim}
    real(wp), dim(:) :: x       Coordinate values.
    real(wp), dim(:) :: newx    New coordinate values.
    real(wp), dim(:) :: array   Data to be interpolated (lives on x).
\end{Verbatim}
\textbf{OUTPUT:}\hspace{0.08in}\begin{Verbatim}
    real(wp), dim(:) :: interp  Interpolated data (lives on newx).
\end{Verbatim}
\textbf{NOTES:}\hspace{0.08in}\begin{Verbatim}
    Array must be strictly positive.

    The coordinate array x must be strictly monotonically increasing. If
    elements of newx are outside the range of x, data will be extrapolated.

    None of the above conditions are checked for, but wrong or unexpected
    results will be obtained if thone of them is not met. 
\end{Verbatim}
\textbf{SEE ALSO:}\hspace{0.08in}\begin{Verbatim}
    ropp_pp_interpol
\end{Verbatim}
\section{IonosphericCorrection/ropp\_pp\_invert\_refraction}
\textsl{[ Subroutines ]}

\label{ch:robo27}
\label{ch:IonosphericCorrection_ropp_pp_invert_refraction}
\index{unsorted!ropp\_pp\_invert\_refraction}\index{Subroutines!ropp\_pp\_invert\_refraction}
\textbf{NAME:}\hspace{0.08in}\begin{Verbatim}
    ropp_pp_invert_refraction - Invert neutral atmosphere bending angle
\end{Verbatim}
\textbf{SYNOPSIS:}\hspace{0.08in}\begin{Verbatim}
    call ropp_pp_invert_refraction(mfile, month, lat, lon, impact, bangle,
                                   geop, refrac, config)
\end{Verbatim}
\textbf{DESCRIPTION:}\hspace{0.08in}\begin{Verbatim}
    This subroutine calculates the refractivity profile for a given location
    from a bending angle profile combined with MSIS climatology
\end{Verbatim}
\textbf{INPUTS:}\hspace{0.08in}\begin{Verbatim}
    character(len=*) :: mfile          Model coefficients file
    integer          :: month          Month of year  
    real(wp)         :: lat            Latitude  (deg)
    real(wp)         :: lon            Longitude (deg)
    real(wp), dim(:) :: impact         Impact parameter (m)
    real(wp), dim(:) :: bangle         Bending angles (rad)
    type(ppConfig)   :: config         Configuration parameters
\end{Verbatim}
\textbf{OUTPUT:}\hspace{0.08in}\begin{Verbatim}
    real(wp), dim(:) :: geop           Geopotential height (m)
    real(wp), dim(:) :: refrac         Refractivity (N-units)
\end{Verbatim}
\section{IonosphericCorrection/ropp\_pp\_ionospheric\_correction}
\textsl{[ Subroutines ]}

\label{ch:robo28}
\label{ch:IonosphericCorrection_ropp_pp_ionospheric_correction}
\index{unsorted!ropp\_pp\_ionospheric\_correction}\index{Subroutines!ropp\_pp\_ionospheric\_correction}
\textbf{NAME:}\hspace{0.08in}\begin{Verbatim}
    ropp_pp_ionospheric_correction - 
                   Calculate neutral and ionospheric bending angle 
                   profile on L1 impact heights from L1/L2 bending angles 
\end{Verbatim}
\textbf{SYNOPSIS:}\hspace{0.08in}\begin{Verbatim}
    call ropp_pp_ionospheric_correction(impact_L1, ba_L1, 
                                        impact_L2, ba_L2,
                                        impact_LM, ba_LM,
                                        config, impact_LC, ba_LC, diag_out)
\end{Verbatim}
\textbf{DESCRIPTION:}\hspace{0.08in}\begin{Verbatim}
    This routine calculates bending angles at a given set of impact parameters
    from vertical profiles of bending angles at the two measurement 
    frequencies (channels) L1 and L2.
\end{Verbatim}
\textbf{INPUTS:}\hspace{0.08in}\begin{Verbatim}
    real(wp), dimension(:) :: impact_L1   ! Impact parameters of channel L1 (m)
    real(wp), dimension(:) :: ba_L1       ! Bending angles for channel L1 (rad)
    real(wp), dimension(:) :: impact_L2   ! Impact parameters of channel L2 (m)
    real(wp), dimension(:) :: ba_L2       ! Bending angles for channel L2 (rad)
    real(wp), dimension(:) :: impact_LM   ! Model impact parameters (m)
    real(wp), dimension(:) :: ba_LM       ! Model bending angles (rad)
    type(ppConfig)         :: config      ! Configuration parameters
\end{Verbatim}
\textbf{OUTPUT:}\hspace{0.08in}\begin{Verbatim}
    real(wp), dimension(:) :: impact_LC   ! Impact parameters of channel L1
    real(wp), dimension(:) :: ba_LC       ! Corrected bending angles 
    type(ppDiag)           :: diag_out    ! Output diagnostics
\end{Verbatim}
\textbf{NOTES:}\hspace{0.08in}\begin{Verbatim}
    Method:
        1. Calculation of strongly smoothed ionospheric signal
           (using the external scale). Further deviations from
           this signal are calculated.
        2. Estimation of ionospheric signal and noise covariances
           using the highest part (> 50 km) of the occultation.
        3. Calculation of relative mean deviation of neutral refraction
           from the model refraction using signal at heights 12-35 km.
        4. Optimal linear combination for the same impact parameters using the
           covariances.
\end{Verbatim}
\textbf{REFERENCES:}\hspace{0.08in}\begin{Verbatim}
     M.E. Gorbunov
     Ionospheric correction and statistical optimization of radio 
     occultation data
     Radio Science, 37(5), 1084
\end{Verbatim}
\section{IonosphericCorrection/ropp\_pp\_linear\_combination}
\textsl{[ Subroutines ]}

\label{ch:robo29}
\label{ch:IonosphericCorrection_ropp_pp_linear_combination}
\index{unsorted!ropp\_pp\_linear\_combination}\index{Subroutines!ropp\_pp\_linear\_combination}
\textbf{NAME:}\hspace{0.08in}\begin{Verbatim}
    ropp_pp_linear_combination - 
                   Calculate a one dimensional bending angle
                   profile on L1 impact heights from L1/L2 bending angles 
                   using linear combination
\end{Verbatim}
\textbf{SYNOPSIS:}\hspace{0.08in}\begin{Verbatim}
    call ropp_pp_linear_combination(impact_L1, bangle_L1, 
                                    impact_L2, bangle_L2,
                                    impact_LC, bangle_LC)
\end{Verbatim}
\textbf{DESCRIPTION:}\hspace{0.08in}\begin{Verbatim}
    This routine calculates bending angles at a given set of impact parameters
    from vertical profiles of bending angles at the two measurement 
    frequencies (channels) L1 and L2.
\end{Verbatim}
\textbf{INPUTS:}\hspace{0.08in}\begin{Verbatim}
    real(wp), dimension(:) :: impact_L1   ! Impact parameters of channel L1
    real(wp), dimension(:) :: bangle_L1   ! Bending angles for channel L1
    real(wp), dimension(:) :: impact_L2   ! Impact parameters of channel L2
    real(wp), dimension(:) :: bangle_L2   ! Bending angles for channel L2
\end{Verbatim}
\textbf{OUTPUT:}\hspace{0.08in}\begin{Verbatim}
    real(wp), dimension(:) :: impact_LC   ! Impact parameters of channel L1
    real(wp), dimension(:) :: bangle_LC   ! Corrected bending angles 
\end{Verbatim}
\textbf{NOTES:}\hspace{0.08in}\begin{Verbatim}
    Uses a linear combination of bending angles at the common impact parameter
    (impact_L1)
                        
                          bangle1(impactL1)*f1*f1  - bangle2(impactL1)*f2*f2
   bangle_LC(impact_L1) = --------------------------------------------------
                                          f1*f1  - f2*f2
\end{Verbatim}
\textbf{REFERENCES:}\hspace{0.08in}\begin{Verbatim}
     M.E. Gorbunov
     Ionospheric correction and statistical optimization of radio 
     occultation data
     Radio Science, 37(5), 1084
\end{Verbatim}
\section{IonosphericCorrection/ropp\_pp\_merge\_profile}
\textsl{[ Subroutines ]}

\label{ch:robo30}
\label{ch:IonosphericCorrection_ropp_pp_merge_profile}
\index{unsorted!ropp\_pp\_merge\_profile}\index{Subroutines!ropp\_pp\_merge\_profile}
\textbf{NAME:}\hspace{0.08in}\begin{Verbatim}
    ropp_pp_merge_profile - 
                   Merge and interpolate bending angles onto standard grid
\end{Verbatim}
\textbf{SYNOPSIS:}\hspace{0.08in}\begin{Verbatim}
    call ropp_pp_merge_profile(impact_L1, bangle_L1, impact_L2, bangle_L2, 
                               impact_I1, bangle_I1, impact_I2, bangle_I2,
                               Pmin, Pmax)                               
\end{Verbatim}
\textbf{DESCRIPTION:}\hspace{0.08in}\begin{Verbatim}
    This routine calculates L1 and L2 bending angles on a standard vertical 
    grid by linear interpolation
\end{Verbatim}
\textbf{INPUTS:}\hspace{0.08in}\begin{Verbatim}
    real(wp), dimension(:) :: impact_L1   ! Impact parameters of channel L1
    real(wp), dimension(:) :: bangle_L1   ! Bending angles for channel L1
    real(wp), dimension(:) :: impact_L2   ! Impact parameters of channel L2
    real(wp), dimension(:) :: bangle_L2   ! Bending angles for channel L2
    real(wp), optional     :: Pmin        ! Minimum impact parameter
    real(wp), optional     :: Pmax        ! Maximum impact parameter
\end{Verbatim}
\textbf{OUTPUT:}\hspace{0.08in}\begin{Verbatim}
    real(wp), dimension(:) :: impact_I1   ! Standard impact parameter grid L1
    real(wp), dimension(:) :: bangle_I1   ! L1 bending angles on impact_I1
    real(wp), dimension(:) :: impact_I2   ! Standard impact parameter grid L1
    real(wp), dimension(:) :: bangle_I2   ! L2 bending angles on impact_I2
\end{Verbatim}
\section{IonosphericCorrection/ropp\_pp\_smooth\_profile}
\textsl{[ Subroutines ]}

\label{ch:robo31}
\label{ch:IonosphericCorrection_ropp_pp_smooth_profile}
\index{unsorted!ropp\_pp\_smooth\_profile}\index{Subroutines!ropp\_pp\_smooth\_profile}
\textbf{NAME:}\hspace{0.08in}\begin{Verbatim}
    ropp_pp_smooth_profile - Filter bending angle profile 
\end{Verbatim}
\textbf{SYNOPSIS:}\hspace{0.08in}\begin{Verbatim}
    call ropp_pp_smooth_profile(impact, bangle, smooth, config)
\end{Verbatim}
\textbf{DESCRIPTION:}\hspace{0.08in}\begin{Verbatim}
    This routine filters a signal by least-square fitting a polynomial
    in sliding windows
\end{Verbatim}
\textbf{INPUTS:}\hspace{0.08in}\begin{Verbatim}
    real(wp), dimension(:) :: impact      ! Impact parameters 
    real(wp), dimension(:) :: bangle      ! Bending angles
    type(ppConfig)         :: config      ! Configuration parameters 
\end{Verbatim}
\textbf{OUTPUT:}\hspace{0.08in}\begin{Verbatim}
    real(wp), dimension(:) :: smooth      ! Smoothed bending angles 
    type(ppConfig)         :: config      ! Configuration parameters 
\end{Verbatim}
\section{Meteo/ropp\_pp\_tdry}
\textsl{[ Subroutines ]}

\label{ch:robo33}
\label{ch:Meteo_ropp_pp_tdry}
\index{unsorted!ropp\_pp\_tdry}\index{Subroutines!ropp\_pp\_tdry}
\textbf{NAME:}\hspace{0.08in}\begin{Verbatim}
    ropp_pp_tdry - Compute temperature (and pressure profile) from 
                   refractivity assuming zero humidity.
\end{Verbatim}
\textbf{SYNOPSIS:}\hspace{0.08in}\begin{Verbatim}
    call ropp_pp_tdry(lat, alt, refrac, shum, t_dry, p_dry, Zmax)
\end{Verbatim}
\textbf{DESCRIPTION:}\hspace{0.08in}\begin{Verbatim}
    Calculate P(z) from numerical integration of barometric formula

            d ln(P(z))                    g(z)
            ---------- = - -------------------------------------
               dz           Rd T(N(z),P(z),Q(z)) (1 + eps(Q(z))

    given N(z) and Q(z) and known dependence T(N, P, Q).
    Calculate T(z) = T(N(z), P(z), Q(z)). 
\end{Verbatim}
\textbf{INPUTS:}\hspace{0.08in}\begin{Verbatim}
    real(wp)                 :: lat       ! latitude
    real(wp), dimension(:)   :: alt       ! altitude (z)
    real(wp), dimension(:)   :: refrac    ! refraction (N(z))
    real(wp), dimension(:)   :: shum      ! specific humidity (Q(z))
    real(wp), optional       :: zmax      ! upper integration height 
\end{Verbatim}
\textbf{OUTPUT:}\hspace{0.08in}\begin{Verbatim}
    real(wp), dimension(:)   :: t_dry     ! dry temperature (T(z))
    real(wp), dimension(:)   :: p_dry     ! dry pressure (P(z))
\end{Verbatim}
\section{ModelRefraction/ropp\_pp\_bangle\_MSIS}
\textsl{[ Subroutines ]}

\label{ch:robo34}
\label{ch:ModelRefraction_ropp_pp_bangle_MSIS}
\index{unsorted!ropp\_pp\_bangle\_MSIS}\index{Subroutines!ropp\_pp\_bangle\_MSIS}
\textbf{NAME:}\hspace{0.08in}\begin{Verbatim}
    ropp_pp_bangle_MSIS - Compute bending angle from MSIS spherical harmonics
\end{Verbatim}
\textbf{SYNOPSIS:}\hspace{0.08in}\begin{Verbatim}
    call ropp_pp_bangle_MSIS(file, month, lat, lon, alt, bangle)
\end{Verbatim}
\textbf{DESCRIPTION:}\hspace{0.08in}\begin{Verbatim}
    This subroutine calculates a climatological bending angle profile for 
    a given month, latitude and longitude from MSIS data. Profiles are 
    computed using Chebyshev polynomials and spherical harmonics with 
    coefficients read from file.
\end{Verbatim}
\textbf{INPUTS:}\hspace{0.08in}\begin{Verbatim}
    character(len=*) :: file     Model coefficients filename
    integer,         :: month    Month of year
    real(wp),        :: lat      Latitude
    real(wp),        :: lon      Longitude
    real(wp), dim(:) :: alt      Altitude levels on which to find N
\end{Verbatim}
\textbf{OUTPUT:}\hspace{0.08in}\begin{Verbatim}
    character(len=*) :: file     Model coefficients filename
    real(wp), dim(:) :: bangle   Bending angle field
\end{Verbatim}
\section{ModelRefraction/ropp\_pp\_bg\_refraction}
\textsl{[ Subroutines ]}

\label{ch:robo35}
\label{ch:ModelRefraction_ropp_pp_bg_refraction}
\index{unsorted!ropp\_pp\_bg\_refraction}\index{Subroutines!ropp\_pp\_bg\_refraction}
\textbf{NAME:}\hspace{0.08in}\begin{Verbatim}
    ropp_pp_bg_refraction - Calculate bending angle profile for BG 
                            refractivity
\end{Verbatim}
\textbf{SYNOPSIS:}\hspace{0.08in}\begin{Verbatim}
    call ropp_pp_bg_refraction(bfile, month, lat, lon, impact,    &
                               bangle_BG, config)
\end{Verbatim}
\textbf{DESCRIPTION:}\hspace{0.08in}\begin{Verbatim}
    This subroutine calculates the bending angle profile for a given location
    from the background t,p,q fields.

 INPUT
    character(len=*) :: bfile          Model coefficients file
    integer,         :: month          Month of year
    real(wp)         :: lat            Latitude  (deg)
    real(wp)         :: lon            Longitude (deg)
    real(wp), dim(:) :: impact         Impact parameter (m)
    type(PPConfig)   :: config         Configuration parameters
\end{Verbatim}
\textbf{OUTPUT:}\hspace{0.08in}\begin{Verbatim}
    real(wp), dim(:) :: bangle_BG      BG bending angles 
                                       (on input impact parameter levels).
\end{Verbatim}
\section{ModelRefraction/ropp\_pp\_fit\_model\_refraction}
\textsl{[ Subroutines ]}

\label{ch:robo36}
\label{ch:ModelRefraction_ropp_pp_fit_model_refraction}
\index{unsorted!ropp\_pp\_fit\_model\_refraction}\index{Subroutines!ropp\_pp\_fit\_model\_refraction}
\textbf{NAME:}\hspace{0.08in}\begin{Verbatim}
    ropp_pp_fit_model_refraction - Fit model bending angle profile with 
                                   observed bending angles
\end{Verbatim}
\textbf{SYNOPSIS:}\hspace{0.08in}\begin{Verbatim}
    call ropp_pp_fit_model_refraction(impact_LC, bangle_LC, 
                                      impact_model, bangle_model, config)
\end{Verbatim}
\textbf{DESCRIPTION:}\hspace{0.08in}\begin{Verbatim}
    This subroutine calculates a fitting factor of model bending angle profile
    with observed bending angles by linear regression in height interval 
    40-60 km.
\end{Verbatim}
\textbf{INPUTS:}\hspace{0.08in}\begin{Verbatim}
    real(wp), dim(:) :: impact_LC     Observed impact parameters (m)   
    real(wp), dim(:) :: bangle_LC     Observed bending angles (rad)
    real(wp), dim(:) :: impact_model  Model impact parameter (m)
    real(wp), dim(:) :: bangle_model  Model bending angles (rad)
    type(ppConfig)   :: config        Configuration parameters
\end{Verbatim}
\textbf{OUTPUT:}\hspace{0.08in}\begin{Verbatim}
    real(wp), dim(:) :: bangle_model  Fitted model bending angles (rad)
\end{Verbatim}
\section{ModelRefraction/ropp\_pp\_model\_refraction}
\textsl{[ Subroutines ]}

\label{ch:robo37}
\label{ch:ModelRefraction_ropp_pp_model_refraction}
\index{unsorted!ropp\_pp\_model\_refraction}\index{Subroutines!ropp\_pp\_model\_refraction}
\textbf{NAME:}\hspace{0.08in}\begin{Verbatim}
    ropp_pp_model_refraction - Calculate bending angle profile for MSIS 
                               refractivity
\end{Verbatim}
\textbf{SYNOPSIS:}\hspace{0.08in}\begin{Verbatim}
    call ropp_pp_model_refraction(mfile, month, lat, lon, impact,    &
                                  bangle_MSIS, config)
\end{Verbatim}
\textbf{DESCRIPTION:}\hspace{0.08in}\begin{Verbatim}
    This subroutine calculates the bending angle profile for a given location
    from the MSIS refractivity field

 INPUT
    character(len=*) :: mfile          Model coefficients file
    integer,         :: month          Month of year
    real(wp)         :: lat            Latitude  (deg)
    real(wp)         :: lon            Longitude (deg)
    real(wp), dim(:) :: impact         Impact parameter (m)
    type(PPConfig)   :: config         Configuration parameters
\end{Verbatim}
\textbf{OUTPUT:}\hspace{0.08in}\begin{Verbatim}
    real(wp), dim(:) :: bangle_MSIS    MSIS bending angles 
                                       (on input impact parameter levels).
\end{Verbatim}
\section{ModelRefraction/ropp\_pp\_read\_MSIS}
\textsl{[ Subroutines ]}

\label{ch:robo38}
\label{ch:ModelRefraction_ropp_pp_read_MSIS}
\index{unsorted!ropp\_pp\_read\_MSIS}\index{Subroutines!ropp\_pp\_read\_MSIS}
\textbf{NAME:}\hspace{0.08in}\begin{Verbatim}
    ropp_pp_read_MSIS - Read spherical harmonic coefficients from file
\end{Verbatim}
\textbf{SYNOPSIS:}\hspace{0.08in}\begin{Verbatim}
    call ropp_pp_read_MSIS(file, month, coeff)
\end{Verbatim}
\textbf{DESCRIPTION:}\hspace{0.08in}\begin{Verbatim}
    This subroutine reads a file of MSIS climatology spherical harmonic
    coefficients for the month specified and fills the elements of a 
    structure of derived type MSIScoeff.
\end{Verbatim}
\textbf{INPUTS:}\hspace{0.08in}\begin{Verbatim}
    character(*)     :: file          Filename
    integer,         :: month         Month of year
\end{Verbatim}
\textbf{OUTPUT:}\hspace{0.08in}\begin{Verbatim}
    type(MSIScoeff)  :: coeff         Coefficients  (ref/ba)
\end{Verbatim}
\section{ModelRefraction/ropp\_pp\_refrac\_BG}
\textsl{[ Subroutines ]}

\label{ch:robo39}
\label{ch:ModelRefraction_ropp_pp_refrac_BG}
\index{unsorted!ropp\_pp\_refrac\_BG}\index{Subroutines!ropp\_pp\_refrac\_BG}
\textbf{NAME:}\hspace{0.08in}\begin{Verbatim}
    ropp_pp_refrac_BG - Compute refractivity from BG atmospheric data
\end{Verbatim}
\textbf{SYNOPSIS:}\hspace{0.08in}\begin{Verbatim}
    call ropp_pp_refrac_BG(file, month, lat, lon, alt, refrac)
\end{Verbatim}
\textbf{DESCRIPTION:}\hspace{0.08in}\begin{Verbatim}
    This subroutine calculates a background refractivity profile from a
    profile of temperature, pressure and humidity data read from a ROPP
    format netCDF file. Processing is applied according to the ropp_fm module
    routines.
\end{Verbatim}
\textbf{INPUTS:}\hspace{0.08in}\begin{Verbatim}
    character(len=*) :: file     Background profile filename
    integer,         :: month    Month of year
    real(wp),        :: lat      Latitude
    real(wp),        :: lon      Longitude
    real(wp), dim(:) :: alt      Altitude levels on which to find N
\end{Verbatim}
\textbf{OUTPUT:}\hspace{0.08in}\begin{Verbatim}
    real(wp), dim(:) :: refrac   Refractivity field
\end{Verbatim}
\section{ModelRefraction/ropp\_pp\_refrac\_MSIS}
\textsl{[ Subroutines ]}

\label{ch:robo40}
\label{ch:ModelRefraction_ropp_pp_refrac_MSIS}
\index{unsorted!ropp\_pp\_refrac\_MSIS}\index{Subroutines!ropp\_pp\_refrac\_MSIS}
\textbf{NAME:}\hspace{0.08in}\begin{Verbatim}
    ropp_pp_refrac_MSIS - Compute refractivity from MSIS spherical harmonics
\end{Verbatim}
\textbf{SYNOPSIS:}\hspace{0.08in}\begin{Verbatim}
    call ropp_pp_refrac_MSIS(file, month, lat, lon, alt, refrac, grad)
\end{Verbatim}
\textbf{DESCRIPTION:}\hspace{0.08in}\begin{Verbatim}
    This subroutine calculates a climatological refractivity profile for 
    a given month, latitude and longitude from MSIS data. Profiles are 
    computed using Chebyshev polynomials and spherical harmonics with 
    coefficients read from file.
\end{Verbatim}
\textbf{INPUTS:}\hspace{0.08in}\begin{Verbatim}
    character(len=*) :: file     Model coefficients filename
    integer,         :: month    Month of year
    real(wp),        :: lat      Latitude
    real(wp),        :: lon      Longitude
    real(wp), dim(:) :: alt      Altitude levels on which to find N
\end{Verbatim}
\textbf{OUTPUT:}\hspace{0.08in}\begin{Verbatim}
    character(len=*) :: file     Model coefficients filename
    real(wp), dim(:) :: refrac   Refractivity field
    real(wp), dim(:) :: grad     Refractivity field gradient (optional)
\end{Verbatim}
\section{ModelRefraction/ropp\_pp\_search\_model\_refraction}
\textsl{[ Subroutines ]}

\label{ch:robo41}
\label{ch:ModelRefraction_ropp_pp_search_model_refraction}
\index{unsorted!ropp\_pp\_search\_model\_refraction}\index{Subroutines!ropp\_pp\_search\_model\_refraction}
\textbf{NAME:}\hspace{0.08in}\begin{Verbatim}
    ropp_pp_search_model_refraction - Calculate best-fit bending angle profile
                                      for MSIS refractivity
\end{Verbatim}
\textbf{SYNOPSIS:}\hspace{0.08in}\begin{Verbatim}
    call ropp_pp_model_refraction(mfile, in_month, in_lat, in_lon, 
                                  in_impact, in_bangle, 
                                  impact, bangle_MSIS, config)
\end{Verbatim}
\textbf{DESCRIPTION:}\hspace{0.08in}\begin{Verbatim}
    This subroutine calculates the best-fit bending angle profile from the
    MSIS refractivity field by regression to an input bending angle profile 

 INPUT
    character(len=*) :: mfile          Model coefficients file
    integer,         :: in_month       Month of year
    real(wp)         :: in_lat         Latitude  (deg)
    real(wp)         :: in_lon         Longitude (deg)
    real(wp), dim(:) :: in_impact      Input impact parameter (m)
    real(wp), dim(:) :: in_bangle      Input bending angle (rad)
    real(wp), dim(:) :: impact         Impact parameter (m)
    type(PPConfig)   :: config         Configuration parameters
\end{Verbatim}
\textbf{OUTPUT:}\hspace{0.08in}\begin{Verbatim}
    real(wp), dim(:) :: bangle_MSIS    Best-fit MSIS bending angles 
                                       (on input impact parameter levels).
\end{Verbatim}
\section{Monotonous/ropp\_pp\_monotonous}
\textsl{[ Subroutines ]}

\label{ch:robo50}
\label{ch:Monotonous_ropp_pp_monotonous}
\index{unsorted!ropp\_pp\_monotonous}\index{Subroutines!ropp\_pp\_monotonous}
\textbf{NAME:}\hspace{0.08in}\begin{Verbatim}
    ropp_pp_monotonous - Find monotonous sequence.
\end{Verbatim}
\textbf{SYNOPSIS:}\hspace{0.08in}\begin{Verbatim}
    call ropp_pp_monotonous(x, d)
\end{Verbatim}
\textbf{DESCRIPTION:}\hspace{0.08in}\begin{Verbatim}
    This subroutine finds a monotonous sequence by an iterative orthogonal
    projections and crawling monotonoization method.
\end{Verbatim}
\textbf{INPUTS:}\hspace{0.08in}\begin{Verbatim}
    real(wp), dim(:)  :: x      Coordinate values to be transformed.
    integer, optional :: d      Sort direction flag 
                                             > 0 - rising
                                             < 0 - setting
\end{Verbatim}
\textbf{OUTPUT:}\hspace{0.08in}\begin{Verbatim}
    real(wp), dim(:)  :: x      Transformed sequence
\end{Verbatim}
\section{Parameters/TPH\_QC\_flag}
\textsl{[ Parameters ]}

\label{ch:robo51}
\label{ch:Parameters_TPH_QC_flag}
\index{unsorted!TPH\_QC\_flag}\index{Parameters!TPH\_QC\_flag}
\textbf{NAME:}\hspace{0.08in}\begin{Verbatim}
    TPH QC flag parameters - Parameters that define the components 
                             of the overall TPH QC flag.
\end{Verbatim}
\textbf{SOURCE:}\hspace{0.08in}\begin{Verbatim}
  INTEGER, PARAMETER :: TPH_QC_data_invalid     =  0  ! Input data validity check
  INTEGER, PARAMETER :: TPH_QC_prof_depth       =  1  ! Input data depth check
  INTEGER, PARAMETER :: TPH_QC_prof_height      =  2  ! Input data height check
  INTEGER, PARAMETER :: TPH_QC_CT_smooth_above  =  3  ! Cov trans sharpness above TPH check
  INTEGER, PARAMETER :: TPH_QC_CT_smooth_below  =  4  ! Cov trans sharpness below TPH check
  INTEGER, PARAMETER :: TPH_QC_double_trop      =  5  ! Double tropopause detected
  INTEGER, PARAMETER :: TPH_QC_too_low          =  6  ! TPH minimum height check
  INTEGER, PARAMETER :: TPH_QC_too_high         =  7  ! TPH maximum height check
  INTEGER, PARAMETER :: TPH_QC_reserved8        =  8  ! Reserved
  INTEGER, PARAMETER :: TPH_QC_reserved9        =  9  ! Reserved
  INTEGER, PARAMETER :: TPH_QC_reserved10       = 10  ! Reserved
  INTEGER, PARAMETER :: TPH_QC_reserved11       = 11  ! Reserved
  INTEGER, PARAMETER :: TPH_QC_reserved12       = 12  ! Reserved
  INTEGER, PARAMETER :: TPH_QC_reserved13       = 13  ! Reserved
  INTEGER, PARAMETER :: TPH_QC_reserved14       = 14  ! Reserved
  INTEGER, PARAMETER :: TPH_QC_reserved15       = 15  ! Reserved
\end{Verbatim}
\section{Parameters/wp}
\textsl{[ Parameters ]}

\label{ch:robo52}
\label{ch:Parameters_wp}
\index{unsorted!wp}\index{Parameters!wp}
\textbf{NAME:}\hspace{0.08in}\begin{Verbatim}
    wp - Kind parameter for double precision floating point numbers.
\end{Verbatim}
\textbf{SOURCE:}\hspace{0.08in}\begin{Verbatim}
  USE typesizes, ONLY: wp => EightByteReal
\end{Verbatim}
\section{Physical/c\_light}
\textsl{[ Parameters ]}

\label{ch:robo53}
\label{ch:Physical_c_light}
\index{unsorted!c\_light}\index{Parameters!c\_light}
\textbf{NAME:}\hspace{0.08in}\begin{Verbatim}
    c_light - Light speed in vaccuum (m/s)
\end{Verbatim}
\textbf{SOURCE:}\hspace{0.08in}\begin{Verbatim}
  REAL(wp), PARAMETER :: c_light = 299792458.0_wp         !   m s^{-1}
\end{Verbatim}
\section{PP/ropp\_pp\_diag2roprof}
\textsl{[ Subroutines ]}

\label{ch:robo54}
\label{ch:PP_ropp_pp_diag2roprof}
\index{unsorted!ropp\_pp\_diag2roprof}\index{Subroutines!ropp\_pp\_diag2roprof}
\textbf{NAME:}\hspace{0.08in}\begin{Verbatim}
    ropp_pp_diag2roprof - Add diagnostic information gathered during 
                          processing an occultation to a ROprof data structure
\end{Verbatim}
\textbf{SYNOPSIS:}\hspace{0.08in}\begin{Verbatim}
    call ropp_pp_diag2roprof(diag, ro_data)
\end{Verbatim}
\textbf{INPUTS:}\hspace{0.08in}\begin{Verbatim}
    type(PPdiag) :: diag       Diagnostic information structure
    type(ROprof) :: ro_data    RO data file structure
\end{Verbatim}
\textbf{OUTPUT:}\hspace{0.08in}\begin{Verbatim}
    type(ROprof) :: ro_data    Updated RO data structure with diagnostics
\end{Verbatim}
\section{PPSpline/ropp\_pp\_basic\_splines}
\textsl{[ Subroutines ]}

\label{ch:robo55}
\label{ch:PPSpline_ropp_pp_basic_splines}
\index{unsorted!ropp\_pp\_basic\_splines}\index{Subroutines!ropp\_pp\_basic\_splines}
\textbf{NAME:}\hspace{0.08in}\begin{Verbatim}
    ropp_pp_basic_splines - Generate a matrix of basic polynomials for
                            polynomial regression
\end{Verbatim}
\textbf{SYNOPSIS:}\hspace{0.08in}\begin{Verbatim}
    call ropp_pp_basic_splines(X, Xs, K)
\end{Verbatim}
\textbf{DESCRIPTION:}\hspace{0.08in}\begin{Verbatim}
    Generation of matrix of basic polynomials for polynomial regression
\end{Verbatim}
\section{PPSpline/ropp\_pp\_init\_spline}
\textsl{[ Subroutines ]}

\label{ch:robo56}
\label{ch:PPSpline_ropp_pp_init_spline}
\index{unsorted!ropp\_pp\_init\_spline}\index{Subroutines!ropp\_pp\_init\_spline}
\textbf{NAME:}\hspace{0.08in}\begin{Verbatim}
    ropp_pp_init_spline - Calculation of 2nd derivative of a spline
\end{Verbatim}
\textbf{SYNOPSIS:}\hspace{0.08in}\begin{Verbatim}
    call ropp_pp_init_spline(x, f, d2)
\end{Verbatim}
\textbf{DESCRIPTION:}\hspace{0.08in}\begin{Verbatim}
    Calculate 2nd derivative of a spline, drive-through solution
\end{Verbatim}
\section{PPSpline/ropp\_pp\_interpol\_spline}
\textsl{[ Subroutines ]}

\label{ch:robo57}
\label{ch:PPSpline_ropp_pp_interpol_spline}
\index{unsorted!ropp\_pp\_interpol\_spline}\index{Subroutines!ropp\_pp\_interpol\_spline}
\textbf{NAME:}\hspace{0.08in}\begin{Verbatim}
    ropp_pp_interpol_spline - Spline interpolation of a gridded function with
                              linear extrapolation outside grid extent
\end{Verbatim}
\textbf{SYNOPSIS:}\hspace{0.08in}\begin{Verbatim}
    call ropp_pp_interpol_spline(x, f, d2, x_int, f_int, fd_int, fd2_int)
\end{Verbatim}
\textbf{DESCRIPTION:}\hspace{0.08in}\begin{Verbatim}
    Search for grid interval containing the interpolation point and summation
    of polynomail with given spline coefficients
\end{Verbatim}
\section{PPUtils/ropp\_pp\_init\_polynomial}
\textsl{[ Subroutines ]}

\label{ch:robo58}
\label{ch:PPUtils_ropp_pp_init_polynomial}
\index{unsorted!ropp\_pp\_init\_polynomial}\index{Subroutines!ropp\_pp\_init\_polynomial}
\textbf{NAME:}\hspace{0.08in}\begin{Verbatim}
    ropp_pp_init_polynomial - Generate matrix of basic polynomials 
                              for regression
\end{Verbatim}
\textbf{SYNOPSIS:}\hspace{0.08in}\begin{Verbatim}
    call ropp_pp_init_polynomial(x, K)
\end{Verbatim}
\textbf{DESCRIPTION:}\hspace{0.08in}\begin{Verbatim}
    Compute basic polynomials as
        K_ij = f_j(x_i) = (x_i)^j     for j=0..UBound(K,2)
\end{Verbatim}
\section{PPUtils/ropp\_pp\_invert\_matrix}
\textsl{[ Subroutines ]}

\label{ch:robo59}
\label{ch:PPUtils_ropp_pp_invert_matrix}
\index{unsorted!ropp\_pp\_invert\_matrix}\index{Subroutines!ropp\_pp\_invert\_matrix}
\textbf{NAME:}\hspace{0.08in}\begin{Verbatim}
    ropp_pp_invert_matrix - Invert a matrix A(dimY, dimX) 
\end{Verbatim}
\textbf{SYNOPSIS:}\hspace{0.08in}\begin{Verbatim}
    call ropp_pp_invert_matrix(A, B)
\end{Verbatim}
\textbf{DESCRIPTION:}\hspace{0.08in}\begin{Verbatim}
    Gauss elimination
\end{Verbatim}
\section{PPUtils/ropp\_pp\_isnan}
\textsl{[ Subroutines ]}

\label{ch:robo60}
\label{ch:PPUtils_ropp_pp_isnan}
\index{unsorted!ropp\_pp\_isnan}\index{Subroutines!ropp\_pp\_isnan}
\textbf{NAME:}\hspace{0.08in}\begin{Verbatim}
    ropp_pp_isnan - Test if variable value is NaN
\end{Verbatim}
\textbf{SYNOPSIS:}\hspace{0.08in}\begin{Verbatim}
    true_or_false = ropp_pp_isnan(x)
\end{Verbatim}
\section{PPUtils/ropp\_pp\_matmul}
\textsl{[ Subroutines ]}

\label{ch:robo61}
\label{ch:PPUtils_ropp_pp_matmul}
\index{unsorted!ropp\_pp\_matmul}\index{Subroutines!ropp\_pp\_matmul}
\textbf{NAME:}\hspace{0.08in}\begin{Verbatim}
    ropp_pp_matmul - Matrix multiplication 
\end{Verbatim}
\textbf{SYNOPSIS:}\hspace{0.08in}\begin{Verbatim}
    C = ropp_pp_matmul(A, B)
\end{Verbatim}
\textbf{DESCRIPTION:}\hspace{0.08in}\begin{Verbatim}
    Replicates intrinsic 'MATMUL' function
\end{Verbatim}
\section{PPUtils/ropp\_pp\_nearest\_power2}
\textsl{[ Subroutines ]}

\label{ch:robo62}
\label{ch:PPUtils_ropp_pp_nearest_power2}
\index{unsorted!ropp\_pp\_nearest\_power2}\index{Subroutines!ropp\_pp\_nearest\_power2}
\textbf{NAME:}\hspace{0.08in}\begin{Verbatim}
    ropp_pp_nearest_power2 - Find power of 2 nearest input integer
\end{Verbatim}
\textbf{SYNOPSIS:}\hspace{0.08in}\begin{Verbatim}
    n2 = ropp_pp_nearest_power2(n)
\end{Verbatim}
\section{PPUtils/ropp\_pp\_polynomial}
\textsl{[ Subroutines ]}

\label{ch:robo63}
\label{ch:PPUtils_ropp_pp_polynomial}
\index{unsorted!ropp\_pp\_polynomial}\index{Subroutines!ropp\_pp\_polynomial}
\textbf{NAME:}\hspace{0.08in}\begin{Verbatim}
    ropp_pp_polynomial - Calculation of a polynomial and its derivative
\end{Verbatim}
\textbf{SYNOPSIS:}\hspace{0.08in}\begin{Verbatim}
    call ropp_pp_polynomial(c, x, P, DP)
\end{Verbatim}
\textbf{DESCRIPTION:}\hspace{0.08in}\begin{Verbatim}
    Horner scheme
\end{Verbatim}
\section{PPUtils/ropp\_pp\_quasi\_invert}
\textsl{[ Subroutines ]}

\label{ch:robo64}
\label{ch:PPUtils_ropp_pp_quasi_invert}
\index{unsorted!ropp\_pp\_quasi\_invert}\index{Subroutines!ropp\_pp\_quasi\_invert}
\textbf{NAME:}\hspace{0.08in}\begin{Verbatim}
    ropp_pp_quasi_invert - Quasi-inverse of a matrix K(dimY, dimX) 
\end{Verbatim}
\textbf{SYNOPSIS:}\hspace{0.08in}\begin{Verbatim}
    call ropp_pp_quasi_invert(K, Q)
\end{Verbatim}
\textbf{DESCRIPTION:}\hspace{0.08in}\begin{Verbatim}
    1. dimY >= dimX:
       x = Qy - vector minimizing ||Kx - y||
       QK = E; Q is left inverse operator.
       Q = (K^T K)^-1 K^T
    2. dimY <= dimX:
       x = Qy - solution minimizing ||x||
       KQ = E; Q is right inverse operator.
       Q = K^T (KK^T)^-1
\end{Verbatim}
\section{PPUtils/ropp\_pp\_regression}
\textsl{[ Subroutines ]}

\label{ch:robo65}
\label{ch:PPUtils_ropp_pp_regression}
\index{unsorted!ropp\_pp\_regression}\index{Subroutines!ropp\_pp\_regression}
\textbf{NAME:}\hspace{0.08in}\begin{Verbatim}
    ropp_pp_regression - Linear regression
\end{Verbatim}
\textbf{SYNOPSIS:}\hspace{0.08in}\begin{Verbatim}
    call ropp_pp_regression(K, y, a)
\end{Verbatim}
\textbf{DESCRIPTION:}\hspace{0.08in}\begin{Verbatim}
    Quasi-inversion of matrix of basic functions:
       || Sum_j a_j f_j(x_i) - y_i || = min
    Solution is a = Q y where Q is left inverse of K_ij = f_j(x_i) 
\end{Verbatim}
\section{PPUtils/ropp\_pp\_seek\_index}
\textsl{[ Subroutines ]}

\label{ch:robo66}
\label{ch:PPUtils_ropp_pp_seek_index}
\index{unsorted!ropp\_pp\_seek\_index}\index{Subroutines!ropp\_pp\_seek\_index}
\textbf{NAME:}\hspace{0.08in}\begin{Verbatim}
    ropp_pp_seek_index - Find grid interval containing given point
\end{Verbatim}
\textbf{SYNOPSIS:}\hspace{0.08in}\begin{Verbatim}
    ip = ropp_pp_seek_index(x, xp)
\end{Verbatim}
\textbf{DESCRIPTION:}\hspace{0.08in}\begin{Verbatim}
    Search for grid interval containing the interpolation point.
    Combination of Newton-like and dichotomic iterative search.
     Result:
         >0   - index of grid interval containing xp (x(ip) <= xp <= x(ip+1))
          0   - xp is not inside grid
         -1   - iterations do not converge
\end{Verbatim}
\section{Preprocessing/Cutoff}
\textsl{[ Topics ]}

\label{ch:robo69}
\label{ch:Preprocessing_Cutoff}
\index{unsorted!Cutoff}\index{Topics!Cutoff}
\textbf{DESCRIPTION:}\hspace{0.08in}\begin{Verbatim}
    Data cutoff based on amplitude, smoothed bending angle and impact parameter
\end{Verbatim}
\textbf{SEE ALSO:}\hspace{0.08in}\section{Preprocessing/Preprocess}
\textsl{[ Topics ]}

\label{ch:robo70}
\label{ch:Preprocessing_Preprocess}
\index{unsorted!Preprocess}\index{Topics!Preprocess}
\textbf{DESCRIPTION:}\hspace{0.08in}\begin{Verbatim}
    Routines for mission-specific pre-processing steps
\end{Verbatim}
\textbf{SEE ALSO:}\hspace{0.08in}\begin{Verbatim}
    ropp_pp_preprocess
\end{Verbatim}
\subsection{Preprocess/ropp\_pp\_amplitude\_go}
\textsl{[ Preprocess ]}
\textsl{[ Subroutines ]}

\label{ch:robo67}
\label{ch:Preprocess_ropp_pp_amplitude_go}
\index{unsorted!ropp\_pp\_amplitude\_go}\index{Subroutines!ropp\_pp\_amplitude\_go}
\textbf{NAME:}\hspace{0.08in}\begin{Verbatim}
    ropp_pp_amplitude_go - Calculate RO signal amplitude in GO approximation.
\end{Verbatim}
\textbf{SYNOPSIS:}\hspace{0.08in}\begin{Verbatim}
    call ropp_pp_amplitude_go(time, r_leo, r_gns, r_coc, roc, impact, snr, 
                              w_smooth, snr_R)
\end{Verbatim}
\textbf{DESCRIPTION:}\hspace{0.08in}\begin{Verbatim}
    This routine calculates L2 amplitude in Geometric optics approximation
\end{Verbatim}
\textbf{INPUTS:}\hspace{0.08in}\begin{Verbatim}
    real(wp), dimension(:)   :: time      ! Relative time of samples (s)
    real(wp), dimension(:,:) :: r_leo     ! LEO coordinates (ECI or ECF) (m)
    real(wp), dimension(:,:) :: r_gns     ! GPS coordinates (ECI or ECF) (m)
    real(wp), dimension(:)   :: r_coc     ! centre curvature coordinates (m)
    real(wp)                 :: roc       ! Radius curvature (m)
    real(wp), dimension(:)   :: impact    ! Impact parameters (m)
    real(wp), dimension(:)   :: snr       ! Observed amplitude
    integer                  :: w_smooth  ! smoothing window (points)
\end{Verbatim}
\textbf{OUTPUT:}\hspace{0.08in}\begin{Verbatim}
    real(wp), dimension(:)   :: snr_R     ! Refractive amplitude
\end{Verbatim}
\textbf{NOTES:}\hspace{0.08in}\textbf{REFERENCES:}\hspace{0.08in}\begin{Verbatim}
   Gorbunov M.E. and Lauritsen K.B. 2004
   Analysis of wave fields by Fourier integral operators and their application
   for radio occultations
   Radio Science (39) RS4010

   Gorbunov M.E., Lauritsen K.B., Rodin A., Tomassini M., Kornblueh L. 2005 
   Analysis of the CHAMP experimental data on radio-occultation sounding of
   the Earth's atmosphere.
   Izvestiya Atmospheric and Oceanic Physics (41) 726-740.
\end{Verbatim}
\subsection{Preprocess/ropp\_pp\_correct\_L2}
\textsl{[ Preprocess ]}
\textsl{[ Subroutines ]}

\label{ch:robo68}
\label{ch:Preprocess_ropp_pp_correct_L2}
\index{unsorted!ropp\_pp\_correct\_L2}\index{Subroutines!ropp\_pp\_correct\_L2}
\textbf{NAME:}\hspace{0.08in}\begin{Verbatim}
    ropp_pp_correct_L2 - Correction of unusable L2 data.
\end{Verbatim}
\textbf{SYNOPSIS:}\hspace{0.08in}\begin{Verbatim}
    call ropp_pp_correct_L2(time, r_leo, r_gns, r_coc, roc, impact_LM, 
                             phase_LM, phase_L1, phase_L2, snr_L1, snr_L2,
                             lcf, Hmid, L2Q)
\end{Verbatim}
\textbf{DESCRIPTION:}\hspace{0.08in}\begin{Verbatim}
    This routine corrects unusable L2 data.
       1) Computation of penalty function for L2 channel from radio optical 
          analysis.
       2) Lower region correction:
             A2 is multiplied with penalty function;
             Corrected S2 is linear combination of S1 and S2.
       3) Upper region correction:
             Corrected S2 is linear combination of S2 and and smoothed S2.
       4) Click removal in L2 phase.
\end{Verbatim}
\textbf{INPUTS:}\hspace{0.08in}\begin{Verbatim}
    real(wp), dimension(:)   :: time      ! Relative time of samples (s)
    real(wp), dimension(:,:) :: r_leo     ! LEO coordinates (ECI or ECF) (m)
    real(wp), dimension(:,:) :: r_gns     ! GPS coordinates (ECI or ECF) (m)
    real(wp), dimension(:)   :: r_coc     ! centre curvature coordinates (m)
    real(wp)                 :: roc       ! Radius curvature (m)
    real(wp), dimension(:)   :: impact_LM ! Model impact parameters (m)
    real(wp), dimension(:)   :: phase_LM  ! Model excess phase (m)
    real(wp), dimension(:)   :: phase_L1  ! L1 excess phase (m)
    real(wp), dimension(:)   :: phase_L2  ! L2 excess phase (m)
    real(wp), dimension(:)   :: snr_L1    ! L1 amplitude
    real(wp), dimension(:)   :: snr_L2    ! L2 amplitude
    integer,  dimension(:)   :: lcf       ! Lost carrier flag  
    real(wp)                 :: hmid      ! Border between upper/lower region 
\end{Verbatim}
\textbf{OUTPUT:}\hspace{0.08in}\begin{Verbatim}
    real(wp), dimension(:)   :: phase_L1  ! L1 excess phase (m)
    real(wp), dimension(:)   :: phase_L2  ! L2 excess phase (m)
    real(wp), dimension(:)   :: snr_L1    ! L1 amplitude
    real(wp), dimension(:)   :: snr_L2    ! L2 amplitude
    real(wp)                 :: L2Q       ! L2 badness score
\end{Verbatim}
\textbf{NOTES:}\hspace{0.08in}\textbf{REFERENCES:}\hspace{0.08in}\begin{Verbatim}
   Gorbunov M.E., Lauritsen K.B., Rhodin A., Tomassini M. and Kornblueh L. 
   2006
   Radio holographic filtering, error estimation, and quality control of 
   radio occultation data
   Journal of Geophysical Research (111) D10105

   Gorbunov M.E., Lauritsen K.B., Rodin A., Tomassini M., Kornblueh L. 2005 
   Analysis of the CHAMP experimental data on radio-occultation sounding of
   the Earth's atmosphere.
   Izvestiya Atmospheric and Oceanic Physics (41) 726-740.
\end{Verbatim}
\section{Preprocessing/ropp\_pp\_bangle2phase}
\textsl{[ Subroutines ]}

\label{ch:robo71}
\label{ch:Preprocessing_ropp_pp_bangle2phase}
\index{unsorted!ropp\_pp\_bangle2phase}\index{Subroutines!ropp\_pp\_bangle2phase}
\textbf{NAME:}\hspace{0.08in}\begin{Verbatim}
    ropp_pp_bangle2phase - Compute excess phase as a function of time from 
                           bending angle as function of impact parameter
\end{Verbatim}
\textbf{SYNOPSIS:}\hspace{0.08in}\begin{Verbatim}
    call ropp_pp_bangle2phase(time, r_leo, r_gns, r_coc, impact, bangle, 
                              phase, dphi, impact_LM, IL, IU)
\end{Verbatim}
\textbf{DESCRIPTION:}\hspace{0.08in}\begin{Verbatim}
    This routine computes model excess phase from bending angle data
      1. Compute t(p) for given satellite trajectories
      2. Compute p(t) -> d(t) -> Phi(t)
\end{Verbatim}
\textbf{INPUTS:}\hspace{0.08in}\begin{Verbatim}
    real(wp), dimension(:)   :: time      ! time of samples (s)
    real(wp), dimension(:,:) :: r_leo     ! cartesian LEO coordinates (ECF)
    real(wp), dimension(:,:) :: r_gns     ! cartesian GPS coordinates (ECF)
    real(wp), dimension(:)   :: r_coc     ! cartesian centre curvature (ECF)
    real(wp), dimension(:)   :: impact    ! input impact parameters (m)
    real(wp), dimension(:)   :: bangle    ! input bending angle profile (rad)
\end{Verbatim}
\textbf{OUTPUT:}\hspace{0.08in}\begin{Verbatim}
    real(wp), dimension(:)   :: bangle    ! output bending angle profile (rad)
    real(wp), dimension(:)   :: phase     ! excess phase as function time (m)
    real(wp), optional, dimension(:) :: dphi      ! derivative phase with time
    real(wp), optional, dimension(:) :: impact_LM ! model impact parameter (m)
    integer,  optional               :: IL  ! lower index valid model interval
    integer,  optional               :: IU  ! upper index valid model interval
\end{Verbatim}
\textbf{NOTES:}\hspace{0.08in}\textbf{REFERENCES:}\hspace{0.08in}\section{Preprocessing/ropp\_pp\_cutoff}
\textsl{[ Subroutines ]}

\label{ch:robo72}
\label{ch:Preprocessing_ropp_pp_cutoff}
\index{unsorted!ropp\_pp\_cutoff}\index{Subroutines!ropp\_pp\_cutoff}
\textbf{NAME:}\hspace{0.08in}\begin{Verbatim}
    ropp_pp_cutoff - Cut off occultation data based on amplitude, smoothed
                     bending angle and impact parameter values
\end{Verbatim}
\textbf{SYNOPSIS:}\hspace{0.08in}\begin{Verbatim}
    call ropp_pp_cutoff(ro_data, config, var1, var2, var3)
\end{Verbatim}
\textbf{DESCRIPTION:}\hspace{0.08in}\begin{Verbatim}
    1) Compute smoothed bending angle profile
    2) Cut off from amplitude (based on config%Acut parameter)
    3) Cut off from bending angle profile (based on config%Bcut, config%Pcut)
\end{Verbatim}
\textbf{INPUTS:}\hspace{0.08in}\begin{Verbatim}
    type(ROprof)                      :: ro_data ! RO data strucuture
    type(PPConfig)                    :: config  ! Configuration options
    real(wp), dimension(:), optional  :: var1    ! Extra variables to shrink
    real(wp), dimension(:), optional  :: var2    ! Extra variables to shrink
    integer,  dimension(:), optional  :: var3    ! Extra variables to shrink
\end{Verbatim}
\textbf{OUTPUT:}\hspace{0.08in}\begin{Verbatim}
    type(ROprof)                      :: ro_data ! Shrunk RO data strucuture
    type(PPConfig)                    :: config  ! Configuration options
    real(wp), dimension(:), optional  :: var1    ! Shrunk extra variables
    real(wp), dimension(:), optional  :: var2    ! Shrunk extra variables
    integer,  dimension(:), optional  :: var3    ! Shrunk extra variables
\end{Verbatim}
\textbf{NOTES:}\hspace{0.08in}\begin{Verbatim}
   Requires ROprof data structure type, defined in ropp_io module. This
   routine therefore requires that the ropp_io module is pre-installed before
   compilation.
\end{Verbatim}
\textbf{REFERENCES:}\hspace{0.08in}\section{Preprocessing/ropp\_pp\_cutoff\_amplitude}
\textsl{[ Subroutines ]}

\label{ch:robo73}
\label{ch:Preprocessing_ropp_pp_cutoff_amplitude}
\index{unsorted!ropp\_pp\_cutoff\_amplitude}\index{Subroutines!ropp\_pp\_cutoff\_amplitude}
\textbf{NAME:}\hspace{0.08in}\begin{Verbatim}
    ropp_pp_cutoff_amplitude - Cut off occultation data based on amplitude
                               and missing data flag
\end{Verbatim}
\textbf{SYNOPSIS:}\hspace{0.08in}\begin{Verbatim}
    call ropp_pp_cutoff_amplitude(ro_data, LCF, config)
\end{Verbatim}
\textbf{DESCRIPTION:}\hspace{0.08in}\begin{Verbatim}
    Cut off from amplitude (based on config%Acut parameter)
\end{Verbatim}
\textbf{INPUTS:}\hspace{0.08in}\begin{Verbatim}
    type(ROprof)                      :: ro_data ! RO data strucuture
    integer,  dimension(:)            :: LCF     ! Lost carrier flag
    type(PPConfig)                    :: config  ! Configuration options
\end{Verbatim}
\textbf{OUTPUT:}\hspace{0.08in}\begin{Verbatim}
    type(ROprof)                      :: ro_data ! Shrunk RO data strucuture
    integer,  dimension(:)            :: LCF     ! Shrunk LCF
    type(PPConfig)                    :: config  ! Configuration options
\end{Verbatim}
\textbf{NOTES:}\hspace{0.08in}\begin{Verbatim}
   Requires ROprof data structure type, defined in ropp_io module. This
   routine therefore requires that the ropp_io module is pre-installed before
   compilation.
\end{Verbatim}
\textbf{REFERENCES:}\hspace{0.08in}\section{Preprocessing/ropp\_pp\_impact2doppler}
\textsl{[ Subroutines ]}

\label{ch:robo74}
\label{ch:Preprocessing_ropp_pp_impact2doppler}
\index{unsorted!ropp\_pp\_impact2doppler}\index{Subroutines!ropp\_pp\_impact2doppler}
\textbf{NAME:}\hspace{0.08in}\begin{Verbatim}
    ropp_pp_impact2doppler - Compute relative doppler shift from geometrical
                             data
\end{Verbatim}
\textbf{SYNOPSIS:}\hspace{0.08in}\begin{Verbatim}
    call ropp_pp_impact2doppler(xleo,vleo,xgns,vgns,impact,doppler,bangle)
\end{Verbatim}
\textbf{DESCRIPTION:}\hspace{0.08in}\textbf{INPUTS:}\hspace{0.08in}\begin{Verbatim}
    real(wp), dimension(:) :: xleo     ! cartesian LEO coordinates
    real(wp), dimension(:) :: vleo     ! cartesian LEO velocity
    real(wp), dimension(:) :: xgns     ! cartesian GPS coordinates
    real(wp), dimension(:) :: vgns     ! cartesian GPS velocity
    real(wp)               :: impact   ! impact parameter
\end{Verbatim}
\textbf{OUTPUT:}\hspace{0.08in}\begin{Verbatim}
    real(wp)               :: doppler  ! Relative doppler frequency shift
    real(wp)               :: bangle   ! Bending angle
\end{Verbatim}
\textbf{NOTES:}\hspace{0.08in}\textbf{REFERENCES:}\hspace{0.08in}\section{Preprocessing/ropp\_pp\_internal\_navbit}
\textsl{[ Subroutines ]}

\label{ch:robo75}
\label{ch:Preprocessing_ropp_pp_internal_navbit}
\index{unsorted!ropp\_pp\_internal\_navbit}\index{Subroutines!ropp\_pp\_internal\_navbit}
\textbf{NAME:}\hspace{0.08in}\begin{Verbatim}
    ropp_pp_internal_navbit - get internal navigation bits
\end{Verbatim}
\textbf{SYNOPSIS:}\hspace{0.08in}\begin{Verbatim}
    call ropp_pp_internal_navbit(time, ds, lcf)
\end{Verbatim}
\textbf{DESCRIPTION:}\hspace{0.08in}\begin{Verbatim}
    Determine navigation bits frame position, polynomial regression of
    phase in frames and detection of phase jumps by Pi
\end{Verbatim}
\section{Preprocessing/ropp\_pp\_interpolate\_trajectory}
\textsl{[ Subroutines ]}

\label{ch:robo76}
\label{ch:Preprocessing_ropp_pp_interpolate_trajectory}
\index{unsorted!ropp\_pp\_interpolate\_trajectory}\index{Subroutines!ropp\_pp\_interpolate\_trajectory}
\textbf{NAME:}\hspace{0.08in}\begin{Verbatim}
    ropp_pp_interpolate_trajectory - Calculate interpolated coordinates,
                                     velocities, satellite-satellite angle
\end{Verbatim}
\textbf{SYNOPSIS:}\hspace{0.08in}\begin{Verbatim}
     call ropp_pp_interpolate_trajectory(time, cleo, cgns, r_coc, t_init,  & 
                                         xleo, vleo, xgns, vgns, theta)
\end{Verbatim}
\textbf{DESCRIPTION:}\hspace{0.08in}\begin{Verbatim}
    This routine calculates interpolated  coordinates, velocities
    and satellite-satellite angle at time t_init by polynomial regression
\end{Verbatim}
\textbf{INPUTS:}\hspace{0.08in}\begin{Verbatim}
    real(wp), dimension(:)   :: time    ! time of sample (s)
    real(wp), dimension(:,:) :: cleo    ! LEO regression coefficients
    real(wp), dimension(:,:) :: cgns    ! GPS regression coefficients
    real(wp), dimension(:,:) :: r_coc   ! cartesian centre curvature
    real(wp)                 :: t_init  ! interpolation time

 OUTPUTS
    real(wp), dimension(:,:) :: xleo    ! LEO positions from regression
    real(wp), dimension(:,:) :: vleo    ! LEO velocities from regression
    real(wp), dimension(:,:) :: xgns    ! GPS positions from regression
    real(wp), dimension(:,:) :: vgns    ! GPS velocities from regression
    real(wp), optional       :: theta   ! Satellite-to-satellite angle
\end{Verbatim}
\textbf{NOTES:}\hspace{0.08in}\textbf{REFERENCES:}\hspace{0.08in}\section{Preprocessing/ropp\_pp\_modelphase}
\textsl{[ Subroutines ]}

\label{ch:robo77}
\label{ch:Preprocessing_ropp_pp_modelphase}
\index{unsorted!ropp\_pp\_modelphase}\index{Subroutines!ropp\_pp\_modelphase}
\textbf{NAME:}\hspace{0.08in}\begin{Verbatim}
    ropp_pp_modelphase - Compute model excess phase from MSIS data
\end{Verbatim}
\textbf{SYNOPSIS:}\hspace{0.08in}\begin{Verbatim}
    call ropp_pp_modelphase(month, lat, lon, time, r_leo, r_gns,
                            r_coc, roc, phase_LM, impact_LM, config)
\end{Verbatim}
\textbf{DESCRIPTION:}\hspace{0.08in}\begin{Verbatim}
    This routine computes model excess phase
      1. Compute MSIS model bending angle as a function of impact parameter p
      2. Compute t(p) for given satellite trajectories
      3. Compute p(t) -> d(t) -> phase(t)
\end{Verbatim}
\textbf{INPUTS:}\hspace{0.08in}\begin{Verbatim}
    integer                  :: month     ! month of year
    real(wp)                 :: lat       ! occultation point latitude
    real(wp)                 :: lon       ! occultation point longitude
    real(wp), dimension(:)   :: time      ! time of samples (s)
    real(wp), dimension(:,:) :: r_leo     ! cartesian LEO coordinates (ECF)
    real(wp), dimension(:,:) :: r_gns     ! cartesian GPS coordinates (ECF)
    real(wp), dimension(:)   :: r_coc     ! cartesian centre curvature (ECF)
    real(wp)                 :: roc       ! radius curvature
    type(PPConfig)           :: config    ! configuration options
\end{Verbatim}
\textbf{OUTPUT:}\hspace{0.08in}\begin{Verbatim}
    real(wp), dimension(:)   :: phase_LM  ! model excess phase (m)
    real(wp), dimension(:)   :: impact_LM ! model impact parameter (m)
\end{Verbatim}
\textbf{NOTES:}\hspace{0.08in}\textbf{REFERENCES:}\hspace{0.08in}\section{Preprocessing/ropp\_pp\_openloop}
\textsl{[ Subroutines ]}

\label{ch:robo78}
\label{ch:Preprocessing_ropp_pp_openloop}
\index{unsorted!ropp\_pp\_openloop}\index{Subroutines!ropp\_pp\_openloop}
\textbf{NAME:}\hspace{0.08in}\begin{Verbatim}
    ropp_pp_openloop - Transform open-loop data into excess phase
\end{Verbatim}
\textbf{SYNOPSIS:}\hspace{0.08in}\begin{Verbatim}
    call ropp_pp_openloop(time, phase_L1, phase_L2, phase_LM, 
                             r_leo, r_gns, r_coc, LCF)
\end{Verbatim}
\textbf{DESCRIPTION:}\hspace{0.08in}\begin{Verbatim}
    This routine transforms open-loop data into excess phase.
      1. Frequency down-conversion by subtracting phase mode.
      2. Navigation bits removal.
      3. Phase accumulation.
      4. Restoring phase variation.
\end{Verbatim}
\textbf{INPUTS:}\hspace{0.08in}\begin{Verbatim}
    real(wp), dimension(:)   :: time      ! time of samples (s)
    real(wp), dimension(:)   :: phase_L1  ! excess phase L1 (m)
    real(wp), dimension(:)   :: phase_L2  ! excess phase L2 (m)
    real(wp), dimension(:)   :: phase_LM  ! model excess phase (m)
    real(wp), dimension(:,:) :: r_leo     ! cartesian LEO coordinates (ECF)
    real(wp), dimension(:,:) :: r_gns     ! cartesian GPS coordinates (ECF)
    real(wp), dimension(:)   :: r_coc     ! cartesian centre curvature (ECF)
    integer,  dimension(:)   :: lcf       ! lost carrier flag
\end{Verbatim}
\textbf{OUTPUT:}\hspace{0.08in}\begin{Verbatim}
    real(wp), dimension(:)   :: phase_L1  ! excess phase L1 (m)
    real(wp), dimension(:)   :: phase_L2  ! excess phase L2 (m)
    integer,  dimension(:)   :: lcf       ! lost carrier flag
\end{Verbatim}
\textbf{NOTES:}\hspace{0.08in}\textbf{REFERENCES:}\hspace{0.08in}\section{Preprocessing/ropp\_pp\_preprocess}
\textsl{[ Subroutines ]}

\label{ch:robo79}
\label{ch:Preprocessing_ropp_pp_preprocess}
\index{unsorted!ropp\_pp\_preprocess}\index{Subroutines!ropp\_pp\_preprocess}
\textbf{NAME:}\hspace{0.08in}\begin{Verbatim}
    ropp_pp_preprocess - Level1a data preprocessing
\end{Verbatim}
\textbf{SYNOPSIS:}\hspace{0.08in}\begin{Verbatim}
    call ropp_pp_preprocess(ro_data, config, diag)
\end{Verbatim}
\textbf{DESCRIPTION:}\hspace{0.08in}\textbf{INPUTS:}\hspace{0.08in}\begin{Verbatim}
    type(ROprof)   :: ro_data      ! Radio occultation data strucuture
    type(PPConfig) :: config       ! Configuration options
\end{Verbatim}
\textbf{OUTPUT:}\hspace{0.08in}\begin{Verbatim}
    type(ROprof)   :: ro_data      ! Corrected radio occultation data
    type(PPConfig) :: config       ! Configuration options
    type(PPDiag)   :: diag         ! Diagnostic output
\end{Verbatim}
\textbf{NOTES:}\hspace{0.08in}\begin{Verbatim}
   Requires ROprof data structure type, defined in ropp_io module. This
   routine therefore requires that the ropp_io module is pre-installed before
   compilation.
\end{Verbatim}
\textbf{REFERENCES:}\hspace{0.08in}\begin{Verbatim}
   Gorbunov M.E., Lauritsen K.B., Rhodin A., Tomassini M. and Kornblueh L.
   2006
   Radio holographic filtering, error estimation, and quality control of
   radio occultation data
   Journal of Geophysical Research (111) D10105
\end{Verbatim}
\section{Preprocessing/ropp\_pp\_preprocess\_CHAMP}
\textsl{[ Subroutines ]}

\label{ch:robo80}
\label{ch:Preprocessing_ropp_pp_preprocess_CHAMP}
\index{unsorted!ropp\_pp\_preprocess\_CHAMP}\index{Subroutines!ropp\_pp\_preprocess\_CHAMP}
\textbf{NAME:}\hspace{0.08in}\begin{Verbatim}
    ropp_pp_preprocess_CHAMP - Mission-specific Level1a data preprocessing
                               for CHAMP
\end{Verbatim}
\textbf{SYNOPSIS:}\hspace{0.08in}\begin{Verbatim}
    call ropp_pp_preprocess_CHAMP(ro_data, config)
\end{Verbatim}
\textbf{DESCRIPTION:}\hspace{0.08in}\begin{Verbatim}
    1) Amplitude correction
\end{Verbatim}
\textbf{INPUTS:}\hspace{0.08in}\begin{Verbatim}
    type(ROprof)   :: ro_data      ! Radio occultation data strucuture
    type(PPConfig) :: config       ! Configuration options
\end{Verbatim}
\textbf{OUTPUT:}\hspace{0.08in}\begin{Verbatim}
    type(ROprof)   :: ro_data      ! Corrected radio occultation data
    type(PPConfig) :: config       ! Configuration options
\end{Verbatim}
\textbf{NOTES:}\hspace{0.08in}\begin{Verbatim}
   Requires ROprof data structure type, defined in ropp_io module. This 
   routine therefore requires that the ropp_io module is pre-installed before
   compilation.
\end{Verbatim}
\textbf{REFERENCES:}\hspace{0.08in}\section{Preprocessing/ropp\_pp\_preprocess\_COSMIC}
\textsl{[ Subroutines ]}

\label{ch:robo81}
\label{ch:Preprocessing_ropp_pp_preprocess_COSMIC}
\index{unsorted!ropp\_pp\_preprocess\_COSMIC}\index{Subroutines!ropp\_pp\_preprocess\_COSMIC}
\textbf{NAME:}\hspace{0.08in}\begin{Verbatim}
    ropp_pp_preprocess_COSMIC - Mission-specific Level1a data preprocessing
                                for COSMIC 
\end{Verbatim}
\textbf{SYNOPSIS:}\hspace{0.08in}\begin{Verbatim}
    call ropp_pp_preprocess_COSMIC(ro_data, config, LCF)
\end{Verbatim}
\textbf{DESCRIPTION:}\hspace{0.08in}\begin{Verbatim}
    Removal of phase discontinuity - updates phase_L1 and phase_L2
\end{Verbatim}
\textbf{INPUTS:}\hspace{0.08in}\begin{Verbatim}
    type(ROprof)   :: ro_data      ! Radio occultation data strucuture
    type(PPConfig) :: config       ! Configuration options
    integer        :: LCF          ! Lost carrier flag
\end{Verbatim}
\textbf{OUTPUT:}\hspace{0.08in}\begin{Verbatim}
    type(ROprof)   :: ro_data      ! Corrected radio occultation data
    type(PPConfig) :: config       ! Configuration options
    integer        :: LCF          ! Lost carrier flag
\end{Verbatim}
\textbf{NOTES:}\hspace{0.08in}\begin{Verbatim}
   Requires ROprof data structure type, defined in ropp_io module. This 
   routine therefore requires that the ropp_io module is pre-installed before
   compilation.
\end{Verbatim}
\textbf{REFERENCES:}\hspace{0.08in}\section{Preprocessing/ropp\_pp\_preprocess\_GRASRS}
\textsl{[ Subroutines ]}

\label{ch:robo82}
\label{ch:Preprocessing_ropp_pp_preprocess_GRASRS}
\index{unsorted!ropp\_pp\_preprocess\_GRASRS}\index{Subroutines!ropp\_pp\_preprocess\_GRASRS}
\textbf{NAME:}\hspace{0.08in}\begin{Verbatim}
    ropp_pp_preprocess_GRASRS - Mission-specific Level1a data preprocessing
                                for GRAS Raw Sampling data
\end{Verbatim}
\textbf{SYNOPSIS:}\hspace{0.08in}\begin{Verbatim}
    call ropp_pp_preprocess_GRASRS(ro_data, config, LCF)
\end{Verbatim}
\textbf{DESCRIPTION:}\hspace{0.08in}\begin{Verbatim}
    Merge and upsample CL and RS data
    1. Select CL and RS records by LCF flag
    2. Generate merged time grid anchored at highest point of RS record
    3. Interpolate CL and RS data to merged time grid
    4. Merge data
    5. Generate phase model and connecting phase
    6. Interpolate data on merged time grid within small gaps
    7. Restore phase variation
\end{Verbatim}
\textbf{INPUTS:}\hspace{0.08in}\begin{Verbatim}
    type(ROprof)   :: ro_data      ! Radio occultation data strucuture
    type(PPConfig) :: config       ! Configuration options
    integer        :: LCF          ! Lost carrier flag
\end{Verbatim}
\textbf{OUTPUT:}\hspace{0.08in}\begin{Verbatim}
    type(ROprof)   :: ro_data      ! Corrected radio occultation data
    type(PPConfig) :: config       ! Configuration options
    integer        :: LCF          ! Lost carrier flag
\end{Verbatim}
\textbf{NOTES:}\hspace{0.08in}\begin{Verbatim}
   Requires ROprof data structure type, defined in ropp_io module. This
   routine therefore requires that the ropp_io module is pre-installed before
   compilation.
\end{Verbatim}
\textbf{REFERENCES:}\hspace{0.08in}\section{Preprocessing/ropp\_pp\_radioholographic\_filter}
\textsl{[ Subroutines ]}

\label{ch:robo83}
\label{ch:Preprocessing_ropp_pp_radioholographic_filter}
\index{unsorted!ropp\_pp\_radioholographic\_filter}\index{Subroutines!ropp\_pp\_radioholographic\_filter}
\textbf{NAME:}\hspace{0.08in}\begin{Verbatim}
    ropp_pp_radioholographic_filter - Radioholographic filtering of radio
                                      occultation data
\end{Verbatim}
\textbf{SYNOPSIS:}\hspace{0.08in}\begin{Verbatim}
    call ropp_pp_radioholographic_filter(time,r_leo,r_gns,r_coc,roc,phase_LM,
                                         phase, snr)
\end{Verbatim}
\textbf{DESCRIPTION:}\hspace{0.08in}\begin{Verbatim}
     1) Multiplication of the radio occulutation data with a reference signal 
     2) Fourier filter combined signal
     3) Removal of reference signal
\end{Verbatim}
\textbf{INPUTS:}\hspace{0.08in}\begin{Verbatim}
    real(wp), dimension(:)   :: time     ! relative time of samples (s)
    real(wp), dimension(:,:) :: r_leo    ! cartesian LEO coordinates (m)
    real(wp), dimension(:,:) :: r_gns    ! cartesian GPS coordinates (m)
    real(wp), dimension(:)   :: r_coc    ! cartesian centre curvature (m)
    real(wp)                 :: roc      ! radius of curvature (m)
    real(wp), dimension(:)   :: phase_LM ! model excess phase (m)
    real(wp), dimension(:,:) :: phase    ! L1 and L2 excess phase (m)
    real(wp), dimension(:,:) :: snr      ! L1 and L2 amplitude
\end{Verbatim}
\textbf{OUTPUT:}\hspace{0.08in}\begin{Verbatim}
    real(wp), dimension(:,:) :: phase    ! L1 and L2 excess phase (m)
    real(wp), dimension(:,:) :: snr      ! L1 and L2 amplitude
\end{Verbatim}
\textbf{NOTES:}\hspace{0.08in}\textbf{REFERENCES:}\hspace{0.08in}\begin{Verbatim}
   Gorbunov M.E., Lauritsen K.B., Rhodin A., Tomassini M. and Kornblueh L. 
   2006
   Radio holographic filtering, error estimation, and quality control of 
   radio occultation data
   Journal of Geophysical Research (111) D10105

   Gorbunov M.E., Lauritsen K.B., Rodin A., Tomassini M., Kornblueh L. 2005 
   Analysis of the CHAMP experimental data on radio-occultation sounding of
   the Earth's atmosphere.
   Izvestiya Atmospheric and Oceanic Physics (41) 726-740.
\end{Verbatim}
\section{Preprocessing/ropp\_pp\_radiooptic\_analysis}
\textsl{[ Subroutines ]}

\label{ch:robo84}
\label{ch:Preprocessing_ropp_pp_radiooptic_analysis}
\index{unsorted!ropp\_pp\_radiooptic\_analysis}\index{Subroutines!ropp\_pp\_radiooptic\_analysis}
\textbf{NAME:}\hspace{0.08in}\begin{Verbatim}
    ropp_pp_radiooptic_analysis - Radiooptical analysis of radio occultation
                                  data - calculation of local spatial spectra
\end{Verbatim}
\textbf{SYNOPSIS:}\hspace{0.08in}\begin{Verbatim}
    call ropp_pp_radiooptic_analysis(time, r_leo, r_gns, r_coc, roc, 
                                         phase_LM, phase, snr, PA, PD, OutRO, filnam)
\end{Verbatim}
\textbf{DESCRIPTION:}\hspace{0.08in}\begin{Verbatim}
     Calculation of local spatial spectra
\end{Verbatim}
\textbf{INPUTS:}\hspace{0.08in}\begin{Verbatim}
    real(wp), dimension(:)     :: time     ! relative time of samples (s)
    real(wp), dimension(:,:)   :: r_leo    ! cartesian LEO coordinates (m)
    real(wp), dimension(:,:)   :: r_gns    ! cartesian GPS coordinates (m)
    real(wp), dimension(:)     :: r_coc    ! cartesian centre curvature (m)
    real(wp)                   :: roc      ! radius of curvature (m)
    real(wp), dimension(:)     :: phase_LM ! model excess phase (m)
    real(wp), dimension(:,:)   :: phase    ! L1 and L2 excess phase (m)
    real(wp), dimension(:,:)   :: snr      ! L1 and L2 amplitude
    logical, optional          :: OutRO    ! Flag to output spectra results
    character(len=*), optional :: filnam   ! Output file name root
\end{Verbatim}
\textbf{OUTPUT:}\hspace{0.08in}\begin{Verbatim}
    real(wp), dimension(:,:)   :: PA       ! Average impact parameter (m)
    real(wp), dimension(:,:)   :: PD       ! RMS deviation of impact parameter
\end{Verbatim}
\textbf{REFERENCES:}\hspace{0.08in}\begin{Verbatim}
   Gorbunov M.E., Lauritsen K.B., Rhodin A., Tomassini M. and Kornblueh L. 
   2006
   Radio holographic filtering, error estimation, and quality control of 
   radio occultation data
   Journal of Geophysical Research (111) D10105

   Gorbunov M.E., Lauritsen K.B., Rodin A., Tomassini M., Kornblueh L. 2005 
   Analysis of the CHAMP experimental data on radio-occultation sounding of
   the Earth's atmosphere.
   Izvestiya Atmospheric and Oceanic Physics (41) 726-740.
\end{Verbatim}
\section{Preprocessing/ropp\_pp\_satellite\_velocities}
\textsl{[ Subroutines ]}

\label{ch:robo85}
\label{ch:Preprocessing_ropp_pp_satellite_velocities}
\index{unsorted!ropp\_pp\_satellite\_velocities}\index{Subroutines!ropp\_pp\_satellite\_velocities}
\textbf{NAME:}\hspace{0.08in}\begin{Verbatim}
    ropp_pp_satellite_velocities - Calculate satellite velocities by
                                   polynomial regression
\end{Verbatim}
\textbf{SYNOPSIS:}\hspace{0.08in}\begin{Verbatim}
    call ropp_pp_satellite_velocities(time, r_leo, r_gns, xleo, vleo,    &
                                      xgns, vgns, abl, abg)
\end{Verbatim}
\textbf{DESCRIPTION:}\hspace{0.08in}\begin{Verbatim}
    This routine calculates satellite velocities by polynomial regression
\end{Verbatim}
\textbf{INPUTS:}\hspace{0.08in}\begin{Verbatim}
    real(wp), dimension(:)   :: time    ! time of sample (s)
    real(wp), dimension(:,:) :: r_leo   ! cartesian LEO position (ECI or ECF)
    real(wp), dimension(:,:) :: r_gns   ! cartesian GPS position (ECI or ECF)

 OUTPUTS
    real(wp), dimension(:,:) :: xleo    ! LEO positions from regression
    real(wp), dimension(:,:) :: vleo    ! LEO velocities from regression
    real(wp), dimension(:,:) :: xgns    ! GPS positions from regression
    real(wp), dimension(:,:) :: vgns    ! GPS velocities from regression
    real(wp), dimension(:,:) :: abl     ! LEO regression coefficients (opt)
    real(wp), dimension(:,:) :: abg     ! GPS regression coefficients (opt)
\end{Verbatim}
\textbf{NOTES:}\hspace{0.08in}\textbf{REFERENCES:}\hspace{0.08in}\section{Preprocessing/ropp\_pp\_spectra}
\textsl{[ Subroutines ]}

\label{ch:robo86}
\label{ch:Preprocessing_ropp_pp_spectra}
\index{unsorted!ropp\_pp\_spectra}\index{Subroutines!ropp\_pp\_spectra}
\textbf{NAME:}\hspace{0.08in}\begin{Verbatim}
    ropp_pp_spectra - Calculation of local spatial spectra
\end{Verbatim}
\textbf{SYNOPSIS:}\hspace{0.08in}\begin{Verbatim}
    call ropp_pp_spectra(time, phase_L1, phase_L2, phase_LM, impact_LM,
                         config, OutRO, filnam)
\end{Verbatim}
\textbf{DESCRIPTION:}\hspace{0.08in}\textbf{INPUTS:}\hspace{0.08in}\begin{Verbatim}
    real(wp), dimension(:)     :: time      ! time of samples (s)
    real(wp), dimension(:)     :: phase_L1  ! excess phase L1 (m)
    real(wp), dimension(:)     :: phase_L2  ! excess phase L2 (m)
    real(wp), dimension(:)     :: phase_LM  ! model excess phase (m)
    real(wp), dimension(:)     :: impact_LM ! model impact parameter (m)
    type(PPConfig)             :: config    ! Configuration options
    logical, optional          :: OutRO     ! Flag to output spectra results
    character(len=*), optional :: filnam    ! Output file name root
\end{Verbatim}
\textbf{OUTPUT:}\hspace{0.08in}\begin{Verbatim}
    Spectra as functions of (doppler,time) written to output ASCII files.
\end{Verbatim}
\textbf{NOTES:}\hspace{0.08in}\textbf{REFERENCES:}\hspace{0.08in}\begin{Verbatim}
   Gorbunov M.E., Lauritsen K.B., Rhodin A., Tomassini M. and Kornblueh L.
   2006
   Radio holographic filtering, error estimation, and quality control of
   radio occultation data
   Journal of Geophysical Research (111) D10105
\end{Verbatim}
\section{Programs/ropp\_pp\_abel\_tool}
\textsl{[ Parameters ]}

\label{ch:robo87}
\label{ch:Programs_ropp_pp_abel_tool}
\index{unsorted!ropp\_pp\_abel\_tool}\index{Parameters!ropp\_pp\_abel\_tool}
\textbf{NAME:}\hspace{0.08in}\begin{Verbatim}
   ropp_pp_abel_tool
\end{Verbatim}
\textbf{SYNOPSIS:}\hspace{0.08in}\begin{Verbatim}
   Tool to test Abel transform and inverse Abel transform routines.

   > ropp_pp_abel_tool [<options>] <infile(s)>
\end{Verbatim}
\textbf{ARGUMENTS:}\hspace{0.08in}\begin{Verbatim}
  <infile(s)>   One (or more) input file names.
\end{Verbatim}
\textbf{OPTIONS:}\hspace{0.08in}\begin{Verbatim}
   -o <output_file>  name of ROPP netCDF output file
   -h                help
   -v                version information
\end{Verbatim}
\textbf{DESCRIPTION:}\hspace{0.08in}\begin{Verbatim}
   This program reads RO refractivity data from the input data files and 
   calculates vertical profiles of bending angles using the Abel transform. 
   The inverse Abel transform is then used to compute a new refractivity 
   profile, which may be compared with the input data.
\end{Verbatim}
\textbf{NOTES:}\hspace{0.08in}\begin{Verbatim}
   If the input file is a multifile, or more than one input files are
   specified, the output file is a multifile.

   Already existing output files will be overwritten.
\end{Verbatim}
\section{Programs/ropp\_pp\_grasrs2ropp}
\textsl{[ Parameters ]}

\label{ch:robo88}
\label{ch:Programs_ropp_pp_grasrs2ropp}
\index{unsorted!ropp\_pp\_grasrs2ropp}\index{Parameters!ropp\_pp\_grasrs2ropp}
\textbf{NAME:}\hspace{0.08in}\begin{Verbatim}
   ropp_pp_grasrs2ropp
\end{Verbatim}
\textbf{SYNOPSIS:}\hspace{0.08in}\begin{Verbatim}
   Program to convert EUMETSAT GRAS Raw Sampling data netcdf file to standard
   ROPP format netCDF structure, suitable for processing with ropp_pp_occ_tool.

   > ropp_pp_grasrs2ropp [<options>] <infile(s)>
\end{Verbatim}
\textbf{ARGUMENTS:}\hspace{0.08in}\begin{Verbatim}
   <infile(s)>   One (or more) input file names.
\end{Verbatim}
\textbf{OPTIONS:}\hspace{0.08in}\begin{Verbatim}
   -o <output_file>  name of ROPP netCDF output file
                     (default: ropp_pp_gras.nc)
   -d                output additional diagnostics
   -h                help
   -v                version information
\end{Verbatim}
\textbf{DESCRIPTION:}\hspace{0.08in}\begin{Verbatim}
   This program reads EUMETSAT format netCDF files containing closed loop and
   raw sampling data from GRAS. Data are translated into the standard ROPP
   data format structures and written to a ROPP formatted output file.
\end{Verbatim}
\textbf{NOTES:}\hspace{0.08in}\begin{Verbatim}
   If the input file is a multifile, or more than one input files are
   specified, the output file is a multifile.

   Already existing output files will be overwritten.
\end{Verbatim}
\section{Programs/ropp\_pp\_invert\_tool}
\textsl{[ Parameters ]}

\label{ch:robo89}
\label{ch:Programs_ropp_pp_invert_tool}
\index{unsorted!ropp\_pp\_invert\_tool}\index{Parameters!ropp\_pp\_invert\_tool}
\textbf{NAME:}\hspace{0.08in}\begin{Verbatim}
    ropp_pp_invert_tool
\end{Verbatim}
\textbf{SYNOPSIS:}\hspace{0.08in}\begin{Verbatim}
   Calculate refractivity and dry temperature profile from L1 and L2
   channel radio occultation bending angle data using ionospheric
   correction, statistical optimization, Abel transform, and 
   hydrostatic equation.

   > ropp_pp_invert_tool infile(s)
                         [-c <config_file>]  [-o <output_file>[
                         [-m <method>]  [-mfile <file>]  [-bfile <file>]
                         [-d] [-h] [-v]
\end{Verbatim}
\textbf{ARGUMENTS:}\hspace{0.08in}\begin{Verbatim}
   infile(s)   One or more input file names.
\end{Verbatim}
\textbf{OPTIONS:}\hspace{0.08in}\begin{Verbatim}
   -c <config_file> name of configuration file
   -o <output_file> name of ROPP netCDF output file
   -m <method>      ionospheric correction method
                    [NONE,MSIS,GMSIS,BG], (default GMSIS)
   -mfile <file>    model refractivity coefficients file
                    (default 'MSIS_coeff.nc')
   -bfile <file>    background model atmospheric profile file path
                    (if using BG ionospheric correction method)
   -d               output additional diagnostics
   -h               help
   -v               version information

    -o <outfile>  Name of output file (default: ropp_pp_inv.nc).
\end{Verbatim}
\textbf{DESCRIPTION:}\hspace{0.08in}\begin{Verbatim}
    This program reads RO L1 and L2 bending angle data on impact parameter 
    levels from the input data files and calculates vertical profiles of 
    ionospheric corrected bending angle, refractivity, and dry temperature. 
    The result is written to a ROPP formatted output file.
\end{Verbatim}
\textbf{NOTES:}\hspace{0.08in}\begin{Verbatim}
   If the input file is a multifile, or more than one input files are
   specified, the output file is a multifile.

   Already existing output files will be overwritten.
\end{Verbatim}
\textbf{EXAMPLE:}\hspace{0.08in}\begin{Verbatim}
   To calculate bending angle, refractivity, and dry temperature 
   from one of the example (single-) files in the data directory:
     > ropp_pp_invert_tool ../data/input.nc -o example_01.nc

   To calculate bending angle, refractivity and dry temperature profiles
   from all singlefiles in the data directory:

     > ropp_pp_occ_tool ../data/*.nc -o example_02.nc

   Note that the resulting example_02.nc file contains processed data from
   all example profiles.

   To calculate bending angle, refractivity, and dry temperature profiles from
   all profiles contained in the multifile multi.nc:
     > ropp_pp_invert_tool ../data/multi.nc -o example_03.nc

   Since the multi_* file was generated by concatenating the other files
   in the data directory, example_02.nc and example_03.nc should be identical
   apart from the file names.
\end{Verbatim}
\section{Programs/ropp\_pp\_occ\_tool}
\textsl{[ Parameters ]}

\label{ch:robo90}
\label{ch:Programs_ropp_pp_occ_tool}
\index{unsorted!ropp\_pp\_occ\_tool}\index{Parameters!ropp\_pp\_occ\_tool}
\textbf{NAME:}\hspace{0.08in}\begin{Verbatim}
   ropp_pp_occ_tool
\end{Verbatim}
\textbf{SYNOPSIS:}\hspace{0.08in}\begin{Verbatim}
   Pre-processing tool to calculate refractivity and dry temperature 
   profile from L1 and L2 excess phase radio occultation data using 
   geometric optics or wave optics, ionospheric correction, statistical 
   optimization, Abel transform and hydrostatic equation.

   > ropp_pp_occ_tool [<options>] <infile(s)>
\end{Verbatim}
\textbf{ARGUMENTS:}\hspace{0.08in}\begin{Verbatim}
   <infile(s)>   One (or more) input file names.
\end{Verbatim}
\textbf{OPTIONS:}\hspace{0.08in}\begin{Verbatim}
   -c <config_file>  name of configuration file
   -o <output_file>  name of ROPP netCDF output file
   -m <method>       ionospheric correction method [NONE,MSIS,GMSIS,BG]
                     (default GMSIS)
   -mfile <file>     model refractivity coefficients file
                     (default 'MSIS_coeff.nc')
   -bfile <file>     background model atmospheric profile file path
                     (if using BG ionospheric correction method)
   -navfile <file>   external navigation bit *_txt file path
                     (default internal correction)
   -occ <method>     processing method, WO or GO (default WO)
   -filter <method>  filtering method, slpoly or optest
                     (default slpoly, sliding polynomial)
   -fit              apply 2-parameter regression fit to model
   -ellipsoid        output height with respect to WGS84 ellipsoid
                     (default output wrt EGM96 geoid)
   -d                output additional diagnostics
   -h                help
   -v                version information
\end{Verbatim}
\textbf{DESCRIPTION:}\hspace{0.08in}\begin{Verbatim}
   This program reads RO L1 and L2 excess phase data as a function of time
   from the input data files and calculates vertical profiles L1 and L2
   bending angles, ionospheric corrected bending angle, refractivity and
   dry temperature.  The result is written to a ROPP output file.
\end{Verbatim}
\textbf{NOTES:}\hspace{0.08in}\begin{Verbatim}
   If the input file is a multifile, or more than one input files are
   specified, the output file is a multifile.

   Already existing output files will be overwritten.
\end{Verbatim}
\textbf{EXAMPLE:}\hspace{0.08in}\begin{Verbatim}
   To calculate bending angle, refractivity and dry temperature from one
   of the example (single-) files in the data directory:

     > ropp_pp_occ_tool ../data/input.nc -o example_01.nc

   To calculate bending angle, refractivity and dry temperature profiles
   from all singlefiles in the data directory:

     > ropp_pp_occ_tool ../data/*.nc -o example_02.nc

   Note that the resulting example_02.nc file contains processed data from
   all example profiles.

   To calculate bending angle, refractivity and dry temperature profiles 
   from all profiles contained in the multifile multi.nc:

     > ropp_pp_occ_tool ../data/multi.nc -o example_03.nc

   Since the multi_* file was generated by concatenating the other files
   in the data directory, example_02.nc and example_03.nc should be identical
   apart from the file names.
\end{Verbatim}
\section{Programs/ropp\_pp\_spectra\_tool}
\textsl{[ Parameters ]}

\label{ch:robo91}
\label{ch:Programs_ropp_pp_spectra_tool}
\index{unsorted!ropp\_pp\_spectra\_tool}\index{Parameters!ropp\_pp\_spectra\_tool}
\textbf{NAME:}\hspace{0.08in}\begin{Verbatim}
   ropp_pp_spectra_tool
\end{Verbatim}
\textbf{SYNOPSIS:}\hspace{0.08in}\begin{Verbatim}
   Pre-processing tool to compute local spatial spectra of the
   complex wave field.

   >  ropp_pp_spectra_tool <infile(s)> [-o <outfile>]
                           [-c <cnfgfile>] [-m <method>] [-mfile <mfile>]
                           [-navfile <nfile>]
                           [-h] [-v]
\end{Verbatim}
\textbf{ARGUMENTS:}\hspace{0.08in}\begin{Verbatim}
   <infile(s)>   One (or more) input file names
\end{Verbatim}
\textbf{OPTIONS:}\hspace{0.08in}\begin{Verbatim}
   -o <outfile>      name of spectra output file
                     (default: ROanalysis_pp_<type>_<freq>.dat)
   -c <cnfgfile>     name of configuration file
   -m <method>       ionospheric correction method
   -mfile <mfile>    model refractivity file
   -navfile <nfile>  name of external navigation bit file
   -h                help
   -v                version information
\end{Verbatim}
\textbf{DESCRIPTION:}\hspace{0.08in}\begin{Verbatim}
   This program reads RO L1 and L2 excess phase data as a function of time
   from the input data files and calculates local spatial spectra as a
   function of doppler frequency shift and time or as a function of bending
   angle and impact parameter. The results are written to four ASCII files, 
   whose contents can be plotted by, for instance, 
   ropp_pp/tests/it_pp_spectra_dt.pro and ropp_pp/tests/it_pp_spectra_dt.pro.
\end{Verbatim}
\textbf{NOTES:}\hspace{0.08in}\begin{Verbatim}
   If the input file is a multifile, or more than one input files are
   specified, output for all profiles are written out in separate files.

   Already existing output files will be overwritten.
\end{Verbatim}
\textbf{EXAMPLE:}\hspace{0.08in}\begin{Verbatim}
   To calculate spectra for one of the example files in the data directory:

     > ropp_pp_spectra_tool ../data/input.nc
\end{Verbatim}
\textbf{SEE ALSO:}\hspace{0.08in}\begin{Verbatim}
   For an example plotting tool to view the resulting spectra see
   ropp_pp/tests_plot_spectra.pro
\end{Verbatim}
\textbf{REFERENCES:}\hspace{0.08in}\begin{Verbatim}
  Gorbunov M.E., Lauritsen K.B., Rhodin A., Tomassini M. and
   Kornblueh L. 2006
   Radio holographic filtering, error estimation, and quality control of
   radio occultation data
   Journal of Geophysical Research (111) D10105
\end{Verbatim}
\section{Programs/ropp\_pp\_tph\_tool}
\textsl{[ Parameters ]}

\label{ch:robo92}
\label{ch:Programs_ropp_pp_tph_tool}
\index{unsorted!ropp\_pp\_tph\_tool}\index{Parameters!ropp\_pp\_tph\_tool}
\textbf{NAME:}\hspace{0.08in}\begin{Verbatim}
   ropp_pp_tph_tool
\end{Verbatim}
\textbf{SYNOPSIS:}\hspace{0.08in}\begin{Verbatim}
   Tropopause height (TPH) diagnostic

   > ropp_pp_tph_tool [<options>] <infile(s)>
\end{Verbatim}
\textbf{ARGUMENTS:}\hspace{0.08in}\begin{Verbatim}
   <infile(s)>   One (or more) input file names.
\end{Verbatim}
\textbf{OPTIONS:}\hspace{0.08in}\begin{Verbatim}
   -o <output_file>  name of ROPP netCDF output file
   -b                calculate bending angle-based TPH
   -n                calculate refractivity-based TPH
   -y                calculate dry-temperature-based TPH
   -t                calculate temperature-based TPH
   -h                help
   -d                output additional diagnostics
   -v                version information
\end{Verbatim}
\textbf{DESCRIPTION:}\hspace{0.08in}\begin{Verbatim}
   Diagnose tropopause height from the kinks in one or more RO profiles, 
   using the covariance transform method of Lewis (GRL 2009).
\end{Verbatim}
\textbf{NOTES:}\hspace{0.08in}\begin{Verbatim}
   If the input file is a multifile, or more than one input files are
   specified, the output file is a multifile.

   Existing output files will be overwritten.

   If none of the {-b, -n, -y, -t} options is specified, the tool will 
   attempt to calculate all four TPHs.
\end{Verbatim}
\section{Setup/ropp\_pp\_read\_config}
\textsl{[ Subroutines ]}

\label{ch:robo95}
\label{ch:Setup_ropp_pp_read_config}
\index{unsorted!ropp\_pp\_read\_config}\index{Subroutines!ropp\_pp\_read\_config}
\textbf{NAME:}\hspace{0.08in}\begin{Verbatim}
    ropp_pp_read_config - Read a configuration file for ROPP_PP
\end{Verbatim}
\textbf{SYNOPSIS:}\hspace{0.08in}\begin{Verbatim}
    call ropp_pp_read_config(file, config)
\end{Verbatim}
\textbf{DESCRIPTION:}\hspace{0.08in}\textbf{INPUTS:}\hspace{0.08in}\begin{Verbatim}
      file
\end{Verbatim}
\textbf{OUTPUT:}\hspace{0.08in}\begin{Verbatim}
     config
\end{Verbatim}
\section{Thermodynamics/C\_p}
\textsl{[ Parameters ]}

\label{ch:robo96}
\label{ch:Thermodynamics_C_p}
\index{unsorted!C\_p}\index{Parameters!C\_p}
\textbf{NAME:}\hspace{0.08in}\begin{Verbatim}
    C_p - Specific heat capacity of dry air at constant pressure
\end{Verbatim}
\textbf{SOURCE:}\hspace{0.08in}\begin{Verbatim}
!!  REAL(wp), PARAMETER :: C_p = 1005.7_wp       !   J K^{-1} kg^{-1}
  REAL(wp), PARAMETER :: C_p = 1004.6_wp       !   J K^{-1} kg^{-1}
\end{Verbatim}
\section{Thermodynamics/epsilon\_water}
\textsl{[ Parameters ]}

\label{ch:robo97}
\label{ch:Thermodynamics_epsilon_water}
\index{unsorted!epsilon\_water}\index{Parameters!epsilon\_water}
\textbf{NAME:}\hspace{0.08in}\begin{Verbatim}
    epsilon_water - Ratio of molecular weight of water to that of dry air.
\end{Verbatim}
\textbf{SOURCE:}\hspace{0.08in}\begin{Verbatim}
  REAL(wp), PARAMETER :: epsilon_water = 287.05_wp / 461.51_wp  !mw_water / mw_dry_air
\end{Verbatim}
\section{Thermodynamics/mw\_dry\_air}
\textsl{[ Parameters ]}

\label{ch:robo98}
\label{ch:Thermodynamics_mw_dry_air}
\index{unsorted!mw\_dry\_air}\index{Parameters!mw\_dry\_air}
\textbf{NAME:}\hspace{0.08in}\begin{Verbatim}
    mw_dry_air - Molecular weight of dry air
\end{Verbatim}
\textbf{SOURCE:}\hspace{0.08in}\begin{Verbatim}
  REAL(wp), PARAMETER :: mw_dry_air = 28.9648e-3_wp !   kg mol^{-1}
\end{Verbatim}
\section{Thermodynamics/mw\_water}
\textsl{[ Parameters ]}

\label{ch:robo99}
\label{ch:Thermodynamics_mw_water}
\index{unsorted!mw\_water}\index{Parameters!mw\_water}
\textbf{NAME:}\hspace{0.08in}\begin{Verbatim}
    mw_water - Molecular weight of water
\end{Verbatim}
\textbf{SOURCE:}\hspace{0.08in}\begin{Verbatim}
  REAL(wp), PARAMETER :: mw_water = 18.01528e-3_wp  !   kg mol^{-1}
\end{Verbatim}
\section{Thermodynamics/R\_dry}
\textsl{[ Parameters ]}

\label{ch:robo100}
\label{ch:Thermodynamics_R_dry}
\index{unsorted!R\_dry}\index{Parameters!R\_dry}
\textbf{NAME:}\hspace{0.08in}\begin{Verbatim}
    R_dry - Gas constant of dry air
\end{Verbatim}
\textbf{SOURCE:}\hspace{0.08in}\begin{Verbatim}
!!  REAL(wp), PARAMETER :: R_dry = 287.0597_wp        !   J K^{-1} kg^{-1}
  REAL(wp), PARAMETER :: R_dry = 287.05_wp        !   J K^{-1} kg^{-1}
\end{Verbatim}
\section{Thermodynamics/R\_vap}
\textsl{[ Parameters ]}

\label{ch:robo101}
\label{ch:Thermodynamics_R_vap}
\index{unsorted!R\_vap}\index{Parameters!R\_vap}
\textbf{NAME:}\hspace{0.08in}\begin{Verbatim}
    R_vap - Gas constant of water vapor
\end{Verbatim}
\textbf{SOURCE:}\hspace{0.08in}\begin{Verbatim}
!!  REAL(wp), PARAMETER :: R_vap = 461.5250_wp        !   J K^{-1} kg^{-1}
  REAL(wp), PARAMETER :: R_vap = 461.51_wp        !   J K^{-1} kg^{-1}
\end{Verbatim}
\section{Tools/Coordinates}
\textsl{[ Topics ]}

\label{ch:robo102}
\label{ch:Tools_Coordinates}
\index{unsorted!Coordinates}\index{Topics!Coordinates}
\textbf{DESCRIPTION:}\hspace{0.08in}\begin{Verbatim}
    Routines for dealing with coordinate frames
\end{Verbatim}
\textbf{SEE ALSO:}\hspace{0.08in}\begin{Verbatim}
    ropp_pp_set_coordinates
\end{Verbatim}
\subsection{Coordinates/ropp\_pp\_set\_coordinates}
\textsl{[ Coordinates ]}
\textsl{[ Subroutines ]}

\label{ch:robo8}
\label{ch:Coordinates_ropp_pp_set_coordinates}
\index{unsorted!ropp\_pp\_set\_coordinates}\index{Subroutines!ropp\_pp\_set\_coordinates}
\textbf{NAME:}\hspace{0.08in}\begin{Verbatim}
    ropp_pp_set_coordinatess - Set internally used position and velocity 
                               reference frames used in ropp_pp processing
\end{Verbatim}
\textbf{SYNOPSIS:}\hspace{0.08in}\begin{Verbatim}
    use ropp_io
    use ropp_pp
      ...
    type(ROprof) :: rodata
      ...
    call ropp_pp_set_coordinates(rodata)
\end{Verbatim}
\textbf{DESCRIPTION:}\hspace{0.08in}\begin{Verbatim}
    This subroutine sets the reference coordinate frame within an ROprof
    data structure to the reference frame used internally in the ropp_pp 
    package. For each variable to be defined a call to ropp_utils function
    coordinates_eci2ecef is made to transform the data if required. 
\end{Verbatim}
\textbf{INPUTS:}\hspace{0.08in}\begin{Verbatim}
    rodata  Radio occultation profile data structure
\end{Verbatim}
\textbf{OUTPUT:}\hspace{0.08in}\begin{Verbatim}
    rodata  As above, but with Level1a data coordinate reference frame 
             modified to reflect standard reference frames as used within 
             ropp_pp.
\end{Verbatim}
\textbf{NOTES:}\hspace{0.08in}\begin{Verbatim}
    Default coordinate frames assumed by ropp_pp processing are:
         r_leo : ECF (Earth Centred Fixed)
         r_gns : ECF (Earth Centred Fixed)
         v_leo : ECI (Earth Centred Inertial)
         v_gns : ECI (Earth Centred Inertial)
\end{Verbatim}
\section{WaveOptics/ropp\_pp\_bending\_angle\_wo}
\textsl{[ Subroutines ]}

\label{ch:robo103}
\label{ch:WaveOptics_ropp_pp_bending_angle_wo}
\index{unsorted!ropp\_pp\_bending\_angle\_wo}\index{Subroutines!ropp\_pp\_bending\_angle\_wo}
\textbf{NAME:}\hspace{0.08in}\begin{Verbatim}
    ropp_pp_bending_angle_wo - Calculate L1 and L2 bending angle profiles from
                               occultation data by WAVE OPTICS
\end{Verbatim}
\textbf{SYNOPSIS:}\hspace{0.08in}\begin{Verbatim}
    call ropp_pp_bending_angle_wo(time, r_leo, r_gns, r_coc, phase_L1,
                                  phase_L2, snr_L1, snr_L2, w_ls,
                                  w_smooth, w_low, hmax, filter, opt_DL2,
                                  cff, dsh, 
                                  impact_L1, bangle_L1, ba_sigma_L1,
                                  impact_L2, bangle_L2, ba_sigma_L2, diag)
\end{Verbatim}
\textbf{DESCRIPTION:}\hspace{0.08in}\begin{Verbatim}
    This routine calculates L1 and L2 bending angles using the CT2 algorithm.
\end{Verbatim}
\textbf{INPUTS:}\hspace{0.08in}\begin{Verbatim}
    real(wp), dimension(:)   :: time      ! Relative time of samples (s)
    real(wp), dimension(:,:) :: r_leo     ! LEO coordinates (m) (ECI or ECF)
    real(wp), dimension(:,:) :: r_gns     ! GPS coordinates (m) (ECI or ECF)
    real(wp), dimension(:)   :: r_coc     ! Centre curvature coords (m)
    real(wp)                 :: roc       ! Radius curvature (m)
    integer                  :: w_ls      ! Large-scale smoothing (points)
    integer                  :: w_smooth  ! Smoothing window above 7km (point)
    integer                  :: w_low     ! Smoothing window below 7km (point)
    real(wp)                 :: hmax      ! Maximum height for WO (m)
    character(len=*)         :: filter    ! Filter method ('optest'/'slpoly')
    logical                  :: opt_DL2   ! Degraded L2 flag
    integer                  :: cff       ! Complex filtering flag
    real(wp)                 :: dsh       ! Shadow border width (m)
    real(wp), dimension(:)   :: phase_L1  ! L1 excess phase (m)
    real(wp), dimension(:)   :: phase_L2  ! L2 excess phase (m)
    real(wp), dimension(:)   :: snr_L1    ! L1 amplitude
    real(wp), dimension(:)   :: snr_L2    ! L2 amplitude
    real(wp), dimension(:)   :: impact_L1 ! L1 impact parameters (m)
    real(wp), dimension(:)   :: bangle_L1 ! L1 bending angles (rad)
    real(wp), dimension(:)   :: impact_L2 ! L2 impact parameters (m)
    real(wp), dimension(:)   :: bangle_L2 ! L2 bending angles (rad)
\end{Verbatim}
\textbf{OUTPUT:}\hspace{0.08in}\begin{Verbatim}
    real(wp), dimension(:)   :: impact_L1 ! L1 impact parameters (m)
    real(wp), dimension(:)   :: bangle_L1 ! L1 bending angles (rad)
    real(wp), dimension(:)   :: ba_sigma_L1 ! L1 bending angle error (rad)
    real(wp), dimension(:)   :: impact_L2 ! L2 impact parameters (m)
    real(wp), dimension(:)   :: bangle_L2 ! L2 bending angles (rad)
    real(wp), dimension(:)   :: ba_sigma_L2 ! L2 bending angle error (rad)
    type(PPdiag), optional   :: diag      ! Additional output diagnostic strt
\end{Verbatim}
\textbf{NOTES:}\hspace{0.08in}\textbf{REFERENCES:}\hspace{0.08in}\begin{Verbatim}
   Gorbunov M.E. and Lauritsen K.B. 2004
   Analysis of wave fields by Fourier integral operators and their application
   for radio occultations
   Radio Science (39) RS4010

   Gorbunov M.E., Lauritsen K.B., Rodin A., Tomassini M., Kornblueh L. 2005 
   Analysis of the CHAMP experimental data on radio-occultation sounding of
   the Earth's atmosphere.
   Izvestiya Atmospheric and Oceanic Physics (41) 726-740.
\end{Verbatim}
\section{WaveOptics/ropp\_pp\_DCT}
\textsl{[ Subroutines ]}

\label{ch:robo104}
\label{ch:WaveOptics_ropp_pp_DCT}
\index{unsorted!ropp\_pp\_DCT}\index{Subroutines!ropp\_pp\_DCT}
\textbf{NAME:}\hspace{0.08in}\begin{Verbatim}
    ropp_pp_DCT - Calculate L1 and L2 bending angle profiles using
                  Canonical Transform (CT2).
\end{Verbatim}
\textbf{SYNOPSIS:}\hspace{0.08in}\begin{Verbatim}
    call ropp_pp_DCT(time, snr, phase, r_leo, r_gns, r_coc, roc, w_ls,
                     w_smooth, hmax, filter, opt_DL2, cff, dsh,
                     impact, bangle, ba_cov, diag)
\end{Verbatim}
\textbf{DESCRIPTION:}\hspace{0.08in}\begin{Verbatim}
    This routine calculates L1 and L2 bending angles using a CT2 algorithm.
\end{Verbatim}
\textbf{INPUTS:}\hspace{0.08in}\begin{Verbatim}
    real(wp), dimension(:)   :: time     ! Relative time of samples (s)
    real(wp), dimension(:,:) :: snr      ! L1,L2 amplitudes [channel, time]
    real(wp), dimension(:,:) :: phase    ! L1,L2 excess phase (m) [ch, time]
    real(wp), dimension(:,:) :: r_leo    ! LEO coordinates (m) (ECI or ECF)
    real(wp), dimension(:,:) :: r_gns    ! GPS coordinates (m) (ECI or ECF)
    real(wp), dimension(:)   :: r_coc    ! Centre curvature coordinates (m)
    real(wp)                 :: roc      ! Radius of curvature (m)
    integer                  :: w_ls     ! Large-scale smoothing window
    integer                  :: w_smooth ! Smoothing window above 7km (points)
    integer                  :: w_low    ! Smoothing window below 7km (points)
    real(wp)                 :: hmax     ! Maximum height for WO processing
    character(len=*)         :: filter   ! Filter method ('optest','slpoly')
    logical                  :: opt_DL2  ! Degraded L2 flag
    integer                  :: cff      ! Complex filtering flag
    real(wp)                 :: dsh      ! Shadow border width (m)
    real(wp), dimension(:,:) :: impact   ! L1,L2 impact parameters (m) [ch,t]
    real(wp), dimension(:,:) :: bangle   ! L1,L2 bending angles (rad) [ch,t]
\end{Verbatim}
\textbf{OUTPUT:}\hspace{0.08in}\begin{Verbatim}
    real(wp), dimension(:,:) :: impact   ! L1,L2 impact parameters (m) [ch, t]
    real(wp), dimension(:,:) :: bangle   ! L1,L2 bending angles (rad) [ch, ip]
    real(wp), dimension(:,:) :: ba_cov   ! Estimate of bangle covariance
    type(PPdiag),   optional :: diag     ! Additional output diagnostics structure
\end{Verbatim}
\textbf{NOTES:}\hspace{0.08in}\begin{Verbatim}
   Impact parameters are calculated with respect to local centre of curvature
   Bending angles are calculated from Doppler shift in the approximation of
   local spherical symmetry.
   Variable names follow Gorbunov and Lauritsen (2004)
                   P - impact parameter
                   E - bending angle (epsilon)
                   Y - new coordinate
\end{Verbatim}
\textbf{REFERENCES:}\hspace{0.08in}\begin{Verbatim}
   Gorbunov M.E. and Lauritsen K.B. 2004
   Analysis of wave fields by Fourier integral operators and their application
   for radio occultations
   Radio Science (39) RS4010

   Gorbunov M.E., Lauritsen K.B., Rhodin A., Tomassini M. and Kornblueh L.
   Radio holographic filtering, error estimation, and quality control of
   radio occultation data
   Journal of Geophysical Research (111) D10105

   Gorbunov M.E., Lauritsen K.B., Rodin A., Tomassini M., Kornblueh L. 2005
   Analysis of the CHAMP experimental data on radio-occultation sounding of
   the Earth's atmosphere.
   Izvestiya Atmospheric and Oceanic Physics (41) 726-740.
\end{Verbatim}
