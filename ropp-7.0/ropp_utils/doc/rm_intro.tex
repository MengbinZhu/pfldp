%%%%%%%%%%%%%%%%%%%%%%%%%%%%%%%%%%%%%%%%%%%%%%%%%%%%%%%%%%%%%%%%%%%%%%%%%%%%%%%%
%                                                                              %
% ROPP Utils Reference Manual: Introduction                                    %
%                                                                              %
% ROM SAF, Met Office, Exeter                                                  %
%                                                                              %
% Version 1.2                                                                  %
%                                                                              %
%%%%%%%%%%%%%%%%%%%%%%%%%%%%%%%%%%%%%%%%%%%%%%%%%%%%%%%%%%%%%%%%%%%%%%%%%%%%%%%%

\chapter{Introduction}\label{chap:introduction}

\section{Purpose of this document}

This document provides a Reference Manual for the Utils
library which is part of the Radio Occultation Processing Package
(ROPP). ROPP's aim is
%
\begin{quote}
  \emph{\ldots{}to provide users with a comprehensive software
    package, containing all necessary functions to pre--process RO
    data from CGS Level 1b files or ROM SAF Level 2 files, plus
    RO--specific components to assist with the assimilation of these
    data in NWP systems.}
\end{quote}
%
ROPP is a collection of software (provided as source code), supporting
data files and documentation, which aids users wishing to assimilate
radio occultation data into their NWP models. As far as is practical,
the software will be generic, in that it can handle any GPS--LEO
configuration radio occultation mission (GRAS, COSMIC, CHAMP, GRACE-A, 
TerraSAR-X, TanDEM-X, C/NOFS, SAC-C, ROSA, PAZ, etc). 

ROPP can generate bending angles and refractivites from excess phase 
measurements, forward model background fields to observations, 
and perform 1DVAR retrievals. 

The \texttt{ropp\_utils} library provides low-level, general-purpose,
routines used by other ROPP modules and application-level tools.
These routines are not intended to be called directly by user
applications, but they are available for this purpose if the user
wishes to maintain consistency with ROPP applications.
This document describes the Application Programming Interface  of
\texttt{ropp\_utils} in detail.

\input{romsaf_acronyms}

\section{Structure of this document}

This document is organised as follows: Chapter~2 provides an
alphabetical reference manual to the user-visible components of the
\texttt{ropp\_utils} library. As the software partially consists of
software developed outside of the ROM SAF,
Chapter~\ref{chap:copyrights} informs about the various copyrights.
Note that a functional overview of the software is given in the ROPP
Overview \citep{romsaf_ov}, while detailed build and install instructions 
are contained in the Release Notes \citep{romsaf_srn}. Testing
procedures the software has undergone are described in a separate Test
Plan Document \citep{romsaf_testplan} and Test Folder Report 
\citep{romsaf_testrep}.


\bibliographystyle{romsaf}
\bibliography{meteorology,rom_saf}

%%% Local Variables:
%%% mode: latex
%%% TeX-master: "romsaf_ropp_utils_rm"
%%% End:

