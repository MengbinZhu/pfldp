% Document: ./rm_utils
% Source: ./mywork/
% Generated with ROBODoc Version 4.99.36 (Jul 28 2009)
\section{Arrays/arrays}
\textsl{[ Modules ]}

\label{ch:robo0}
\label{ch:Arrays_arrays}
\index{unsorted!arrays}\index{Modules!arrays}
\textbf{NAME:}\hspace{0.08in}\begin{Verbatim}
    arrays - Tools for handling arrays in Fortran 90.
\end{Verbatim}
\textbf{SYNOPSIS:}\hspace{0.08in}\begin{Verbatim}
    use arrays
\end{Verbatim}
\textbf{DESCRIPTION:}\hspace{0.08in}\begin{Verbatim}
    This module provides interfaces to several arrays related
    functions and subroutines within the tools90 library. This
    does not only include utilities to get indices of array
    elements fulfilling certain conditions, but also tools for
    sorting and unifying values of arrays, reversing fields
    along one of their coordinates, reallocation of fields,
    and simple mathematical operations like vector products
    or the calculation of the norm and unit vector from a given
    one dimensional vector.
\end{Verbatim}
\textbf{NOTES:}\hspace{0.08in}\begin{Verbatim}
    All of the routines are at least implemented for float and
    double, some also for single and double complex arguments
    as well as for integer and string arguments (if applicable).
    Routines returning different numbers of array elements or
    changing the size of arrays and fields implemented via
    pointers. This may cause significant performance problems
    with some compilers. Also make sure that functions returning
    pointers are not used with normal arrays, as this is likely
    to cause memory leaks.
\end{Verbatim}
\textbf{SEE ALSO:}\hspace{0.08in}\begin{Verbatim}
    Allocation:             copy_allocate, callocate
    Reallocation:           reallocate, preallocate
    Index functions:        where, nruns, getrun, uniq, unique, locate,
                              iminloc, imaxloc, setminus
    Sorting et al.:         sort, sorted, quick_sort
    Array manipulation:     reverse, blend, swap, isinrange
    Simple maths:           cross_product, outer_product, outer_and, 
                              l2norm, unit_vector
\end{Verbatim}
\section{Arrays/blend}
\textsl{[ Functions ]}

\label{ch:robo1}
\label{ch:Arrays_blend}
\index{unsorted!blend}\index{Functions!blend}
\textbf{NAME:}\hspace{0.08in}\begin{Verbatim}
    blend - Sets up weighting array for blending arrays together.
\end{Verbatim}
\textbf{SYNOPSIS:}\hspace{0.08in}\begin{Verbatim}
    w = blend(n, i, j)
\end{Verbatim}
\textbf{DESCRIPTION:}\hspace{0.08in}\begin{Verbatim}
    This function calculates cosine weights for blending two arrays.
\end{Verbatim}
\textbf{INPUTS:}\hspace{0.08in}\begin{Verbatim}
    n   size of array to make
    i   index of first value to blend
    j   index of last value to blend
\end{Verbatim}
\textbf{OUTPUT:}\hspace{0.08in}\begin{Verbatim}
    w   weigthing array (0.0 until i, 1.0 after j, cos-weight in between)
\end{Verbatim}
\textbf{NOTES:}\hspace{0.08in}\begin{Verbatim}
    To blend two arrays A and B:

        blended =  A * (1 - w) + B * w

    The result starts with pure A and ends with pure B.
\end{Verbatim}
\textbf{REFERENCES:}\hspace{0.08in}\begin{Verbatim}
    This function is reimplemented in Fortran 90 from the original
    IDL function blend.pro written by R. Sterner; this function is
    is part of the John Hopkins University / Applied Physics
    Laboratory (JHU/APL) IDL library.
\end{Verbatim}
\section{Arrays/callocate}
\textsl{[ Subroutines ]}

\label{ch:robo2}
\label{ch:Arrays_callocate}
\index{unsorted!callocate}\index{Subroutines!callocate}
\textbf{NAME:}\hspace{0.08in}\begin{Verbatim}
    callocate - Allocate arrays and initialise them with a given value.
\end{Verbatim}
\textbf{SYNOPSIS:}\hspace{0.08in}\begin{Verbatim}
    call callocate(array, newsize, [value])
\end{Verbatim}
\textbf{DESCRIPTION:}\hspace{0.08in}\begin{Verbatim}
    This subroutine allocates an array or field to the given size and
    initialises the array with a given value (or zero if no value is
    given).
\end{Verbatim}
\textbf{INPUTS:}\hspace{0.08in}\begin{Verbatim}
    ..., dim(:[,:[...]]), pointer :: array
    integer [, dim(:[...])]       :: newsize
\end{Verbatim}
\textbf{OUTPUT:}\hspace{0.08in}\begin{Verbatim}
    ..., dim(:[,:[...]]), pointer :: array
\end{Verbatim}
\textbf{NOTES:}\hspace{0.08in}\begin{Verbatim}
    This subroutine supports integer, float, double, complex,
    double complex arguments for up to seven dimensions.
\end{Verbatim}
\textbf{SEE ALSO:}\hspace{0.08in}\begin{Verbatim}
    reallocate
\end{Verbatim}
\section{Arrays/copy\_alloc}
\textsl{[ Subroutines ]}

\label{ch:robo3}
\label{ch:Arrays_copy_alloc}
\index{unsorted!copy\_alloc}\index{Subroutines!copy\_alloc}
\textbf{NAME:}\hspace{0.08in}\begin{Verbatim}
    copy_alloc - Copy data into a newly allocated array.
\end{Verbatim}
\textbf{SYNOPSIS:}\hspace{0.08in}\begin{Verbatim}
    call copy_alloc(array, newarray)
\end{Verbatim}
\textbf{DESCRIPTION:}\hspace{0.08in}\begin{Verbatim}
    This subroutine copies the values in array into the newly allocated
    newarray. Previous contents of newaray are lost.
\end{Verbatim}
\textbf{INPUTS:}\hspace{0.08in}\begin{Verbatim}
    ..., dim(:[,:[...]]) :: array
\end{Verbatim}
\textbf{OUTPUT:}\hspace{0.08in}\begin{Verbatim}
    ..., dim(:[,:[...]]), pointer :: newarray
\end{Verbatim}
\textbf{NOTES:}\hspace{0.08in}\begin{Verbatim}
    On exit, newarray will have the same shape as array.

    This subroutine supports integer, float, double, complex,
    double complex and character/string arguments for up to
    seven dimensions.
\end{Verbatim}
\section{Arrays/cross\_product}
\textsl{[ Functions ]}

\label{ch:robo4}
\label{ch:Arrays_cross_product}
\index{unsorted!cross\_product}\index{Functions!cross\_product}
\textbf{NAME:}\hspace{0.08in}\begin{Verbatim}
    cross_product - Calculate the cross product between two vectors.
\end{Verbatim}
\textbf{SYNOPSIS:}\hspace{0.08in}\begin{Verbatim}
    z = cross_product(x, y)
\end{Verbatim}
\textbf{DESCRIPTION:}\hspace{0.08in}\begin{Verbatim}
    This function returns the cross product between two 1D vectors.
\end{Verbatim}
\textbf{INPUTS:}\hspace{0.08in}\begin{Verbatim}
    real(wp), dimension(3) :: x, y
\end{Verbatim}
\textbf{OUTPUT:}\hspace{0.08in}\begin{Verbatim}
    real(wp), dimension(3) :: z
\end{Verbatim}
\textbf{NOTES:}\hspace{0.08in}\begin{Verbatim}
    This function is obviously only useful for vectors from R^3.
    Both float and double arguments are supported.
\end{Verbatim}
\textbf{EXAMPLE:}\hspace{0.08in}\begin{Verbatim}
    To calculate the cross product between vectors representing
    the x- and y-coordinate axis, try

      use arrays
        ...
      real(wp), dimension (3) :: x, y, z
        ...
      x = (/ 1., 0., 0./)
      y = (/ 0., 1., 0./)
        ...
      z = cross_product(x, y)
\end{Verbatim}
\textbf{SEE ALSO:}\hspace{0.08in}\begin{Verbatim}
    outer_product
\end{Verbatim}
\section{Arrays/getrun}
\textsl{[ Functions ]}

\label{ch:robo5}
\label{ch:Arrays_getrun}
\index{unsorted!getrun}\index{Functions!getrun}
\textbf{NAME:}\hspace{0.08in}\begin{Verbatim}
    getrun - Gives N'th run of consecutive integers from a given array.
\end{Verbatim}
\textbf{SYNOPSIS:}\hspace{0.08in}\begin{Verbatim}
    run => getrun(iarray, n, m, longest)
\end{Verbatim}
\textbf{DESCRIPTION:}\hspace{0.08in}\begin{Verbatim}
    This function returns the N'th run of consecutive integers in a 
    given (integer) array. This function is meant to be used with
    arrays obtained from the where function.
\end{Verbatim}
\textbf{INPUTS:}\hspace{0.08in}\begin{Verbatim}
    int     iarray(:)    Integer array to process.
    int     n            Run number to get (first = 1 = def).
    int     m            Optional last run number to get.
    logical last         If .true., n is offset from last run.
    logical longest      Returns the longest run.
\end{Verbatim}
\textbf{OUTPUT:}\hspace{0.08in}\begin{Verbatim}
    run
\end{Verbatim}
\textbf{NOTES:}\hspace{0.08in}\begin{Verbatim}
    getrun is meant to be used on the output from where. If m > n,
    run will be a set of runs from run n to run m.  If no m is given,
    run will be a single run. If n<0, the routine returns runs starting
    at run abs(n) to end of iarray. If n is out of range, a -1 is
    returned.

    last = .true. means that n is an offset from last run. So n = -1
    gives the last run, n = -2 gives next to last, ... If n = -3 and
    m = -1, the last 3 runs are returned.

    longest = .true. returns the longest run. n and m are both ignored.
\end{Verbatim}
\textbf{SEE ALSO:}\hspace{0.08in}\begin{Verbatim}
    nruns
\end{Verbatim}
\textbf{REFERENCES:}\hspace{0.08in}\begin{Verbatim}
    This function is reimplemented in Fortran 90 from the original
    IDL function getrun.pro written by R. Sterner; this function is
    is part of the John Hopkins University / Applied Physics
    Laboratory (JHU/APL) IDL library.
\end{Verbatim}
\section{Arrays/imaxloc}
\textsl{[ Functions ]}

\label{ch:robo6}
\label{ch:Arrays_imaxloc}
\index{unsorted!imaxloc}\index{Functions!imaxloc}
\textbf{NAME:}\hspace{0.08in}\begin{Verbatim}
    imaxloc - Index of first maximum value.
\end{Verbatim}
\textbf{SYNOPSIS:}\hspace{0.08in}\begin{Verbatim}
    idx = imaxloc(array)
\end{Verbatim}
\textbf{DESCRIPTION:}\hspace{0.08in}\begin{Verbatim}
    This function returns the index of the first maximum value.
\end{Verbatim}
\textbf{INPUTS:}\hspace{0.08in}\begin{Verbatim}
    array - 1D array of type real, double, or integer.
\end{Verbatim}
\textbf{OUTPUT:}\hspace{0.08in}\begin{Verbatim}
    idx   - index position of the first maximum in array.
\end{Verbatim}
\textbf{NOTES:}\hspace{0.08in}\begin{Verbatim}
    The routine is based on the Fortran 90 intrinsic function maxloc;
    it just returns a scalar index for 1 dimensional arrays instead of
    a rank one array.
\end{Verbatim}
\textbf{SEE ALSO:}\hspace{0.08in}\begin{Verbatim}
    iminloc
\end{Verbatim}
\section{Arrays/iminloc}
\textsl{[ Functions ]}

\label{ch:robo7}
\label{ch:Arrays_iminloc}
\index{unsorted!iminloc}\index{Functions!iminloc}
\textbf{NAME:}\hspace{0.08in}\begin{Verbatim}
    iminloc - Index of first minimum value.
\end{Verbatim}
\textbf{SYNOPSIS:}\hspace{0.08in}\begin{Verbatim}
    idx = iminloc(array)
\end{Verbatim}
\textbf{DESCRIPTION:}\hspace{0.08in}\begin{Verbatim}
    This function returns the index of the first minimum value.
\end{Verbatim}
\textbf{INPUTS:}\hspace{0.08in}\begin{Verbatim}
    array - 1D array of type real, double, or integer.
\end{Verbatim}
\textbf{OUTPUT:}\hspace{0.08in}\begin{Verbatim}
    idx   - index position of the first maximum in array.
\end{Verbatim}
\textbf{NOTES:}\hspace{0.08in}\begin{Verbatim}
    The routine is based on the Fortran 90 intrinsic function minloc;
    it just returns a scalar index for 1 dimensional arrays instead of
    a rank one array.
\end{Verbatim}
\textbf{SEE ALSO:}\hspace{0.08in}\begin{Verbatim}
    iminloc
\end{Verbatim}
\section{Arrays/isinrange}
\textsl{[ Subroutines ]}

\label{ch:robo8}
\label{ch:Arrays_isinrange}
\index{unsorted!isinrange}\index{Subroutines!isinrange}
\textbf{NAME:}\hspace{0.08in}\begin{Verbatim}
    isinrange - Check if data values are within a given range of numbers.
\end{Verbatim}
\textbf{SYNOPSIS:}\hspace{0.08in}\begin{Verbatim}
    ... = isinrange(data, range)
\end{Verbatim}
\textbf{DESCRIPTION:}\hspace{0.08in}\begin{Verbatim}
    This function returns .true. if all elements of an array are within
    a given range, .false. if at least one element is outside the given
    range.
\end{Verbatim}
\textbf{INPUTS:}\hspace{0.08in}\begin{Verbatim}
    ..., dim(:[,:[...]]) :: data
    ..., dimension(2)    :: range
\end{Verbatim}
\textbf{OUTPUT:}\hspace{0.08in}\begin{Verbatim}
    logical              :: isinrange
\end{Verbatim}
\textbf{NOTES:}\hspace{0.08in}\begin{Verbatim}
    The array range is a two element vector giving the minimum and
    maximum value for the allowed (valid) range.

    This subroutine supports integer, float and double arguments for
    up to seven dimensions. Scalar data values can also be tested.
\end{Verbatim}
\section{Arrays/l2norm}
\textsl{[ Functions ]}

\label{ch:robo9}
\label{ch:Arrays_l2norm}
\index{unsorted!l2norm}\index{Functions!l2norm}
\textbf{NAME:}\hspace{0.08in}\begin{Verbatim}
    l2norm - Calculate the L2 (Euclidian) norm of a vector.
\end{Verbatim}
\textbf{SYNOPSIS:}\hspace{0.08in}\begin{Verbatim}
    use arrays
      ...
    norm = l2norm(vector)
\end{Verbatim}
\textbf{DESCRIPTION:}\hspace{0.08in}\begin{Verbatim}
    This function computes the L2 (i.e., the usual Euclidian) norm of
    a 1D vector.
\end{Verbatim}
\textbf{INPUTS:}\hspace{0.08in}\begin{Verbatim}
    ..., dim(:) :: vector    1D array.

 RESULT
    ...         :: norm      Scalar value of the norm of vector.
\end{Verbatim}
\textbf{NOTES:}\hspace{0.08in}\begin{Verbatim}
    Vector can be a float, double, complex or double complex
    array, while the norm will always be real (float or double).
\end{Verbatim}
\textbf{EXAMPLE:}\hspace{0.08in}\begin{Verbatim}
    To calculate the norm of an arbitrary vector in physical space
    (like a coordinate vector), try

      use arrays
        ...
      real(wp), dimension(3) :: vec
      real                   :: length
        ...
      vec    = ...
      length = l2norm(vec)
\end{Verbatim}
\textbf{SEE ALSO:}\hspace{0.08in}\begin{Verbatim}
      unit_vector
\end{Verbatim}
\section{Arrays/locate}
\textsl{[ Functions ]}

\label{ch:robo10}
\label{ch:Arrays_locate}
\index{unsorted!locate}\index{Functions!locate}
\textbf{NAME:}\hspace{0.08in}\begin{Verbatim}
    locate - locate where a (list of) number(s) can be found in an array.
\end{Verbatim}
\textbf{SYNOPSIS:}\hspace{0.08in}\begin{Verbatim}
    index = locate(array, point(s))
\end{Verbatim}
\textbf{DESCRIPTION:}\hspace{0.08in}\begin{Verbatim}
    This function returns the index or indices of point(s) within an array.
    If the point is outside the array bounds, either 0 or n + 1 will be
    returned, n being the number of elements in array. If the point is
    within the range spanned by the array, the returned index will give the
    position of the sought for point such that

       array(i) <= point <= array(i+1)
\end{Verbatim}
\textbf{INPUTS:}\hspace{0.08in}\begin{Verbatim}
    ...,    :: array(:)
    ...,    :: points(:)
\end{Verbatim}
\textbf{OUTPUT:}\hspace{0.08in}\begin{Verbatim}
    integer :: index(:)
\end{Verbatim}
\textbf{NOTES:}\hspace{0.08in}\begin{Verbatim}
    The data array must be either monotonically increasing or decreasing
    for the algorithm to work.

    The index array must have the same number of elements as the point(s)
    array.
\end{Verbatim}
\textbf{REFERENCES:}\hspace{0.08in}\begin{Verbatim}
    This routine has been copied (by hand) and slightly edited from the
    fgauss routine found in

       Press, W.H, Teukolsky, S.A. Vetterling, W.T. and Flannery, B.P.,
          Numerical Recipes in Fortran 90, Cambridge University Press, 
          Cambridge, 1999.
\end{Verbatim}
\section{Arrays/nruns}
\textsl{[ Functions ]}

\label{ch:robo11}
\label{ch:Arrays_nruns}
\index{unsorted!nruns}\index{Functions!nruns}
\textbf{NAME:}\hspace{0.08in}\begin{Verbatim}
    nruns - Gives # of runs of consecutive integers in an array.
\end{Verbatim}
\textbf{SYNOPSIS:}\hspace{0.08in}\begin{Verbatim}
    n_runs = nruns(iarray)
\end{Verbatim}
\textbf{DESCRIPTION:}\hspace{0.08in}\begin{Verbatim}
    This function returns the number of runs of consecutive integers
    in a given (integer) array. This routine is meant to be used with
    arrays obtained from the where function.
\end{Verbatim}
\textbf{INPUTS:}\hspace{0.08in}\begin{Verbatim}
    int :: iarray(:)
\end{Verbatim}
\textbf{OUTPUT:}\hspace{0.08in}\begin{Verbatim}
    int :: n_runs
\end{Verbatim}
\textbf{SEE ALSO:}\hspace{0.08in}\begin{Verbatim}
    getrun
\end{Verbatim}
\textbf{REFERENCES:}\hspace{0.08in}\begin{Verbatim}
    This function is reimplemented in Fortran 90 from the original
    IDL function nruns.pro written by R. Sterner; this function is
    is part of the John Hopkins University / Applied Physics
    Laboratory (JHU/APL) IDL library.
\end{Verbatim}
\section{Arrays/outer\_and}
\textsl{[ Functions ]}

\label{ch:robo12}
\label{ch:Arrays_outer_and}
\index{unsorted!outer\_and}\index{Functions!outer\_and}
\textbf{NAME:}\hspace{0.08in}\begin{Verbatim}
    outer_product - Calculate the outer logical and of two vectors.
\end{Verbatim}
\textbf{SYNOPSIS:}\hspace{0.08in}\begin{Verbatim}
    z = outer_and(x, y)
\end{Verbatim}
\textbf{DESCRIPTION:}\hspace{0.08in}\begin{Verbatim}
    This function returns the outer (dyadic) and between two
    logical 1D vectors.
\end{Verbatim}
\textbf{INPUTS:}\hspace{0.08in}\begin{Verbatim}
    logical, dimension(:)   :: x, y
\end{Verbatim}
\textbf{OUTPUT:}\hspace{0.08in}\begin{Verbatim}
    logical, dimension(:,:) :: z
\end{Verbatim}
\textbf{NOTES:}\hspace{0.08in}\begin{Verbatim}
    The result of an outer and of two 1d vectors is matrix!
\end{Verbatim}
\textbf{SEE ALSO:}\hspace{0.08in}\begin{Verbatim}
      outer_product
\end{Verbatim}
\section{Arrays/outer\_product}
\textsl{[ Functions ]}

\label{ch:robo13}
\label{ch:Arrays_outer_product}
\index{unsorted!outer\_product}\index{Functions!outer\_product}
\textbf{NAME:}\hspace{0.08in}\begin{Verbatim}
    outer_product - Calculate the outer product of two vectors.
\end{Verbatim}
\textbf{SYNOPSIS:}\hspace{0.08in}\begin{Verbatim}
    z = outer_product(x, y)
\end{Verbatim}
\textbf{DESCRIPTION:}\hspace{0.08in}\begin{Verbatim}
    This function returns the outer (dyadic) product between two
    1D vectors.
\end{Verbatim}
\textbf{INPUTS:}\hspace{0.08in}\begin{Verbatim}
    real(wp), dimension(:)   :: x, y
\end{Verbatim}
\textbf{OUTPUT:}\hspace{0.08in}\begin{Verbatim}
    real(wp), dimension(:,:) :: z
\end{Verbatim}
\textbf{NOTES:}\hspace{0.08in}\begin{Verbatim}
    The result of an outer product of two 1d vectors is matrix!
    Both float and double arguments are supported.
\end{Verbatim}
\textbf{EXAMPLE:}\hspace{0.08in}\begin{Verbatim}
    To calculate the dyadic (outer) product of vectors representing
    the x- and y-coordinate axis, try

      use arrays
        ...
      real(wp), dimension(3)   :: x, y
      real(wp), dimension(3,3) :: A
        ...
      x = (/ 1., 0., 0./)
      y = (/ 0., 1., 0./)
        ...
      A = outer_product(x, y)
\end{Verbatim}
\textbf{SEE ALSO:}\hspace{0.08in}\begin{Verbatim}
      cross_product
\end{Verbatim}
\section{Arrays/preallocate}
\textsl{[ Functions ]}

\label{ch:robo14}
\label{ch:Arrays_preallocate}
\index{unsorted!preallocate}\index{Functions!preallocate}
\textbf{NAME:}\hspace{0.08in}\begin{Verbatim}
    preallocate - Reallocate 1d arrays (function).
\end{Verbatim}
\textbf{SYNOPSIS:}\hspace{0.08in}\begin{Verbatim}
    use arrays
      ...
    array => preallocate(array, newsize)
\end{Verbatim}
\textbf{DESCRIPTION:}\hspace{0.08in}\begin{Verbatim}
    This subroutine reallocates an array or field to a new size and
    rescues the previous contents if possible (i.e., the array
    is enlarged, the old values will be available in the first
    elements of the new, enlarged array. If an array is shrinked,
    superfluous elements will be lost. If an array is enlarged,
    the additional elements are undetermined.
\end{Verbatim}
\textbf{INPUTS:}\hspace{0.08in}\begin{Verbatim}
    ..., dim(:[,:[...]]), pointer :: array
    integer [, dim(:[...])]       :: newsize
\end{Verbatim}
\textbf{OUTPUT:}\hspace{0.08in}\begin{Verbatim}
    ..., dim(:[,:[...]]), pointer :: array
\end{Verbatim}
\textbf{NOTES:}\hspace{0.08in}\begin{Verbatim}
    If an array is shrinked, superfluous elements will be lost.
    If an array is enlarged, the compiler will fill in whatever
    is available into the additional elements; what is filled in
    depends solely on the state of the heap and the compiler's
    internals and cannot be relied on.
    If the pointer receiving the reallocated arrays is different from
    from the argument given to the function, preallocate will allocate
    a new pointer, copy newsize elements from the old pointer to the
    new one and destroy the old pointer. Any previous contents of the
    receiving pointer will be lost.
    Make sure to use pointer notation for receiving the reallocated
    pointer array, or a memory leak and performance reduction (due to
    copying the old data twice) will almost certainly occure.
    This function supports integer, float, double, complex,
    double complex and character/string arguments for up to
    seven dimensions.
\end{Verbatim}
\textbf{EXAMPLE:}\hspace{0.08in}\begin{Verbatim}
    To enlarge a 1d vector with m elements to n elements (with n > m),
    try

      use arrays
         ...
      real, dimension(:), pointer :: vector
      integer                     :: m, n
         ...
      allocate(vector(m))
         ...
      vector => preallocate(vector, n)
\end{Verbatim}
\textbf{SEE ALSO:}\hspace{0.08in}\begin{Verbatim}
    reallocate
\end{Verbatim}
\section{Arrays/quick\_sort}
\textsl{[ Subroutines ]}

\label{ch:robo15}
\label{ch:Arrays_quick_sort}
\index{unsorted!quick\_sort}\index{Subroutines!quick\_sort}
\textbf{NAME:}\hspace{0.08in}\begin{Verbatim}
    quick_sort - Sort an array. 
\end{Verbatim}
\textbf{SYNOPSIS:}\hspace{0.08in}\begin{Verbatim}
    call quick_sort(array, idx)
\end{Verbatim}
\textbf{DESCRIPTION:}\hspace{0.08in}\begin{Verbatim}
    This function sorts an array.
\end{Verbatim}
\textbf{INPUTS:}\hspace{0.08in}\begin{Verbatim}
    ...     :: array(:)  Array to be sorted; will be overwritten.
\end{Verbatim}
\textbf{OUTPUT:}\hspace{0.08in}\begin{Verbatim}
    ...     :: array(:)  Sorted array.
    int     :: idx(:)    
\end{Verbatim}
\textbf{NOTES:}\hspace{0.08in}\begin{Verbatim}
    The routine implements the quick_sort algorithm; the sort order is
    ascending. array and idx have the same length.

    This function supports and single and double precison arguments.
\end{Verbatim}
\textbf{SEE ALSO:}\hspace{0.08in}\begin{Verbatim}
    sort
    sorted
\end{Verbatim}
\textbf{REFERENCES:}\hspace{0.08in}\begin{Verbatim}
    The quick sort routine originates from:

       Brainerd, W.S., Goldberg, C.H. & Adams, J.C. (1990) Programmers
          Guide to Fortran 90, McGraw-Hill  ISBN 0-07-000248-7, pages
          149-150.

    and was modified by Alan Miller to include an associated integer array
    which gives the positions of the elements in the original order.
\end{Verbatim}
\section{Arrays/reallocate}
\textsl{[ Functions ]}

\label{ch:robo16}
\label{ch:Arrays_reallocate}
\index{unsorted!reallocate}\index{Functions!reallocate}
\textbf{NAME:}\hspace{0.08in}\begin{Verbatim}
    reallocate - Reallocate 1d arrays (subroutine).
\end{Verbatim}
\textbf{SYNOPSIS:}\hspace{0.08in}\begin{Verbatim}
    use arrays
      ...
    call reallocate(array, newsize)
\end{Verbatim}
\textbf{DESCRIPTION:}\hspace{0.08in}\begin{Verbatim}
    This subroutine reallocates an array or field to a new size and
    rescues the previous contents if possible (i.e., the array
    is enlarged, the old values will be available in the first
    elements of the new, enlarged array. If an array is shrinked,
    superfluous elements will be lost. If an array is enlarged,
    the additional elements are undetermined.
\end{Verbatim}
\textbf{INPUTS:}\hspace{0.08in}\begin{Verbatim}
    ..., dim(:[,:[...]]), pointer :: array
    integer [, dim(:[...])]       :: newsize
\end{Verbatim}
\textbf{OUTPUT:}\hspace{0.08in}\begin{Verbatim}
    ..., dim(:[,:[...]]), pointer :: array
\end{Verbatim}
\textbf{NOTES:}\hspace{0.08in}\begin{Verbatim}
    If an array is shrinked, superfluous elements will be lost.
    If an array is enlarged, the compiler will fill in whatever
    is available into the additional elements; what is filled in
    depends solely on the state of the heap and the compiler's
    internals and cannot be relied on.
    This subroutine supports integer, float, double, complex,
    double complex and character/string arguments for up to
    seven dimensions.
\end{Verbatim}
\textbf{EXAMPLE:}\hspace{0.08in}\begin{Verbatim}
    To enlarge a 1d vector with m elements to n elements (with n > m),
    try

      use arrays
         ...
      real, dimension(:), pointer :: vector
      integer                     :: m, n
         ...
      allocate(vector(m))
         ...
      call reallocate(vector, n)
\end{Verbatim}
\textbf{SEE ALSO:}\hspace{0.08in}\begin{Verbatim}
    preallocate
\end{Verbatim}
\section{Arrays/reverse}
\textsl{[ Functions ]}

\label{ch:robo17}
\label{ch:Arrays_reverse}
\index{unsorted!reverse}\index{Functions!reverse}
\textbf{NAME:}\hspace{0.08in}\begin{Verbatim}
    reverse - Reverse an array along a given dimension.
\end{Verbatim}
\textbf{SYNOPSIS:}\hspace{0.08in}\begin{Verbatim}
    use arrays
      ...
    array = reverse(array [, dim])
\end{Verbatim}
\textbf{INPUTS:}\hspace{0.08in}\begin{Verbatim}
    array   array to be reversed.
    dim     dimension along which the array is to be reversed.
\end{Verbatim}
\textbf{OUTPUT:}\hspace{0.08in}\begin{Verbatim}
    array   reversed data.
\end{Verbatim}
\textbf{DESCRIPTION:}\hspace{0.08in}\begin{Verbatim}
    This subroutine reverses an array along a given dimension (or along
    the first dimension is dim is not given.
\end{Verbatim}
\textbf{EXAMPLE:}\hspace{0.08in}\begin{Verbatim}
    To reverse the order of a 1-dimensional array, use

       array = reverse(array)

    which is identical to

       array = array(size(array):1:-1)

    To invert a 2-dimensional array along its second dimension,

       array(:,:) = reverse(array, 2)
\end{Verbatim}
\section{Arrays/setminus}
\textsl{[ Functions ]}

\label{ch:robo18}
\label{ch:Arrays_setminus}
\index{unsorted!setminus}\index{Functions!setminus}
\textbf{NAME:}\hspace{0.08in}\begin{Verbatim}
    setminus - Eliminate elements from a set that are also in
               another set.
\end{Verbatim}
\textbf{SYNOPSIS:}\hspace{0.08in}\begin{Verbatim}
    c = setminus(a, b)
\end{Verbatim}
\textbf{DESCRIPTION:}\hspace{0.08in}\begin{Verbatim}
    This function removes all elements in a which are also part of b. It
    is intended to be used with index arrays.
\end{Verbatim}
\textbf{INPUTS:}\hspace{0.08in}\begin{Verbatim}
    a = set (array) to process.
    b = set (array) of elements to eliminate.
\end{Verbatim}
\textbf{OUTPUT:}\hspace{0.08in}\begin{Verbatim}
    c = resulting set (array).
\end{Verbatim}
\textbf{NOTES:}\hspace{0.08in}\begin{Verbatim}
    This function should work for all data types, however: an equality
    constraint for floating point data, regardless of single of double
    precision, is a somewhat suspect concept.
\end{Verbatim}
\textbf{REFERENCES:}\hspace{0.08in}\begin{Verbatim}
    This function is reimplemented in Fortran 90 from the original
    IDL function setminus.pro written by R. Sterner; this function is
    is part of the John Hopkins University / Applied Physics
    Laboratory (JHU/APL) IDL library.
\end{Verbatim}
\section{Arrays/sort}
\textsl{[ Functions ]}

\label{ch:robo19}
\label{ch:Arrays_sort}
\index{unsorted!sort}\index{Functions!sort}
\textbf{NAME:}\hspace{0.08in}\begin{Verbatim}
    sort - Indices for a sorted array. 
\end{Verbatim}
\textbf{SYNOPSIS:}\hspace{0.08in}\begin{Verbatim}
    idx = sort(array [, reverse = .true.|.false.])
\end{Verbatim}
\textbf{DESCRIPTION:}\hspace{0.08in}\begin{Verbatim}
    This function returns the indices of a sorted array.
\end{Verbatim}
\textbf{INPUTS:}\hspace{0.08in}\begin{Verbatim}
    ...     :: array(:)  Array to be sorted.
    logical :: reverse   If true, do an descending sort (default is to
                           sort the array with ascending values).
\end{Verbatim}
\textbf{OUTPUT:}\hspace{0.08in}\begin{Verbatim}
    int     :: idx(:)    Indices of array after sorting.
\end{Verbatim}
\textbf{NOTES:}\hspace{0.08in}\begin{Verbatim}
    The routine is based on the quick_sort algorithm; the default sort order
    is ascending. The returned index array has the same length as the array
    to be sorted.

    This function supports and single and double precison arguments.
\end{Verbatim}
\textbf{SEE ALSO:}\hspace{0.08in}\begin{Verbatim}
    sorted
    quick_sort
\end{Verbatim}
\textbf{REFERENCES:}\hspace{0.08in}\begin{Verbatim}
    The quick sort routine originates from:

       Brainerd, W.S., Goldberg, C.H. & Adams, J.C. (1990) Programmers
          Guide to Fortran 90, McGraw-Hill  ISBN 0-07-000248-7, pages
          149-150.

    and was modified by Alan Miller to include an associated integer array
    which gives the positions of the elements in the original order.
\end{Verbatim}
\section{Arrays/sorted}
\textsl{[ Functions ]}

\label{ch:robo20}
\label{ch:Arrays_sorted}
\index{unsorted!sorted}\index{Functions!sorted}
\textbf{NAME:}\hspace{0.08in}\begin{Verbatim}
    sorted - Sort an array. 
\end{Verbatim}
\textbf{SYNOPSIS:}\hspace{0.08in}\begin{Verbatim}
    array = sort(array [, reverse = .true.|.false.])
\end{Verbatim}
\textbf{DESCRIPTION:}\hspace{0.08in}\begin{Verbatim}
    This function returns a sorted array.
\end{Verbatim}
\textbf{INPUTS:}\hspace{0.08in}\begin{Verbatim}
    ...     :: array(:)  Array to be sorted.
    logical :: reverse   If true, do an descending sort (default is to
                           sort the array with ascending values).
\end{Verbatim}
\textbf{OUTPUT:}\hspace{0.08in}\begin{Verbatim}
    ...     :: array(:)  Sorted array.
\end{Verbatim}
\textbf{NOTES:}\hspace{0.08in}\begin{Verbatim}
    The routine is based on the quick_sort algorithm; the default sort order
    is ascending. The returned array has the same length as the array to be
    to be sorted.

    This function supports and single and double precison arguments.
\end{Verbatim}
\textbf{SEE ALSO:}\hspace{0.08in}\begin{Verbatim}
    sort
    quick_sort
\end{Verbatim}
\textbf{REFERENCES:}\hspace{0.08in}\begin{Verbatim}
    The quick sort routine originates from:

       Brainerd, W.S., Goldberg, C.H. & Adams, J.C. (1990) Programmers
          Guide to Fortran 90, McGraw-Hill  ISBN 0-07-000248-7, pages
          149-150.

    and was modified by Alan Miller to include an associated integer array
    which gives the positions of the elements in the original order.
\end{Verbatim}
\section{Arrays/swap}
\textsl{[ Subroutines ]}

\label{ch:robo21}
\label{ch:Arrays_swap}
\index{unsorted!swap}\index{Subroutines!swap}
\textbf{NAME:}\hspace{0.08in}\begin{Verbatim}
    swap - Swap the two arguments.
\end{Verbatim}
\textbf{SYNOPSIS:}\hspace{0.08in}\begin{Verbatim}
    call swap(a, b)
\end{Verbatim}
\textbf{DESCRIPTION:}\hspace{0.08in}\begin{Verbatim}
    This subroutine swaps the content of the two arguments. This is an
    elemental subroutine.
\end{Verbatim}
\textbf{INPUTS:}\hspace{0.08in}\begin{Verbatim}
    a   -  First argument, to be swapped with...
    b   -  ... second argument.
\end{Verbatim}
\textbf{OUTPUT:}\hspace{0.08in}\begin{Verbatim}
    a   -  is now the former b.
    b   -  is now the former a.
\end{Verbatim}
\section{Arrays/uniq}
\textsl{[ Functions ]}

\label{ch:robo22}
\label{ch:Arrays_uniq}
\index{unsorted!uniq}\index{Functions!uniq}
\textbf{NAME:}\hspace{0.08in}\begin{Verbatim}
    uniq - Return the subscripts of the unique elements in an array.
\end{Verbatim}
\textbf{SYNOPSIS:}\hspace{0.08in}\begin{Verbatim}
    idx => uniq(array [, index])
\end{Verbatim}
\textbf{DESCRIPTION:}\hspace{0.08in}\begin{Verbatim}
    This function returns the subscripts of the unique elements in an
    array. Note that repeated elements must be adjacent in order to be
    found.  This routine is intended to be used with the sort function;
    see the discussion of the index argument below.
\end{Verbatim}
\textbf{INPUTS:}\hspace{0.08in}\begin{Verbatim}
    ...          :: array(:)   The array to be scanned.
    int          :: index(:)   An array of indices into array that order
                                 the elements into monotonic order.
\end{Verbatim}
\textbf{OUTPUT:}\hspace{0.08in}\begin{Verbatim}
    ..., pointer :: idx(:)     An array of indicies into array.
\end{Verbatim}
\textbf{SEE ALSO:}\hspace{0.08in}\begin{Verbatim}
    unique
\end{Verbatim}
\textbf{REFERENCES:}\hspace{0.08in}\begin{Verbatim}
    This is a Fortran 90 implementation of IDL's uniq() function.
\end{Verbatim}
\section{Arrays/unique}
\textsl{[ Functions ]}

\label{ch:robo23}
\label{ch:Arrays_unique}
\index{unsorted!unique}\index{Functions!unique}
\textbf{NAME:}\hspace{0.08in}\begin{Verbatim}
    uniq - Return the unique elements in an array.
\end{Verbatim}
\textbf{SYNOPSIS:}\hspace{0.08in}\begin{Verbatim}
    array => uniq(array [, index])
\end{Verbatim}
\textbf{DESCRIPTION:}\hspace{0.08in}\begin{Verbatim}
    This function returns the unique elements in an array. Note that
    repeated elements must be adjacent in order to be found.  This
    routine is intended to be used with the sort function; see the
    discussion of the index argument below.
\end{Verbatim}
\textbf{INPUTS:}\hspace{0.08in}\begin{Verbatim}
    ...          :: array(:)   The array to be scanned.
    int          :: index(:)   An array of indices into array that order
                                 the elements into monotonic order.
\end{Verbatim}
\textbf{OUTPUT:}\hspace{0.08in}\begin{Verbatim}
    ..., pointer :: uarray(:)  An array with the unique elements of array.
\end{Verbatim}
\textbf{SEE ALSO:}\hspace{0.08in}\begin{Verbatim}
    uniq
\end{Verbatim}
\textbf{REFERENCES:}\hspace{0.08in}\begin{Verbatim}
    This function was inspired by the unique() function in Ray Sterner's 
    JHUA library.
\end{Verbatim}
\section{Arrays/unit\_vector}
\textsl{[ Functions ]}

\label{ch:robo24}
\label{ch:Arrays_unit_vector}
\index{unsorted!unit\_vector}\index{Functions!unit\_vector}
\textbf{NAME:}\hspace{0.08in}\begin{Verbatim}
    unit_vector
\end{Verbatim}
\textbf{SYNOPSIS:}\hspace{0.08in}\begin{Verbatim}
    uvector = unit_vector(vector)
\end{Verbatim}
\textbf{DESCRIPTION:}\hspace{0.08in}\begin{Verbatim}
    This function calculates the unit vector of a 1D vector.
\end{Verbatim}
\textbf{INPUTS:}\hspace{0.08in}\begin{Verbatim}
    real, dim(:) :: vector   1D array to be normalized.

 RESULT
    real, dim(:) :: uvector  Normalized version of vector, i.e.
                               vector divided by its norm.
\end{Verbatim}
\textbf{NOTES:}\hspace{0.08in}\begin{Verbatim}
    Vector can be a float, double, complex or double complex
    array; the calculated unit vector will be of the same type.
\end{Verbatim}
\textbf{EXAMPLE:}\hspace{0.08in}\begin{Verbatim}
    To calculate a unit vector from an arbitray vector in physical
    space (like a coordinate vector), try

      [use tools90|use arrays]
        ...
      real(wp), dimension(3) :: vec, uvec
        ...
      vec  = ...
      uvec = unit_vector(vec)
\end{Verbatim}
\textbf{SEE ALSO:}\hspace{0.08in}\begin{Verbatim}
      l2norm
\end{Verbatim}
\section{Arrays/where}
\textsl{[ Functions ]}

\label{ch:robo25}
\label{ch:Arrays_where}
\index{unsorted!where}\index{Functions!where}
\textbf{NAME:}\hspace{0.08in}\begin{Verbatim}
    where - Return the subscripts for which a condition is fullfilled.
\end{Verbatim}
\textbf{SYNOPSIS:}\hspace{0.08in}\begin{Verbatim}
    idx => where(mask [, n])
\end{Verbatim}
\textbf{DESCRIPTION:}\hspace{0.08in}\begin{Verbatim}
    This function returns the subscripts of the true elements of mask.
\end{Verbatim}
\textbf{INPUTS:}\hspace{0.08in}\begin{Verbatim}
    logical      :: mask(:)   The array to be scanned.
\end{Verbatim}
\textbf{OUTPUT:}\hspace{0.08in}\begin{Verbatim}
    int, pointer :: idx(:)    An array of indices where mask == .true.
\end{Verbatim}
\textbf{REFERENCES:}\hspace{0.08in}\begin{Verbatim}
    This is a Fortran 90 implementation of IDL's where() function.
\end{Verbatim}
\section{Common/file\_delete}
\textsl{[ Subroutines ]}

\label{ch:robo26}
\label{ch:Common_file_delete}
\index{unsorted!file\_delete}\index{Subroutines!file\_delete}
\textbf{NAME:}\hspace{0.08in}\begin{Verbatim}
       file_delete - delete a file if it exists
\end{Verbatim}
\textbf{SYNOPSIS:}\hspace{0.08in}\begin{Verbatim}
       use ropp_utils
         ...
       call file_delete(file, ierr)

 INPUT
    file - filename to be deleted
\end{Verbatim}
\textbf{OUTPUT:}\hspace{0.08in}\begin{Verbatim}
    ierr - error message with status
\end{Verbatim}
\textbf{NOTES:}\hspace{0.08in}\begin{Verbatim}
    Output status flag depending on outcome
    -2 : task failed 
    -1 : file does not exist - task failed
     0 : task successful
\end{Verbatim}
\section{Common/get\_io\_unit}
\textsl{[ Subroutines ]}

\label{ch:robo27}
\label{ch:Common_get_io_unit}
\index{unsorted!get\_io\_unit}\index{Subroutines!get\_io\_unit}
\textbf{NAME:}\hspace{0.08in}\begin{Verbatim}
       get_io_unit - Obtain a free Fortran unit number
\end{Verbatim}
\textbf{SYNOPSIS:}\hspace{0.08in}\begin{Verbatim}
       use ropp_utils
        ...
       unit = get_io_unit()
\end{Verbatim}
\textbf{DESCRIPTION:}\hspace{0.08in}\begin{Verbatim}
       This function returns the first available unit number larger than 10.
       BASED ON MARQUARDT COLLECTION FILES LIBRARY ROUTINE
\end{Verbatim}
\section{Common/ropp\_utils\_version}
\textsl{[ Functions ]}

\label{ch:robo28}
\label{ch:Common_ropp_utils_version}
\index{unsorted!ropp\_utils\_version}\index{Functions!ropp\_utils\_version}
\textbf{NAME:}\hspace{0.08in}\begin{Verbatim}
   ropp_utils_version
\end{Verbatim}
\textbf{SYNOPSIS:}\hspace{0.08in}\begin{Verbatim}
   Return ROPP_UTILS version ID string

   USE ropp_utils
   version = ropp_utils_version()
\end{Verbatim}
\textbf{DESCRIPTION:}\hspace{0.08in}\begin{Verbatim}
   This function returns the (common) version string for the ROPP_UTILS
   module. By default, this function should be called by all ROPP_UTILS
   tools to display a version ID when the '-v' command-line switch is
   used.
\end{Verbatim}
\section{Common/To\_Lower}
\textsl{[ Subroutines ]}

\label{ch:robo29}
\label{ch:Common_To_Lower}
\index{unsorted!To\_Lower}\index{Subroutines!To\_Lower}
\textbf{NAME:}\hspace{0.08in}\begin{Verbatim}
       To_Lower - Convert alpha characters in a string from upper to
                  lower case
\end{Verbatim}
\textbf{SYNOPSIS:}\hspace{0.08in}\begin{Verbatim}
       use ropp_utils
        ...
       call To_Lower( string )
\end{Verbatim}
\section{Common/To\_Upper}
\textsl{[ Subroutines ]}

\label{ch:robo30}
\label{ch:Common_To_Upper}
\index{unsorted!To\_Upper}\index{Subroutines!To\_Upper}
\textbf{NAME:}\hspace{0.08in}\begin{Verbatim}
       To_Upper - Convert alpha characters in a string from lower to
                  upper case
\end{Verbatim}
\textbf{SYNOPSIS:}\hspace{0.08in}\begin{Verbatim}
       use ropp_utils
        ...
       call To_Upper( string )
\end{Verbatim}
\section{Coordinates/cart2geod}
\textsl{[ Functions ]}

\label{ch:robo32}
\label{ch:Coordinates_cart2geod}
\index{unsorted!cart2geod}\index{Functions!cart2geod}
\textbf{NAME:}\hspace{0.08in}\begin{Verbatim}
    cart2geod - Transform occultation geometry in cartesian coordinates
                to geodetic latitude, longitude and height data.
\end{Verbatim}
\textbf{SYNOPSIS:}\hspace{0.08in}\begin{Verbatim}
    call cart2geod(cart, lat, lon, height)
\end{Verbatim}
\textbf{DESCRIPTION:}\hspace{0.08in}\begin{Verbatim}
    This subroutine calculates occultation latitude, longitude and height 
    relative to the WGS-84 ellipsoid from a cartesian vector relative to the
    ECF frame. 
\end{Verbatim}
\textbf{INPUTS:}\hspace{0.08in}\begin{Verbatim}
    cart        cartesian position vector (relative to ECF frame)
\end{Verbatim}
\textbf{OUTPUT:}\hspace{0.08in}\begin{Verbatim}
    lat           latitude (deg)
    lon           longitude (deg)
    height        geometric height relative to WGS-84 ellipsoid (m)
\end{Verbatim}
\textbf{NOTES:}\hspace{0.08in}\begin{Verbatim}
    Fast approximate solution
\end{Verbatim}
\textbf{SEE ALSO:}\hspace{0.08in}\textbf{REFERENCES:}\hspace{0.08in}\begin{Verbatim}
    Escobal, Methods of orbit determination
    1965, Wiley & sons, Inc. Pp. 27 - 29.

    Mahoney, M.J., A discussion of various measures of altitudes,
       URL: http://mtp.jpl.nasa.gov/notes/altitude/altitude.html,
       cited 13/01/2009.
\end{Verbatim}
\section{Coordinates/coordinates}
\textsl{[ Modules ]}

\label{ch:robo33}
\label{ch:Coordinates_coordinates}
\index{unsorted!coordinates}\index{Modules!coordinates}
\textbf{NAME:}\hspace{0.08in}\begin{Verbatim}
    coordinates - Module interfacing various geometric routines and functions.
\end{Verbatim}
\textbf{SYNOPSIS:}\hspace{0.08in}\begin{Verbatim}
    use coordinates
\end{Verbatim}
\textbf{DESCRIPTION:}\hspace{0.08in}\begin{Verbatim}
    This module provides interfaces to some geometric funtions and
    subroutines of the ropp_utils library.
\end{Verbatim}
\textbf{SEE ALSO:}\hspace{0.08in}\begin{Verbatim}
    ecf2eci
    eci2ecf
    impact_parameter
    occ_point
    plane_coordinates
    rotate
    earth
\end{Verbatim}
\section{Coordinates/curvature}
\textsl{[ Functions ]}

\label{ch:robo34}
\label{ch:Coordinates_curvature}
\index{unsorted!curvature}\index{Functions!curvature}
\textbf{NAME:}\hspace{0.08in}\begin{Verbatim}
    curvature - Determine the section curvature, its centre for the Earth
                reference ellipsoid, and the curvature radius
\end{Verbatim}
\textbf{SYNOPSIS:}\hspace{0.08in}\begin{Verbatim}
    call curvature(lat, lon, theta, r_coc, roc)
\end{Verbatim}
\textbf{DESCRIPTION:}\hspace{0.08in}\begin{Verbatim}
    This subroutine calculates the cartesian coordinates of the local centre
    of curvature and the radius for the reference ellipsoid.
\end{Verbatim}
\textbf{INPUTS:}\hspace{0.08in}\begin{Verbatim}
    lat           latitude of surface point
    lon           longitude of surface point
    theta         azimuth direction of the cross-section
\end{Verbatim}
\textbf{OUTPUT:}\hspace{0.08in}\begin{Verbatim}
    r_coc         cartesian centre of curvature vector (relative to ECF frame)
    roc           radius of curvature value           
\end{Verbatim}
\textbf{NOTES:}\hspace{0.08in}\begin{Verbatim}
    Uses theorems of Meusnier and Euler
\end{Verbatim}
\textbf{REFERENCES:}\hspace{0.08in}\section{Coordinates/deg2rad}
\textsl{[ Parameters ]}

\label{ch:robo35}
\label{ch:Coordinates_deg2rad}
\index{unsorted!deg2rad}\index{Parameters!deg2rad}
\textbf{NAME:}\hspace{0.08in}\begin{Verbatim}
    deg2rad - Degrees to radians conversion
    rad2deg - Radians to degrees conversion
\end{Verbatim}
\textbf{SOURCE:}\hspace{0.08in}\begin{Verbatim}
  real(wp), parameter  :: deg2rad = Pi / 180.0_wp

  real(wp), parameter  :: rad2deg = 180.0_wp / Pi
\end{Verbatim}
\section{Coordinates/ecf2eci}
\textsl{[ Functions ]}

\label{ch:robo36}
\label{ch:Coordinates_ecf2eci}
\index{unsorted!ecf2eci}\index{Functions!ecf2eci}
\textbf{NAME:}\hspace{0.08in}\begin{Verbatim}
    ecf2eci - Transform occultation geometry relative to ECF
              reference frame to geometry relative to ECI frame.
\end{Verbatim}
\textbf{SYNOPSIS:}\hspace{0.08in}\begin{Verbatim}
    r_eci = ecf2eci(year, month, day, hour, minute, sec, dsec, r_ecf)
\end{Verbatim}
\textbf{DESCRIPTION:}\hspace{0.08in}\begin{Verbatim}
    This subroutine calculates occultation coordinates relative to the ECI 
    frame.
\end{Verbatim}
\textbf{INPUTS:}\hspace{0.08in}\begin{Verbatim}
    year          occultation start year
    month         occultation start month
    day           occultation start day
    hour          occultation start hour
    minute        occultation start minute
    sec           occultation start second
    dsec          time since occultation start
    r_ecf         cartesian position vector (relative to ECF frame)
\end{Verbatim}
\textbf{OUTPUT:}\hspace{0.08in}\begin{Verbatim}
    r_eci         cartesian position vector (relative to ECI frame)  
\end{Verbatim}
\textbf{NOTES:}\hspace{0.08in}\textbf{SEE ALSO:}\hspace{0.08in}\begin{Verbatim}
    eci2ecf
\end{Verbatim}
\textbf{REFERENCES:}\hspace{0.08in}\begin{Verbatim}
     Astronomical Alamanus, 1993
\end{Verbatim}
\section{Coordinates/eci2ecf}
\textsl{[ Functions ]}

\label{ch:robo37}
\label{ch:Coordinates_eci2ecf}
\index{unsorted!eci2ecf}\index{Functions!eci2ecf}
\textbf{NAME:}\hspace{0.08in}\begin{Verbatim}
    eci2ecf - Transform occultation geometry relative to ECI
              reference frame to geometry relative to ECF frame.
\end{Verbatim}
\textbf{SYNOPSIS:}\hspace{0.08in}\begin{Verbatim}
    r_ecf = eci2ecf(year, month, day, hour, minute, sec, dsec, r_eci)
\end{Verbatim}
\textbf{DESCRIPTION:}\hspace{0.08in}\begin{Verbatim}
    This subroutine calculates occultation coordinates relative to the ECF 
    frame.
\end{Verbatim}
\textbf{INPUTS:}\hspace{0.08in}\begin{Verbatim}
    year           occultation start year
    month          occultation start month
    day            occultation start day
    hour           occultation start hour
    minute         occultation start minute
    sec            occultation start second
    dsec           time since occultation start
    r_eci          cartesian position vector (relative to ECI frame)
\end{Verbatim}
\textbf{OUTPUT:}\hspace{0.08in}\begin{Verbatim}
    r_ecf          cartesian position vector (relative to ECF frame)  
\end{Verbatim}
\textbf{NOTES:}\hspace{0.08in}\textbf{SEE ALSO:}\hspace{0.08in}\begin{Verbatim}
    ecf2eci
\end{Verbatim}
\textbf{REFERENCES:}\hspace{0.08in}\begin{Verbatim}
     Astronomical Alamanus, 1993
\end{Verbatim}
\section{Coordinates/geod2cart}
\textsl{[ Functions ]}

\label{ch:robo38}
\label{ch:Coordinates_geod2cart}
\index{unsorted!geod2cart}\index{Functions!geod2cart}
\textbf{NAME:}\hspace{0.08in}\begin{Verbatim}
    geod2cart - Transform occultation geodetic latitude, longitude and height 
                data to geometry in cartesian coordinates.
\end{Verbatim}
\textbf{SYNOPSIS:}\hspace{0.08in}\begin{Verbatim}
    call geod2cart(lat, lon, height, cart)
\end{Verbatim}
\textbf{DESCRIPTION:}\hspace{0.08in}\begin{Verbatim}
    This subroutine calculates a cartesian vector relative to the
    ECF frame from occultation latitude, longitude and height 
    relative to the WGS-84 ellipsoid.
\end{Verbatim}
\textbf{INPUTS:}\hspace{0.08in}\begin{Verbatim}
    lat           latitude (deg)
    lon           longitude (deg)
    height        geometric height relative to WGS-84 ellipsoid (m)
\end{Verbatim}
\textbf{OUTPUT:}\hspace{0.08in}\begin{Verbatim}
    cart          cartesian position vector (relative to ECF frame)
\end{Verbatim}
\textbf{NOTES:}\hspace{0.08in}\textbf{SEE ALSO:}\hspace{0.08in}\begin{Verbatim}
   cart2geod
\end{Verbatim}
\textbf{REFERENCES:}\hspace{0.08in}\begin{Verbatim}
    Escobal, Methods of orbit determination
    1965, Wiley & sons, Inc. Pp. 27 - 29.

    Mahoney, M.J., A discussion of various measures of altitudes,
       URL: http://mtp.jpl.nasa.gov/notes/altitude/altitude.html,
       cited 13/01/2009.
\end{Verbatim}
\section{Coordinates/global}
\textsl{[ Parameters ]}

\label{ch:robo39}
\label{ch:Coordinates_global}
\index{unsorted!global}\index{Parameters!global}
\textbf{NAME:}\hspace{0.08in}\begin{Verbatim}
    Re - Equatorial semiaxis (m)
    Rp - Polar semiaxis (m)
    f  - Flatness
    e  - Squared eccentricity
\end{Verbatim}
\textbf{REFERENCES:}\hspace{0.08in}\begin{Verbatim}
   http://earth-info.nga.mil/GandG/wgs84/gravitymod/index.html
\end{Verbatim}
\textbf{SOURCE:}\hspace{0.08in}\begin{Verbatim}
  real(wp), parameter  :: Re = 6378137.0_wp

  real(wp), parameter  :: f  = 1.0_wp / 298.2572235630_wp

  real(wp), parameter  :: Rp = Re * (1.0_wp - f)

  real(wp), parameter  :: e  = (Re**2 - Rp**2)/(Re**2)
\end{Verbatim}
\section{Coordinates/impact\_parameter}
\textsl{[ Functions ]}

\label{ch:robo40}
\label{ch:Coordinates_impact_parameter}
\index{unsorted!impact\_parameter}\index{Functions!impact\_parameter}
\textbf{NAME:}\hspace{0.08in}\begin{Verbatim}
    impact parameter - Determine the impact parameter
\end{Verbatim}
\textbf{SYNOPSIS:}\hspace{0.08in}\begin{Verbatim}
    impact =  impact_parameter(r_leo, r_gns, bangle)
\end{Verbatim}
\textbf{DESCRIPTION:}\hspace{0.08in}\begin{Verbatim}
    This function calculates the impact parameter connecting two given points
    assuming spherical symmetry using trigonometric formulae.
    With no knowledge of the bending angle (i.e. default processing), this 
    routine returns the straight line tangent altitude.
\end{Verbatim}
\textbf{INPUTS:}\hspace{0.08in}\begin{Verbatim}
    r_leo         cartesian LEO position vector (relative to ECF frame)
    r_gns         cartesian GPS position vector (relative to ECF frame)
    bangle        bending angle
\end{Verbatim}
\textbf{OUTPUT:}\hspace{0.08in}\begin{Verbatim}
    impact        impact parameter
\end{Verbatim}
\textbf{SEE ALSO:}\hspace{0.08in}\textbf{REFERENCES:}\hspace{0.08in}\section{Coordinates/occ\_point}
\textsl{[ Functions ]}

\label{ch:robo41}
\label{ch:Coordinates_occ_point}
\index{unsorted!occ\_point}\index{Functions!occ\_point}
\textbf{NAME:}\hspace{0.08in}\begin{Verbatim}
    occ_point - Determine the occultation point
\end{Verbatim}
\textbf{SYNOPSIS:}\hspace{0.08in}\begin{Verbatim}
    call occ_point(r_leo, r_gns, lat, lon, r_coc, roc, azimuth, undulation, 
                   cfile, efile)
\end{Verbatim}
\textbf{DESCRIPTION:}\hspace{0.08in}\begin{Verbatim}
    This subroutine calculates the lowest occultation perigree point projected
    to the Earth's surface
\end{Verbatim}
\textbf{INPUTS:}\hspace{0.08in}\begin{Verbatim}
    r_leo         cartesian LEO position vector (relative to ECF frame)
    r_gns         cartesian GPS position vector (relative to ECF frame)
    cfile         path to geoid potential coefficients file (optional)
    efile         path to geoid potential corrections file (optional)
\end{Verbatim}
\textbf{OUTPUT:}\hspace{0.08in}\begin{Verbatim}
    lat           Occultation point latitude
    lon           Occultation point longitude
    r_coc         Cartesian centre of curvature vector for occ point (ECF)
    roc           Radius of curvature value for occultation point
    azimuth       GPS to LEO azimuth direction wrt true North (deg)
    undulation    Difference between ellipsoid (WGS-84) and EGM-96 geoid (m)
\end{Verbatim}
\textbf{SEE ALSO:}\hspace{0.08in}\textbf{REFERENCES:}\hspace{0.08in}\section{Coordinates/plane\_coordinates}
\textsl{[ Functions ]}

\label{ch:robo42}
\label{ch:Coordinates_plane_coordinates}
\index{unsorted!plane\_coordinates}\index{Functions!plane\_coordinates}
\textbf{NAME:}\hspace{0.08in}\begin{Verbatim}
    plane_coordinates - Calculation of occultation plane coordinates of GSP
                        and LEO
\end{Verbatim}
\textbf{SYNOPSIS:}\hspace{0.08in}\begin{Verbatim}
    call plane_coordinates(r_leo, r_gns, r_coc, roc, xleo, yleo, xgns, ygns, ax, ay)
\end{Verbatim}
\textbf{DESCRIPTION:}\hspace{0.08in}\begin{Verbatim}
    This subroutine calculates the GPS and LEO coordinates in GPS-O-LEO plane
    transformed to unmoving GPS.
\end{Verbatim}
\textbf{INPUTS:}\hspace{0.08in}\begin{Verbatim}
    r_leo         cartesian LEO position vector (m)
    r_gns         cartesian GPS position vector (m)
    r_coc         Cartesian centre of curvature vector (m) 
    roc           Local radius of curvature (m)
\end{Verbatim}
\textbf{OUTPUT:}\hspace{0.08in}\begin{Verbatim}
    xleo          X coordinates of LEO (m)
    yleo          Y coordinates of LEO (m)
    xgns          X coordinates of GPS (m)
    ygns          Y coordinates of GPS (m)
    ax            Occultation plane X basis vector
    ay            Occultation plane Y basis vector
\end{Verbatim}
\textbf{SEE ALSO:}\hspace{0.08in}\textbf{REFERENCES:}\hspace{0.08in}\section{Coordinates/rotate}
\textsl{[ Functions ]}

\label{ch:robo43}
\label{ch:Coordinates_rotate}
\index{unsorted!rotate}\index{Functions!rotate}
\textbf{NAME:}\hspace{0.08in}\begin{Verbatim}
    rotate - Rotate a vector in cartesian coordinates around
             a given axis by a given angle  
\end{Verbatim}
\textbf{SYNOPSIS:}\hspace{0.08in}\begin{Verbatim}
    Rotate = rotate(X, A, phi)
\end{Verbatim}
\textbf{DESCRIPTION:}\hspace{0.08in}\begin{Verbatim}
    This function rotates a vector X around a given axis A by angle phi.
       N*(N,X) + [N,X]*Sin(Phi) + (X-N*(N,X))*Cos(Phi),   where N=A/|A|
\end{Verbatim}
\textbf{INPUTS:}\hspace{0.08in}\begin{Verbatim}
    X             Vector to rotate
    A             Rotation axis
    Phi           Rotation angle (rad)
\end{Verbatim}
\textbf{OUTPUT:}\hspace{0.08in}\begin{Verbatim}
    Rotate        Rotated vector
\end{Verbatim}
\textbf{NOTES:}\hspace{0.08in}\textbf{SEE ALSO:}\hspace{0.08in}\textbf{REFERENCES:}\hspace{0.08in}\section{Coordinates/tangent\_point}
\textsl{[ Functions ]}

\label{ch:robo44}
\label{ch:Coordinates_tangent_point}
\index{unsorted!tangent\_point}\index{Functions!tangent\_point}
\textbf{NAME:}\hspace{0.08in}\begin{Verbatim}
    tangent_point - Determine tangent point coordinates
\end{Verbatim}
\textbf{SYNOPSIS:}\hspace{0.08in}\begin{Verbatim}
    call occ_point(r_leo, r_gns, lat_tp, lon_tp, azimuth_tp)
\end{Verbatim}
\textbf{DESCRIPTION:}\hspace{0.08in}\begin{Verbatim}
    This subroutine calculates the latitude, longitude and azimuth of
    each tangent point for an occultation
\end{Verbatim}
\textbf{INPUTS:}\hspace{0.08in}\begin{Verbatim}
    r_leo         cartesian LEO position vector (relative to ECF frame)
    r_gns         cartesian GPS position vector (relative to ECF frame)
\end{Verbatim}
\textbf{OUTPUT:}\hspace{0.08in}\begin{Verbatim}
    lat_tp        tangent point latitude
    lon_tp        tangent point longitude
    azimuth_tp    GPS to LEO azimuth direction wrt true North at tangent (deg)
\end{Verbatim}
\textbf{SEE ALSO:}\hspace{0.08in}\textbf{REFERENCES:}\hspace{0.08in}\section{Coordinates/vector\_angle}
\textsl{[ Functions ]}

\label{ch:robo45}
\label{ch:Coordinates_vector_angle}
\index{unsorted!vector\_angle}\index{Functions!vector\_angle}
\textbf{NAME:}\hspace{0.08in}\begin{Verbatim}
    vector_angle - Find the angle between two cartesian vectors
\end{Verbatim}
\textbf{SYNOPSIS:}\hspace{0.08in}\begin{Verbatim}
    angle = vector_angle(X, Y, A)
\end{Verbatim}
\textbf{INPUTS:}\hspace{0.08in}\begin{Verbatim}
    X             Vector 1
    Y             Vector 2
    A             Orientation axis (optional)
\end{Verbatim}
\textbf{OUTPUT:}\hspace{0.08in}\begin{Verbatim}
    Angle       Angle between vectors
\end{Verbatim}
\section{Coordinates/vector\_product}
\textsl{[ Functions ]}

\label{ch:robo46}
\label{ch:Coordinates_vector_product}
\index{unsorted!vector\_product}\index{Functions!vector\_product}
\textbf{NAME:}\hspace{0.08in}\begin{Verbatim}
    vector_product - Compute a vector product of two cartesian vectors
\end{Verbatim}
\textbf{SYNOPSIS:}\hspace{0.08in}\begin{Verbatim}
    product = vector_product(X, Y)
\end{Verbatim}
\textbf{INPUTS:}\hspace{0.08in}\begin{Verbatim}
    X             Vector 1
    Y             Vector 2
\end{Verbatim}
\textbf{OUTPUT:}\hspace{0.08in}\begin{Verbatim}
    Product       Vector product
\end{Verbatim}
\section{Datetime/CalToJul}
\textsl{[ Subroutines ]}

\label{ch:robo47}
\label{ch:Datetime_CalToJul}
\index{unsorted!CalToJul}\index{Subroutines!CalToJul}
\textbf{NAME:}\hspace{0.08in}\begin{Verbatim}
   CalToJul  (caltojul.f90)
\end{Verbatim}
\textbf{SYNOPSIS:}\hspace{0.08in}\begin{Verbatim}
   Subroutine to convert between calendar date/time &
   Julian date & fraction

   REAL(dp) :: JDF
   INTEGER  :: CDT(8), inv
   CALL CalToJul ( CDT, JDF, inv )
\end{Verbatim}
\textbf{INPUTS:}\hspace{0.08in}\begin{Verbatim}
   CDT  int   Array(8): Calender date & clock time [if inv>0]
   JDF  dflt  Julian date & fraction (UTC)         [if inv<=0]
   inv  int   Indicator for direction of conversion
               > 0 : CDT --> JDF
              <= 0 : JDF --> CDT

 OUTPUTS
   CDT  int   Array(8): Calender date & clock time [if inv<=0]
   JDF  dflt  Julian date & fraction (UTC)         [if inv>0]
\end{Verbatim}
\textbf{DESCRIPTION:}\hspace{0.08in}\begin{Verbatim}
   Converts between:
     - CDT: Gregorian calendar date & clock time as an 8-element array
       (year,month,day,zone,hour,minute,second,millisecond)
       as returned by the F90 DATE_AND_TIME(VALUE=array) intrinsic
   and:
     - JDF: Julian Date (UTC, in days and fractions) [See NOTES]
   or vice-versa.
   inv indicates the direction of conversion: >0 for CDT-->JDF else
   JDF-->CDT.
   The zone parameter (CDT(4) - offset from UTC in minutes) is
   applied to CDT-->JDF conversions so that JDF is always in UTC.
   JDF-->CDT conversions assume JDF is in UTC and CDT(4) is
   applied to obtain the appropriate time zone (all elements of CDT
   except CDT(4) are overwritten). Set CDT(4)=0 to keep the
   JDF-->CDT conversion in UTC.
   NB JDF argument's KIND must be declared equivalent to
   DOUBLE PRECISION (REAL*8).
   The Date/Julian day number conversions apply the algorithms from
   Ref.1. All dates are assumed to be in the Gregorian Calendar.
\end{Verbatim}
\textbf{NOTES:}\hspace{0.08in}\begin{Verbatim}
   From Ref.2, Julian period or dates:
   - Number of consecutive days since 1-Jan-4713 BC.
   - Devised by Joseph Justus Scaliger in 1582. Thought to be named after
     his father, Julius Caesar Scaliger, but some now think it named
     after the Julian calendar (ie after Julius Caesar) - see Ref.3.
   - Based on 'cycles of history' of period 7980 years, being the
     product of:
      28 = so-called solar cycle in Julian calendar
      19 = lunar or metonic cycle
      15 = 'cycle of indiction', an ancient Roman taxation period.
     4713 BC was the nearest past year on which all three cycles started
     together. See Refs.2,3.
   - By convention, Julian dates begin at midday, the astronomical start
     of the day (Ref.3)
   - Epochs other than 12:00 1-Jan-4713 BC may be used, but these strictly
     should not be called 'Julian' dates. See the TimeSince() subroutine in
     this package.
\end{Verbatim}
\textbf{REFERENCES:}\hspace{0.08in}\begin{Verbatim}
   1) Meeus, Jan (1998). Astronomical Algorithms.
      2nd ed., Willmann-Bell Inc, ISBN 0-943396-61-1
   2) Encyclpaedia Brittanica, Microcropaedia Vol.V,
      15th Ed. 1975, p.632.
   3) Wikipaedia: http://en.wikipedia.org/wiki/Julian_Date
\end{Verbatim}
\textbf{SEE ALSO:}\hspace{0.08in}\begin{Verbatim}
   TimeSince()
\end{Verbatim}
\section{DateTime/ConvertDT}
\textsl{[ Subroutines ]}

\label{ch:robo48}
\label{ch:DateTime_ConvertDT}
\index{unsorted!ConvertDT}\index{Subroutines!ConvertDT}
\textbf{NAME:}\hspace{0.08in}\begin{Verbatim}
   ConvertDT     (convertdt.f90)
\end{Verbatim}
\textbf{SYNOPSIS:}\hspace{0.08in}\begin{Verbatim}
   Convert a Date & Time string between long1 or long2 & short formats

   USE DateTime
   CHARACTER (LEN=24) :: DTLong1
   CHARACTER (LEN=19) :: DTLong2
   CHARACTER (LEN=14) :: DTshort
   INTEGER            :: inv msecs, lfmt
   CALL ConvertDT ( DTlong, DTshort, inv[, LongFmt=lfmt][, Msec=msecs] )
\end{Verbatim}
\textbf{INPUTS:}\hspace{0.08in}\begin{Verbatim}
   DTlong   chr  Date & Time as dd-MMM-yyyy hh:mm:ss.ttt
                 or yyyy-MM-dd hh:mm:ss.ttt    [if inv>0]
   DTshort  chr  Date & Time as yyyyMMddhhmmss [if inv<=0]
   inv      int  Inversion flag
                  >0 : long --> short
                 <=0 : short --> long
   Msec     int  (Optional) millisecs value (ttt: 0-999)
   LongFmt  int  (optional) Output long format style. Ignored if inv>0
                  1 = dd-MMM-yyyy hh:mm:ss.ttt (default)
                  2 = yyyy-MM-dd hh:mm:ss.ttt

 OUTPUTS
   DTlong   chr  Date & Time as dd-MMM-yyyy hh:mm:ss.ttt [if inv<=0]
   DTshort  chr  Date & Time as yyyyMMddhhmmss           [if inv>0]
   Msec     int  (Optional) millisecs value (ttt: 0-999)
\end{Verbatim}
\textbf{CALLS:}\hspace{0.08in}\begin{Verbatim}
   MonthOfYear
\end{Verbatim}
\textbf{DESCRIPTION:}\hspace{0.08in}\begin{Verbatim}
   When inv > 0, converts a date & time string from long1
   (dd-MMM-yyyy hh:mm:ss.ttt) or long2 ( yyyy-MM-dd hh:mm:ss.ttt)
   formats to short (yyyymmddhhmmss) format. For both long formats,
   the time portion may be truncated, in which case the short format
   will be padded with '0's. The ttt portion is anyway ignored.
   When inv <= 0, converts a date & time string from short
   (yyyymmddhhmmss) format to long1 (dd-MMM-yyyy hh:mm:ss.ttt) format.
   The ttt portion will be '0's.
   While the long formats are suitable for human reading, the
   short format is more efficient to use for comparisons and sorting.

 EXAMPLES
   USE DateTime
   CHARACTER (LEN=24) :: DTlong1
   CHARACTER (LEN=19) :: DTlong2
   CHARACTER (LEN=14) :: DTshort
   INTEGER            :: msecs
   DTlong2 = "2010-04-14"
   CALL ConvertDT ( DTlong2, DTshort, 1 )
   --> DTshort = "20100414000000"
   DTshort = "20100414093456"
   msecs = 789
   CALL ConvertDT ( DTlong1, DTshort, -1, Msec=msecs )
   --> DTlong = "14-Apr-2010 09:34:56.789"
   DTshort = "20100414093456"
   CALL ConvertDT ( DTlong1, DTshort, -1, LongFmt=2 )
   --> DTlong = "2010-04-14 09:34:56.000"
\end{Verbatim}
\section{DateTime/Date\_and\_Time\_UTC}
\textsl{[ Subroutines ]}

\label{ch:robo49}
\label{ch:DateTime_Date_and_Time_UTC}
\index{unsorted!Date\_and\_Time\_UTC}\index{Subroutines!Date\_and\_Time\_UTC}
\textbf{NAME:}\hspace{0.08in}\begin{Verbatim}
   Date_and_Time_UTC     (date_and_time_utc.f90)
\end{Verbatim}
\textbf{SYNOPSIS:}\hspace{0.08in}\begin{Verbatim}
   Mimics the F90 intrinsic DATE_AND_TIME, but returned
   elements are in UTC

   USE DateTime
   CHARACTER (LEN=8)  :: Date
   CHARACTER (LEN=10) :: Time
   CHARACTER (LEN=5)  :: Zone
   INTEGER            :: Values(8)
   CALL Date_and_Time_UTC ( [DATE=Date][, TIME=Time]
                          [, ZONE=zone][, VALUES=Values] )
\end{Verbatim}
\textbf{INPUTS:}\hspace{0.08in}\begin{Verbatim}
   None

 OUTPUTS
   Date    chr   (Optional) Date (CCYYMMDD)   in UTC
   Time    chr   (Optional) Time (hhmmss.sss) in UTC
   Zone    chr   (Optional) Zone (+/-hhmm)    always "+00:00"
   Values  int   (Optional) Array(8): Calendar date & clock time in UTC
\end{Verbatim}
\textbf{CALLS:}\hspace{0.08in}\begin{Verbatim}
   CalToJul
\end{Verbatim}
\textbf{DESCRIPTION:}\hspace{0.08in}\begin{Verbatim}
   Get current system date/time as YR,MN,DY,ZO,HR,MN,SC,MS.
   and using the zone element (minutes relative to UTC),
   offset local time to UTC. NB: 'local' may already in UTC!
   All arguments are optional, and can be any or all of Date,
   Time, Zone (not useful!) as strings and Values (integer array
   of date/time elements), as per the DATE_AND_TIME() intrinsic.
\end{Verbatim}
\section{DateTime/DateTimeOffset}
\textsl{[ Subroutines ]}

\label{ch:robo50}
\label{ch:DateTime_DateTimeOffset}
\index{unsorted!DateTimeOffset}\index{Subroutines!DateTimeOffset}
\textbf{NAME:}\hspace{0.08in}\begin{Verbatim}
   DateTimeOffset     (datetimeoffset.f90)
\end{Verbatim}
\textbf{SYNOPSIS:}\hspace{0.08in}\begin{Verbatim}
   Offset a date/time by a given amount

   USE DateTime
   INTEGER :: CDT1(8), CDT2(8)
   CALL DateTimeOffset ( CDT1 "+|-", CDT2 )
\end{Verbatim}
\textbf{INPUTS:}\hspace{0.08in}\begin{Verbatim}
   CDT1  int   Array(8): Calendar date & clock time in UTC
   oper  char  Operation ("+" or "-")
   CDT2  int   Array(8): Calender date & clock time in UTC

 OUTPUTS
   CDT1  int   Array(8): Calendar date & clock time in UTC
\end{Verbatim}
\textbf{CALLS:}\hspace{0.08in}\begin{Verbatim}
   CalToJul
\end{Verbatim}
\textbf{DESCRIPTION:}\hspace{0.08in}\begin{Verbatim}
   Appplies an offset to the given date & time array using
   a second date/time array, which may be a small delta or
   a full calendar value. The offset operation may be
   to add (oper="+") or subtract (oper="-"). The result is
   returned in the first date/time array.
   This routine handles two types of case:
   1) CDT1 is an absolute date/time and CDT2 is a relatively
      small (<10 years) delta offset (oper is "+" or "-"); CDT1
      will be the absolute date/time with the offset applied.
   2) CDT1 and CDT2 are both absolute date/times and oper is
      "-"; CDT1 will be the time difference in days, hours,
      minutes, seconds & msecs (year, month & zone elements
      will be zero). If CDT2 is later than CDT1, all of these
      non-zero time elements will be negative.
   Note that the case of CDT1 + CDT2 where both are absolute
   date/times will work - but this is not a sensible operation!

 EXAMPLES
   1) Difference between two date/times:
      USE DateTime
      INTEGER :: CDT1(8)
      INTEGER :: CDT2(8) = (/2010,1,1,0,0,0,0,0/)
      CALL DATE_AND_TIME_UTC(VALUES=CDT1)
      --> CDT1 = 2010,4,9,15,17,14,234
      CALL DateTimeOffset ( CDT1, "-", CDT2 )
      --> CDT1 = 0,0,98,0,15,17,14,234

   2) Date/time three days, 12 hours ago from now (which just
      happens to be 06:30 2-Jan-2011):
      USE DateTime
      INTEGER :: CDT1(8) = (/2011,1,2,0,06,30,0,0/)
      INTEGER :: CDT2(8) = (/0,0,3,0,12,0,0,0/)
      CALL DateTimeOffset ( CDT1, "+", CDT2 )
      --> CDT1 = 2010,12,29,0,18,30,0,0

   3) Next whole hour after an observaton timestamped at
      23:14 31-Dec-2010:
      USE DateTime
      INTEGER :: CDT1(8) = (/2010,12,31,0,23,14,0,0/)
      INTEGER :: CDT2(8) = (/0,0,0,0,1,0,0,0/)
      CALL DateTimeOffset ( CDT1, "+", CDT2 )
      CDT1(6:8) = 0
      --> CDT1 = 2011,1,1,0,0,0,0,0

   4) Day-of-year for today
      USE DateTime
      INTEGER :: CDT1(8), CDT2(8)
      CALL Date_and_Time_UTC ( VALUES=CDT1 )
      --> CDT1 = 2010,4,9,15,17,14,234
      CDT2 = (/CDT1(1),1,0,0,0,0,0,0/) ! NB day=0 so 1-Jan returns 1
      CALL DateTimeOffset ( CDT1,'-',CDT2)
      --> CDT1(IdxDay) = 99
\end{Verbatim}
\textbf{SEE ALSO:}\hspace{0.08in}\begin{Verbatim}
    TimeSince()
\end{Verbatim}
\section{DateTime/DateTimeProgs}
\textsl{[ Modules ]}

\label{ch:robo51}
\label{ch:DateTime_DateTimeProgs}
\index{unsorted!DateTimeProgs}\index{Modules!DateTimeProgs}
\textbf{NAME:}\hspace{0.08in}\begin{Verbatim}
    DateTimeProgs    (datetimeprogs.f90)
\end{Verbatim}
\textbf{SYNOPSIS:}\hspace{0.08in}\begin{Verbatim}
   Interface module for date & time manipulation routines

    USE DateTimeProgs
\end{Verbatim}
\textbf{DESCRIPTION:}\hspace{0.08in}\begin{Verbatim}
    This module provides interfaces to some date and time manipulation 
    routines
\end{Verbatim}
\textbf{SEE ALSO:}\hspace{0.08in}\begin{Verbatim}
   DateTimeTypes
   CalToJul
   ConvertDT
   Date_and_Time_UTC
   DateTimeOffset
   MonthOfYear
   TimeSince
\end{Verbatim}
\section{DateTime/DateTimeTypes}
\textsl{[ Modules ]}

\label{ch:robo52}
\label{ch:DateTime_DateTimeTypes}
\index{unsorted!DateTimeTypes}\index{Modules!DateTimeTypes}
\textbf{NAME:}\hspace{0.08in}\begin{Verbatim}
    DateTimeTypes    (datetimetypes.f90) 
\end{Verbatim}
\textbf{SYNOPSIS:}\hspace{0.08in}\begin{Verbatim}
   Module of definitions for date & time manipulation routines

    USE DateTimeTypes
\end{Verbatim}
\textbf{DESCRIPTION:}\hspace{0.08in}\begin{Verbatim}
    This module provides definitions for date & time manipulation 
    routines
\end{Verbatim}
\textbf{SEE ALSO:}\hspace{0.08in}\begin{Verbatim}
   DateTimeProgs
   CalToJul
   ConvertDT
   Date_and_Time_UTC
   DateTimeOffset
   MonthOfYear
   TimeSince
\end{Verbatim}
\section{DateTime/MonthOfYear}
\textsl{[ Subroutines ]}

\label{ch:robo53}
\label{ch:DateTime_MonthOfYear}
\index{unsorted!MonthOfYear}\index{Subroutines!MonthOfYear}
\textbf{NAME:}\hspace{0.08in}\begin{Verbatim}
   MonthOfYear     (monthofyear.f90)
\end{Verbatim}
\textbf{SYNOPSIS:}\hspace{0.08in}\begin{Verbatim}
   Convert a Month between string and numeric form

   USE DateTime
   CHARACTER (LEN=10) :: monthname
   INTEGER            :: monthnumber, inv
   CALL MonthOfYear ( monthnumber, monthname, inv )
\end{Verbatim}
\textbf{INPUTS:}\hspace{0.08in}\begin{Verbatim}
   MonthNumber  int  Month as number (1-12) {if inv>=0}
   MonthString  chr  Month as name ("JAN"-"DEC", "January"-"December")
                     [if inv<0]
   inv          int  Inversion flag
                      > 0 : number --> string (full name)
                      = 0 : number --> string (3chr uppercase)
                      < 0 : string --> number

 OUTPUTS
   MonthNumber  int  Month as number (1-12 or 0 if string invalid)
                     [if inv<0]
   MonthString  chr  Month as name ("JAN"-"DEC", "January"-"December")
                     [if inv>=0]

 DECRIPTION
   Converts a month name (e.g. "February") to it's numeric
   equivalent (e.g 2) or vice-versa depending on the value
   of inv.
   inv = 0 : an input numeric value must be in the range 1-12
             and is converted to its equivalent month name (returned
             as a 3-chr uppercase string, e.g. "FEB"). Invalid values
             return the month name for absolute modulo-12 value (e.g.
             13 returns "JAN" and 0 returns "DEC").
   inv > 0 : as above, but returns the full, capitalised, month name
             (e.g. "February").
   inv < 0 : For input, month names may be full or abbreviated to a
             minimum of three characters, and may be in upper,
             lower or mixed case (the match is done using only
             the first three significant characters in upper case).
             Invalid months return a numeric value of zero, else
             the month in the range 1 to 12.

 EXAMPLES
   USE DateTime
   CHARACTER (LEN=10) :: monthname
   INTEGER            :: monthnumber
   monthnumber = 12
   CALL MonthOfYear ( monthnumber, monthname, 1 )
   --> monthname = "December"
   CALL MonthOfYear ( monthnumber, monthname, 0 )
   --> monthname = "DEC"
   monthname = "March"
   CALL MonthOfYear ( monthnumber, monthname, -1 )
   --> monthnumber = 3
\end{Verbatim}
\section{DateTime/TimeSince}
\textsl{[ Subroutines ]}

\label{ch:robo54}
\label{ch:DateTime_TimeSince}
\index{unsorted!TimeSince}\index{Subroutines!TimeSince}
\textbf{NAME:}\hspace{0.08in}\begin{Verbatim}
    TimeSince   (timesince.f90)
\end{Verbatim}
\textbf{SYNOPSIS:}\hspace{0.08in}\begin{Verbatim}
   Subroutine to convert between calendar date & time since some
   arbitrary base date time, or vice-versa

   USE  DateTime
   CHARACTER (LEN=40) :: Tbase
   INTEGER            :: CDT(8)
   REAL(dp)           :: Tsince
   INTEGER            :: inv
   CALL TimeSince ( CDT, Tsince, inv[, Base=Tbase] )
\end{Verbatim}
\textbf{INPUTS:}\hspace{0.08in}\begin{Verbatim}
   CDT     int   Array(8): Calender date & clock time  [if inv>0]
   Tsince  dflt  Value in 'units since base date/time' [if inv<=0]
   inv     int   Indicator for direction of conversion
                  > 0 : Calendar --> time since base
                 <= 0 : time since base --> Calendar
   Base    char  (Optional) Base units/date/time
                 Default: 'JD2000'

 OUTPUTS
   CDT     int   Array(8): Calendar date & clock time  [if inv<=0]
   Tsince  dflt  Value in 'units since base date/time' [if inv>0]
\end{Verbatim}
\textbf{CALLS:}\hspace{0.08in}\begin{Verbatim}
   CalToJul
   ConvertDT
\end{Verbatim}
\textbf{DESCRIPTION:}\hspace{0.08in}\begin{Verbatim}
   Converts between:
     - CDT: Gregorian calendar date & clock time as an 8-element array
       (year,month,day,zone,hour,minute,second,millisecond)
       as returned by the F90 DATE_AND_TIME(VALUE=array) intrinsic
   and:
     - Tsince: time relative to an arbitary base date/time and in
       base units (seconds, minutes, hours or days)
   or vice-versa.
   inv indicates the direction of conversion: >0 for CDT-->Tsince else
   Tsince-->CDT.
   NB Tsince argument's KIND must be declared equivalent to
   DOUBLE PRECISION (REAL*8).

   The (optional) Base string may specify an arbitrary base
   units and date/time as one of:
   1) '<units> SINCE <[d]d-MMM-yyyy [hh[:mm[:ss]]]>'
      where <units> can be 'seconds', 'minutes', 'hours' or 'days'.
      If the time portion is truncated, the missing elements are
      assumed to be zero, so '1-Jan-2010' is the same as
      '01-Jan-2010 00:00:00'
   2) '<units> SINCE <yyyy-mm-dd [hh[:mm[:ss]]]>'
      otherwise the same as (1)
   3) '<units> SINCE <yyyymmdd[hh[mm[ss]]]'
      otherwise the same as (1)
   4) 'J<u>[<yyyy>]'
      where <u> is one of 'S', 'M', 'H' or 'D' and the date is
      implicitly 1st January of the year yyyy at midnight.
      If yyyy is blank, 2000 is assumed.
   5) one of the following short-hand keywords:
      a) 'GPS[SECONDS]' eqiv 'seconds since 6-Jan-1980 00:00:00',
          - the base date for counting GPS time in seconds
      b) 'MJD' equiv 'days since 16-Nov-1858 00:00:00'
         - Modified Julian Date (defined as JDF - 2,400,000.5)
      c) 'UNIX' equiv 'seconds since 1-Jan-1970 00:00:00'
         - the base for UNIX time counting
      d) 'ZERO' eqiv 'days since 1-Jan-4317 12:00:00'
         - the base JDF=0 (standard Julian Days); same as CalToJul().

   If the Base units argument is not present, or the string is given
   but is blank or incorrectly formatted, a default of JD2000
   (days since 1-Jan-2000 00:00;00) is used.
   The 'zone' element is ignored for input and set to zero on output
   (implying UTC).

 EXAMPLES
   1) Current time in seconds since midnight, 1-Jan-2000
      USE DateTime
      INTEGER  :: CDT(8)
      REAL(dp) :: JDF
      CALL Date_and_Time_UTC ( Values=CDT )
      --> CDT = 2010,4,9,15,17,14,234
      CALL TimeSince ( CDT, JDF, Base="JS2000" )
      --> JDF = 324141434.23402011

   2) Day-of-year for today
      USE DateTime
      CHARACTER (LEN=20) :: Tbase
      INTEGER  :: CDT(8), DoY
      REAL(dp) :: JDF
      CALL Date_and_Time_UTC ( Values=CDT )
      --> CDT = 2010,4,9,15,17,14,234
      WRITE ( Tbase, "(A2,I4.4)" ) "JD", CDT(1)
      CALL TimeSince ( CDT, JDF, 1, Base=Tbase )
      DoY = INT ( JDF1 + HalfDay ) ! allow for JDFs from midday
      --> DoY = 99

   3) Calendar date for 1 million hours from 11:32 on 8-Nov-1952
      USE DateTime
      INTEGER  :: CDT(8)=0
      REAL(dp) :: OMH=1000000.0_dp
      CALL TimeSince ( CDT, OMH, -1, Base="Hours since 1952-11-08 11:32" )
      --> CDT = 2066,12,7,0,3,32,0,0 (03:32 7-Dec-2066)
\end{Verbatim}
\textbf{SEE ALSO:}\hspace{0.08in}\begin{Verbatim}
    CalToJul(), DateTimeOffset()
\end{Verbatim}
\section{EARTH/Datum\_Erad}
\textsl{[ Subroutines ]}

\label{ch:robo55}
\label{ch:EARTH_Datum_Erad}
\index{unsorted!Datum\_Erad}\index{Subroutines!Datum\_Erad}
\textbf{NAME:}\hspace{0.08in}\begin{Verbatim}
   Datum_Erad     (earth.f90)
\end{Verbatim}
\textbf{SYNOPSIS:}\hspace{0.08in}\begin{Verbatim}
   Calculate the Earth's radii of curvature on ellipsoid datum

   USE EarthMod
   real(wp) :: Rm, Rp, Re
   CALL Datum_Erad ( "WGS84", 50.0D0, Rm, Rp, Re )
\end{Verbatim}
\textbf{ARGUMENTS:}\hspace{0.08in}\begin{Verbatim}
   Datum  (in)   chr   Name of datum
   Lat    (in)   dflt  Latitude (degrees)
   Rm     (out)  dflt  Radius of curvature along meridian
                       i.e. N-S at given latitude (metres)
   Rp     (out)  dflt  Radius of curvature along parallel
                       i.e. E-W at given latitude (metres)
   Re     (out)  dflt  Distance from earth centre to
                       ellipsoid surface (metres)
\end{Verbatim}
\textbf{CALLS:}\hspace{0.08in}\begin{Verbatim}
   Datum_Param
\end{Verbatim}
\textbf{CALLED BY:}\hspace{0.08in}\begin{Verbatim}
   Datum_Trans
   LLtoNE

 ALGORITHM
    Rm   = a (1-e^2) / (1-e^2.sin^2{lat})^(3/2)
    Rp   = a / (1-e^2.sin^2{lat})^(1/2)
    Re^2 = a^2.(1-e^2) / (1 - e^2 + e^2.sin^2{lat})
   where a is the equatorial radius,
   and e is the eccentricity:
    e^2  = f(2-f)
   where f is the flattening parameter, and a,f define the ellipsoid.
\end{Verbatim}
\textbf{DESCRIPTION:}\hspace{0.08in}\begin{Verbatim}
   Given the latitude of a point (in degrees), calculates the Earth's
   radii of curvature in both the meridional (N/S) and parallel (E/W)
   directions, and the Earth's radius (i.e. the distance from the
   Earth's centre to surface at the given latitude) using an ellipsoid
   model. Datum is specified as character string for any of the
   supported types:
    "ED50"   - European datum, 1950
    "GEM6"   - used for ERS orbits
    "GRS80"  - Geodetic Reference System, 1980 [5]
    "KL86"   - Kaye & Laby, 1986 [3]
    "NAD27"  - North American Datum, 1927 (Clark 1866)
    "OSGB36" - Airy 1936, UK National Grid (Ordnance Survey)
    "OSU91"  - OSU, 1991-a
    "WGS84"  - World Geodetic System 1984 [2]
   plus the non-elliposid cases:
    "SPHERE" - spherical earth (latitude independent)
    "ECF"    - earth-centred fixed (x,y,z in m)
   Datum names are case-insensitive. If datum name is not one of the
   above, Rm, Rp and Re are all returned as -9999999.0. For "SPHERE" and
   "ECF", all 3 radii will be the same and equal to a mean earth radius.
\end{Verbatim}
\textbf{REFERENCES:}\hspace{0.08in}\begin{Verbatim}
   [1] ACIC (1962). Geodesy for the layman. Geosciences Branch, Chart
       Research Division, Aeronautical Chart and Information Centre,
       St.Louis, Miss. January 1992.
   [2] DMA (1987). Department of Defense World Geodetic System 1984 -
       its definition and relationships with local geodetic systems.
       DMA Technical Report 8350.2, DMA, September 1987.
   [3] Kaye, CGW and TH Laby (1986). Tables of Physical and Chemical
       Constants, 15th Edn. Longman, UK.
   [4] Ruffhead, AC (1989). A 5-Parameter Transformation for OSGB36
       to WGS84. MCE Working Paper No. 7/89, Mapping and Charting
       Establishment, Computer and Geodesy Support Group, September 1989.
   [5] Geodetic Reference System 1980, Bulletin G�od�sique, Vol 54:3, 1980.
\end{Verbatim}
\section{EARTH/Datum\_HMSL}
\textsl{[ Subroutines ]}

\label{ch:robo56}
\label{ch:EARTH_Datum_HMSL}
\index{unsorted!Datum\_HMSL}\index{Subroutines!Datum\_HMSL}
\textbf{NAME:}\hspace{0.08in}\begin{Verbatim}
   Datum_HMSL     (earth.f90)
\end{Verbatim}
\textbf{SYNOPSIS:}\hspace{0.08in}\begin{Verbatim}
   Convert height on datum to geoid (msl)

   USE EarthMod
   real(wp) :: GEO(3), HMSL
   GEO(:) = (/ <lat>, <lon>, <ht> /) ! on WGS-84 ellipsoid
   CALL Datum_HMSL ( "WGS84", GEO, HMSL )
\end{Verbatim}
\textbf{ARGUMENTS:}\hspace{0.08in}\begin{Verbatim}
   Datum  (in)   chr   Name of datum
   GEO    (in)   dflt  Point coordinates wrt datum
                       for ellipsoid: lat/lon (deg)/ht (m)
                       for ECF: x/y/z (m)
   HMSL   (out)  dflt  Height above geoid (msl)
   FCOEFF (in, optional) chr Path to geoid potential coeffs file
   FCORR  (in, optional) chr Path to geoid potential corrections file

 ENVIRONMENT VARIABLES
   GEOPOT_COEF - path to geoid potential coeffs file (if not specified arg)
   GEOPOT_CORR - path to geoid corrections file (if not specified arg)
\end{Verbatim}
\textbf{CALLS:}\hspace{0.08in}\begin{Verbatim}
   Datum_Param
   Datum_Trans
   GEOPOT_LEGDFN
\end{Verbatim}
\textbf{USES:}\hspace{0.08in}\begin{Verbatim}
 ALGORITHM
   Uses NIMA EGM96 geoid coefficients wrt WGS-84 datum. These are
   in the form of geoid potential and correction coefficients to
   order and degree 360. The coefficients are expanded as Legendre
   polynomials and applied to the given location to calculate the
   local EGM96 undulation wrt WGS-84; the location Datum is also
   converted to the same ellipsoid if not already in that system.
   The HMSL value (strictly orthometric height) is simply equal to
   WGS-84 minus EGM96 heights.
   This routine can be validated by comparing its output with the
   online calculator linked from the NIMA/NASA website URL noted
   in Refs.
\end{Verbatim}
\textbf{DESCRIPTION:}\hspace{0.08in}\begin{Verbatim}
   Given the location of a point wrt an ellipsoid datum (e.g. WGS-84),
   or ECF X,Y,Z coordinates, return the location's height above mean
   sea level. MSL here is assumed to correspond with the local
   geoid undulation, given by a geoid model. The potential coeffs.
   for this model are read from a file pointed to by the environment
   variable 'GEOPOT_COEF' and the corrections file from 'GEOPOT_CORR'.
   Alternatively, it is possible to specify the files by passing optional
   arguments FCOEF and FCORR.
\end{Verbatim}
\textbf{REFERENCES:}\hspace{0.08in}\begin{Verbatim}
   Original package from Ohio State University for OSU91.
   EGM96 coefficients files can be obtained from the National Image
   and Mapping Agency (NIMA/NASA) website at:
     http://earth-info.nima.mil/GandG/wgs84/gravitymod/egm96/egm96.html
   For more information on EGM96, see also:
     http://cddis.gsfc.nasa.gov/926/egm96/egm96.html
\end{Verbatim}
\section{EARTH/Datum\_Param}
\textsl{[ Subroutines ]}

\label{ch:robo57}
\label{ch:EARTH_Datum_Param}
\index{unsorted!Datum\_Param}\index{Subroutines!Datum\_Param}
\textbf{NAME:}\hspace{0.08in}\begin{Verbatim}
   Datum_Param     (earth.f90)
\end{Verbatim}
\textbf{SYNOPSIS:}\hspace{0.08in}\begin{Verbatim}
   Return geoid parameters

   USE EarthMod
   real(wp) :: delXYZ(3), a, f
   CALL Datum_Param ( "OSGB36", delXYZ, a, f )
\end{Verbatim}
\textbf{ARGUMENTS:}\hspace{0.08in}\begin{Verbatim}
   Datum   (in)   chr   Name of datum
   delXYZ  (out)  dflt  Array(3) delta offsets in x,y,z (m)
   a       (out)  dflt  Semi-major axis (m)
   f       (out)  dflt  Flattening parameter
\end{Verbatim}
\textbf{CALLED BY:}\hspace{0.08in}\begin{Verbatim}
   Datum_HMSL
   Datum_Trans
   Datum_Erad
   ECFtoGEO
\end{Verbatim}
\textbf{DESCRIPTION:}\hspace{0.08in}\begin{Verbatim}
   Returns ellipsoid datum parameters. Datum is specified as a character
   string for any of the supported types:
    "ED50"   - European datum, 1950
    "GEM6"   - used for ERS orbits
    "GRS80"  - Geodetic Reference System, 1980 [5]
    "KL86"   - Kaye & Laby, 1986 [3]
    "NAD27"  - North American Datum, 1927 (Clark 1866)
    "OSGB36" - Airy 1936, UK National Grid (Ordnance Survey)
    "OSU91"  - OSU, 1991-a
    "WGS84"  - World Geodetic System 1984 [2]
   plus the non-elliposid cases:
    "SPHERE" - spherical earth (latitude independent)
    "ECF"    - earth-centred fixed (x,y,z in m)
   Datum names are case-insensitive. If datum is blank, WGS-84 is used;
   if datum is not blank but is not one of the above names, all
   parameters are all returned as -9999999. For "SPHERE" and "ECF",
   'a' represents mean earth radius value and 'f' is zero. NB: 'a' is
   returned as a -ve value for "ECF" type.
\end{Verbatim}
\textbf{REFERENCES:}\hspace{0.08in}\begin{Verbatim}
   [1] ACIC (1962). Geodesy for the layman. Geosciences Branch, Chart
       Research Division, Aeronautical Chart and Information Centre,
       St.Louis, Miss. January 1992.
   [2] DMA (1987). Department of Defense World Geodetic System 1984 -
       its definition and relationships with local geodetic systems.
       DMA Technical Report 8350.2, DMA, September 1987.
   [3] Kaye, CGW and TH Laby (1986). Tables of Physical and Chemical
       Constants, 15th Edn. Longman, UK.
   [4] Ruffhead, AC (1989). A 5-Parameter Transformation for OSGB36
       to WGS84. MCE Working Paper No. 7/89, Mapping and Charting
       Establishment, Computer and Geodesy Support Group, September 1989.
   [5] Geodetic Reference System 1980, Bulletin G�od�sique, Vol 54:3, 1980.
\end{Verbatim}
\section{EARTH/Datum\_Trans}
\textsl{[ Subroutines ]}

\label{ch:robo58}
\label{ch:EARTH_Datum_Trans}
\index{unsorted!Datum\_Trans}\index{Subroutines!Datum\_Trans}
\textbf{NAME:}\hspace{0.08in}\begin{Verbatim}
   Datum_Trans     (earth.f90)
\end{Verbatim}
\textbf{SYNOPSIS:}\hspace{0.08in}\begin{Verbatim}
   Transform geographic coordinates from one ellipsoid datum to another

   USE EarthMod
   real(wp) :: GEO1(3), GEO2(3)
   GEO1(:) = (/ <lat>, <lon>, <ht> /) ! on WGS-84 ellipsoid
   CALL Datum_Trans ( "WGS84", GEO1, "OSGB36", GEO2 )
   [GEO2 contains lat/lon/ht on OS-GB (Airy 1936) spheroid]
\end{Verbatim}
\textbf{ARGUMENTS:}\hspace{0.08in}\begin{Verbatim}
   Datum1  (in)   chr   Name of datum to transform 'from'
   GEO1    (in)   dflt  Array(3) cordinates in Datum1
                        (1) latitude  (deg) [X (m) if ECF]
                        (2) longitude (deg) [Y (m) if ECF]
                        (3) height    (m)   [Z (m) if ECF]
   Datum2  (in)   chr   Name of datum to transform 'to'
   GEO2    (out)  dflt  Array(3) coordinates in Datum2
                        (1) latitude  (deg) [X (m) if ECF]
                        (2) longitude (deg) [Y (m) if ECF]
                        (3) height    (m)   [Z (m) if ECF]
\end{Verbatim}
\textbf{CALLS:}\hspace{0.08in}\begin{Verbatim}
   Datum_Erad
   Datum_Param
   ECFtoGEO
\end{Verbatim}
\textbf{CALLED BY:}\hspace{0.08in}\begin{Verbatim}
   Datum_HMSL

 ALGORITHM
   Simplified Molodensky formulae [4]
\end{Verbatim}
\textbf{DESCRIPTION:}\hspace{0.08in}\begin{Verbatim}
   Transforms latitude, longitude and height relative to one (ellipsoid)
   datum to another datum. Input and output datums are specified as
   character strings for any of the supported types:
    "ED50"   - European datum, 1950
    "GEM6"   - used for ERS orbits
    "GRS80"  - Geodetic Reference System, 1980 [5]
    "KL86"   - Kaye & Laby, 1986 [3]
    "NAD27"  - North American Datum, 1927 (Clark 1866)
    "OSGB36" - Airy 1936, UK National Grid (Ordnance Survey)
    "OSU91"  - OSU, 1991-a
    "WGS84"  - World Geodetic System 1984 [2]
   plus the non-ellipsoid cases:
    "SPHERE" - spherical earth (latitude independent)
    "ECF"    - earth-centred fixed (x,y,z in m)
   For datum "ECF", the associated 'GEO1' or 'GEO2' array elements are
   interpreted as X,Y,Z distances from the Earth's centre.
   Datum names are case-insensitive. If Datum1 name is not one of the
   above, GEO2 array is returned with all elemnts -999999.0; if Datum2
   name is unknown, the returned values are +999999.0.
\end{Verbatim}
\textbf{REFERENCES:}\hspace{0.08in}\begin{Verbatim}
   [1] ACIC (1962). Geodesy for the layman. Geosciences Branch, Chart
       Research Division, Aeronautical Chart and Information Centre,
       St.Louis, Miss. January 1992.
   [2] DMA (1987). Department of Defense World Geodetic System 1984 -
       its definition and relationships with local geodetic systems.
       DMA Technical Report 8350.2, DMA, September 1987.
   [3] Kaye, CGW and TH Laby (1986). Tables of Physical and Chemical
       Constants, 15th Edn. Longman, UK.
   [4] Ruffhead, AC (1989). A 5-Parameter Transformation for OSGB36
       to WGS84. MCE Working Paper No. 7/89, Mapping and Charting
       Establishment, Computer and Geodesy Support Group, September 1989.
   [5] Geodetic Reference System 1980, Bulletin G�od�sique, Vol 54:3, 1980.
\end{Verbatim}
\section{EARTH/EarthMod}
\textsl{[ Modules ]}

\label{ch:robo59}
\label{ch:EARTH_EarthMod}
\index{unsorted!EarthMod}\index{Modules!EarthMod}
\textbf{NAME:}\hspace{0.08in}\begin{Verbatim}
   EarthMod     (earth.f90)
\end{Verbatim}
\textbf{SYNOPSIS:}\hspace{0.08in}\begin{Verbatim}
   Module defining Earth-related constants
   and datum conversion routines.

   USE EarthMod
   CHARACTER (LEN=6)  :: Datum1,  Datum2
   CHARACTER (LEn=12) :: NGref
   DOUBLE PRECISION   :: GEO1(3), GEO2(3), ECF(3), delXYZ(3)
   DOUBLE PRECISION   :: rm, rp, re, a, f
   DOUBLE PRECISION   :: rlat, rleg(nmax+1)
   DOUBLE PRECISION   :: Latitude1, Longitude1
   DOUBLE PRECISION   :: Latitude2, Longitude2
   real(wp)   :: DistNorth, DistEast, Easting, Northing
   REAL               :: Lat1, Lon1, Lat2, Lon2, Distance, Bearing, Ht
   INTEGER            :: m, nmax, iflg

  - Ellipsoid/geoid datum routines:

   CALL Datum_Trans  ( Datum1, GEO1, Datum2, GEO2 )
   CALL Datum_Erad   ( Datum1, Latitude1, rm, rp, re )
   CALL Datum_Param  ( Datum1, delXYZ, A, F )
   CALL Datum_HMSL   ( Datum1, GEO1, HMSL )
   CALL Geopot_LegFn ( m, rlat, nmax, rleg )
   CALL ECFtoGEO     ( ECF, Datum1, GEO, iflg )

  - Earth physics:

   RAD = EradL      ( Lat1 )         [See also Datum_Erad]
   g   = EGravity    ( Lat1, Ht )

 CONTAINS
   Datum_Trans
   Datum_Erad
   Datum_Param
   Datum_HMSL
   Geopot_LegFn
   ECFtoGEO
   EradL
   EGravity

 PARAMETERS
   BIGNEG   = -9999999.0D0             ! Missing data value
   SMALL    = 1.0D-8                   ! Something close to zero
   Pi       = 3.14159265358979323846D0 ! Pi
   HalfPi   = Pi / 2.0D0               ! Pi / 2
   TwoPi    = Pi * 2.0D0               ! Pi * 2
   RadToDeg = 180.0D0 / Pi             ! Radians to degrees
   DegToRad = Pi / 180.0D0             ! Degrees to radians
\end{Verbatim}
\textbf{DESCRIPTION:}\hspace{0.08in}\begin{Verbatim}
   Defines (a) physical constants related to angle conversion,
           (b) various routines to transform a datum (location
               on the Earth) between different coordinate systems
           (c) various routines to calculate point to point parameters
               (distance, bearing...)
           (d) Earth's physical parameters (radius, gravity)
\end{Verbatim}
\section{EARTH/ECFtoGEO}
\textsl{[ Subroutines ]}

\label{ch:robo60}
\label{ch:EARTH_ECFtoGEO}
\index{unsorted!ECFtoGEO}\index{Subroutines!ECFtoGEO}
\textbf{NAME:}\hspace{0.08in}\begin{Verbatim}
   ECFtoGEO     (earth.f90)
\end{Verbatim}
\textbf{SYNOPSIS:}\hspace{0.08in}\begin{Verbatim}
   Convert between ECF coordinates & geodetic position

   USE EarthMod
   real(wp) :: EFC(3), GEO(3)
   ECF(:) = (/ <X>, <Y>, <Z> /)   !   location wrt Earth's centre
   CALL ECFtoGEO ( ECF, "WGS84", GEO, 1 )
   [GEO contains lat/lon/ht on WGS-84 ellipsoid]
\end{Verbatim}
\textbf{ARGUMENTS:}\hspace{0.08in}\begin{Verbatim}
   ECF    (inout)  dflt  Array(3) ECF x/y/z coordinates (m)
   Datum  (in)     chr   Name of datum for geo coords
   GEO    (inout)  dflt  Array(3) geodetic position
                         Lat/lon (deg) & altitude (m)
   iflg   (in)     int   Conversion flag:
                          >  0 : ecf --> geo
                          <= 0 : geo --> ecf
\end{Verbatim}
\textbf{CALLS:}\hspace{0.08in}\begin{Verbatim}
   Datum_Param
\end{Verbatim}
\textbf{CALLED BY:}\hspace{0.08in}\begin{Verbatim}
   Datum_Trans
\end{Verbatim}
\textbf{DESCRIPTION:}\hspace{0.08in}\begin{Verbatim}
   Converts between earth-centred fixed (ECF), X,Y,Z coordinates
   (in m from Earth's centre) and geodetic latitude, longitude
   (deg) and altitude above the Earth's surface (m) relative to
   the specified (ellipsoid) datum. The direction of conversion
   depends on iflg.
   Datum is specified as a character string for any of the supported
   types:
    "ED50"   - European datum, 1950
    "GEM6"   - used for ERS orbits
    "GRS80"  - Geodetic Reference System, 1980 [5]
    "KL86"   - Kaye & Laby, 1986 [3]
    "NAD27"  - North American Datum, 1927 (Clark 1866)
    "OSGB36" - Airy 1936, UK National Grid (Ordnance Survey)
    "OSU91"  - OSU, 1991-a
    "WGS84"  - World Geodetic System 1984 [2]
   plus the non-elliposid cases:
    "SPHERE" - spherical earth (latitude independent)
    "ECF"    - earth-centred fixed (x,y,z in m)
   Datum names are case-insensitive. If datum is blank, WGS-84 is used
\end{Verbatim}
\textbf{REFERENCES:}\hspace{0.08in}\begin{Verbatim}
   The 1992 Astronomical Almanac, page K12.
   See also numbered references under routine Datum_Param
\end{Verbatim}
\section{EARTH/EGravity}
\textsl{[ Subroutines ]}

\label{ch:robo61}
\label{ch:EARTH_EGravity}
\index{unsorted!EGravity}\index{Subroutines!EGravity}
\textbf{NAME:}\hspace{0.08in}\begin{Verbatim}
   EGravity     (earth.f90)
\end{Verbatim}
\textbf{SYNOPSIS:}\hspace{0.08in}\begin{Verbatim}
   Gravity (m.s^-2) as a function of latitude (deg) and height
   (m amsl).

   USE EarthMod
   REAL :: g
   g = EGravity ( 50.0, 5000.0 )

 ALGORITHM
   g = g0[1-c1.cos(lat)-c2.ht]  (see ref.)
\end{Verbatim}
\textbf{DESCRIPTION:}\hspace{0.08in}\begin{Verbatim}
   Calculates gravity, g, as a simple function of latitude and
   height above a reference surface (nominally WGS-84).
\end{Verbatim}
\textbf{REFERENCES:}\hspace{0.08in}\begin{Verbatim}
   Baker, HC (1998): GPS water vapour estimation for meteorological
          applications. PhD Thesis, University of Nottingham.
\end{Verbatim}
\section{EARTH/EradL}
\textsl{[ Subroutines ]}

\label{ch:robo62}
\label{ch:EARTH_EradL}
\index{unsorted!EradL}\index{Subroutines!EradL}
\textbf{NAME:}\hspace{0.08in}\begin{Verbatim}
   EradL     (earth.f90)
\end{Verbatim}
\textbf{SYNOPSIS:}\hspace{0.08in}\begin{Verbatim}
   Calculate the Earth's radius as a simple function of latitude.

   USE EarthMod
   REAL Re
   Re = EradL ( 45.0 )
\end{Verbatim}
\textbf{ARGUMENTS:}\hspace{0.08in}\begin{Verbatim}
   Latitude  (in)  flt  Latitude (deg)

 OUTPUTS
   EradL   (func)  flt  Earth's radius at given latitude (km)
\end{Verbatim}
\textbf{CALLED BY:}\hspace{0.08in}\begin{Verbatim}
   LLtoDB

 ALGORITHM
   re(lat) = r0 + r1.cos(2.lat) * r2.cos(4.lat)
\end{Verbatim}
\textbf{DESCRIPTION:}\hspace{0.08in}\begin{Verbatim}
   Returns radius of earth given the latitude (north or south).
   Uses a simple non-ellipsoid model. For ellipsoid radius, see
   Datum_Erad
\end{Verbatim}
\section{EARTH/Geopot\_LegFn}
\textsl{[ Subroutines ]}

\label{ch:robo63}
\label{ch:EARTH_Geopot_LegFn}
\index{unsorted!Geopot\_LegFn}\index{Subroutines!Geopot\_LegFn}
\textbf{NAME:}\hspace{0.08in}\begin{Verbatim}
   Geopot_LegFn     (earth.f90)
\end{Verbatim}
\textbf{SYNOPSIS:}\hspace{0.08in}\begin{Verbatim}
   Compute legendre polynomials for geopotential

   USE EarthMod
   real(wp) :: COLAT, RLEG(NMAX+1)
   INTEGER          :: M, NMAX
   CALL Geopot_LegFn ( M, COLAT, NMAX, RLEG )
\end{Verbatim}
\textbf{ARGUMENTS:}\hspace{0.08in}\begin{Verbatim}
   M      (in)   int   Order of legendre functions
   COLAT  (in)   dflt  Co-latitude (rad)
   NMAX   (in)   int   Maximum degree of functions
   RLEG   (out)  dflt  Array(nmax+1) of legendre function values
\end{Verbatim}
\textbf{CALLED BY:}\hspace{0.08in}\begin{Verbatim}
   Datum_HMSL
\end{Verbatim}
\textbf{DESCRIPTION:}\hspace{0.08in}\begin{Verbatim}
   This subroutine computes all normalized Legendre functions in RLEG
   for order M and co-latitude COLAT (radians). Maximum degree is NMAX.
   The dimensions of array RLEG must be at least equal to NMAX+1.
   Original algorithm & code from Ohio State University.
\end{Verbatim}
\section{Geodesy/geodesy}
\textsl{[ Modules ]}

\label{ch:robo64}
\label{ch:Geodesy_geodesy}
\index{unsorted!geodesy}\index{Modules!geodesy}
\textbf{NAME:}\hspace{0.08in}\begin{Verbatim}
    geodesy - Module interfacing various geodetic routines and functions.
\end{Verbatim}
\textbf{SYNOPSIS:}\hspace{0.08in}\begin{Verbatim}
    use geodesy
\end{Verbatim}
\textbf{DESCRIPTION:}\hspace{0.08in}\begin{Verbatim}
    This module provides interfaces to some geodetic funtions and
    subroutines of the ropp_utils library.
\end{Verbatim}
\textbf{SEE ALSO:}\hspace{0.08in}\begin{Verbatim}
    R_eff
    gravity
    geometric2geopotential
    geopotential2geometric
\end{Verbatim}
\section{Geodesy/geometric2geopotential}
\textsl{[ Functions ]}

\label{ch:robo65}
\label{ch:Geodesy_geometric2geopotential}
\index{unsorted!geometric2geopotential}\index{Functions!geometric2geopotential}
\textbf{NAME:}\hspace{0.08in}\begin{Verbatim}
    geometric2geopotential - Calculate geopotential height as function of 
                             altitude using the expressions derived from 
                             the Somigliana equation
\end{Verbatim}
\textbf{SYNOPSIS:}\hspace{0.08in}\begin{Verbatim}
    geop = geometric2geopotential(lat, h)
\end{Verbatim}
\textbf{DESCRIPTION:}\hspace{0.08in}\begin{Verbatim}
    This subroutine calculates geopotential height as
    function of altitude.
\end{Verbatim}
\textbf{INPUTS:}\hspace{0.08in}\begin{Verbatim}
    lat           geographical latitude (in degree north).
    h             altitude (in m).
\end{Verbatim}
\textbf{OUTPUT:}\hspace{0.08in}\begin{Verbatim}
    geopotential  geopotential height (in geopotential metres).
\end{Verbatim}
\textbf{NOTES:}\hspace{0.08in}\begin{Verbatim}
    Note Somigliana's Equation is defined for normal gravity on the surface 
    of an ellipsoid while geopotential height is strictly defined relative to
    an equi-potential surface for the geopotential (i.e. the geoid). 
    These routines preserve the datum used for the input height. 
      i.e.
 
        If INPUT: h wrt ellipsoid -> OUTPUT: Z wrt ellipsoid
        The output is not strictly a geopotential height (relative to an equi-
        potential surface). To compute the geopotential height relative to 
        the EGM96 geoid, it is necessary for users to subtract the undulation
        from the output 'geopotential height'.

        If INPUT: h wrt geoid -> OUTPUT: Z wrt geoid
        In this case, the output is a geopotential height relative to the 
        geoid. Note Somagliana's equation is strictly based on computations
        relative to a reference ellipsoid, but the error in assuming these 
        apply for conversions wrt geoid is small.
\end{Verbatim}
\textbf{SEE ALSO:}\hspace{0.08in}\begin{Verbatim}
    R_eff
    gravity_msl
    geopotential2geometric
\end{Verbatim}
\textbf{REFERENCES:}\hspace{0.08in}\begin{Verbatim}
    Mahoney, M.J., A discussion of various measures of altitudes,
       URL: http://mtp.jpl.nasa.gov/notes/altitude/altitude.html,
       cited 29/06/2004.
\end{Verbatim}
\section{Geodesy/geopotential2geometric}
\textsl{[ Functions ]}

\label{ch:robo66}
\label{ch:Geodesy_geopotential2geometric}
\index{unsorted!geopotential2geometric}\index{Functions!geopotential2geometric}
\textbf{NAME:}\hspace{0.08in}\begin{Verbatim}
    geopotential2geometric - Calculate geometric altitude 
                             from geopotential height using expressions 
                             derived from the Somigliana equation
\end{Verbatim}
\textbf{SYNOPSIS:}\hspace{0.08in}\begin{Verbatim}
    h = geopotential2geometric(lat, geop)
\end{Verbatim}
\textbf{DESCRIPTION:}\hspace{0.08in}\begin{Verbatim}
    This subroutine calculates geometric altitude
    from geopotential height 
    [see http://mtp.jpl.nasa.gov/notes/altitude/altitude.html for details]
\end{Verbatim}
\textbf{INPUTS:}\hspace{0.08in}\begin{Verbatim}
    lat           geographical latitude (in degrees north).
    geopotential  geopotential height (in geopotential metres).
\end{Verbatim}
\textbf{OUTPUT:}\hspace{0.08in}\begin{Verbatim}
    h             altitude above reference ellipsoid (in m).
\end{Verbatim}
\textbf{NOTES:}\hspace{0.08in}\begin{Verbatim}
    Note Somigliana's Equation is defined for normal gravity on the surface 
    of an ellipsoid while geopotential height is strictly defined relative to
    an equi-potential surface for the geopotential (i.e. the geoid). 
    These routines preserve the datum used for the input height. 
      i.e.
 
        If INPUT: Z wrt ellipsoid -> OUTPUT: h wrt ellipsoid
        Typically, geometric altitude is expressed relative to a reference 
        geoid. To compute the geometric height relative to 
        the EGM96 geoid, it is necessary for users to subtract the undulation
        from the output geometric height in this case.

        If INPUT: Z wrt geoid -> OUTPUT: h wrt geoid
        In this case, the output is a geometric height relative to the 
        geoid. Note Somagliana's equation is strictly based on computations
        relative to a reference ellipsoid, but the error in assuming these 
        apply for conversions wrt geoid is small.
\end{Verbatim}
\textbf{SEE ALSO:}\hspace{0.08in}\begin{Verbatim}
    R_eff
    gravity
    geopotential2geometric
\end{Verbatim}
\textbf{REFERENCES:}\hspace{0.08in}\begin{Verbatim}
    Mahoney, M.J., A discussion of various measures of altitudes,
       URL: http://mtp.jpl.nasa.gov/notes/altitude/altitude.html,
       cited 29/06/2004.
\end{Verbatim}
\section{Geodesy/gravity}
\textsl{[ Functions ]}

\label{ch:robo67}
\label{ch:Geodesy_gravity}
\index{unsorted!gravity}\index{Functions!gravity}
\textbf{NAME:}\hspace{0.08in}\begin{Verbatim}
    gravity - Calculate gravity following the Somigliana equation
\end{Verbatim}
\textbf{SYNOPSIS:}\hspace{0.08in}\begin{Verbatim}
    g_lat = gravity( lat, [ h ] )
\end{Verbatim}
\textbf{DESCRIPTION:}\hspace{0.08in}\begin{Verbatim}
    This function returns gravity as function of geographic latitude and 
    (optionally) altitude above the WGS-84 reference ellipsoid as derived 
    from the Somigliana equation 
    [see http://mtp.jpl.nasa.gov/notes/altitude/altitude.html for details]
\end{Verbatim}
\textbf{INPUTS:}\hspace{0.08in}\begin{Verbatim}
    lat     geographical latitude (in degree north).
    h       altitude above reference ellipsod (in m, optional). If z is not
               given, the function returns the gravity at the ellipsoid.
\end{Verbatim}
\textbf{OUTPUT:}\hspace{0.08in}\begin{Verbatim}
    g       gravity (in m/s^2).
\end{Verbatim}
\textbf{NOTES:}\hspace{0.08in}\textbf{SEE ALSO:}\hspace{0.08in}\begin{Verbatim}
    R_eff
    geopotential2geometric
    geometric2geopotential
\end{Verbatim}
\textbf{REFERENCES:}\hspace{0.08in}\begin{Verbatim}
    Mahoney, M.J., A discussion of various measures of altitudes,
       URL: http://mtp.jpl.nasa.gov/notes/altitude/altitude.html,
       cited 29/06/2004.
\end{Verbatim}
\section{Geodesy/great\_circle\_distance}
\textsl{[ Functions ]}

\label{ch:robo68}
\label{ch:Geodesy_great_circle_distance}
\index{unsorted!great\_circle\_distance}\index{Functions!great\_circle\_distance}
\textbf{NAME:}\hspace{0.08in}\begin{Verbatim}
    great_circle_distance - Calculate great circle distance between two
                              points on Earth.
\end{Verbatim}
\textbf{SYNOPSIS:}\hspace{0.08in}\begin{Verbatim}
    dist = great_circle_distance(lon1, lat1, lon2, lat2)
\end{Verbatim}
\textbf{DESCRIPTION:}\hspace{0.08in}\begin{Verbatim}
    This function calculates the great circle distance between two points
    on a sphere, given by their (geodetic) longitude and latitude coordinates,
    respectively
\end{Verbatim}
\textbf{INPUTS:}\hspace{0.08in}\begin{Verbatim}
    lon1   Longitude of first point on the sphere / Earth (in degree east).
    lat1   Latitude of the first point (in degree north).
    lon2   Longitude of the second point (in degree east).
    lat2   Latitude of the second point (in degree north).
\end{Verbatim}
\textbf{OUTPUT:}\hspace{0.08in}\begin{Verbatim}
    dist   Great circle distance between (lon1, lat1) and (lon2, lat2) (in meter).
\end{Verbatim}
\textbf{NOTES:}\hspace{0.08in}\begin{Verbatim}
    It is assumed that Earth is a sphere with an average great-circle radius
    of 6372.795 km. No corrections are applied for the ellipsoidal shape of
    the Earth.
\end{Verbatim}
\textbf{EXAMPLE:}\hspace{0.08in}\textbf{SEE ALSO:}\hspace{0.08in}\textbf{REFERENCES:}\hspace{0.08in}\begin{Verbatim}
    Formulation and default values were taken from

       http://en.wikipedia.org/wiki/Great-circle_distance

    cited in December 2006.
\end{Verbatim}
\section{Geodesy/R\_eff}
\textsl{[ Functions ]}

\label{ch:robo69}
\label{ch:Geodesy_R_eff}
\index{unsorted!R\_eff}\index{Functions!R\_eff}
\textbf{NAME:}\hspace{0.08in}\begin{Verbatim}
    R_eff - Effective radius of Earth for geopotential height conversion
               following Somigliana's equation
\end{Verbatim}
\textbf{SYNOPSIS:}\hspace{0.08in}\begin{Verbatim}
    R_eff = R_eff(lat)
\end{Verbatim}
\textbf{DESCRIPTION:}\hspace{0.08in}\begin{Verbatim}
    This function calculates the effective radius of Earth relative to 
    WGS-84 reference ellipsoid
\end{Verbatim}
\textbf{INPUTS:}\hspace{0.08in}\begin{Verbatim}
    lat    geographical latitude (in degree north).
\end{Verbatim}
\textbf{OUTPUT:}\hspace{0.08in}\begin{Verbatim}
    R_lat  effective Earth radius (in m).
\end{Verbatim}
\textbf{NOTES:}\hspace{0.08in}\textbf{SEE ALSO:}\hspace{0.08in}\begin{Verbatim}
    gravity_msl
    geopotential2geometric
    geometric2geopotential
\end{Verbatim}
\textbf{REFERENCES:}\hspace{0.08in}\begin{Verbatim}
    Mahoney, M.J., A discussion of various measures of altitudes,
       URL: http://mtp.jpl.nasa.gov/notes/altitude/altitude.html,
       cited 29/06/2004.
\end{Verbatim}
\section{Initialisation/ropp\_unit\_conversion}
\textsl{[ Functions ]}

\label{ch:robo70}
\label{ch:Initialisation_ropp_unit_conversion}
\index{unsorted!ropp\_unit\_conversion}\index{Functions!ropp\_unit\_conversion}
\textbf{NAME:}\hspace{0.08in}\begin{Verbatim}
   ropp_unit_conversion - unit conversion
\end{Verbatim}
\textbf{SYNOPSIS:}\hspace{0.08in}\begin{Verbatim}
   CALL ropp_unit_conversion ( from_unit, to_unit, slope, intercept )
\end{Verbatim}
\textbf{INPUTS:}\hspace{0.08in}\begin{Verbatim}
   from_unit      Original data unit
   to_unit        Required output data unit

 OUTPUTS
   slope          Scaling factor for unit conversion
   intercept      Offset for unit conversion
\end{Verbatim}
\textbf{USES:}\hspace{0.08in}\begin{Verbatim}
   typsesizes
   To_Lower
\end{Verbatim}
\textbf{DESCRIPTION:}\hspace{0.08in}\section{Mathematics/Pi}
\textsl{[ Parameters ]}

\label{ch:robo71}
\label{ch:Mathematics_Pi}
\index{unsorted!Pi}\index{Parameters!Pi}
\textbf{NAME:}\hspace{0.08in}\begin{Verbatim}
    Pi - Mathematical constant
\end{Verbatim}
\textbf{SOURCE:}\hspace{0.08in}\begin{Verbatim}
  real(wp), parameter :: Pi = 3.141592653589793238_wp
\end{Verbatim}
\section{Messages/assert}
\textsl{[ Subroutines ]}

\label{ch:robo72}
\label{ch:Messages_assert}
\index{unsorted!assert}\index{Subroutines!assert}
\textbf{NAME:}\hspace{0.08in}\begin{Verbatim}
    assert - Make sure that a condition is fulfilled.
\end{Verbatim}
\textbf{SYNOPSIS:}\hspace{0.08in}\begin{Verbatim}
    call assert(condition, string [, severity])
\end{Verbatim}
\textbf{DESCRIPTION:}\hspace{0.08in}\begin{Verbatim}
    This subroutine provides an assertion facility, i.e. it issues a
    (fatal by default) error message if the condition if the condition
    is NOT fullfilled.
\end{Verbatim}
\textbf{INPUTS:}\hspace{0.08in}\begin{Verbatim}
    logical condition
    char*   string
    int     severity
\end{Verbatim}
\textbf{OUTPUT:}\hspace{0.08in}\begin{Verbatim}
    If the test fails (i.e., the condition evaluates to .false.), the
    text string is printed as part of a fatal error message. This also
    terminates the program.
\end{Verbatim}
\textbf{NOTES:}\hspace{0.08in}\begin{Verbatim}
    If an informational or warning message is desired instead of a fatal
    error, the severity of the failure can be specified with the optional
    argument. It is intended to receive one of msg_info, msg_warn, or
    msg_error (all defined in the messages module).
\end{Verbatim}
\textbf{SEE ALSO:}\hspace{0.08in}\begin{Verbatim}
   message
   message_set_program
   message_get_program
   message_set_routine
   message_get_routine
   message_set_addinfo
   message_get_addinfo
\end{Verbatim}
\section{Messages/message}
\textsl{[ Subroutines ]}

\label{ch:robo73}
\label{ch:Messages_message}
\index{unsorted!message}\index{Subroutines!message}
\textbf{NAME:}\hspace{0.08in}\begin{Verbatim}
    message - Issue an information / warning / error message.
            --> BASED ON MARQUARDT COLLECTION LIBRARY ROUTINE
\end{Verbatim}
\textbf{SYNOPSIS:}\hspace{0.08in}\begin{Verbatim}
   use ropp_utils
     ...
   call message(type, text [, status])
     ...
\end{Verbatim}
\textbf{DESCRIPTION:}\hspace{0.08in}\begin{Verbatim}
    This subroutine prints an error message. For fatal errors, the program
    is aborted, and an exit status is set. The output will look like this:

       MESSAGE_TYPE (<prog>/<routine>/<add>):
          Text of the message, which might have multiple lines of text.

    where the MESSAGE_TYPE is determined by the value of the type argument.
    Fatal error messages result in termination of the program.
\end{Verbatim}
\textbf{INPUTS:}\hspace{0.08in}\begin{Verbatim}
   int   type    The type of the message which can be msg_cont, msg_info,
                   msg_warn, msg_error or msg_fatal (as defined in the
                   module messages). The following values are supported:

                     msg_cont    This will continue the previous message
                                   (and provides a way to build up multi
                                   line informational or warning messages).
                     msg_info    Informational message; message starts with
                                   'INFO:'
                     msg_warn    Warning message; message starts with
                                   'WARNING:'
                     msg_error   Error message; message starts with
                                   'ERROR:'
                     msg_fatal   Fatal error message; message starts with
                                   'FATAL ERROR:'

                   For fatal errors, the program will terminate with an
                   an error code as given in status if the Fortran compiler
                   supports a 'call exit(status)'.

   char* text    Error (or informational) message to be printed. This is
                   a text string which may have embedded C-style control
                   characters. Currently supported are the following ones:

                     \n   newline; use for multi-line messages
                     \t   tabulator
                     \'   single and...
                     \"   ...double hyphens
                     \a   terminal bell (i.e., a 'beep')
                     \b   backspace
                     \\   backslash

   int   status  Unix error status in case of a fatal error. This argument is
                   optional.
\end{Verbatim}
\textbf{OUTPUT:}\hspace{0.08in}\begin{Verbatim}
   Text is printed to stdout or stderr.
\end{Verbatim}
\textbf{NOTES:}\hspace{0.08in}\begin{Verbatim}
   The MESSAGE_TYPE qualifiers in the first line of the message are determined
   by the value of the type parameter; use one of the parameters provided in
   the module messages (as described above). The additional qualifiers after
   the main one can be set via the additional routines message_set_program(),
   message_set_routine() and message_set_addinfo(). If those qualifiers are
   set to ' ', they will not be printed. Similar *_set_* routines exist to
   enquire about the current setting of the additional qualifiers.

   Multiline texts can be generated by adding '\n' for line breaks with a
   single call to message. Note, however, that the routine prints three blanks
   in front of the error string; for multiline error messages, these must be
   added manually after the '\n'. An alternative way to create non-fatal 
   multiline messages is to use the routine repeatedly with the 'msg_cont'
   type parameter to write out individual lines.

   The default exit status (if status is not specified) in case of fatal errors
   is -1.
\end{Verbatim}
\textbf{SEE ALSO:}\hspace{0.08in}\begin{Verbatim}
   message_set_program
   message_get_program
   message_set_routine
   message_get_routine
   message_set_addinfo
   message_get_addinfo
\end{Verbatim}
\section{Messages/message\_bs\_escape}
\textsl{[ Subroutines ]}

\label{ch:robo74}
\label{ch:Messages_message_bs_escape}
\index{unsorted!message\_bs\_escape}\index{Subroutines!message\_bs\_escape}
\textbf{NAME:}\hspace{0.08in}\begin{Verbatim}
    message - Issue an information / warning / error message.
            --> BASED ON MARQUARDT COLLECTION LIBRARY ROUTINE
\end{Verbatim}
\textbf{SYNOPSIS:}\hspace{0.08in}\begin{Verbatim}
   use ropp_utils
     ...
   call message(type, text [, status])
     ...
\end{Verbatim}
\textbf{DESCRIPTION:}\hspace{0.08in}\begin{Verbatim}
    This subroutine prints an error message. For fatal errors, the program
    is aborted, and an exit status is set. The output will look like this:

       MESSAGE_TYPE (<prog>/<routine>/<add>):
          Text of the message, which might have multiple lines of text.

    where the MESSAGE_TYPE is determined by the value of the type argument.
    Fatal error messages result in termination of the program.
\end{Verbatim}
\textbf{INPUTS:}\hspace{0.08in}\begin{Verbatim}
   int   type    The type of the message which can be msg_cont, msg_info,
                   msg_warn, msg_error or msg_fatal (as defined in the
                   module messages). The following values are supported:

                     msg_cont    This will continue the previous message
                                   (and provides a way to build up multi
                                   line informational or warning messages).
                     msg_info    Informational message; message starts with
                                   'INFO:'
                     msg_warn    Warning message; message starts with
                                   'WARNING:'
                     msg_error   Error message; message starts with
                                   'ERROR:'
                     msg_fatal   Fatal error message; message starts with
                                   'FATAL ERROR:'

                   For fatal errors, the program will terminate with an
                   an error code as given in status if the Fortran compiler
                   supports a 'call exit(status)'.

   char* text    Error (or informational) message to be printed. This is
                   a text string which may have embedded C-style control
                   characters. Currently supported are the following ones:

                     \n   newline; use for multi-line messages
                     \t   tabulator
                     \'   single and...
                     \"   ...double hyphens
                     \a   terminal bell (i.e., a 'beep')
                     \b   backspace
                     \\   backslash

   int   status  Unix error status in case of a fatal error. This argument is
                   optional.
\end{Verbatim}
\textbf{OUTPUT:}\hspace{0.08in}\begin{Verbatim}
   Text is printed to stdout or stderr.
\end{Verbatim}
\textbf{NOTES:}\hspace{0.08in}\begin{Verbatim}
   The MESSAGE_TYPE qualifiers in the first line of the message are determined
   by the value of the type parameter; use one of the parameters provided in
   the module messages (as described above). The additional qualifiers after
   the main one can be set via the additional routines message_set_program(),
   message_set_routine() and message_set_addinfo(). If those qualifiers are
   set to ' ', they will not be printed. Similar *_set_* routines exist to
   enquire about the current setting of the additional qualifiers.

   Multiline texts can be generated by adding '\n' for line breaks with a
   single call to message. Note, however, that the routine prints three blanks
   in front of the error string; for multiline error messages, these must be
   added manually after the '\n'. An alternative way to create non-fatal 
   multiline messages is to use the routine repeatedly with the 'msg_cont'
   type parameter to write out individual lines.

   The default exit status (if status is not specified) in case of fatal errors
   is -1.
\end{Verbatim}
\textbf{SEE ALSO:}\hspace{0.08in}\begin{Verbatim}
   assert
   message_set_program
   message_get_program
   message_set_routine
   message_get_routine
   message_set_addinfo
   message_get_addinfo
\end{Verbatim}
\section{Messages/message\_get\_addinfo}
\textsl{[ Subroutines ]}

\label{ch:robo75}
\label{ch:Messages_message_get_addinfo}
\index{unsorted!message\_get\_addinfo}\index{Subroutines!message\_get\_addinfo}
\textbf{NAME:}\hspace{0.08in}\begin{Verbatim}
    message_get_addinfo - Get an additional information string used by the
                          ropp_utils message anderror handling system.
\end{Verbatim}
\textbf{SYNOPSIS:}\hspace{0.08in}\begin{Verbatim}
    call message_get_addinfo(progname)
\end{Verbatim}
\textbf{DESCRIPTION:}\hspace{0.08in}\begin{Verbatim}
    This routine gets the additional info string that appears in error messages
    issued by the message and error handling system of the ropp_utils library.
\end{Verbatim}
\textbf{INPUTS:}\hspace{0.08in}\begin{Verbatim}
    character(len = *) :: addinfo
\end{Verbatim}
\textbf{SEE ALSO:}\hspace{0.08in}\begin{Verbatim}
    assert
    message
    message_set_program
    message_get_program
    message_set_routine
    message_get_routine
    message_set_addinfo
\end{Verbatim}
\section{Messages/message\_get\_program}
\textsl{[ Subroutines ]}

\label{ch:robo76}
\label{ch:Messages_message_get_program}
\index{unsorted!message\_get\_program}\index{Subroutines!message\_get\_program}
\textbf{NAME:}\hspace{0.08in}\begin{Verbatim}
    message_get_program - Get the program name for the ropp_utils message and
                          error handling system.
\end{Verbatim}
\textbf{SYNOPSIS:}\hspace{0.08in}\begin{Verbatim}
    call message_get_program(progname)
\end{Verbatim}
\textbf{DESCRIPTION:}\hspace{0.08in}\begin{Verbatim}
    This routine sets the name of the program that appears in error messages
    issued by the message and error handling system of the ropp_utils library.
\end{Verbatim}
\textbf{INPUTS:}\hspace{0.08in}\begin{Verbatim}
    character(len = *) :: progname
\end{Verbatim}
\textbf{SEE ALSO:}\hspace{0.08in}\begin{Verbatim}
    assert
    message
    message_set_program
    message_set_routine
    message_get_routine
    message_set_addinfo
    message_get_addinfo
\end{Verbatim}
\section{Messages/message\_get\_routine}
\textsl{[ Subroutines ]}

\label{ch:robo77}
\label{ch:Messages_message_get_routine}
\index{unsorted!message\_get\_routine}\index{Subroutines!message\_get\_routine}
\textbf{NAME:}\hspace{0.08in}\begin{Verbatim}
    message_get_routine - Get the routine name for the ropp_utils message and
                          error handling system.
\end{Verbatim}
\textbf{SYNOPSIS:}\hspace{0.08in}\begin{Verbatim}
    call message_get_routine(routine)
\end{Verbatim}
\textbf{DESCRIPTION:}\hspace{0.08in}\begin{Verbatim}
    This routine gets the name of the routine that appears in error messages
    issued by the message and error handling system of the ropp_utils library.
\end{Verbatim}
\textbf{INPUTS:}\hspace{0.08in}\begin{Verbatim}
    character(len = *) :: routine
\end{Verbatim}
\textbf{SEE ALSO:}\hspace{0.08in}\begin{Verbatim}
    assert
    message
    message_set_program
    message_get_program
    message_set_routine
    message_set_addinfo
    message_get_addinfo
\end{Verbatim}
\section{Messages/message\_set\_addinfo}
\textsl{[ Subroutines ]}

\label{ch:robo78}
\label{ch:Messages_message_set_addinfo}
\index{unsorted!message\_set\_addinfo}\index{Subroutines!message\_set\_addinfo}
\textbf{NAME:}\hspace{0.08in}\begin{Verbatim}
    message_set_addinfo - Set an additional information string for the ropp_utils
                          message and error handling system.
\end{Verbatim}
\textbf{SYNOPSIS:}\hspace{0.08in}\begin{Verbatim}
    call message_set_addinfo(addinfo)
\end{Verbatim}
\textbf{DESCRIPTION:}\hspace{0.08in}\begin{Verbatim}
    This routine sets an additional information string that appears in error
    messages issued by the message and error handling system of the ropp_utils
    library.
\end{Verbatim}
\textbf{INPUTS:}\hspace{0.08in}\begin{Verbatim}
    character(len = *) :: addinfo
\end{Verbatim}
\textbf{SEE ALSO:}\hspace{0.08in}\begin{Verbatim}
    assert
    message
    message_set_program
    message_get_program
    message_set_routine
    message_get_routine
    message_get_addinfo
\end{Verbatim}
\section{Messages/message\_set\_program}
\textsl{[ Subroutines ]}

\label{ch:robo79}
\label{ch:Messages_message_set_program}
\index{unsorted!message\_set\_program}\index{Subroutines!message\_set\_program}
\textbf{NAME:}\hspace{0.08in}\begin{Verbatim}
    message_set_program - Set the program name for the ropp_utils message and
                          error handling system.
\end{Verbatim}
\textbf{SYNOPSIS:}\hspace{0.08in}\begin{Verbatim}
    call message_set_program(progname)
\end{Verbatim}
\textbf{DESCRIPTION:}\hspace{0.08in}\begin{Verbatim}
    This routine sets the name of the program that appears in error messages
    issued by the message and error handling system of the ropp_utils library.
\end{Verbatim}
\textbf{INPUTS:}\hspace{0.08in}\begin{Verbatim}
    character(len = *) :: progname
\end{Verbatim}
\textbf{SEE ALSO:}\hspace{0.08in}\begin{Verbatim}
    assert
    message
    message_get_program
    message_set_routine
    message_get_routine
    message_set_addinfo
    message_get_addinfo
\end{Verbatim}
\section{Messages/message\_set\_routine}
\textsl{[ Subroutines ]}

\label{ch:robo80}
\label{ch:Messages_message_set_routine}
\index{unsorted!message\_set\_routine}\index{Subroutines!message\_set\_routine}
\textbf{NAME:}\hspace{0.08in}\begin{Verbatim}
    message_set_routine - Set the routine name for the ropp_utils message and
                          error handling system.
\end{Verbatim}
\textbf{SYNOPSIS:}\hspace{0.08in}\begin{Verbatim}
    call message_set_routine(routine)
\end{Verbatim}
\textbf{DESCRIPTION:}\hspace{0.08in}\begin{Verbatim}
    This routine sets the name of the routine that appears in error messages
    issued by the message and error handling system of the ropp_utils library.
\end{Verbatim}
\textbf{INPUTS:}\hspace{0.08in}\begin{Verbatim}
    character(len = *) :: routine
\end{Verbatim}
\textbf{SEE ALSO:}\hspace{0.08in}\begin{Verbatim}
    assert
    message
    message_set_program
    message_get_program
    message_get_routine
    message_set_addinfo
    message_get_addinfo
\end{Verbatim}
\section{Messages/messages}
\textsl{[ Modules ]}

\label{ch:robo81}
\label{ch:Messages_messages}
\index{unsorted!messages}\index{Modules!messages}
\textbf{NAME:}\hspace{0.08in}\begin{Verbatim}
    messages - The tools90 message- and error handling system.
\end{Verbatim}
\textbf{SYNOPSIS:}\hspace{0.08in}\begin{Verbatim}
    use messages
\end{Verbatim}
\textbf{DESCRIPTION:}\hspace{0.08in}\begin{Verbatim}
    This module provides constants and interfaces to the Tools90
    message and error handling system.
\end{Verbatim}
\textbf{SEE ALSO:}\hspace{0.08in}\begin{Verbatim}
   assert
   message
   message_set_program
   message_get_program
   message_set_routine
   message_get_routine
   message_set_addinfo
   message_get_addinfo
\end{Verbatim}
\section{Modules/ropp\_utils}
\textsl{[ Modules ]}

\label{ch:robo82}
\label{ch:Modules_ropp_utils}
\index{unsorted!ropp\_utils}\index{Modules!ropp\_utils}
\textbf{NAME:}\hspace{0.08in}\begin{Verbatim}
    ropp_utils - The ROPP utils library
\end{Verbatim}
\textbf{SYNOPSIS:}\hspace{0.08in}\begin{Verbatim}
    use ropp_utils
\end{Verbatim}
\textbf{DESCRIPTION:}\hspace{0.08in}\begin{Verbatim}
    This Fortran module provides interfaces required for the use of the
    ROPP utils library
\end{Verbatim}
\textbf{SEE ALSO:}\hspace{0.08in}\begin{Verbatim}
    arrays
    geodesy
    coordinates
    datetimeprogs
    messages
    unitconvert
\end{Verbatim}
\section{Modules/unitconvert}
\textsl{[ Modules ]}

\label{ch:robo83}
\label{ch:Modules_unitconvert}
\index{unsorted!unitconvert}\index{Modules!unitconvert}
\textbf{NAME:}\hspace{0.08in}\begin{Verbatim}
    unitconvert - A simple interface to the unit conversion library.
                 --> BASED ON MARQUARDT COLLECTION LIBRARY ROUTINE
                     FOR INTERFACE TO THIRD-PARTY UDUNITS LIBRARY
\end{Verbatim}
\textbf{SYNOPSIS:}\hspace{0.08in}\begin{Verbatim}
    use unitconvert
\end{Verbatim}
\textbf{DESCRIPTION:}\hspace{0.08in}\begin{Verbatim}
    This module provides a simple interface to a unit conversion library and
    allows straightford conversion between physical units.
\end{Verbatim}
\textbf{SEE ALSO:}\hspace{0.08in}\begin{Verbatim}
    ut_convert
    ropp_unit_conversion
\end{Verbatim}
\section{tools90/nag\_interfaces}
\textsl{[ Functions ]}

\label{ch:robo84}
\label{ch:tools90_nag_interfaces}
\index{unsorted!nag\_interfaces}\index{Functions!nag\_interfaces}
\textbf{NAME:}\hspace{0.08in}\begin{Verbatim}
    nag_interfaces - Interfaces to the NAG f90/f95 f90_unix_* routines.
\end{Verbatim}
\textbf{SYNOPSIS:}\hspace{0.08in}\begin{Verbatim}
    When Fortran 90 routines call standard Unix system calls but are
    compiled with the NAGWare Fortran 90/95 compiler, link the resulting
    object code with a compiled version of this file.
\end{Verbatim}
\textbf{DESCRIPTION:}\hspace{0.08in}\begin{Verbatim}
    The NAG f90 / f95 compiler provides access to most standard Unix
    system routines via a set of modules named f90_unix_*. While this
    is a clear and well defined interface to system routines, it leaves
    source code developed for the NAG compilers incompatible with other
    Fortran compilers (which usually provide access to the same routines
    as part of their standard library), as the corresponding NAG modules
    are missing in other compiler distributions. This causes a lot of
    unnecessary recoding work.

    As an alternative, this file collects (part of) the system routines
    addressed by the various f90_unix_* modules and provides an interface
    to them. Instead of adding the appropriate use f90_unix_* entries
    in each affected source file, this file can be compiled with the NAG
    compiler and linked to the other routines.
\end{Verbatim}
\textbf{NOTES:}\hspace{0.08in}\begin{Verbatim}
    Not all routines are currently implemented; i will add them if they
    will be requested (or I have a need for them).

    Some of the NAG routines do have optional arguments, like ALARM.
    Obviously, this is not implemented. Also, all parameter and type
    definitions available from the NAG modules have been omitted.
\end{Verbatim}
\textbf{EXAMPLE:}\hspace{0.08in}\textbf{SEE ALSO:}\hspace{0.08in}\begin{Verbatim}
    For the NAG compiler:
       f90_unix
       f90_unix_dir
       f90_unix_dirent
       f90_unix_env
       f90_unix_file
       f90_unix_proc
\end{Verbatim}
\textbf{REFERENCES:}\hspace{0.08in}\section{Units/Conversion}
\textsl{[ Subroutines ]}

\label{ch:robo85}
\label{ch:Units_Conversion}
\index{unsorted!Conversion}\index{Subroutines!Conversion}
\textbf{DESCRIPTION:}\hspace{0.08in}\begin{Verbatim}
    The interface to a ROPP-specific unit conversion routine
    is contained in a single subroutine, ut_convert(), and has been coded
    with simplicity in mind, not efficiency.
\end{Verbatim}
\textbf{SEE ALSO:}\hspace{0.08in}\begin{Verbatim}
    ut_convert
\end{Verbatim}
\subsection{Conversion/ut\_convert}
\textsl{[ Conversion ]}
\textsl{[ Subroutines ]}

\label{ch:robo31}
\label{ch:Conversion_ut_convert}
\index{unsorted!ut\_convert}\index{Subroutines!ut\_convert}
\textbf{NAME:}\hspace{0.08in}\begin{Verbatim}
    ut_convert - Convert between physical units.
\end{Verbatim}
\textbf{SYNOPSIS:}\hspace{0.08in}\begin{Verbatim}
    use unitconvert
      ...
    call ut_convert(data, from_unit, converted, to_unit)
\end{Verbatim}
\textbf{DESCRIPTION:}\hspace{0.08in}\begin{Verbatim}
    This module provides a simple interface to the udunits library and
    allows straightford conversion between physical units. If either
    from_unit or to_unit is blank or '1' (indicating no or dimensionless
    quantities) or are the same, output values are merely copied from
    the inputs.
\end{Verbatim}
\textbf{INPUTS:}\hspace{0.08in}\begin{Verbatim}
    data       Integer, real or double precision scalar or array.
    from_unit  Udunits conformant unit specification of the data to be
                  converted.
    to_unit    Udunits conformant unit specification of the target unit.
\end{Verbatim}
\textbf{OUTPUT:}\hspace{0.08in}\begin{Verbatim}
    converted  Converted data; must have same type and shape as data.
\end{Verbatim}
\textbf{EXAMPLE:}\hspace{0.08in}\begin{Verbatim}
    To convert a length from mm to km:

       call ut_convert(length, 'mm', length, 'km')
\end{Verbatim}
